% ------------------------------------------------------------------------
%% abtex2-modelo-trabalho-academico.tex, v-1.7.1 laurocesar
%% Copyright 2012-2013 by abnTeX2 group at http://abntex2.googlecode.com/ 
% ------------------------------------------------------------------------

\documentclass[
	% -- opções da classe memoir --
	12pt,				% tamanho da fonte
	openright,			% capítulos começam em pág ímpar (insere página vazia caso preciso)
	oneside,			% para impressão em verso e anverso. Oposto a oneside
	a4paper,			% tamanho do papel. 
	% -- opções da classe abntex2 --
	chapter=TITLE,		% títulos de capítulos convertidos em letras maiúsculas
	section=TITLE,		% títulos de seções convertidos em letras maiúsculas
	%subsection=TITLE,	% títulos de subseções convertidos em letras maiúsculas
	%subsubsection=TITLE,% títulos de subsubseções convertidos em letras maiúsculas
	% -- opções do pacote babel --
	french,				% idioma adicional para hifenização
	spanish,			% idioma adicional para hifenização
	english,
	brazil,
	]{abntex2/abntex2}
	\renewcommand{\baselinestretch}{1.5} %para customizar o espaço entre as linhas do texto


% -----------------------------------------------------------
% PACOTES
% -----------------------------------------------------------
\usepackage{abntex2/abntex2-cin-ufpe}
\usepackage{pdfpages}           %para incluir pdf como páginas
\usepackage{float}
\usepackage{cmap}				% Mapear caracteres especiais no PDF
\usepackage{lmodern}			% Usa a fonte Latin Modern			
\usepackage[T1]{fontenc}		% Selecao de codigos de fonte.
\usepackage[utf8]{inputenc}		% Codificacao do documento (conversão automática dos acentos)
\usepackage{lastpage}			% Usado pela Ficha catalográfica
\usepackage{indentfirst}		% Indenta o primeiro parágrafo de cada seção.
\usepackage{xcolor}				% Controle das cores
\usepackage{graphicx}			% Inclusão de gráficos
\usepackage{lipsum}				% para geração de dummy text
\usepackage[versalete,alf,abnt-and-type=e,abnt-etal-list=0,abnt-etal-cite=3]{abntex2/abntex2cite} 
\usepackage{multirow}
\usepackage[section]{placeins}
\usepackage{lscape} 
\usepackage{rotating}           %rotates the figures, page
\usepackage{tikz}
\usepackage[section]{placeins}
\usepackage{setspace} 
\usepackage{inconsolata}
\usepackage{listings}
\usepackage{adjustbox}          % ajustar tabela ao tamanho da pagina
\usepackage{subcaption}
\usepackage{tcolorbox}
\usepackage{courier}
\usepackage{comandos}



% -----------------------------------------------------------
% CONFIGURAÇÃO DO DOCUMENTO
% -----------------------------------------------------------
% \usepackage[noframe]{showframe}
% \usepackage{showframe}

%\overfullrule=4mm %para identificar onde existem os alertas de linhas grandes mal formatada pelo LaTex, basta comentar para não aparecer a barra lateral preta na linha em questão.

\renewcommand*\arraystretch{1.2} %para customizar o espaço entre as linhas das tabelas



% -----------------------------------------------------------
% ABREVIATURAS E SIGLAS
% -----------------------------------------------------------
% \usepackage[noredefwarn,acronym]{glossaries} %GLOSSÁRIO
\usepackage[acronym,nonumberlist,nogroupskip,noredefwarn]{glossaries}
% \usepackage{glossary-superragged}

\newcolumntype{L}[1]{>{\raggedright\let\newline\\\arraybackslash\hspace{0pt}}m{#1}}
\newcolumntype{C}[1]{>{\centering\let\newline\\\arraybackslash\hspace{0pt}}m{#1}}
\newcolumntype{R}[1]{>{\raggedleft\let\newline\\\arraybackslash\hspace{0pt}}m{#1}}

\newglossarystyle{modsuper}{%
  %\glossarystyle{super}%
  \setglossarystyle{super}%
  \renewcommand{\glsgroupskip}{}
  
  % put the glossary in a longtable environment:
 \renewenvironment{theglossary}%
  {
    \begin{longtable}
        {L{0.2\textwidth}L{0.8\textwidth}}}%
    {\end{longtable}
  }%
}



% ----------------------------------------------------------
% CORES
% ----------------------------------------------------------
\definecolor{blue}{RGB}{41,5,195}
\definecolor{gray}{rgb}{.4,.4,.4}
\definecolor{gray}{rgb}{0.5, 0.5, 0.5}
\definecolor{graybackground}{rgb}{242,242,242}
\definecolor{greencomment}{RGB}{0, 128, 0}


% ----------------------------------------------------------
% PERSONALIZAÇÃO DO USUÁRIO
% ----------------------------------------------------------
% ----------------------------------------------------------
% DADOS DO TRABALHO - CAPA e FOLHA DE ROSTO
% ----------------------------------------------------------
\titulo{Uma Abordagem Linguística para o Reconhecimento de Línguas de Sinais}
\autor{CLEISON CORREIA DE AMORIM}
\local{Recife}
\data{2022}
\areaconcentracao{\textbf{Área de Concentração}: Inteligência Computacional}
\orientador{\textbf{Orientador}: Cleber Zanchettin}

\instituicao{UNIVERSIDADE FEDERAL DE PERNAMBUCO \\ CENTRO DE INFORMÁTICA \\ PROGRAMA DE PÓS-GRADUAÇÃO EM CIÊNCIA DA COMPUTAÇÃO}
\departamento{Centro de Informática}
\programa{Pós-graduação em Ciência da Computação}
\emailprograma{contato@cin.ufpe.br}
\siteprograma{http://www.cin.ufpe.br}

\tipotrabalho{Dissertação de Mestrado}

\preambulo{Trabalho apresentado ao Programa de Pós-Graduação em Ciência da Computação do Centro de Informática da Universidade Federal de Pernambuco, como requisito para obtenção do grau de Mestre em Ciência da Computação.}



% ----------------------------------------------------------
% COMPILA O ÍNDICE
% ----------------------------------------------------------
\makeindex
% ---


% ----------------------------------------------------------
% ACRONIMOS
% ----------------------------------------------------------
% Lista de acrônimos:
\newacronym{oms}{OMS}{Organização Mundial da Saúde}
\newacronym{cfm}{CFM}{Conselho Federal de Medicina}
\newacronym{ibge}{IBGE}{Instituto Brasileiro de Geografia e Estatística}
\newacronym{ines}{INES}{Instituto Nacional de Educação de Surdos}
\newacronym{slr}{RLS}{Reconhecimento de Língua de Sinais}
\newacronym{ia}{IA}{Inteligência Artificial}
\newacronym{nlp}{PLN}{Processamento de Linguagem Natural}
\newacronym{mt}{TA}{Tradução Automática}

% Línguas de sinais
\newacronym{asl}{ASL}{\textit{American Sign Language} (Língua de Sinais Americana)}
\newacronym{csl}{CSL}{\textit{Chinese Sign Language} (Língua de Sinais Chinesa)}
\newacronym{dgs}{DGS}{\textit{Deutsche Gebärdensprache} (Língua de Sinais Alemã)}
\newacronym{bsl}{BSL}{\textit{British Sign Language} (Língua de Sinais Britânica)}
\newacronym{arsl}{ArSL}{\textit{Arabic Sign Language} (Língua de Sinais Britânica)}
\newacronym{jsl}{JSL}{\textit{Japanese Sign Language} (Língua de Sinais Japonesa)}
\newacronym{isl}{ISL}{\textit{Indian Sign Language} (Língua de Sinais Indiana)}
\newacronym{gsl}{GSL}{\textit{Greek Sign Language} (Língua de Sinais Grega)}
\newacronym{tid}{TID}{\textit{Türk İşaret Dili} (Língua de Sinais Turca)}
\newacronym{ngt}{NGT}{\textit{Nederlandse Gebarentaal} (Língua de Sinais Holandesa)}
\newacronym{vgt}{VGT}{\textit{Vlaamse Gebarentaal} (Língua de Sinais Flamenga)}
\newacronym{lis}{LIS}{\textit{Lingua dei Segni Italiana} (Língua de Sinais Italiana)}
\newacronym{auslan}{AUSLAN}{\textit{Australian Sign Language} (Língua de Sinais Austrália)}
\newacronym{lsa}{LSA}{\textit{Lengua de Señas Argentina} (Língua de Sinais Argentina)}
\newacronym{tsl}{TSL}{\textit{Taiwan Sign Language} (Língua de Sinais de Taiwan)}
\newacronym{pjm}{PJM}{\textit{Polski Język Migowy} (Língua de Sinais Polonesa)}
\newacronym{libras}{Libras}{Língua Brasileira de Sinais}
\newacronym{tch}{TCH}{\textit{Teanga Chomharthaíochta na hÉireann} (Língua de Sinais Irlandesa)}
\newacronym{ksl}{KSL}{\textit{Korean Sign Language} (Língua de Sinais Coreana)}
\newacronym{bisindo}{BISINDO}{\textit{Bahasa Isyarat Indonesia} (Língua de Sinais Indonésia)}

\makenoidxglossaries
\renewcommand*{\glsseeformat}[3][\seename]{\textit{#1}  
\glsseelist{#2}}
\renewcommand*{\glspostdescription}{} % remove trailing dot
\renewcommand{\glsnamefont}[1]{\textbf{#1}}
\renewcommand{\familydefault}{\sfdefault}


% ----------------------------------------------------------
% GLOSSÁRIO
% ----------------------------------------------------------
% % Glossario:

%\newglossaryentry{libras}
%{
%  name=\textit{Libras},
%  description={Língua Brasileira de Sinais},
%  plural=\textit{Libras}
%}




% ----------------------------------------------------------
% INÍCIO DO DOCUMENTO
% ----------------------------------------------------------
\begin{document}

\frenchspacing % Retira espaço extra obsoleto entre as frases.

\imprimircapa
\imprimirfolhaderosto*~
% A ficha deve ser passada pelo setor da biblioteca e sobrescrito no formato PDF
\includepdf[pages=-]{outros/biblioteca/ficha_catalografica.pdf}

%\newpage
% Ata de defesa
\includepdf[pages=-]{outros/biblioteca/ata_defesa.pdf}

% A folha de aprovação deve ser um PDF que a secretaria encaminha sem assinaturas
\includepdf[pages=-]{outros/biblioteca/folha_aprovacao}

% ----------------------------------------------------------
% DEDICATÓRIA
% ----------------------------------------------------------
\begin{dedicatoria}
   \vspace*{\fill}
   % \centering
   % \noindent
   \textit{
      Dedico este trabalho ao meu pai, que partiu ao longo da jornada deste mestrado, mas que deixou um grande exemplo de integridade e simplicidade pelo qual me espelharei em minha caminhada.
   }
\end{dedicatoria}
% ---

% ----------------------------------------------------------
% AGRADECIMENTOS
% ----------------------------------------------------------
\begin{agradecimentos}
    Primeiramente, agradeço a Deus por sua infinita bondade e por me conceber essa oportunidade de completar mais um importante ciclo em minha vida.
    É impossível não recordar como os últimos anos foram difíceis para todos nós, bem como nos trouxeram desafios e perdas que tivemos que superar. 
    No entanto, hoje respirando mais aliviado e com esperanças renovadas, celebro mais essa conquista acadêmica.
    
    Em segundo lugar, agradeço à minha família por todo apoio e compreensão nessa jornada. De um modo especial, sou muito grato aos meus pais por acreditarem, persistirem e ensinarem a cada um de seus filhos que a educação é o legado mais valioso que eles poderiam nos transmitir e que jamais alguém poderá nos tirar.
    Seus ensinamentos são muito sábios e fontes de muitas realizações.

    Por fim, agradeço ao meu orientador por acreditar em meu potencial desde o primeiro momento e por me apoiar compartilhando muito de seu tempo e de seu conhecimento para que eu pudesse concretizar esta pesquisa.
\end{agradecimentos}
\input{outros/epigrafe}
% -----------------------------------------------------------
% PORTUGUÊS
% -----------------------------------------------------------
\begin{resumo}[Resumo]
  \noindent
  A língua de sinais é uma ferramenta essencial na vida do Surdo, capaz de assegurar seu acesso à comunicação, educação e desenvolvimento cognitivo e socio-emocional.
  Na verdade, ela é a principal força que une essa comunidade e o símbolo de identificação entre seus membros.
  Contudo, atualmente o número de indivíduos ouvintes que conseguem se comunicar por meio dessa língua é pequeno e, na prática, isso impõe obstáculos ao cotidiano do Surdo.
  Tarefas simples como utilizar o transporte público, comprar roupas, ir ao cinema ou obter assistência médica acabam se tornando um desafio por conta dessa limitação.
  O Reconhecimento de Língua de Sinais é uma das áreas de pesquisa que objetiva desenvolver tecnologias capazes de reduzir essas barreiras linguísticas e facilitar a comunicação entre ambos indivíduos.
  Apesar disso, ao analisarmos sua evolução ao longo das últimas décadas, percebe-se que seu progresso ainda não é suficiente para disponibilizar soluções efetivamente aplicáveis ao mundo real.
  Isso ocorre principalmente porque várias pesquisas nessa área acabam não apropriando-se ou abordando adequadamente as particularidades linguísticas das línguas de sinais, decorrentes de sua natureza visual.
  % Tendo isso em vista, este trabalho introduz uma abordagem linguística ao reconhecimento computacional da língua de sinais, a qual objetiva estabelecer uma direção capaz de conduzir a avanços mais efetivos para essa área e, consequentemente, contribuir com a superação dos obstáculos hoje enfrentados pelo Surdo.
  Tendo isso em vista, este trabalho apresenta uma abordagem que aplica modelos sequenciais de aprendizagem de máquina para realizar o reconhecimento computacional dos sinais através de seus atributos linguísticos. Além disso, são introduzidos dois novos \textit{datasets} para a língua de sinais, dentre os quais está um \textit{dataset} de atributos linguísticos computados a partir do ASLLVD.
  Com isso, objetiva-se estabelecer uma direção capaz de conduzir a avanços mais efetivos para essa área e, consequentemente, contribuir com a superação dos obstáculos hoje enfrentados pelo Surdo.

  % %FIXME: [cz] acho que faltou dizer o que na pratica vc fez. Falta um paragrafo apresentando seu trabalho, o que o leitor vai encontrar se decidir ler --. [cca5] feito


  % Em meio a tantos desafios, a língua de sinais surge como uma ferramenta poderosa que é capaz de assegurar o desenvolvimento cognitivo, facilitar a comunicação, e possibilitar que esses indivíduos obtenham educação e socio-emocional adequado (World Health Organization, 2021).

  % A língua de sinais é a língua utilizada pela maioria dos Surdos em sua vida diária. Muito mais do que isso, ela é a principal força que une essa comunidade e o símbolo de identificação entre seus membros

  % Apesar disso, Bragg et al. (2019), Agência Senado (2019) observam que atualmente ainda são poucos os ouvintes que conseguem se comunicar por meio dessa língua. Isso traz obstáculos adicionais aos Surdos e transforma muitas de suas atividades corriqueiras num grande desafio. 
  % Por exemplo, no transporte público é difícil solicitar ajuda ou ter acesso às instruções divulgadas nos alto-falantes; em lojas, é raro encontrar vendedores preparados para interagir através dessa língua ou que não os trate com preconceito; no cinema, eles apenas podem consumir filmes estrangeiros, uma vez que os nacionais não dispõem de legenda; no serviço de saúde, não são raros os relatos de pacientes que saem de consultas com prescrições médicas erradas porque o médico não entendeu corretamente seus sintomas; entre outras situações

  % Isso deve-se, de um modo geral, a um conjunto de particularidades que as línguas de sinais apresentam quando comparadas às línguas faladas, bem como à forma com que as pesquisas em RLS têm abordado elas, afirmam Bragg et al. (2019), Cooper, Holt e Bowden (2011). Diferentemente das faladas, as línguas sinalizadas possuem uma natureza visual e transmitem significado através de múltiplos canais ao mesmo tempo, como mãos, corpo, face, entre outros de granularidade ainda menor. Essa natureza faz com que sua linguística seja estruturada de uma forma muito específica, demandando que novas técnicas sejam desenvolvidas para abordar tais particularidades. Contudo, um grande número de pesquisas nessa área não aborda essa linguística ou suas complexidades e, como consequência, acabam não produzindo avanços realmente efetivos.

  % Tendo em vista isso, este trabalho busca contribuir com a área de Reconhecimento de Língua de Sinais (RLS) por meio de uma proposta que aborda a língua de sinais através de sua linguística e aplica técnicas de Processamento de Linguagem Natural (PLN) para estabelecer uma direção que possa conduzir a avanços efetivos na área e, consequentemente, ajude a superar alguns dos desafios cotidianos atualmente encarados pelos Surdos.

  \vspace{\onelineskip}

  \noindent
  \textbf{Palavras-chaves}: Língua de Sinais. Linguística. Processamento de Linguagem Natural.
\end{resumo}



% -----------------------------------------------------------
% INGLÊS
% -----------------------------------------------------------
\begin{resumo}[Abstract]
  \begin{otherlanguage*}{english}
    \noindent
    Sign language is an essential resource to ensure that the Deaf have access to communication, education, as well as to cognitive and socio-emotional development. In fact, it is the main force that unites this community and an identifying trait among its members.
    On the other hand, the number of hearing individuals who are able to communicate through this language is currently small and, in practice, this imposes obstacles to the daily life of the Deaf.
    Simple tasks like using public transportation, shopping for clothes, going to the movies, or getting medical assistance end up becoming challenges due to such limitation.
    The Sign Language Recognition, in turn, is one of the research areas dedicated to developing technologies that aim to reduce such language barriers and facilitate communication between these individuals.
    However, when analyzing its evolution over the last decades, we realize that it has not progressed enough to provide solutions effectively applicable to the real world.
    This is mainly because several researches in this field do not appropriate or address the linguistic particularities presented by the sign languages, which stem from their visual nature.
    % With this in mind, this work introduces a linguistic approach to sign language recognition that aims to establish a direction that can lead to more effective advances in this research area and, consequently, contribute to overcoming the obstacles faced by the Deaf today.
    Considering this problem, the present work introduces an approach that adopts sequential machine learning models to recognize signs through their linguistic attributes. In addition, we introduce two new sign language datasets, among which is a novel dataset of linguistic attributes computed from the ASLLVD.
    Thus, we aim to establish a direction that can lead to more effective advances in this research area and, consequently, contribute to overcoming the obstacles faced by the Deaf today.

    \vspace{\onelineskip}

    \noindent
    \textbf{Keywords}: Sign Language. Linguistics. Natural Language Processing.
  \end{otherlanguage*}
\end{resumo}



% ----------------------------------------------------------
% LISTA DE FIGURAS
% ----------------------------------------------------------
\pdfbookmark[0]{\listfigurename}{lof}
\listoffigures*
\cleardoublepage


% ----------------------------------------------------------
% LISTA DE CÓDIGOS
% ----------------------------------------------------------
\lstdefinelanguage{ASLDataset}{
  sensitive=true,
  commentstyle=\color{greencomment},
  breaklines=true,
  keywords={ x, y, z },
  otherkeywords={ label, gloss, consultant, session, scene, frame_start, frame_end, passive_arm, fps, mode, name, value, score, frames, frame_index, handshape_dh, handshape_dh_start, handshape_dh_end, handshape_ndh, handshape_ndh_start, handshape_ndh_end, phono_attributes, skeleton, movement_dh, movement_ndh, orientation_dh, orientation_ndh, non_manual, mouth_opening, body, face, hand_left, hand_right },
  keywordstyle=\color{blue},
  upquote=true,
  stringstyle=\color{purple},
  showstringspaces=false,
  morestring=[b]',
  morestring=[b]",
  comment=[l]{//},
  morecomment=[s]{/*}{*/},
  literate=
    *{0}{{{\color{teal}0}}}{1}
    {1}{{{\color{teal}1}}}{1}
    {2}{{{\color{teal}2}}}{1}
    {3}{{{\color{teal}3}}}{1}
    {4}{{{\color{teal}4}}}{1}
    {5}{{{\color{teal}5}}}{1}
    {6}{{{\color{teal}6}}}{1}
    {7}{{{\color{teal}7}}}{1}
    {8}{{{\color{teal}8}}}{1}
    {9}{{{\color{teal}9}}}{1}
    {.}{{{\color{teal}.}}}{1}
}

\lstdefinestyle{mystyle}{
  backgroundcolor=\color{graybackground},
  basicstyle=\fontsize{9}{9}\ttfamily,
  captionpos=t,
  keepspaces=true,
  numbers=left,
  numbersep=10pt,
  numberstyle=\tiny\color{gray},
  showspaces=false,
  showstringspaces=false,
  showtabs=false,
  tabsize=4,
  frame=single,
  rulecolor=\color{gray},
}

\lstset{style=mystyle}


% Altera o nome padrão do rótulo usado no comando \autoref{}
\renewcommand{\lstlistingname}{Código-fonte}

% Altera o rótulo a ser usando no elemento pré-textual "Lista de código"
\renewcommand{\lstlistlistingname}{Lista de códigos-fontes}

\pdfbookmark[0]{\lstlistlistingname}{lol} % caso não tenha código fonte, comente esta linha 
\counterwithout{lstlisting}{chapter}

% Configura a 'Lista de Códigos' conforme as regras da ABNT (para abnTeX2)
\begingroup\makeatletter
\let\newcounter\@gobble\let\setcounter\@gobbletwo
\globaldefs\@ne \let\c@loldepth\@ne
\newlistof{listings}{lol}{\lstlistlistingname}
\newlistentry{lstlisting}{lol}{0}
\endgroup

\renewcommand{\cftlstlistingaftersnum}{\hfill--\hfil}

\let\oldlstlistoflistings\lstlistoflistings
{
  \let\oldnumberline\numberline
  \newcommand{\algnumberline}[1]{Código-fonte~#1~\enspace--~\enspace}
  \renewcommand{\numberline}{\algnumberline}

  \begin{KeepFromToc}
    \lstlistoflistings
  \end{KeepFromToc}
}
\cleardoublepage



% ----------------------------------------------------------
% LISTA DE QUADROS
% ----------------------------------------------------------
% \pdfbookmark[0]{\listofquadrosname}{loq} % caso não tenha quadros, comente esta linha 
% \listofquadros* % caso não tenha quadros, comente esta linha 
% \cleardoublepage



% ----------------------------------------------------------
% LISTA DE TABELAS
% ----------------------------------------------------------
\pdfbookmark[0]{\listtablename}{lot}
\listoftables*
\cleardoublepage




% ----------------------------------------------------------
% LISTA E ABREVIATURAS E SIGLAS
% ----------------------------------------------------------
%\printglossary[type=\acronymtype,title={\listadesiglasname},nonumberlist]
%\printglossaries
% compile uma vez com o comando \printglossaries e depois compile novamente com o comando \printglossaries comentado para as páginas glossário e siglas serem ocultadas.


% \setglossarystyle{modsuper}
\printnoidxglossary[style=modsuper,type=\acronymtype,title={\listadesiglasname},nonumberlist]
% \printglossary[style=super, type=\acronymtype]
\cleardoublepage



% ----------------------------------------------------------
% LISTA DE SIMBOLOS
% ----------------------------------------------------------
% \input{outros/simbolos}


% ----------------------------------------------------------
% SUMÁRIO
% ----------------------------------------------------------
\pdfbookmark[0]{\contentsname}{toc}
\tableofcontents*
% \begingroup\intoctrue
% \tableofcontents*
% \endgroup
\cleardoublepage

% \setcounter{page}{13}
\setcounter{tocdepth}{2}
\setcounter{table}{0}



% ----------------------------------------------------------
% ELEMENTOS TEXTUAIS
% ----------------------------------------------------------
\textual
% Introdução
\chapter{Introdução}
\label{cap:introducao}

Segundo a \citeonline{who-2021-report-hearing}, o mundo possui hoje cerca de 1,5 bilhões de pessoas com algum grau de perda auditiva, o que corresponde a 19\% da população mundial.
Desse número, 450 milhões se referem a perda de grau moderado a total\footnote{
    O grau de perda refere-se à intensidade mínima de som que um ouvido pode detectar.
    Na perda leve, essa intensidade está entre 20 e 34 dB; na moderada, entre 35 e 49 dB; na moderadamente severa, entre 50 e 64 dB; na severa, entre 65 e 79 dB; na profunda, entre 80 e 94 dB; e na total (ou surdez), 95 dB ou mais.
    \cite[p. 38]{who-2021-report-hearing}
}, as quais necessitarão de acesso a cuidados auditivos e outros serviços de reabilitação.
No Brasil, o número dos que têm perda auditiva é de 10,7 milhões e o dos que apresentam perda moderada a total é de 2,3 milhões, de acordo com \citeonline{ebc-2019-10-milhoes-pessoas, ibge-2021-pns, ibge-2021-projecao-populacao}.


Ao analisar os dispositivos e intervenções que são capazes de auxiliar no diagnóstico e reabilitação desses indivíduos, a \citeonline{who-2021-report-hearing} afirma que as últimas décadas testemunharam avanços revolucionários, como nos campos da tecnologia auditiva, diagnóstico e telemedicina, com inovações que possibilitam problemas relacionadas à audição serem identificadas em qualquer idade e ambiente.
Contudo, uma vez que a grande maioria daqueles com perda auditiva vive em locais de baixa renda, onde profissionais especializados e serviços para cuidados auditivos não estão comumente disponíveis, existe uma disparidade no acesso a tais recursos:

\begin{citacao}
    Cerca de 78\% dos países de baixa renda têm menos de um otorrinolaringologista por milhão de habitantes; 93\% têm menos de um audiologista por milhão; 17\% têm um ou mais fonoaudiólogos por milhão; e 50\% têm um ou mais professores para pessoas com perda auditiva por milhão.

    Mesmo em países com proporções relativamente altas desses especialistas, há uma distribuição desigual que, além de trazer desafios para essa população, impõe uma sobrecarga excessiva aos quadros que prestam esses serviços.~\cite{opas-2021-oms-estima}
\end{citacao}

Essa carência de cuidados adequados gera deficiências estruturais que dificultam o acesso a oportunidades básicas como educação e emprego, e resultam numa pior qualidade de vida para eles.
Isso estende-se a todas as gerações, afirmam \citeonline{ebc-2021-oms-estima,ebc-2019-10-milhoes-pessoas}:

\begin{citacao}
    Essas deficiências estruturais refletem-se na educação das crianças. Uma criança que ouve mal, aprende mal e torna-se um adulto menos capaz do que poderia ser, e assim por diante.~\cite{ebc-2021-oms-estima}
\end{citacao}

\begin{citacao}
    Uma vez que esses indivíduos têm menos oportunidades de estudar e acessar o mercado de trabalho do que a população ouvinte, o dinheiro para conseguir o aparelho auditivo é ainda mais difícil. Esse conjunto de preconceitos acaba criando um círculo vicioso que não possibilita que eles tenham as mesmas oportunidades de se dar bem na vida.~\cite{ebc-2019-10-milhoes-pessoas}
\end{citacao}


Em meio a tantos desafios, a língua de sinais surge como uma ferramenta poderosa que é capaz de assegurar o desenvolvimento cognitivo, facilitar a comunicação, e possibilitar que esses indivíduos obtenham educação e desenvolvimento socio-emocional adequado \cite{who-2021-report-hearing}.
Ela pode ser aprendida através de membros da própria família, da comunidade Surda, de conteúdos geralmente gratuitos na internet (como livros, cursos e vídeos) disponibilizados por instituições como o \acrshort{ines}\footnote{
    O \acrfull{ines}, fundado em 1857, é o centro de referência nacional que subsidia a formulação de políticas públicas para o Surdo. Ele atende a estudantes da educação infantil até o ensino superior e também apoia a pesquisa de novas metodologias de ensino nesse contexto. \cite{mec-2021-conheca-ines}
} ou em escolas públicas que ofertam seu ensino.
Por conta disso, ela torna-se uma alternativa acessível para a inclusão desses indivíduos, uma vez que muitas das barreiras como a demanda por recursos financeiros ou profissionais especializados são removidas.


\citeonline{stewart-2021-barrons-asl} afirmam que a língua de sinais é a chave para acessar a cultura Surda.
O termo Surdo (escrito com ``s'' capitalizado), por sua vez, não refere-se apenas a uma condição clínica, mas a um grupo de indivíduos que, além de possuírem perda auditiva, utilizam a língua de sinais como principal meio de comunicação e compartilham experiências culturais associadas à surdez e ao uso dessa língua.
Há um aspecto cultural fundamental, reiteram \citeonline{pereira-2011-conhecimento-alem-sinais}, acompanhado de um forte sentimento de identidade grupal que faz com que esses indivíduos compartilhem valores, crenças, comportamentos e uma língua própria.


Apesar disso, \citeonline{bragg-2019-slr-interdisciplinary,senado-2019-baixo-alcance-lingua-sinais} observam que atualmente ainda são poucos os ouvintes que conseguem se comunicar por meio dessa língua.
Isso traz obstáculos adicionais aos Surdos e transforma muitas de suas atividades corriqueiras num grande desafio.
Por exemplo, no transporte público é difícil solicitar ajuda ou ter acesso às instruções divulgadas nos alto-falantes;
em lojas, é raro encontrar vendedores preparados para interagir através dessa língua ou que não os trate com preconceito;
no cinema, eles apenas podem consumir filmes estrangeiros, uma vez que os nacionais não dispõem de legenda;
no serviço de saúde, não são raros os relatos de pacientes que saem de consultas com prescrições médicas erradas porque o médico não entendeu corretamente seus sintomas; entre outras situações.


Para contribuir com a superação desses desafios é importante, entre outros fatores, que a comunidade acadêmica esteja mobilizada para impulsionar o desenvolvimento de alternativas e tecnologias.
O \acrfull{slr} é um dos campos de pesquisa que se dedica a desenvolver algumas delas. Segundo \citeonline{wadhawan-2019-slr-literature-review}, trata-se de uma área colaborativa e multidisciplinar que envolve \acrlong{cv} \cite{szeliski-2022-computer-vision}, \acrlong{nlp} \cite{jurafsky-2022-speech-lang-processing}, Reconhecimento de Padrões \cite{bishop-2006-pattern-recognition} e Linguística \cite{quadros-2004-estudos-linguisticos} para construir métodos e algoritmos capazes de identificar sinais produzidos pelo articulador e compreender seu significado.
Por meio deles, seria possível reduzir a barreira linguística entre Surdos e ouvintes permitindo que mensagens transmitidas utilizando-se a língua de sinais fossem transcritas automaticamente e compreendidas por aqueles que não a conhecem.


No entanto, apesar do potencial que o \acrshort{slr} possui, \citeonline{selvaraj-2022-openhands,yin-2021-sl-in-nlp,cooper-2011-slr} acreditam que o progresso apresentado por essa área ao longo das últimas décadas foi insuficiente para conduzir a avanços expressivos:

\begin{citacao}
    Quando comparado com a pesquisa de \acrlong{nlp} baseada em texto e fala, o progresso das pesquisas para línguas de sinais está significativamente atrasado. \cite[tradução nossa]{selvaraj-2022-openhands,yin-2021-sl-in-nlp}
\end{citacao}


\begin{citacao}
    Enquanto sistemas de reconhecimento da fala avançaram ao ponto de estarem comercialmente disponíveis, o reconhecimento de sinais ainda está em sua infância.
    Atualmente, todos os serviços comerciais de tradução de sinais são baseados em humanos e requerem que pessoal especializado esteja disponível, o que os tornam caros e pouco acessíveis. \cite[tradução nossa]{cooper-2011-slr}
\end{citacao}



Isso deve-se, de um modo geral, a um conjunto de particularidades que as línguas de sinais apresentam quando comparadas às línguas faladas, bem como à forma com que as pesquisas em \acrshort{slr} têm abordado elas, afirmam \citeonline{bragg-2019-slr-interdisciplinary,cooper-2011-slr}.
Diferentemente das faladas, as línguas sinalizadas possuem uma natureza visual e transmitem significado através de múltiplos canais ao mesmo tempo, como mãos, corpo, face, entre outros de granularidade ainda menor.
Essa natureza faz com que sua linguística seja estruturada de uma forma muito específica, demandando que novas técnicas sejam desenvolvidas para abordar tais particularidades.


Contudo, segundo \citeonline{cooper-2011-slr,yin-2021-sl-in-nlp}, um grande número de pesquisas nessa área trata o \acrshort{slr} como uma tarefa de reconhecimento de gestos não-estruturados ou poses de mãos estáticas, que são mapeados a partir de imagens RGB, dados de luvas eletrônicas ou coordenadas dos corpos dos indivíduos.
Isso faz com que elas deixem de abordar aspectos essenciais da língua de sinais -- como sua linguística e suas particularidades -- e desviem o foco para um conjunto de desafios pertinentes à área de \acrfull{cv}, como a detecção, segmentação e rastreamento de partes do corpo; a interação entre mãos e oclusões decorrentes disso; variações de tom de pele; entre outros que comumente já são abordados ou solucionados por outras subáreas da \acrshort{cv}.
Como consequência, essas pesquisas acabam não produzindo avanços realmente efetivos para a \acrshort{slr}.


Tendo em vista isso, este trabalho busca contribuir com a área de \acrfull{slr} por meio de uma proposta que aborda a língua de sinais através de sua linguística, pela introdução de um novo \textit{dataset} de atributos linguísticos e pela adoção de técnicas de \acrfull{nlp} nesse contexto afim de estabelecer uma direção que possa conduzir a avanços efetivos na área e, consequentemente, ajude a superar alguns dos desafios cotidianos atualmente encarados pelos Surdos.


\section{Objetivos}
\label{sec:introducao-objetivos}

O objetivo geral deste trabalho consiste em propor uma abordagem de \acrlong{slr} baseada na linguística, a qual considere as complexidades de sua natureza visual e preencha algumas das lacunas deixadas em aberto por pesquisas na área, afim de contribuir com avanços mais efetivos.

Como objetivos específicos, este trabalho busca alcançar:

\begin{itemize}
    \item Propor uma estratégia para computar atributos linguísticos a partir de \textit{datasets} existentes da língua de sinais;

    \item Disponibilizar um \textit{dataset} de atributos linguísticos, o qual atualmente é inexistente, para suportar o desenvolvimento de novas técnicas para as línguas de sinais;

    \item Identificar, aplicar e avaliar algoritmos que possibilitem abordar o reconhecimento da língua de sinais através de sua linguística, acomodando as complexidades de sua natureza visual.
    
\end{itemize}


\section{Metodologia}
\label{sec:introducao-metodologia}

A metodologia aplicada neste trabalho buscou primeiro compreender os desafios atuais do Surdo, da língua de sinais e da área de pesquisa para, em seguida, estabelecer e avaliar uma proposta que aborde adequadamente as lacunas encontradas.

% A metodologia aplicada nesta dissertação concentra-se em compreender os desafios atuais da área de pesquisa para assim introduzir uma proposta que suporte avanços futuros coerentes com as necessidades do mundo real.
% As etapas percorridas aqui podem ser sumarizadas como:

As etapas percorridas para isso compreendem:

\begin{itemize}
    \item Revisão do panorama do Surdo, das línguas de sinais e de sua linguística;
    \item Revisão da área de \acrlong{slr} e das lacunas existentes;
    \item Elaboração de uma proposta que aborde as lacunas acima e produza artefatos que suportem novas pesquisas nessa direção;
    \item Realização de experimentos e análise dos resultados.
\end{itemize}

% \begin{itemize}
%     \item Revisão do panorama atual da deficiência auditiva e do papel que as línguas de sinais desempenham aqui;
%     \item Revisão do panorama das pesquisas atuais em processamento de língua de sinais e das lacunas que têm limitado progressos mais expressivos na área.
%     \item Desenvolvimento de uma proposta que aborde as lacunas acima, contribuindo para preencher algumas delas e produzindo artefatos que suportem novas pesquisas a evoluir nessa direção;
%     \item Realização de experimentos e análise dos resultados coletados.
% \end{itemize}

% - análise do panorama atual das línguas de sinais 
% - análise do panorama atual das pesquisas na área e identificação de principais lacunas para seu avanço
% - revisão da literatura das línguas de sinais e de sua linguísticas
% - seleção de um conjunto de atributos linguísticos
% - análise de técnicas algébricas para suportar o processamento de atributos linguísticos selecionados
% - seleção de um modelos sequenciais de aprendizagem de máquina para os experimentos
% - execução dos experimentos e análise dos resultados coletados

\section{Organização do trabalho}
\label{sec:introducao-organizacao-trabalho}

Além do capítulo atual, esta dissertação está estruturada em mais quatro capítulos, que estão organizados da seguinte forma:

O \autoref{cap:fundamentacao} apresenta o referencial teórico, que contém conceitos importantes que fundamentam a dissertação.
A abordagem adotada nesta pesquisa é discutida no \autoref{cap:metodos}, no qual também são apresentadas as hipóteses e técnicas utilizadas, bem como a preparação dos experimentos realizados.
No \autoref{cap:avaliacao}, por sua vez, são analisados os resultados desses experimentos.
Por fim, no \autoref{cap:consideracoes-finais} são discutidas as conclusões e um levantamento de propostas para trabalhos futuros, com base nas descobertas obtidas.


% Fundamentação
\chapter{Fundamentação teórica}
\label{cap:fundamentacao}

Este capítulo apresenta os principais conceitos necessários para compreender o contexto das línguas de sinais e a abordagem proposta por esta pesquisa.
Primeiro, introduziremos na \autoref{sec:lingua-sinais} uma discussão acerca do Surdo e da língua de sinais.
Em seguida, na \autoref{sec:linguistica}, nos aprofundaremos na linguística e nas particularidades dessa língua.
Na \autoref{sec:slr}, abordaremos o panorama atual e os desafios existentes para a área de \acrfull{slr}.
Por fim, na \autoref{sec:modelos-sequenciais}, discutiremos sobre os modelos sequenciais de \acrlong{dl}.


% Este capítulo apresenta os principais conceitos necessários para compreender o trabalho desenvolvido nesta pesquisa.
% Este capítulo aborda o conteúdo teórico necessário para desenvolver a solução para o problema descrito na introdução:

% Primeiro, abordaremos uma visão geral acerca dos Surdos e das línguas 
% Em seguida, ... 
% Por fim, ... 


% - Línguas de sinais
% - Linguística da língua de sinais
% - Estado atual do SLR (survey com avanços atuais)

% - Reconhecimento de línguas faladas e texto x línguas de sinais
% - NLP: discutir evolução no tempo das abordagens utilizadas (para texto e voz) 
% - Utilização da fonologia + semântica das palavras para reconhecimento?

% - Modelos sequenciais (aprendizagem de máquina)
%     - Transformer: breve introdução (arquitetura e funcionamento)
%     - RNNs (GRU, LSTM, etc)



% Língua de sinais
\section{Línguas de sinais}
\label{sec:lingua-sinais}

A língua de sinais é a língua utilizada pela maioria dos Surdos em sua vida diária. Ela é a principal força que une a comunidade Surda, e o  símbolo de identificação entre seus membros~\cite{pereira-2011-conhecimento-alem-sinais}. 

% A língua de sinais é a língua utilizada pela maioria dos Surdos em sua vida diária. É a principal força que une a comunidade Surda, o  símbolo de identificação entre seus membros. 
%\cite{pereira-2011-conhecimento-alem-sinais}


Quando citamos Surdos ou comunidade Surda (com ``s'' capitalizado), estamos nos referindo ao grupo de indivíduos que possuem perda auditiva, utilizam a língua de sinais como principal meio de comunicação e compartilham experiências associadas à perda auditiva e ao uso da língua de sinais. Esses três elementos estão interconectados e não pode-se definir comunidade Surda sem considerá-los em conjunto. Por exemplo, ter uma perda auditiva não significa que uma pessoa saiba automaticamente sinalizar e, consequentemente, tenha acesso às experiências culturais associadas à comunidade Surda~\cite{stewart-2021-barrons-asl}.

% Having a hearing loss does not mean that a person automatically knows how to sign. If a deaf person does not know sign language, then that person will not be able to access the varied cultural experiences associated with the Deaf community. Communication is basic, and ASL is the communication of the Deaf community
% In fact, having a hearing loss does not mean that a person is a member of the Deaf community, although it is certainly an important requirement.


\begin{citacao}
    Não há como apontar para uma pessoa sentada lendo uma revista em um saguão que você nunca conheceu antes e dizer: ``essa pessoa é Surda''. Mesmo que a pessoa esteja usando aparelhos auditivos, não sabemos com qual comunidade ela se identifica. Da mesma forma, não importa se a pessoa é europeia, afro-americana, asiática ou de outra origem étnica. A idade não é relevante, nem a classe social ou o gênero da pessoa.
    A comunidade Surda não é moldada por nenhuma dessas características. De fato, ter uma perda auditiva não significa que uma pessoa seja membro da comunidade Surda, embora certamente seja um requisito importante.~\cite[tradução nossa]{stewart-2021-barrons-asl}
    
    % There is no way that you can point to a person sitting and reading a magazine in a lobby whom you have never met before and say, "that person is Deaf". Even if the person is wearing hearing aids, we don't know which community the person identifies with. Similarly, it's not important whether the person is European, African-American, Asian, or of some other ethnic origin. Age is not relevant, and neither is the social class or gender of the person. 
    % The Deaf community is not shaped by any of these characteristics. In fact, having a hearing loss does not mean that a person is a member of the Deaf community, although it is certainly an important requirement.
\end{citacao}


Existe um aspecto cultural fundamental, acompanhado de um forte sentimento de identidade grupal com a comunidade Surda, que fazem com que a surdez seja percebida como uma diferença e não mais como uma deficiência, tornando o grau de perda auditiva um fator irrelevante aqui. Como ocorre em qualquer cultura, esses indivíduos compartilham valores, crenças, comportamentos e, além disso, uma língua própria que nesse contexto é diferente daquela utilizada majoritariamente em sociedade~\cite{pereira-2011-conhecimento-alem-sinais}.

% Pertencer à comunidade Surda pode ser definido pelo domínio da língua de sinais e pelos sentimentos de identidade grupal, fatores que consideram a surdez  como uma diferença, e não como uma deficiência.

% Self-identification
% Deaf ("big D"):
% - identify as a culturally Deaf and part of the Deaf community
% - take pride in Deaf identity
% - may have an auditory device, such as cochlear implant, hearing aid, or FM system
% - may have a more severe hearing loss
% - use sign language as their primary source of communication
% - most likely attend a Deaf school/program
% - feel more comfortable in the Deaf world


\citeonline{stewart-2021-barrons-asl} consideram que a língua de sinais é a chave para acessar a cultura Surda. Aprender uma língua de sinais não é simplesmente aprender uma nova língua; é também sobre receber acesso. Embora seja possível aprender sobre qualquer cultura lendo sobre ela, adquirimos uma compreensão mais profunda quando somos capazes de experimentá-la ou ouvir relatos em primeira mão das pessoas inseridas nela.



Historicamente, no entanto, os Surdos tiveram sua identidade estigmatizada e se sentiram desvalorizados pela sociedade ouvinte, que não aceitava a língua de sinais e a considerava como sendo meramente mímica ou gestos. De acordo com \citeonline{hill-2019-sign-languages}, até há relativamente pouco tempo essas comunidades foram ditas (explícita e implicitamente) que sua ``comunicação de sinais'' era inferior, quebrada, sem importância ou insuficiente. Os sistemas educacionais e a comunidade majoritariamente ouvinte enfatizariam o valor de aprender a língua falada, mesmo às custas da língua de sinais~\cite{hill-2019-sign-languages}. 

De fato, tais atitudes persistem, tanto em áreas onde a língua de sinais não foi profundamente estudada linguisticamente quanto em áreas onde foi estudada, mas o foco para o avanço econômico está na língua falada. 
\citeonline{pereira-2011-conhecimento-alem-sinais} comentam que muitos ouvintes tentam  diminuir os Surdos para que vivam isolados e tendo de assumir a  cultura ouvinte como se ela fosse a única existente; ser ``normal'' significaria ouvir e falar oralmente. Segundo as autoras, os ouvintes não prestam atenção aos Surdos e, consequentemente, não acreditam que eles sejam capazes de estudar em faculdades ou realizar mestrado e doutorado, por exemplo. Os indivíduos ouvintes veem os Surdos com curiosidade e, às vezes, zombam por eles serem diferentes~\cite{pereira-2011-conhecimento-alem-sinais, hill-2019-sign-languages}. 

São muitas as lutas e histórias nas comunidades Surdas, nas quais esses indivíduos tem se unido contra práticas que não respeitam sua cultura. Essas lutas tem conduzido a várias vitórias sobretudo nos últimos anos, como o reconhecimento das línguas de sinais como línguas oficiais em seus países (no Brasil, por exemplo, a \acrfull{libras} foi reconhecida em 2002; nos Estados Unidos, a \acrfull{asl} foi reconhecida em 1989), o direito a tradutores e intérpretes em eventos e canais públicos de comunicação e o acesso a uma educação bilíngue para as crianças Surdas, entre outras conquistas~\cite{pereira-2011-conhecimento-alem-sinais, brasil-2002-lei10436, jay-2011-dont-just-sign}.



% No entanto, as línguas de sinais naturais das comunidades surdas são completamente linguísticas, governadas por regras, capazes de expressar qualquer coisa e totalmente valiosas. \cite{pereira-2011-conhecimento-alem-sinais, hill-2019-sign-languages}.

% \cite{pereira-2011-conhecimento-alem-sinais}
% os Surdos tiveram, historicamente,  sua identidade estigmatizada e se sentiram desvalorizados pela  sociedade ouvinte, que não aceitava a língua de sinais, considerada apenas mímica e gesto .
% Na história, constata-se que os Surdos sofreram perseguições  pelas pessoas ouvintes, que não aceitavam as diferenças e exigiam  uma cultura única por meio do modelo  ouvintista ou ouvintismo . São muitas  as lutas e histórias nas comunidades Surdas, em que o povo Surdo se une contra  as práticas dos ouvintes que não respeitam a cultura Surda (Strobel, 2008) .  
% Ainda hoje, muitos ouvintes tentam  diminuir os Surdos para que vivam isolados e tendo de assumir a  cultura ouvinte, como se esta fosse uma cultura única; ser “normal” para a sociedade significa ouvir e falar oralmente . Os ouvintes não prestam atenção aos Surdos que se comunicam por  meio da Libras . Consequentemente, não acreditam que os Surdos sejam capazes de estudar em faculdade ou realizar mestrado e  doutorado, por exemplo . “Os sujeitos ouvintes veem os sujeitos  surdos com curiosidade e, às vezes, zombam por eles serem diferentes” (Strobel, 2008, p . 22) .  
% A luta dos Surdos tem conduzido a várias vitórias, como o reconhecimento da Libras, o direito a tradutores e intérpretes da  língua brasileira de sinais–língua portuguesa e a uma educação  bilíngue para as crianças Surdas, que contemple a Libras e o português, este na modalidade escrita, entre muitas outras conquistas 


%\cite{hill-2019-sign-languages}
% It is important to recognize the connection between sign languages and Deaf communities. Until relatively recently, Deaf communities have been told (explicitly and implicitly) that their “sign communication” was inferior, broken, unimportant, or insufficient. Educational systems and the broader hearing majority community would stress the value of learning the spoken language, even at the expense of the sign language. In fact, such attitudes persist, both in areas where the national sign language has not been deeply studied linguistically and in areas where it has been studied but the focus for economic advancement is on the spoken language. However, the natural sign languages of Deaf communities are completely linguistic, rule-governed, capable of expressing anything, and fully worthwhile. We unreservedly endorse such affirmations of the value of sign languages and promote their use in all aspects of the lives of Deaf people. 


% \cite{pereira-2011-conhecimento-alem-sinais}
% [...] os Surdos tiveram, historicamente,  sua identidade estigmatizada e se sentiram desvalorizados pela  sociedade ouvinte, que não aceitava a língua de sinais, considerada apenas mímica e gesto . O uso ou não da língua de sinais  seria, portanto, o que definiria basicamente a identidade do sujeito, que só seria adquirida quando em contato com outro Surdo .  O que ocorre, segundo Santana e Bérgamo (2005), é que, nesse  contato com outro Surdo que também use a língua de sinais, surgem novas possibilidades interativas, de compreensão, de diálogo  e de aprendizagem que não são possíveis por meio apenas da linguagem oral . A aquisição de uma língua — e de todos os mecanismos afeitos a ela — faz creditar à língua de sinais a capacidade  de ser a única que pode oferecer uma identidade ao Surdo. 
%
% [Pág 33]:
% Na história, constata-se que os Surdos sofreram perseguições  pelas pessoas ouvintes, que não aceitavam as diferenças e exigiam  uma cultura única por meio do modelo  ouvintista ou ouvintismo . São muitas  as lutas e histórias nas comunidades Surdas, em que o povo Surdo se une contra  as práticas dos ouvintes que não respeitam a cultura Surda (Strobel, 2008) .  
% Ainda hoje, muitos ouvintes tentam  diminuir os Surdos para que vivam isolados e tendo de assumir a  cultura ouvinte, como se esta fosse uma cultura única; ser “normal” para a sociedade significa ouvir e falar oralmente . Os ouvintes não prestam atenção aos Surdos que se comunicam por  meio da Libras . Consequentemente, não acreditam que os Surdos sejam capazes de estudar em faculdade ou realizar mestrado e  doutorado, por exemplo . “Os sujeitos ouvintes veem os sujeitos  surdos com curiosidade e, às vezes, zombam por eles serem diferentes” (Strobel, 2008, p . 22) .  
% A luta dos Surdos tem conduzido a várias vitórias, como o reconhecimento da Libras, o direito a tradutores e intérpretes da  língua brasileira de sinais–língua portuguesa e a uma educação  bilíngue para as crianças Surdas, que contemple a Libras e o português, este na modalidade escrita, entre muitas outras conquistas 


% \cite{stewart-2021-barrons-asl}
% Signing as a Choice of Communication
% [...] why, until recently, did so many hearing people know so little about signing? There are at least three reasons for this. First, Deaf people make up just a small fraction of the population in any area. Therefore, many hearing people never encounter a Deaf person in their through life. Second, speech is the dominant form of communication in society and gets the most attention. Third, Deaf people tend to socialize with one another and with hearing people who know how to sign.




Línguas de sinais são línguas naturais, ou seja, que emergem (e não são inventadas) ``naturalmente'' quando os Surdos formam uma comunidade, muitas vezes por meio de sistemas educacionais. Assim como ocorre com as línguas na modalidade oral, cada país tem sua língua de sinais e não há uma universalidade. Devido à estreita relação entre língua e cultura, essas línguas acabam por refletir a cultura dos diferentes países onde são utilizadas~\cite{pereira-2011-conhecimento-alem-sinais, hill-2019-sign-languages}.

Elas são produzidas no espaço pelas mãos, rosto e corpo e percebidas principalmente visualmente, em contraste com as línguas faladas, que são produzidas pela boca e trato vocal e percebidas principalmente auditivamente (embora gestos manuais e percepção visual de gestos e movimentos da boca sejam também importante para as línguas faladas). 
Devido a isso, elas são denominadas línguas de modalidade gestual-visual (ou visual-espacial). Também por esse motivo, não existe uma forma escrita conveniente da língua de sinais, mas apenas glosas que representam uma aproximação do significado dos sinais~\cite{hill-2019-sign-languages, pereira-2011-conhecimento-alem-sinais,stewart-2021-barrons-asl}.

No entanto, línguas faladas e línguas de sinais seguem os mesmos princípios com relação ao fato de que têm um léxico e uma gramática, ou seja, elas apresentam um conjunto de símbolos convencionais e um sistema de regras que rege a combinação desses símbolos em unidades maiores~\cite{pereira-2011-conhecimento-alem-sinais}. Na seção seguinte discutiremos um pouco mais sobre esses elementos e a linguística da língua de sinais.


%\cite{hill-2019-sign-languages}
% Sign languages are produced by the hands, face, and body and perceived primarily visually, in contrast to spoken languages, which are produced by the mouth and vocal tract and perceived primarily auditorily (although manual gestures and visual perception of gestures and mouth movements are also important for spoken languages). Natural sign languages emerge (are not invented) when Deaf people form a community, often through educational systems. Sign languages are, therefore, primarily the languages of Deaf people, who cherish them for their cultural and community-building value. 

%\cite{pereira-2011-conhecimento-alem-sinais}
% ---
% As línguas de sinais distinguem-se das línguas orais porque utilizam o canal visual-espacial em vez do oral-auditivo . Por esse motivo, são denominadas línguas de modalidade gestual-visual  (ou visual-espacial), uma vez que a informação linguística é recebida pelos olhos e produzida no espaço, pelas mãos, pelo movimento do corpo e pela expressão facial .  
% Apesar da diferença existente entre línguas de sinais e línguas  orais, ambas seguem os mesmos princípios com relação ao fato  de que têm um léxico, isto é, um conjunto de símbolos convencionais, e uma gramática, ou seja, um sistema de regras que rege  o uso e a combinação desses símbolos em unidades maiores .












% ########################################################################################

% ===================================================
% \cite{pereira-2011-conhecimento-alem-sinais}
% ===================================================
%
% * Seguindo convenção proposta por James Woodward (1982), neste livro será  usado o termo “surdo” para se referir à condição audiológica de não ouvir, e o termo “Surdo” para se referir a um grupo particular de pessoas surdas que partilham  uma língua e uma cultura. 
% ---
% A língua de sinais é a língua usada pela maioria dos Surdos, na  vida diária . É a principal força que une a comunidade Surda, o  símbolo de identificação entre seus membros 
% ---
% Cada país tem sua língua de sinais, como tem sua língua na  modalidade oral. As línguas de sinais são línguas naturais, ou  seja, nasceram “naturalmente” nas comunidades Surdas. Uma vez  que não se pode falar em comunidade universal, tampouco está  correto falar em língua universal.  
% Outro aspecto a considerar é a relação estreita que existe entre  língua e cultura . As línguas de sinais refletem a cultura dos diferentes países onde são usadas, e esse é mais um argumento contra  a ideia de uma língua de sinais universal 
%---
% As línguas de sinais distinguem-se das línguas orais porque utilizam o canal visual-espacial em vez do oral-auditivo . Por esse motivo, são denominadas línguas de modalidade gestual-visual  (ou visual-espacial), uma vez que a informação linguística é recebida pelos olhos e produzida no espaço, pelas mãos, pelo movimento do corpo e pela expressão facial .  
% Apesar da diferença existente entre línguas de sinais e línguas  orais, ambas seguem os mesmos princípios com relação ao fato  de que têm um léxico, isto é, um conjunto de símbolos convencionais, e uma gramática, ou seja, um sistema de regras que rege  o uso e a combinação desses símbolos em unidades maiores .  
% ---
% Primeiros estudos (intro a linguística):
% As primeiras pesquisas linguísticas sobre as línguas de sinais,  mais especificamente sobre a língua de sinais americana, foram realizadas por William Stokoe, no início dos anos 1960, e tiveram como objetivo mostrar que os sinais poderiam ser vistos  como mais do que gestos holísticos aos quais faltava uma estrutura interna (Stokoe, 1960) . Ao contrário do que se poderia pensar à primeira vista, eles poderiam ser descritos em termos de um  conjunto limitado de elementos formacionais que se combinavam para formar os sinais .  
% A análise das propriedades formais da língua de sinais americana revelou que ela apresenta organização formal nos mesmos níveis encontrados nas línguas faladas, incluindo um nível  sublexical de estruturação interna do sinal (análoga ao nível fonológico das línguas orais) e um nível gramatical, que especifica os  modos como os sinais devem ser combinados para formarem frases e orações (Klima e Bellugi, 1979) .  
% Aos estudos sobre a língua de sinais americana se seguiram outros, cujo objeto eram as línguas de sinais usadas pelas comunidades de surdos em diferentes países, como França, Itália, Uruguai,  Argentina, Suécia, Brasil e muitos outros .  
% Essas línguas são diferentes umas das outras e independem  das línguas orais utilizadas nesses países 
% -----------------
% Concepções de surdez e de surdos 
% - Concepção clínico-patológica
% - Concepção socioantropológica (tentar adotar essa aqui)

% [Pág 30]:
% Martins (2004, p . 204-205) afirma que: “Sem língua não existem nem os surdos nem o modo de ser, cultural, surdo . Existiriam  apenas deficientes auditivos .” E segue com uma boa afirmação  em defesa da língua: “[ . . .] não é simplesmente o nível de audição  que vai definir quem é surdo ou deficiente auditivo” (op . cit .) . 

% [Pág 33]:
% Na história, constata-se que os Surdos sofreram perseguições  pelas pessoas ouvintes, que não aceitavam as diferenças e exigiam  uma cultura única por meio do modelo  ouvintista ou ouvintismo . São muitas  as lutas e histórias nas comunidades Surdas, em que o povo Surdo se une contra  as práticas dos ouvintes que não respeitam a cultura Surda (Strobel, 2008) .  
% Ainda hoje, muitos ouvintes tentam  diminuir os Surdos para que vivam isolados e tendo de assumir a  cultura ouvinte, como se esta fosse uma cultura única; ser “normal” para a sociedade significa ouvir e falar oralmente . Os ouvintes não prestam atenção aos Surdos que se comunicam por  meio da Libras . Consequentemente, não acreditam que os Surdos sejam capazes de estudar em faculdade ou realizar mestrado e  doutorado, por exemplo . “Os sujeitos ouvintes veem os sujeitos  surdos com curiosidade e, às vezes, zombam por eles serem diferentes” (Strobel, 2008, p . 22) .  
% A luta dos Surdos tem conduzido a várias vitórias, como o reconhecimento da Libras, o direito a tradutores e intérpretes da  língua brasileira de sinais–língua portuguesa e a uma educação  bilíngue para as crianças Surdas, que contemple a Libras e o português, este na modalidade escrita, entre muitas outras conquistas 

% [Pág 34] Cultura Surda
% Os Surdos constituem uma comunidade linguística minoritária, cujos elementos identificatórios são a língua de sinais e uma  cultura própria eminentemente visual . Têm um espírito gregário  muito importante que se manifesta em vários espaços . Esses espa-  ços “dos Surdos” são associações e clubes de Surdos onde desenvolvem suas próprias atividades . Constituem refúgios naturais da  língua de sinais e da identidade Surda (Strobel, 2008, p . 45) .  
% Diante da comunidade majoritariamente ouvinte, as comunidades Surdas apresentam suas próprias condutas linguísticas e seus  valores culturais . A comunidade Surda tem uma atitude diferente diante do déficit auditivo, já que não leva em conta o grau de  perda auditiva de seus membros . Pertencer à comunidade Surda pode ser definido pelo domínio da língua de sinais e pelos sentimentos de identidade grupal, fatores que consideram a surdez  como uma diferença, e não como uma deficiência .  
% Como ocorre com qualquer outra cultura, os membros das comunidades de Surdos compartilham valores, crenças, comportamentos e, o mais importante, uma língua diferente da utilizada  pelo restante da sociedade .  A língua de sinais, uma língua visual-espacial com gramática  própria, é uma das maiores produções culturais dos Surdos (Perlin, 2006) . Lane, Hoffmeister e Bahan (1996) referem que a língua de sinais tem basicamente três papéis para os Surdos: ela é símbolo da identidade social, é um meio de interação social e é  um depositário de conhecimento cultural.

% [Pág 97]
% A pessoa surda é definida como aquela que, por ter perda  auditiva, compreende o mundo e interage com ele por meio de  experiências visuais, manifestando sua cultura principalmente  pelo uso da Libras .



% ===================================================
% \cite{stewart-2021-barrons-asl}
% ===================================================
% What is ASL?
% American Sign Language, or ASL, is the language of the American Deaf Community. It is used in North America, and it is the only complete and natural sign language recognized by te Deaf community. This is a simple definition. Once it is understood that ASL maintains grammatical structure, syntax, and rules entirely separate from English grammar, we can begin to understand how to use it properly -- the way the American Deaf community intents.
% As you venture through this book, you will notice that ASL grammar is not only comprised of signed words or concepts, but it heavily involves the use of facial grammar to give information or meaning to signs. Facial grammar, including eye contact, facial expressions, eye gazing, and head movements are part of the unique grammatical structure of ASL. Because of this, and many other reasons, the visual language of ASL is fascinating for nonsigners to observe and quickly becomes a desired language to learn.

% English gloss
% ASL is an expressive and receptive language only. Because of the spatial and gestural qualities of ASL, there can be no convenient written form of ASL. What we can do is write English glosses of ASL signs. An English gloss is the best approximation of the meaning of a sign. It gives us a way of laying out ASL so that it can be studied and discussed, but is not a written form of ASL.

% Signing as a Choice of Communication
% [...] why, until recently, did so many hearing people know so little about signing? There are at least three reasons for this. First, Deaf people make up just a small fraction of the population in any area. Therefore, many hearing people never encounter a Deaf person in their through life. Second, speech is the dominant form of communication in society and gets the most attention. Third, Deaf people tend to socialize with one another and with hearing people who know how to sign.

% ASL Awareness
% Awareness of ASL has been growing since Professor William C. Stokoe, Jr., of Gallaudet University, known as the father of ASL linguistics, published his research on the linguistics of ASL about sixty years ago. His first paper, published in 1960, is titled "Sign Language Structure". This was followed by the first dictionary of ASL in 1965, Dictionary of American Sign Language on Linguistic Principles. Stokoe compiled the dictionary with two Deaf colleagues at Gallaudet, Carl Croneberg and Dorothy Casterline. In 1971, Stokoe established the Linguistic Research Laboratory at Gallaudet. Stokoe's work had a profound impact on ASL awareness in the United States and throughout the world, and we've even come a long way since then.
% ASL courses in high schools and colleges are booming. The television and movies industry has discovered the value of including Deaf actors and actresses in films. [...]

% The Physical Dimensions of ASL: The Five Parameters
% ASL is a visual-gestural language. It is visual because we see it and gestural because the signs are formed by the hands. Signing alone, however, is not an accurate picture of ASL. How signs are formed in space is important to understanding what they mean. The critical space is called the signing space and extends from the waist to just above the head and to just beyond the sides of the body. This is also the space in which the hands can move comfortably. As you will learn in this book, the signing space has a role in ASL grammar. Two or more concepts can be simultaneously expressed in ASL. This feat cannot be accomplished in a spoken language because speech is temporal in that one word rolls off the tongue at a time. One further dimension of ASL is the movement of the head and facial expressions, which help shape the meaning of ASL sentences.

% How Are Signs Formed?
% The five parameters below come together to create a sign.
% 1) Handshape is the shape of the hands when the sign is formed. The handshape may remain the same throughout the sign or it can change. If two hands are use to make a sign, both hands can have the same handshape of be different.
% 2) Orientation is the position of the hand(s) relative to the body. For example, the palms can be facing the body or away from the body, facing the ground, or facing upward.
% 3) Location is the place in the signing space where a sign is formed. Signs can be stationary [...] or they can move from one location in the signing space to another [...].
% 4) Movement of a sign is the direction in which the hand moves relative to the body. There is a variety of movements that range from a simple sliding movement [...] to a complex movement.
% 5) Nonmanual markers add to signs to create meaning. They consist of various facial expressions, head tilting, shoulder movement, and mouth movements. With a non-manual marker, the meaning of the sign can change completely.

% Cultural Importance
% ASL gives us access to Deaf culture. Learning ASL is not simply about learning another language. It is also about access. Even though we can learn something about any culture from reading about it, we acquire a deeper understanding when we can experience the culture or hear firsthand accounts from the people who are a part of the culture.
% ASL is one of the defining characteristics of the Deaf community. Although groups exist withing the community, such as the Black Deaf community and LGBTQAI+ Deaf communities. Deaf community members are bound instead by their language: ASL. To learn more about Deaf culture and tap into the resources of the Deaf community, you need a solid grasp of ASL.

% -----
% The Deaf Community
% Many hearing people view the world of the Deaf as a place where people don't hear -- where silence is a loud reminder of the difference between the two groups of people. But for Deaf people, silence is not the focus. What is important to us is that we obtain a lot of pleasure by being with other Deaf people. We relish the tales about other Deaf people's experiences in the Deaf community [...]

% Self-identification
% Deaf ("big D"):
% - identify as a culturally Deaf and part of the Deaf community
% - take pride in Deaf identity
% - may have an auditory device, such as cochlear implant, hearing aid, or FM system
% - may have a more severe hearing loss
% - use sign language as their primary source of communication
% - most likely attend a Deaf school/program
% - feel more comfortable in the Deaf world
%
% deaf ("little d"):
% - do not typically associate as members of the Deaf community
% - may have an auditory device, such as a cochlear implant, hearing aid, or FM system
% - may refer to hearing loss as a medical condition
% - may not use sign language as their primary choice of communication
% - may attend a mainstream school
% - may feel more comfortable in the hearing world
%
% Hard of hearing (HoH):
% - do not associate as members of the Deaf community
% - have hearing loss but may have residual hearing
% - possibly use an auditory device, such as a hearing or FM system, to access sounds
% - refer to hearing loss as a medical condition
% - may use sign language as their primary choice of communication
% - may attend a mainstream school
% - feel more comfortable in the hearing world

% Who Are Deaf People?
% Deaf people are a group of people who have a hearing loss, use a sign language as their primary means of communication, and have shared experiences associated with the hearing loss and the use of sign language.
% There is no way that you can point to a person sitting and reading a magazine in a lobby whom you have never met before and say, "that person is Deaf". Even if the person is wearing hearing aids, we don't know which community the person identifies with. Similarly, it's not important whether the person is European, African-American, Asian, or of some other ethnic origin. Age is not relevant, and neither is the social class or gender of the person. 
% The Deaf community is not shaped by any of these characteristics. In fact, having a hearing loss does not mean that a person is a member of the Deaf community, although it is certainly an important requirement.

% The pivotal mark of a Deaf person is how this person communicates. A Deaf person uses sign language [...]. This does not mean that the person cannot use other forms of communication, such as writing and speaking. Rather, ASL is the linguistic trademark that sets Deaf people apart from the communication behavior of all other groups of people. It is the reason we say that Deaf people represent a linguistic minority. It is also why some people who are deaf do not see themselves as belonging to the Deaf community. 
% We use the lowercase spelling of deaf to refer to a person or a group of people who have a substantial degree of hearing loss. Having a hearing loss does not mean that a person automatically knows how to sign. If a deaf person does not know sign language, then that person will not be able to access the varied cultural experiences associated with the Deaf community. Communication is basic, and ASL is the communication of the Deaf community.

% Can a nondeaf person who is fluent in ASL be a member of the Deaf community? No. Deaf people do not view nondeaf people as members of their community because nondeaf people lack a third critical characteristic, which is shared experiences. If you have normal hearing, then you will never have the experiences of a life that is centered on seeing. 

% Let's put all of this in perspective. Let's say you have a young friend who acquired a hearing loss and was fitted with hearing aids. WOuld we say that he was Deaf? No, we would say that he is hard of hearing because speaking is still his main means of communication. If his hearing continues to deteriorate, then we might say that he is becoming deaf; that is, he is acquiring a substantial degree of hearing loss. What if his difficulty with hearing leads him to learn to sign ASL, which becomes his primary language of communication, and he begins participating in some activities in the Deaf community? Would we say then that he is Deaf? We would probably say, "friend, welcome to the club".


% Technology and Other Adaptations
% "Many of the technological advances for the majority in our society have penalized Deaf people. This irony emerges most clearly in telecommunications. The invention of the telephone made it difficult for Deaf people to compete in the labor market. Radio became an important means of broadcasting information, whether commercial, political, governmental, or whatever, further cutting off Deaf people from the larger society surrounding them. Television did little to improve the situation, though it embraced the technology that could have (and to some extent does) include Deaf people. Talking pictures were a blow to the entertainment and education of Deaf people; they could enjoy the "silents" on par with the rest of the audience. But Deaf people and their supporters have not passively accepted the status quo. They have taken steps to reduce the handicap the new technologies have imposed". -- Jerome Schein (At Home Among Strangers)

% -----
% ASL Grammar
% ASL has visual, spatial, and gestural features that combine to create some grammatical structures that are unparalleled in the world of spoken languages. As you learn about ASL grammar, look for similarities in the world of spoken languages. [...]
% In particular, the spatial qualities of ASL grammar allow a signer to express more than one thought simultaneously -- a characteristic that cannot be duplicated in English. Facial grammar, or nonmanual signals, is also a significant component of ASL.

% Facial Grammar
% What's in a sign may not be what's in the mind. To capture the sense of what a signer is signing, you must read the signer's face and body. When you listen to someone speak, you listen not only to the words but also to how the words are spoken. The tone of the voice, the rise and fall of the pitch, the length of the pause, and the steadiness of voice are all features that you latch onto with little effort in your spoken communication.
% These traits are nonexistent in signing, but they do have parallel traits that are crucial to ASL's grammar. The raised eyebrow, the tilted head, the open mouth, hunching of the shoulders, and a sign held slightly longer than others shape the meaning of the signs that are made by the hands. We call these nonmanual signals (NMS) facial grammar, which allows you to use facial expressions, your body, and gestures to add meaning and additional information to your signing. Mastering ASL cannot occur without a mastery of facial grammar.


% ===================================================
% \cite{hill-2019-sign-languages}
% ===================================================
%
% [Pág 2]
% Sign languages are produced by the hands, face, and body and perceived primarily visually, in contrast to spoken languages, which are produced by the mouth and vocal tract and perceived primarily auditorily (although manual gestures and visual perception of gestures and mouth movements are also important for spoken languages). Natural sign languages emerge (are not invented) when Deaf people form a community, often through educational systems. Sign languages are, therefore, primarily the languages of Deaf people, who cherish them for their cultural and community-building value. 
% It is important to recognize the connection between sign languages and Deaf communities. Until relatively recently, Deaf communities have been told (explicitly and implicitly) that their “sign communication” was inferior, broken, unimportant, or insufficient. Educational systems and the broader hearing majority community would stress the value of learning the spoken language, even at the expense of the sign language. In fact, such attitudes persist, both in areas where the national sign language has not been deeply studied linguistically and in areas where it has been studied but the focus for economic advancement is on the spoken language. However, the natural sign languages of Deaf communities are completely linguistic, rule-governed, capable of expressing anything, and fully worthwhile. We unreservedly endorse such affirmations of the value of sign languages and promote their use in all aspects of the lives of Deaf people. 

% Who belongs to the Deaf community? The “d” is capitalized to reinforce the view that Deaf communities form cultural groups with practices and values that are in some cases distinct from those of non-Deaf communities. These cultural effects are passed down within the community, from parents to children in some cases, but more often through interactions of Deaf people from different families. The leaders of Deaf communities are usually Deaf adults who were raised with Deaf parents or within the community from a very early age. Generally, members of the Deaf community are audiologically deaf or hard-of-hearing (and they shun the label “hearing impaired”). The hearing children born to Deaf parents are often known as Codas (from the name of an organization, CODA, ‘children of Deaf adults’), and they are sometimes part of the Deaf community. 

% It is important to note that people have many identities with intersectional effects, and in this respect, not all Deaf people have the same experiences, values, and life view. A Deaf person’s identity as Deaf will be affected by their identity in other ways, including race, ethnicity, gender identity, etc. Almost all research on the American Deaf community has focused only on a subset of Deaf people, so it is important to bear in mind that others might share some but not all of the characteristics described here. 
% Sign languages are, then, Deaf languages. Just as with the languages of other minority groups who have experienced oppression, hearing researchers who benefit from the study of sign languages (both in personal satisfaction and in economic, career, and other means) must acknowledge the primacy of Deaf signers and treat their language with the utmost respect.


% Linguística
\section{Linguística da língua de sinais}
\label{sec:linguistica}

\citeonline{quadros-2004-estudos-linguisticos} afirmam que a linguística é o estudo científico das línguas naturais e humanas. Trata-se de uma ciência que procura desvendar os princípios independentes da lógica e da informação que determinam essa linguagem, bem como todas as formas criativas da comunicação.
Ela busca respostas para problemas essenciais relacionados à linguagem como, por exemplo: ``qual a natureza da linguagem humana?'',  ``como a comunicação se constitui?'', ``quais os princípios que determinam a habilidade dos seres humanos de produzir e compreender a linguagem?''.


Segundo \citeonline {stewart-2021-barrons-asl}, os primeiros estudos linguísticos sobre as línguas de sinais foram realizados pelo professor Dr. William C. Stokoe Jr. da Universidade de Gallaudet, em ~\citeyear{stokoe-1960-sl-structure}. 
Seu primeiro artigo, intitulado ``\textit{Sign Language Structure}'' (ou Estrutura da Língua de Sinais), foi seguido pela publicação do primeiro dicionário da \acrfull{asl} em \citeyear{stokoe-1965-dictionary-asl} -- o ``\textit{Dictionary of American Sign Language on Linguistic Principles}'' (ou Dicionário da Língua de Sinais Americana em Princípios Linguísticos) -- que foi compilado em parceria com dois colegas Surdos. Em 1971, por sua vez, ele estabeleceu o Laboratório de Pesquisa em Linguística da Universidade de Gallaudet.

Por conta disso, Stokoe ficou conhecido como o pai da linguística das línguas de sinais e seu trabalho teve um impacto profundo na conscientização acerca da \acrshort{asl} nos Estados Unidos e no restante do mundo.

Em seus estudos, ele comprovou que a \acrshort{asl} atendia a todos os critérios linguísticos de uma língua genuína -- no léxico, na sintaxe e na capacidade de  gerar uma quantidade infinita de sentenças.
A análise de suas propriedades revelou que a língua de sinais apresenta organização formal nos mesmos níveis encontrados nas línguas faladas, incluindo um nível sublexical de estruturação interna do sinal (análoga ao nível fonológico das línguas orais) e um nível gramatical (morfossintático), que especifica os modos como os sinais devem ser combinados para formarem frases e orações. Dessa forma, comprovou-se que os sinais não são meras imagens, mas símbolos abstratos com uma complexa estrutura interior. Aos estudos de Stokoe seguiram-se outros, que estenderam o escopo de análise às línguas de sinais utilizadas em diferentes países, como França, Itália, Uruguai, Argentina, Suécia e Brasil~\cite{stokoe-1960-sl-structure,quadros-2004-estudos-linguisticos, pereira-2011-conhecimento-alem-sinais}.

Serão abordados nas sessões seguintes aspectos importantes da organização gramatical da língua de sinais, segundo sua fonologia, morfologia e sintaxe.


\subsection{Fonologia}
\label{sec:linguistica-fonologia}

Fonologia é o estudo das menores unidades constituintes de uma língua -- denominados fonemas -- e das regras que regem sua produção. Ela objetiva compreender essas unidades, bem como elas são articuladas, para compor unidades maiores com significado, como as palavras, de acordo com \citeonline{quadros-2004-estudos-linguisticos,hill-2019-sign-languages}. 


\citeonline{stokoe-1960-sl-structure} definiu inicialmente três tipos de fonemas (ou parâmetros) para a língua de sinais, os quais são articulados simultaneamente para compor um sinal: configuração de mão, locação e movimento. Em \citeyear{battison-1974-phono-deletion}, \citeauthor{battison-1974-phono-deletion} introduziu um quarto parâmetro, referente à orientação da palma da mão. Posteriormente, estudos como o de \citeonline{baker-padden-1978-nonmanual-components}, adicionaram as expressões não-manuais, como expressões faciais, movimentos da boca e direção do olhar.

Dessa forma, atualmente a fonologia da língua de sinais compreende que os sinais são compostos pelos seguintes parâmetros~\cite{stewart-2021-barrons-asl,jay-2011-dont-just-sign,quadros-2004-estudos-linguisticos}:

\begin{enumerate}
   \item \textbf{Configuração de mão}: configuração assumida pelas mãos ao produzir o sinal, a qual pode permanecer estática ou variar durante a articulação do sinal. É possível que as mãos apresentem configurações distintas nesse processo. A \autoref{fig:config-mao-comuns-asl} ilustra algumas configurações utilizadas na \acrshort{asl}.
   
    \figura[p. 72]
        {fig:config-mao-comuns-asl} % Label
        {capitulos/fundamentacao/imagens/configuracoes_mao_asl} % Path
        {height=8cm} % Size
        {Exemplos de configurações de mãos utilizadas na \acrshort{asl}} % Caption
        {jay-2011-dont-just-sign} % Citation


    \item \textbf{Orientação}: direção apontada pelas palmas das mãos na articulação do sinal. Por exemplo, as palmas podem estar voltadas para o corpo, para fora, para o chão, para cima, entre outras ilustradas na \autoref{fig:orientacoes}. Cada uma das palmas pode também assumir uma orientação distinta.
    
    \figura[p. 59]
        {fig:orientacoes} % Label
        {capitulos/fundamentacao/imagens/orientacoes} % Path
        {height=8cm} % Size
        {Exemplos de orientações que podem ser assumidas pelas palmas das mãos} % Caption
        {quadros-2004-estudos-linguisticos} % Citation
    

    \item \textbf{Movimento}: corresponde à trajetória percorrida pelas mãos em relação ao corpo para articular o sinal. 
    É um parâmetro complexo que pode envolver uma ampla variedade de modos e direções, desde um sutil deslizar entre as mãos ou um movimento interno das mãos e punhos, até uma trajetória complexa desenhada no espaço, por exemplo.
    Além disso, os sinais podem envolver movimentos de uma ou de ambas as mãos.

    
    \item \textbf{Locação}: é o local onde as mãos são posicionadas dentro do espaço de enunciação para articular o sinal. O espaço de enunciação, por sua vez, é uma área que contém todos os pontos possíveis dentro do raio de alcance das mãos, como ilustra a \autoref{fig:espaco-enunciacao}. Nesse espaço, há um número limitado de locações, sendo que algumas são mais exatas -- tais como a ponta do nariz --, e outras são mais abrangentes -- como a frente do tórax. Por fim, as mãos podem permanecer fixas ou se deslocar de uma locação para outro durante a articulação de um sinal.
    
    \figura[p. 57]
        {fig:espaco-enunciacao} % Label
        {capitulos/fundamentacao/imagens/espaco_enunciacao} % Path
        {height=6cm} % Size
        {Espaço de enunciação da língua de sinais} % Caption
        {quadros-2004-estudos-linguisticos} % Citation

        
    \item \textbf{Expressões não-manuais}: consistem nas expressões faciais e movimentos corporais incorporados aos sinais para provê significado adicional. Elas desempenham duas funções essenciais: marcar construções sintáticas (como frases interrogativas, orações relativas, tópicos, concordância e foco) e diferenciar componentes lexicais (como referências específicas, referências pronominais, partículas negativas, advérbios, grau ou aspecto).
    
    De um modo geral, expressões faciais ajudam a prover mais clareza ou alterar o significado de um sinal. 
    Movimentos corporais, por sua vez, são importantes para descrever pessoas em diferentes posições ou locais, ou narrar histórias envolvendo personagens com diferentes papéis, por exemplo.
 
\end{enumerate}

\subsection{Morfologia}
\label{sec:linguistica-morfologia}

Morfologia é o estudo da estrutura interna das palavras e das regras que determinam sua formação. Um morfema é a menor unidade indivisível de sintaxe que retém significado e, na língua de sinais, é tido como a combinação da configuração de mão, orientação, locação e movimento, afirmam \citeonline{quadros-2004-estudos-linguisticos,jay-2011-dont-just-sign,hill-2019-sign-languages}.


Em línguas faladas, palavras complexas são muitas vezes formadas adicionando-se um prefixo ou sufixo a uma raiz. Por exemplo, o adjetivo ``infeliz'' é constituído de dois morfemas: o prefixo negativo \textit{in-} e o adjetivo \textit{feliz}; o substantivo ``capacidade'', por sua vez, é composto pelo adjetivo \textit{capaz} acrescido do sufixo \textit{-idade}; já o substantivo ``guarda-chuva'' é constituído pelos morfemas \textit{guarda} e \textit{chuva}.



De acordo com \citeonline{klima-1975-wit-poetry-asl,quadros-2004-estudos-linguisticos}, nas línguas de sinais essas formações resultam frequentemente de processos em que uma raiz é enriquecida com movimentos e contornos no espaço de sinalização. Também são utilizadas expressões não-manuais, alterações nos parâmetros fonológicos ou sinais específicos para indicar tempo, grau, intensidade, pluralidade, aspecto, entre outros.


Serão discutidos a seguir os dois principais processos de formação de palavras apresentados pela morfologia tradicional, a derivação e a flexão, sob a perspectiva das línguas de sinais \cite{quadros-2004-estudos-linguisticos,hill-2019-sign-languages,klima-1979-signs-of-language}:

\begin{enumerate}
    \item \textbf{Derivação}: consiste na formação de novas palavras a partir de uma mesma base lexical, como nos exemplos ``infeliz'' e ``capacidade'' introduzidos acima. No contexto das línguas de sinais, esses processos derivacionais podem incluir:

          \begin{enumerate}
              \item \underline{Nominalização}: consiste na derivação de substantivos a partir de verbos, e é um dos processos mais comuns para mudança de classe na morfologia.

                    Na língua de sinais, os substantivos apresentam basicamente os mesmos parâmetros fonológicos que os verbos, mas diferenciam-se pela repetição (ou reduplicação) do seu movimento. A \autoref{fig:sinais-ouvir-ouvinte} ilustra um exemplo de derivação do substantivo OUVINTE a partir do verbo OUVIR.

                    \figura[p. 98]
                    {fig:sinais-ouvir-ouvinte} % Label
                    {capitulos/fundamentacao/imagens/sinais_ouvir_ouvinte} % Path
                    {height=4cm} % Size
                    {O verbo OUVIR (à esquerda) é utilizado para derivar o substantivo OUVINTE (à direita)} % Caption
                    {quadros-2004-estudos-linguisticos} % Citation


              \item \underline{Composição}: consiste na criação de um novo sinal através da junção de duas bases preexistentes.
                    Existem três regras para a composição de sinais: a regra do contato, na qual o contato existente no primeiro ou segundo sinal da composição é mantido; a regra da sequência única, na qual o movimento interno ou repetição dos sinais é eliminada para formar um composto; e a regra da antecipação da mão não-dominante, em que a mão passiva antecipa o segundo sinal no processo de composição.

                    Observe na \autoref{fig:sinal-acidente} o exemplo do sinal ACIDENTE, que é composto a partir dos sinais CARRO e BATER utilizando-se a regra da antecipação da mão não-dominante.


                    \figura[p. 105]
                    {fig:sinal-acidente} % Label
                    {capitulos/fundamentacao/imagens/sinal_acidente} % Path
                    {height=4cm} % Size
                    {Composição do sinal ACIDENTE a partir dos sinais CARRO e BATER} % Caption
                    {quadros-2004-estudos-linguisticos} % Citation



              \item \underline{Incorporação de numeral}: combinação da configuração de mão de numeral a um sinal para especificar variação de quantidade em seu significado. Isso é útil, por exemplo, para representar número de anos, dias, horas, minutos, entre outros.

                    A \autoref{fig:sinais-numero-meses} ilustra o uso desse mecanismo para especificar o número de meses no sinal.

                    \figura[p. 107]
                    {fig:sinais-numero-meses} % Label
                    {capitulos/fundamentacao/imagens/sinais_numero_meses} % Path
                    {height=4cm} % Size
                    {Incorporação de numeral para especificar o número de meses no sinal} % Caption
                    {quadros-2004-estudos-linguisticos} % Citation



              \item \underline{Incorporação de negação}: geração da contraparte negativa de um sinal através da alteração de um de seus parâmetros, que comumente é o seu movimento.

                    A \autoref{fig:sinais-saber-naosaber} ilustra a negação do sinal SABER adicionando-se um movimento e uma expressão não-manual específica de negação.

                    \figura[p. 111]
                    {fig:sinais-saber-naosaber} % Label
                    {capitulos/fundamentacao/imagens/sinais_saber_naosaber} % Path
                    {height=4cm} % Size
                    {Incorporação da negação ao sinal SABER} % Caption
                    {quadros-2004-estudos-linguisticos} % Citation

          \end{enumerate}


    \item \textbf{Flexão}: consiste na adição de informação gramatical a palavras existentes para fazer com que elas se adequem melhor ao contexto em que são utilizadas. Nas línguas de sinais, alguns desses processos flexionais incluem:

          \begin{enumerate}
              \item \underline{Pessoa}: também conhecida como dêixis\footnote{
                        Dêixis: palavra grega que significa apontar ou indicar, e representa uma forma de estabelecer referenciais no espaço que são utilizados para flexionar verbos com concordância.~\cite{quadros-2004-estudos-linguisticos}
                    }, consiste na modificação da referência de pessoa para os verbos.
                    Na prática, isso é feito pelo interlocutor apontando-se para diferentes pontos no espaço à sua frente, os quais serão utilizados como referenciais que representam pessoas (ou objetos) envolvidas no discurso.

                    A concordância do verbo se dará pela articulação do movimento partindo de um desses referenciais para o outro, conforme ilustrado na \autoref{fig:verbo-entregar-deixis}.

                    \figura[p. 114]
                    {fig:verbo-entregar-deixis} % Label
                    {capitulos/fundamentacao/imagens/verbo_entregar_deixis} % Path
                    {height=4cm} % Size
                    {Flexão de pessoa para o verbo ENTREGAR, envolvendo dois referenciais} % Caption
                    {quadros-2004-estudos-linguisticos} % Citation


              \item \underline{Número}: é utilizada para indicar: a forma plural do sinal, a qual é marcada pela sua repetição; ou a existência de múltiplos referentes no discurso, pela articulação da ação na direção dos respectivos referentes no espaço.

                    Veja na \autoref{fig:verbo-entregar-flexao-numero} a flexão do verbo ENTREGAR para um, três e vários referentes.

                    \figura[p. 120]
                    {fig:verbo-entregar-flexao-numero} % Label
                    {capitulos/fundamentacao/imagens/verbo_entregar_flexao_numero} % Path
                    {height=4cm} % Size
                    {Flexão de número do verbo ENTREGAR para um (à esquerda), três (ao centro) e vários referentes (à direita)} % Caption
                    {quadros-2004-estudos-linguisticos} % Citation


              \item \underline{Grau}: adiciona variação de grau ou intensidade ao sinal, a qual geralmente é transmitida utilizando-se expressões não-manuais.
                    A \autoref{fig:sinais-lindo-lindinho-lindissimo} ilustra a flexão do sinal LINDO para os graus de pouco (LINDINHO) e muito (LINDÍSSIMO).

                    \figura[p. 65]
                    {fig:sinais-lindo-lindinho-lindissimo} % Label
                    {capitulos/fundamentacao/imagens/sinais_lindo_lindinho_lindissimo} % Path
                    {height=4cm} % Size
                    {Flexões de grau para o sinal LINDO} % Caption
                    {pereira-2011-conhecimento-alem-sinais} % Citation


              \item \underline{Modo}: especifica a maneira com que uma ação é realizada e também se utiliza de expressões não-manuais.
                    Por exemplo, poderia-se detalhar que uma ação foi realizada ``facilmente'' ou ``com dificuldade''.


              \item \underline{Aspecto temporal}: determina a forma com que uma ação relaciona-se com o tempo, a qual pode ser uma das seguintes: incessante, ininterrupta, habitual (recorrente), contínua, ou duradoura (permanente).

                    Observe na \autoref{fig:verbo-cuidar-flexao-temporal} alguns exemplos de flexões do verbo CUIDAR para os aspectos temporais incessante, ininterrupto e habitual.

                    \figura[p. 123]
                    {fig:verbo-cuidar-flexao-temporal} % Label
                    {capitulos/fundamentacao/imagens/verbo_cuidar_flexao_temporal} % Path
                    {height=4cm} % Size
                    {Flexões temporal do verbo CUIDAR para os aspectos incessante (à esquerda), ininterrupto (ao centro) e habitual (à direita)} % Caption
                    {quadros-2004-estudos-linguisticos} % Citation


              \item \underline{Aspecto distributivo}: determina a forma com que uma ação é distribuída entre diferentes pessoas ou objetos no discurso.
                    Ela pode ser: exaustiva, quando a ação é repetida exaustivamente; específica, quando direciona-se a pessoas ou objetos específicos; não-específica, quando é generalizada ou indeterminada.

                    Exemplos de flexão distributiva podem ser encontrados na \autoref{fig:verbo-entregar-flexao-numero} para os aspectos específico (ao centro) e não-específico (à direita).


              \item \underline{Reciprocidade}: especifica que uma ação (ou relação) ocorre de forma mútua. Ela é representada pela duplicação do sinal, a qual é articulada simultaneamente.
                    Observe na \autoref{fig:sinal-olhar-reciproco} um exemplo para o sinal OLHAR.

                    \figura[p. 122]
                    {fig:sinal-olhar-reciproco} % Label
                    {capitulos/fundamentacao/imagens/sinal_olhar_reciproco} % Path
                    {height=4cm} % Size
                    {Flexão de reciprocidade para o sinal OLHAR} % Caption
                    {quadros-2004-estudos-linguisticos} % Citation

          \end{enumerate}

\end{enumerate}


\subsubsection{Sintaxe}
\label{linguistica-gramatica-sintaxe}

A sintaxe é o estudo da construção de sentenças numa linguagem, bem como dos princípios e regras descritivas que regem esse processo. Na língua de sinais, a sintaxe é transmitida através da ordem das palavras e das expressões não-manuais.
Observemos a seguir alguns dos componentes básicos da sintaxe dessa língua \cite{jay-2011-dont-just-sign,hill-2019-sign-languages}:

\begin{enumerate}
    \item \textbf{Ordem das palavras}:
    \item \textbf{Tipos de sentenças}:
        Questões
            Wh-questions
            Sim/não
            Retórica
        Condicionais
        Topicalização
    \item \textbf{Negação}:
    \item \textbf{Preposição}:
    \item \textbf{Conjunções}:
\end{enumerate}




% \cite{hill-2019-sign-languages} ----------------------------
% Syntax
% Syntax is the study of the descriptive rules that are needed to build a sentence in a given language.
% Now we will look at some basic components of ASL syntax and learn how to build a sentence.

% Word order
% Syntatic structures with brow raise
%     Yes/no interrogative
%     Conditionals
%     Topicalization
% Wh-questions
% Negation


% \cite{jay-2011-dont-just-sign} -------------------------------
% Syntax
% Syntax is the study of constructing sentences. Syntax also refers to the rules and principles of sentence structure.
% In ASL, syntax is conveyed through word order and non-manual markers. This section can be confusing, so don’t get discouraged if you don’t understand the first time.

% • Word Order 
%         Word order with plain Verbs
%         Object-subject-verb word order
%         Word order without objects
%         Word order with directional Verbs
%         time-topic-comment
% • Sentence Types
%         Questions
%             Wh-questions
%             Yes/no questions
%             Rethorical questions
%         Declarative sentences
%             Affirmative Declarative ...
%             Negative Declarative ...
%             Neutral Declarative ...
%         Conditional sentences
%         Topicalization
%             Topicalized statements
%             Topicalized "Wh" question
% • Negation
%         Reversal of orientation
% • Pronouns and Indexing
%         Indexing on your non-dominant hand
%         Personal Pronouns
%         Possessive Pronouns
%         Directional Verbs
%         Plural Directional Verbs
% • Nouns
%         Pluralization
% • Adjectives
% • Auxiliary Verbs
% • Prepositions
% • Conjunctions
% • Articles




% \cite{quadros-2004-estudos-linguisticos} ------------------------
% A Sintaxe Espacial
% A língua de sinais brasileira, usada pela comunidade surda brasileira  espalhada por todo o País, é organizada espacialmente de forma tão complexa quanto às línguas orais-auditivas. Analisar alguns aspectos da sintaxe  de uma língua de sinais requer “enxergar” esse sistema que é visuoespacial  e não oral-auditivo. De certa forma, tal desafio apresenta certo grau de dificuldade aos lingüistas; no entanto, abre portas para as investigações no campo  da Teoria da Gramática enquanto manifestação possível da capacidade da  linguagem humana. A organização espacial dessa língua, assim como da  ASL – Língua de Sinais Americana – (Siple, 1978; Lillo-Martin, 1986; Fischer,  1990; Bellugi, Lillo-Martin, O’Grady e van Hoek, 1990), apresenta possibilidades de estabelecimento de relações gramaticais no espaço, através de diferentes formas.
% No espaço em que são realizados os sinais, o estabelecimento nominal e  o uso do sistema pronominal são fundamentais para tais relações sintáticas.  Qualquer referência usada no discurso requer o estabelecimento de um local  no espaço de sinalização (espaço definido na frente do corpo do sinalizador),  observando várias restrições. 












% SLR
\section{Reconhecimento de línguas de sinais}
\label{sec:slr}

De acordo com \citeonline{wadhawan-2019-slr-literature-review,cooper-2011-slr}, o \acrfull{slr} é uma área de pesquisa colaborativa e multidisciplinar que tem por objetivo elaborar métodos e algoritmos para identificar os sinais articulados pelos usuários dessa língua e compreender seu significado.

É uma área capaz de contribuir com a quebra de barreiras existentes para os usuários dessa língua e facilitar a comunicação cotidiana entre Surdos e ouvintes, segundo \citeonline{rastgoo-2021-slr-deep-survey,papastratis-2021-ai-technologies-sl}.
Isso é importante porque, além de promover a inclusão dos Surdos em sociedade, aborda o problema atual de que as tecnologias de comunicação são, em sua maioria, desenvolvidas para suportar línguas faladas ou escritas, mas excluem as línguas sinalizadas. Por exemplo, o WhatsApp, Telegram e iMessage tornaram-se ferramentas imprescindíveis em nossas vidas, porém, a comunidade Surda enfrenta diversos desafios para utilizá-las.

Ainda segundo os autores, apesar dessas necessidades terem sido identificadas há muito tempo pela comunidade acadêmica, apenas recentemente a área de \acrshort{slr} passou a receber mais atenção.
Isso deve-se principalmente aos avanços ocorridos nas tecnologias de sensoriamento e dos algoritmos de \acrshort{ia}, que abriram caminho para o desenvolvimento de aplicações capazes de abordar tais demanda de maneira mais efetiva. 
Além disso, o advento das arquiteturas de \acrshort{dl} proporcionou uma melhora significativa no desempenho dos algoritmos utilizados nesta área.

\citeonline{koller-2020-quantitative-survey-slr} realizou uma análise baseada nos estudos mais relevantes em \acrshort{slr} publicados desde 1983, a qual possibilita delinear melhor essa evolução recente e o estado da arte atual.
Além disso, as revisões apresentadas por \citeonline{rastgoo-2021-slr-deep-survey,papastratis-2021-ai-technologies-sl,wadhawan-2019-slr-literature-review} também contribuem para estender essa análise. Esse panorama será discutido na seção a seguir.



% Modelos
\section{Modelos sequenciais de aprendizagem de máquina}
\label{sec:modelos-sequenciais}

Introduziremos aqui uma breve discussão acerca dos modelos sequenciais de \acrlong{ml}, bem como sua importância dentro da área de \acrfull{nlp} para lidar com tarefas envolvendo linguagens.

\citeonline{jurafsky-2022-speech-lang-processing} afirmam que a linguagem é um fenômeno inerentemente temporal e que pode ser compreendida como uma sequência de eventos que se desdobram ao longo do tempo, como um fluxo contínuo de dados.
Os modelos sequenciais lidam diretamente com essa natureza, e são capazes de capturar e explorar tal aspecto temporal.
Os tipos de arquiteturas mais populares desses modelos são a \acrfull{rnn} (e suas extensões, dentre as quais o \acrfull{lstm} e a \acrfull{gru} são as mais utilizadas) e o \textit{Transformer}.

% Devido a isso, neste trabalho selecionamos algumas dessas arquiteturas para avaliar a eficácia da abordagem proposta, as quais são: \acrfull{lstm}~\cite{hochreiter-1997-lstm} e \acrfull{gru}~\cite{cho-2014-gru} -- que são extensões da arquitetura \acrfull{rnn}~\cite{mikolov-2010-rnn} -- e o \textit{Transformer}~\cite{vaswani-2017-transformer}. 

As \acrshortpl{rnn} consistem de redes neurais criadas por \citeonline{mikolov-2010-rnn} que contêm ciclos (ou recorrências) em suas conexões, que fazem com que o valor de suas unidades sejam direta ou indiretamente dependentes de suas próprias saídas anteriores.
De um modo geral, elas funcionam processando cada palavra da sequência e combinando ela com o contexto (ou estado oculto) anterior para tentar prever a próxima palavra da sequência. Esse contexto, por sua vez, é capaz de representar as informações de todas as palavras anteriores da sequência.

% Elas processam sequências uma palavra por vez, tentando prever a próxima palavra com base na atual e no contexto (ou estado oculto) anterior que, por sua vez, pode representar as informações de todas as palavras anteriores da sequência \cite{jurafsky-2022-speech-lang-processing}.

No entanto, \citeonline{lecun-2015-deep-learning,goodfellow-2016-deep-learning} ressaltam que essas redes apresentaram limitações em armazenar informações por um período muito longo de tempo, dentre as quais estão os problemas conhecidos de \textit{gradient vanishing} (ou desaparecimento do gradiente) e \textit{gradient exploding} (ou explosão do gradiente). Isso demandou com que extensões dessa arquitetura fossem desenvolvidas no decorrer dos anos com o intuito abordar melhor essas questões.

% \citeonline{lecun-2015-deep-learning} afirmam que apesar do principal objetivo das \acrshortpl{rnn} ser aprender dependências de longo prazo, evidências teóricas e empíricas mostram que há desafios em aprender a armazenar informações por um tempo muito longo. Entre esses desafios estão os problemas de desaparecimento e explosão de gradientes, aos quais as \acrshortpl{rnn} estão passíveis. Devido a isso, várias extensões dessa arquitetura foram desenvolvidas no decorrer dos anos com o intuito de abordar melhor esses problemas.

Dentre as mais populares dessas extensões está a \acrshort{lstm}, que foi introduzida por \citeonline{hochreiter-1997-lstm}. De acordo com \citeonline{jurafsky-2022-speech-lang-processing}, a principal inovação dessa rede é a capacidade de aprender a gerenciar o contexto (ou estado oculto) de forma automática, decidindo quando informações são necessárias e quando podem ser removidas, sem necessitar que uma estratégia explícita seja codificada para isso. Ele utiliza uma camada específica para representar esse contexto e um conjunto de portas, as quais controlam o fluxo de informações para dentro e para fora de suas células.
Segundo \citeonline{goodfellow-2016-deep-learning,lecun-2015-deep-learning}, essas redes são extremamente bem-sucedidas em diferentes tipos de aplicações, como reconhecimento e geração de caligrafia, reconhecimento de fala, \acrfull{mt}, legendagem de imagens e análise sintática.

%O \acrshort{lstm} é a mais popular delas e divide o gerenciamento do contexto em duas partes: na primeira, está a adição de informação que provavelmente será necessária para tomada de decisão posterior ao contexto; na segunda, está a remoção de informação que não é mais necessária. Com isso, o \acrshort{lstm} é capaz de aprender como gerenciar esse contexto e lidar com ambas as partes, e não exige que uma estratégia para isso seja codificada na arquitetura.
%Isso é feito utilizando-se uma camada explícita para representar o contexto e também unidades neurais especializadas que utilizam três portas (\textit{update gate}, \textit{forget gate} e \textit{output gate}) para controlar o fluxo de informações para dentro e para fora das unidades que compõem as camadas da rede neural. Essas portas, por sua vez, são implementadas como pesos adicionais que são ajustados durante o processo de treinamento e operam sequencialmente na entrada, na camada oculta e nas camadas de contexto anteriores \cite{jurafsky-2022-speech-lang-processing}.

% \citeonline{goodfellow-2016-deep-learning,lecun-2015-deep-learning} ressaltam que as redes \acrshort{lstm} mostraram ser extremamente bem-sucedidas em diferentes tipos de aplicações, como reconhecimento de caligrafia, reconhecimento de fala, geração de caligrafia, tradução automática, legendagem de imagens e análise sintática.


A \acrshort{gru} é também uma extensão bastante popular das \acrshortpl{rnn}, e foi criada por \citeonline{cho-2014-gru} com o intuito de simplificar o desenho das unidades internas do \acrshort{lstm}. 
De acordo com \citeonline{goodfellow-2016-deep-learning,ravanelli-2018-li-gru}, elas diferenciam-se apenas pela forma como controlam o fluxo de informações entre suas camadas: enquanto o \acrshort{lstm} utiliza três portas em suas células internas (\textit{update gate}, \textit{forget gate} e \textit{output gate}), o \acrshort{gru} propõe a adoção de apenas duas portas para isso (\textit{update gate} e \textit{reset gate}).

% A \acrshort{gru}~\cite{cho-2014-gru}, por sua vez, é uma evolução do \acrshort{lstm}. De acordo com \citeonline{goodfellow-2016-deep-learning}, a principal diferença está na forma como elas controlam o fluxo de informações entre suas camadas: enquanto o \acrshort{lstm} adota três portas em suas células internas para isso (\textit{update gate}, \textit{forget gate} e \textit{output gate}), o \acrshort{gru} propõe uma simplificação das células e utiliza apenas duas portas (\textit{update gate} e \textit{reset gate}) \cite{ravanelli-2018-li-gru,goodfellow-2016-deep-learning}.

% A arquitetura \acrshort{gru} surgiu com o objetivo de simplificar o desenho das unidades internas do \acrshort{lstm} e, devido a isso, é considerada uma evolução desta. Segundo \citeonline{goodfellow-2016-deep-learning}, a principal diferença entre elas está na forma como elas controlam o fluxo de informação entre suas camadas. Enquanto que nas unidades do \acrshort{lstm} são utilizadas três portas para controlar a atualização, esquecimento e a saída de informação, no \acrshort{gru} uma única porta realiza o controle simultâneo do fator de esquecimento e da atualização do estado da unidade -- o que faz com que ela apresentem um total duas portas (denominadas \textit{update gate} e \textit{reset gate}) \cite{ravanelli-2018-li-gru,goodfellow-2016-deep-learning}.



Um outro tipo de arquitetura utilizada no \acrshort{nlp} é a \textit{Sequence-to-Sequence} (ou Sequência para Sequência), também comumente conhecida como \textit{Encoder-Decoder} (ou Codificador Decodificador). Ela foi apresentada por \citeonline{cho-2014-encoder-decoder,sutskever-2014-seq-to-seq} e sua estrutura é composta por duas redes neurais: uma codificadora, que recebe uma sequência de entrada e gera uma representação contextualizada dela (que seria o contexto); e uma decodificadora, que produz uma sequência de saída específica para a tarefa em questão, conforme ilustrado na \autoref{fig:encoder-decoder-arquitetura}. 

\figura[p. 220]
{fig:encoder-decoder-arquitetura}
{capitulos/fundamentacao/imagens/encoder_decoder_arquitetura}
{width=0.90\textwidth}
{Arquitetura do \textit{Encoder-Decoder}: a sequência de entrada \(\{x_1, x_2, x_3, x_n\}\) é recebida pelo \textit{encoder} (à esquerda) e utilizada para gerar o contexto \(c\) (em verde), o qual é utilizado pelo \textit{decoder} (à direita) para produzir a sequência \(\{y_1, y_2, y_3, y_n\}\)}
{jurafsky-2022-speech-lang-processing}


Essas redes codificadoras e decodificadoras são implementadas utilizando-se redes \acrshortpl{rnn}, e algumas otimizações consideram também a adição de uma camada de \textit{attention} antes do decodificador para eliminar um gargalo observado ali por \citeonline{bahdanau-2015-mt-align-translate}.

% Por fim, se compararmos a estrutura do \textit{Transformer} (\autoref{fig:transformer-arquitetura}) com a do \textit{Encoder-Decoder} (\autoref{fig:encoder-decoder-arquitetura}), perceberemos que elas compartilham algumas semelhanças.




O \textit{Transformer}, por sua vez, consiste num tipo de arquitetura que não é recorrente e, ao invés disso, baseia-se num mecanismo de \textit{attention} (ou atenção) para estabelecer dependências globais entre os dados de entrada e saída.
Ele foi introduzido por \citeonline{vaswani-2017-transformer} e baseia-se na estrutura do \textit{Encoder-Decoder}, porém seus codificadores e decodificadores são compostos por blocos empilhados de redes multicamadas que combinam camadas lineares simples, redes \textit{feed-forward} e camadas de \textit{self-attention} (ou auto-atenção) -- as quais são a principal inovação aqui --, conforme ilustra a \autoref{fig:transformer-arquitetura}.
Devido à sua estrutura escalável e capaz de capturar o contexto de sequências muito longas, \citeonline{wolf-2020-transformers,jurafsky-2022-speech-lang-processing} afirmam que os \textit{Transformers} tornaram-se rapidamente a arquitetura dominante entre tarefas de \acrshort{nlp}, superando redes como a \acrfull{cnn} e as \acrshortpl{rnn}.

\figura[p. 3]
{fig:transformer-arquitetura}
{capitulos/fundamentacao/imagens/transformer_arquitetura}
{width=0.45\textwidth}
{Arquitetura do \textit{Transformer}: são utilizados blocos empilhados que combinam redes \textit{feed-forward} e camadas de \textit{self-attention} para o \textit{encoder} (à esquerda) e o \textit{decoder} (à direita); os \textit{embeddings} (abaixo) recebem uma codificação posicional para que seja considerada a ordem de suas sequências}
{vaswani-2017-transformer}



% Por outro lado, o \textit{Transformer}~\cite{vaswani-2017-transformer} é uma arquitetura que não é recorrente -- ao invés disso, baseia-se inteiramente num mecanismo de \textit{attention} (ou atenção) para estabelecer dependências globais entre os dados de entrada e saída.
% Ele é composta por blocos empilhados de redes multicamadas, as quais combinam camadas lineares simples, redes \textit{feed-forward} e camadas de \textit{self-attention} (ou auto-atenção) -- que, por sua vez, é a principal inovação desse tipo de arquitetura.
% \citeonline{wolf-2020-transformers} afirmam que o \textit{Transformer} tornou-se rapidamente a arquitetura dominante para o \acrshort{nlp} e tem superado modelos alternativos como \acrshortpl{cnn} e \acrshortpl{rnn} sobretudo em tarefas de compreensão e geração de linguagem natural. Além disso, sua arquitetura escalável facilita o treinamento paralelo eficiente e a captura do contexto de sequências muito longas
% \cite{vaswani-2017-transformer,jurafsky-2022-speech-lang-processing,wolf-2020-transformers}.




% Dentro da área de \acrshort{nlp}, tarefas que lidam com a linguagem num contexto semelhante ao que estamos endereçando neste trabalho comumente adotam arquiteturas conhecidas como \textit{Encoder-Decoder} (Codificador-Decodificador) ou \textit{Sequence-to-Sequence} (Sequência-para-Sequência) \cite{cho-2014-encoder-decoder,sutskever-2014-seq-to-seq}.
% Essas arquiteturas são compatíveis com diferentes tipos de modelagens sequenciais onde a sequência de saída é uma função complexa da sequência completa de entrada e ambas podem possuir comprimentos e ordens distintas \cite{jurafsky-2022-speech-lang-processing,goodfellow-2016-deep-learning}.

% O \textit{Encoder-Decoder} é composto por uma rede codificadora que recebe uma sequência de entrada e gera uma representação contextualizada dela -- que seria o contexto. Essa representação é então passada para um decodificador que produz uma sequência de saída específica para a tarefa em questão, conforme ilustra a \autoref{fig:encoder-decoder-arquitetura}. Uma otimização proposta por \citeonline{bahdanau-2015-mt-align-translate} também adota uma camada de \textit{attention} antes do decodificador para eliminar um gargalo observado ali.
% Por fim, o \textit{Encoder-Decoder} pode ser implementado utilizando-se \acrshortpl{rnn} e os \textit{Transformers}, por sua vez, já possuem uma arquitetura baseada nele \cite{jurafsky-2022-speech-lang-processing}.



% Encoder-decoder or sequence-to-sequence models are used for a different kind of sequence modeling in which the output sequence is a complex function of the entire input sequencer; we must map from a sequence of input words or tokens to a sequence of tags that are not merely direct mappings from individual words. 
% Machine translation is exactly such a task: the words of the target language don’t necessarily agree with the words of the source language in number or order.

% Encoder-decoder networks are very successful at handling these sorts of complicated cases of sequence mappings. Indeed, the encoder-decoder algorithm is not just for MT; it’s the state of the art for many other tasks where complex mappings between two sequences are involved. These include summarization (where we map from a long text to its summary, like a title or an abstract), dialogue (where we map from what the user said to what our dialogue system should respond), semantic parsing (where we map from a string of words to a semantic representation like logic or SQL), and many others.
% Encoder-decoder networks, or sequence-to-sequence networks, are models capable of generating contextually appropriate, arbitrary length, output sequences. Encoder-decoder networks have been applied to a very wide range of applications including machine translation, summarization, question answering, and dialogue.

% \cite{goodfellow-2016-deep-learning}
% This comes up inmany applications, such as speech recognition, machine translation and questionanswering, where the input and output sequences in the training set are generallynot of the same length (although their lengths might be related).








% This chapter introduces two important deep learning architectures designed to address these challenges: recurrent neural networks and transformer networks. Both approaches have mechanisms to deal directly with the sequential nature of language that allow them to capture and exploit the temporal nature of language. The recurrent network offers a new way to represent the prior context, allowing the model’s decision to depend on information from hundreds of words in the past. The transformer offers new mechanisms (self-attention and positional encodings) that help represent time and help focus on how words relate to each other over long distances.


% LSTM
% LSTM networks have been shown to learn long-term dependencies more easilythan the simple recurrent architectures, first on artificial datasets designed fortesting the ability to learn long-term dependencies (Bengio et al., 1994; Hochreiterand Schmidhuber, 1997; Hochreiter et al., 2001), then on challenging sequenceprocessing tasks where state-of-the-art performance was obtained (Graves, 2012;Graves et al., 2013; Sutskever et al., 2014). Variants and alternatives to the LSTMthat have been studied and used are discussed next.
% The LSTM has been found extremely successfulin many applications, such as unconstrained handwriting recognition (Graveset al., 2009), speech recognition (Graves et al., 2013; Graves and Jaitly, 2014),handwriting generation (Graves, 2013), machine translation (Sutskever et al., 2014),image captioning (Kiros et al., 2014b; Vinyals et al., 2014b; Xu et al., 2015), andparsing (Vinyals et al., 2014a).
% \cite{goodfellow-2016-deep-learning}


% GRU
% Which pieces of the LSTM architecture are actually necessary? What other successful architectures could be designed that allow the network to dynamicallycontrol the time scale and forgetting behavior of different units? Some answers to these questions are given with the recent work on gated RNNs, whose units are also known as gated recurrent units, or GRUs (Cho et al., 2014b;Chung et al., 2014, 2015a; Jozefowicz et al., 2015; Chrupala et al., 2015).
% \cite{goodfellow-2016-deep-learning}

% The main difference with the LSTM is that a single gating unit simultaneously controls the forgetting factor and the decision to update the state unit.
% \cite{goodfellow-2016-deep-learning}

% This evolution has recently led to a novel architecture called Gated Recurrent Unit (GRU) [8], that simplifies the complex LSTM cell design.
% [...]
% A noteworthy attempt to simplify LSTMs has recently led to a novel model called Gated Recurrent Unit (GRU) [8], [47], that is based on just two multiplicative gates.
% \cite{ravanelli-2018-li-gru}


% TRANSFORMER
% \cite{jurafsky-2022-speech-lang-processing}
% transformers – an approach to sequence processing that eliminates recurrent connections and returns to architectures reminiscent of the fully connected networks described earlier in Chapter 7.
% Transformers map sequences of input vectors (x1; :::;xn) to sequences of output vectors (y1; :::;yn) of the same length. 
% Transformers are made up of stacks of transformer blocks, which are multilayer networks made by combining simple linear layers, feedforward networks, and self-attention layers, the key innovation of transformers. Self-attention allows a network to directly extract and use information from arbitrarily large contexts without the need to pass it through intermediate recurrent connections as in RNNs. We’ll start by describing how self-attention works and then return to how it fits into larger transformer blocks.


% \cite{vaswani-2017-transformer}
% In this work we propose the Transformer, a model architecture eschewing recurrence and instead relying entirely on an attention mechanism to draw global dependencies between input and output. The Transformer allows for significantly more parallelization and can reach a new state of the art in translation quality after being trained for as little as twelve hours on eight P100 GPUs.


% \cite{wolf-2020-transformers}
% Transformer architectures have facilitated building higher-capacity models and pretraining has made it possible to effectively utilize this capacity for a wide variety of tasks.

% The Transformer (Vaswani et al., 2017) has rapidly become the dominant architecture for natural language processing, surpassing alternative neural models such as convolutional and recurrent neural networks in performance for tasks in both natural language understanding and natural language generation. The architecture scales with training data and model size, facilitates efficient parallel training, and captures long-range sequence features



% 10. MACHINE TRANSLATION AND ENCODER-DECODER MODELS
% \cite{jurafsky-2022-speech-lang-processing}

% This chapter introduces machine translation (MT), the use of computers to translate from one language to another.
% Of course translation, in its full generality, such as the translation of literature, or poetry, is a difficult, fascinating, and intensely human endeavor, as rich as any other area of human creativity.
% Machine translation in its present form therefore focuses on a number of very practical tasks. Perhaps the most common current use of machine translation is information for information access.

% Another common use of machine translation is to aid human translators. MT systems are routinely used to produce a draft translation that is fixed up in a post-editing phase by a human translator. This task is often called computer-aided translation or CAT. CAT is commonly used as part of localization: the task of adapting content or a product to a particular language community.

% Finally, a more recent application of MT is to in-the-moment human communication needs. This includes incremental translation, translating speech on-the-fly before the entire sentence is complete, as is commonly used in simultaneous interpretation. Image-centric translation can be used for example to use OCR of the text on a phone camera image as input to an MT system to translate menus or street signs.

% The standard algorithm for MT is the encoder-decoder network, also called the sequence to sequence network, an architecture that can be implemented with RNNs or with Transformers. We’ve seen in prior chapters that RNN or Transformer architecture can be used to do classification (for example to map a sentence to a positive or negative sentiment tag for sentiment analysis), or can be used to do sequence labeling (for example to assign each word in an input sentence with a part-of-speech, or with a named entity tag). For part-of-speech tagging, recall that the output tag is associated directly with each input word, and so we can just model the tag as output yt for each input word xt .
% Encoder-decoder or sequence-to-sequence models are used for a different kind of sequence modeling in which the output sequence is a complex function of the entire input sequencer; we must map from a sequence of input words or tokens to a sequence of tags that are not merely direct mappings from individual words. 
% Machine translation is exactly such a task: the words of the target language don’t necessarily agree with the words of the source language in number or order.
% [...]
% Encoder-decoder networks are very successful at handling these sorts of complicated cases of sequence mappings. Indeed, the encoder-decoder algorithm is not just for MT; it’s the state of the art for many other tasks where complex mappings between two sequences are involved. These include summarization (where we map from a long text to its summary, like a title or an abstract), dialogue (where we map from what the user said to what our dialogue system should respond), semantic parsing (where we map from a string of words to a semantic representation like logic or SQL), and many others.

% 10.2 The Encoder-Decoder Model
% Encoder-decoder networks, or sequence-to-sequence networks, are models capable of generating contextually appropriate, arbitrary length, output sequences. Encoder-decoder networks have been applied to a very wide range of applications including machine translation, summarization, question answering, and dialogue.
% The key idea underlying these networks is the use of an encoder network that takes an input sequence and creates a contextualized representation of it, often called the context. This representation is then passed to a decoder which generates a task specific output sequence. Fig. 10.3 illustrates the architecture
% [FIGURE]

% Encoder-decoder networks consist of three components:
% 1. An encoder that accepts an input sequence, xn1, and generates a corresponding sequence of contextualized representations, hn1. LSTMs, convolutional networks, and Transformers can all be employed as encoders.
% 2. A context vector, c, which is a function of hn1, and conveys the essence of the input to the decoder.
% 3. A decoder, which accepts c as input and generates an arbitrary length sequence of hidden states hm1, from which a corresponding sequence of output states ym1, can be obtained. Just as with encoders, decoders can be realized by any kind of sequence architecture.

% 10.3 Encoder-Decoder with RNNs
% [...]

% 10.4 Attention
% The simplicity of the encoder-decoder model is its clean separation of the encoder—
% which builds a representation of the source text—from the decoder, which uses this
% context to generate a target text. In the model as we’ve described it so far, this
% context vector is hn, the hidden state of the last (nth) time step of the source text.
% This final hidden state is thus acting as a bottleneck: it must represent absolutely
% everything about the meaning of the source text, since the only thing the decoder
% knows about the source text is what’s in this context vector (Fig. 10.8). Information
% at the beginning of the sentence, especially for long sentences, may not be equally
% well represented in the context vector.
% The attention mechanism is a solution to the bottleneck problem, a way of
% allowing the decoder to get information from all the hidden states of the encoder,
% not just the last hidden state.

% The idea of attention is instead to create the single fixed-length vector c by taking
% a weighted sum of all the encoder hidden states. The weights focus on (‘attend
% to’) a particular part of the source text that is relevant for the token the decoder is
% currently producing. Attention thus replaces the static context vector with one that
% is dynamically derived from the encoder hidden states, different for each token in
% decoding.
% [...]
% The weights Ws, which are then trained during normal end-to-end training, give the
% network the ability to learn which aspects of similarity between the decoder and
% encoder states are important to the current application. This bilinear model also
% allows the encoder and decoder to use different dimensional vectors, whereas the
% simple dot-product attention requires that the encoder and decoder hidden states
% have the same dimensionality.

% 10.5 Beam Search
% [...]

% 10.6 Encoder-Decoder with Transformers
% The encoder-decoder architecture can also be implemented using transformers (rather
% than RNN/LSTMs) as the component modules. At a high-level, the architecture,
% sketched in Fig. 10.15, is quite similar to what we saw for RNNs. It consists of an
% encoder that takes the source language input words X = x1; :::;xT and maps them
% to an output representation Henc = h1; :::;hT ; usually via N = 6 stacked encoder
% blocks. The decoder, just like the encoder-decoder RNN, is essentially a conditional
% language model that attends to the encoder representation and generates the target
% words one by one, at each timestep conditioning on the source sentence and the
% previously generated target language words.
% [IMAGE]
% But the components of the architecture differ somewhat from the RNN and also
% from the transformer block we’ve seen. First, in order to attend to the source language,
% the transformer blocks in the decoder has an extra cross-attention layer.
% Recall that the transformer block of Chapter 9 consists of a self-attention layer that
% attends to the input from the previous layer, followed by layer norm, a feed forward
% layer, and another layer norm. The decoder transformer block includes an
% extra layer with a special kind of attention, cross-attention (also sometimes called
% encoder-decoder attention or source attention). Cross-attention has the same form
% as the multi-headed self-attention in a normal transformer block, except that while
% the queries as usual come from the previous layer of the decoder, the keys and values
% come from the output of the encoder.




% \cite{goodfellow-2016-deep-learning}
% https://www.deeplearningbook.org/contents/rnn.html

% 10.4 Encoder-Decoder Sequence-to-Sequence Architectures
% Here we discuss how an RNN can be trained to map an input sequence to anoutput sequence which is not necessarily of the same length. This comes up inmany applications, such as speech recognition, machine translation and questionanswering, where the input and output sequences in the training set are generallynot of the same length (although their lengths might be related).
% [IMAGE]
% The simplest RNN architecture for mapping a variable-length sequence toanother variable-length sequence was first proposed by Cho et al. (2014a) [https://aclanthology.org/D14-1179/] and shortly after by Sutskever et al. (2014) [https://arxiv.org/abs/1409.3215], who independently developed that architecture and were the first to obtain state-of-the-art translation using this approach.

% The former system is based on scoring proposals generated by another machinetranslation system, while the latter uses a standalone recurrent network to generatethe translations. These authors respectively called this architecture, illustratedin figure 10.12, the encoder-decoder or sequence-to-sequence architecture. Theidea is very simple: (1) AnencoderorreaderorinputRNN processes the inputsequence. The encoder emits the contextC, usually as a simple function of itsfinal hidden state. (2) AdecoderorwriteroroutputRNN is conditioned onthat fixed-length vector (just as in figure 10.9) to generate the output sequenceY= (y(1), . . . , y(ny)). 

% One clear limitation of this architecture is when the contextCoutput by theencoder RNN has a dimension that is too small to properly summarize a longsequence. This phenomenon was observed by Bahdanau et al. (2015) in the contextof machine translation. They proposed to makeCa variable-length sequence ratherthan a fixed-size vector. Additionally, they introduced anattention mechanismthat learns to associate elements of the sequenceCto elements of the outputsequence. See section 12.4.5.1 for more details.







% \cite{jurafsky-2022-speech-lang-processing}
% The most commonly used such extension to RNNs is the Long short-term
% memory (LSTM) network (Hochreiter and Schmidhuber, 1997). LSTMs divide the context management problem into two sub-problems: removing information no longer needed from the context, and adding information likely to be needed for later decision making. The key to solving both problems is to learn how to manage this context rather than hard-coding a strategy into the architecture. LSTMs accomplish this by first adding an explicit context layer to the architecture (in addition to the usual recurrent hidden layer), and through the use of specialized neural units that make use of gates to control the flow of information into and out of the units that comprise the network layers. These gates are implemented through the use of additional weights that operate sequentially on the input, and previous hidden layer, and previous context layers.
% The gates in an LSTM share a common design pattern; each consists of a feedforward layer, followed by a sigmoid activation function, followed by a pointwise multiplication with the layer being gated. The choice of the sigmoid as the activation function arises from its tendency to push its outputs to either 0 or 1. Combining this with a pointwise multiplication has an effect similar to that of a binary mask. Values in the layer being gated that align with values near 1 in the mask are passed through nearly unchanged; values corresponding to lower values are essentially erased.



% ======================================
% \cite{goodfellow-2016-deep-learning}
% 
% Chapter 10
% https://www.deeplearningbook.org/contents/rnn.html
%
% 10.10 The Long Short-Term Memory and Other GatedRNNs (pg 404)
% As of this writing, the most effective sequence models used in practical applications are called gated RNNs. These include the long short-term memory and networks based on the gated recurrent unit.
% Like leaky units, gated RNNs are based on the idea of creating paths through time that have derivatives that neither vanish nor explode. Leaky units did this with connection weights that were either manually chosen constants or were parameters. Gated RNNs generalize this to connection weights that may change at each time step.
% Leaky units allow the network to accumulate information (such as evidence fora particular feature or category) over a long duration. Once that information has been used, however, it might be useful for the neural network to forget the old state. For example, if a sequence is made of subsequences and we want a leaky unit to accumulate evidence inside each sub-subsequence, we need a mechanism to forget the old state by setting it to zero. Instead of manually deciding when to clear the state, we want the neural network to learn to decide when to do it. This is what gated RNNs do.

% 10.10.1 LSTM
% The clever idea of introducing self-loops to produce paths where the gradientcan flow for long durations is a core contribution of the initiallong short-termmemory(LSTM) model (Hochreiter and Schmidhuber, 1997). A crucial additionhas been to make the weight on this self-loop conditioned on the context, rather thanfixed (Gers et al., 2000). By making the weight of this self-loop gated (controlledby another hidden unit), the time scale of integration can be changed dynamically.In this case, we mean that even for an LSTM with fixed parameters, the time scaleof integration can change based on the input sequence, because the time constantsare output by the model itself. 
% The LSTM has been found extremely successfulin many applications, such as unconstrained handwriting recognition (Graveset al., 2009), speech recognition (Graves et al., 2013; Graves and Jaitly, 2014),handwriting generation (Graves, 2013), machine translation (Sutskever et al., 2014),image captioning (Kiros et al., 2014b; Vinyals et al., 2014b; Xu et al., 2015), andparsing (Vinyals et al., 2014a).
% [...]
% Deeper architectures have also been successfully used (Graves et al.,2013; Pascanu et al., 2014a). Instead of a unit that simply applies an element-wisenonlinearity to the affine transformation of inputs and recurrent units, LSTMrecurrent networks have “LSTM cells” that have an internal recurrence (a self-loop),in addition to the outer recurrence of the RNN. Each cell has the same inputs andoutputs as an ordinary recurrent network, but also has more parameters and asystem of gating units that controls the flow of information.
% LSTM networks have been shown to learn long-term dependencies more easilythan the simple recurrent architectures, first on artificial datasets designed fortesting the ability to learn long-term dependencies (Bengio et al., 1994; Hochreiterand Schmidhuber, 1997; Hochreiter et al., 2001), then on challenging sequenceprocessing tasks where state-of-the-art performance was obtained (Graves, 2012;Graves et al., 2013; Sutskever et al., 2014). Variants and alternatives to the LSTMthat have been studied and used are discussed next.

% 10.10.2 Other Gated RNNs
% Which pieces of the LSTM architecture are actually necessary? What other successful architectures could be designed that allow the network to dynamically control the time scale and forgetting behavior of different units? Some answers to these questions are given with the recent work on gated RNNs,whose units are also known as gated recurrent units, or GRUs (Cho et al., 2014b;Chung et al., 2014, 2015a; Jozefowicz et al., 2015; Chrupala et al., 2015). The main difference with the LSTM is that a single gating unit simultaneously controls the forgetting factor and the decision to update the state unit.



% ================================
% \cite{lecun-2015-deep-learning}
% RNNs, once unfolded in time (Fig. 5), can be seen as very deep feedforward networks in which all the layers share the same weights. Although their main purpose is to learn long-term dependencies, theoretical and empirical evidence shows that it is difficult to learn to store information for very long78.
% To correct for that, one idea is to augment the network with an explicit memory. The first proposal of this kind is the long short-term memory (LSTM) networks that use special hidden units, the natural behaviour of which is to remember inputs for a long time79. A special unit called the memory cell acts like an accumulator or a gated leaky neuron: it has a connection to itself at the next time step that has a weight of one, so it copies its own real-valued state and accumulates the external signal, but this self-connection is multiplicatively gated by another unit that learns to decide when to clear the content of the memory.
% LSTM networks have subsequently proved to be more effective than conventional RNNs, especially when they have several layers for each time step87, enabling an entire speech recognition system that goes all the way from acoustics to the sequence of characters in the transcription. LSTM networks or related forms of gated units are also currently used for the encoder and decoder networks that perform so well at machine translation17,72,76


% Materiais e Métodos
\chapter{Materiais e métodos}
\label{cap:metodos}

Neste capítulo será discutida em mais detalhes a abordagem proposta, as justificativas para sua adoção, bem como as técnicas que foram aplicadas e a preparação dos experimentos realizados.

Esta pesquisa propõe-se a realizar o reconhecimento da linguagem de sinais a partir de uma perspectiva estritamente linguística, baseada nos constituintes fonológicos mínimos que descrevem os sinais, e centrada no aprendizado das complexidades e regras que convém contexto e dá significado a eles.

Isso assemelha-se à forma como hoje outras línguas são abordadas com sucesso pelo \acrfull{nlp} e diferencia-se daquela predominante no \acrfull{slr} que, por sua vez, trata os sinais como gestos não-estruturados, mapeados a partir de dados brutos capturados dos indivíduos -- como pixels de imagens ou \textit{frames} de vídeos, pontos lidos de luvas eletrônicas, coordenadas 2D ou 3D, entre outros -- e colocam em segundo plano a importância linguística do sinal.


A hipótese deste trabalho assume que, além de deixar de abordar uma parte muito importante dessa língua, lidar com esse tipo de dados brutos traz complexidades adicionais que extrapolam o escopo que deveria ser efetivamente abordado pelo \acrshort{slr}.
Em outras palavras, esse foco inadequado faz com que pesquisas em \acrshort{slr} deixem de solucionar um problema intrinsecamente de \acrshort{nlp} e passem a investir esforços consideráveis tentando lidar com um conjunto de desafios pertinentes à área de \acrfull{cv} -- os quais comumente já estão abordados ou solucionados por uma de suas subáreas, como a detecção, segmentação e rastreamento de partes do corpo em imagens e vídeos; a interação entre mãos e oclusões decorrentes disso; as variações de tom de pele, cores de roupa e luminosidade do ambiente; entre outros listados nas revisões literárias elaboradas no decorrer da última década para a \acrshort{slr} \cite{papastratis-2021-ai-technologies-sl,rastgoo-2021-slr-deep-survey,koller-2020-quantitative-survey-slr,bragg-2019-slr-interdisciplinary,wadhawan-2019-slr-literature-review,suharjito-2018-feature-extraction-survey,joksimoski-2022-scoping-review,cooper-2011-slr}.


Alguns exemplos populares desses problemas sendo consistentemente endereçados dentro da \acrshort{cv} incluem as ferramentas OpenPose~\cite{wei-2016-conv-machines-openpose,cao-2017-openpose,simon-2017-openpose-hand-face} e MediaPipe~\cite{lugaresi-2019-mediapipe,bazarevsky-2019-mediapipe-blazeface,vakunov-2020-mediapipe-hands,bazarevsky-2020-mediapipe-blazepose}, desenvolvidas pela Carnegie Mellon University e Google Research, respectivamente.
Ambas são o resultado de anos de pesquisa em torno de tais questões, as quais alcançaram um nível de maturidade elevado capaz de abordar em tempo real tarefas de estimativa de pose e rastreamento do corpo, mãos e face (inclusive envolvendo múltiplos indivíduos) de forma robusta a variações corporais, de luminosidade e de ambientes, utilizando apenas uma câmera comum RGB.
Elas estão disponíveis abertamente\footnote{
    O OpenPose está disponível em \url{https://github.com/CMU-Perceptual-Computing-Lab/openpose} e o MediaPipe em \url{https://mediapipe.dev/}.
} e a reutilização desse conhecimento nas etapas para capturar e gerar \textit{features} de níveis mais elevados para o \acrshort{slr} certamente contribuirá para progressos mais efetivos.


Fazendo uma analogia com outras tarefas de \acrshort{nlp}, abordar a língua de sinais por meio dos dados brutos como discutido acima e lidar com os desafios apresentados, por exemplo, possui uma complexidade equivalente a tentar interpretar textos manuscritos apenas rastreando-se o movimento da mão do autor enquanto ele desenha as letras no papel -- ao invés de simplesmente escanear o texto final escrito como entrada para isso; ou ainda, tentar reconhecer a fala de um indivíduo apenas realizando a detecção e o rastreamento dos movimentos de seus lábios -- ao invés de considerar os sinais de áudio capturados para tal.


Como resultado desse enquadramento inadequado por parte das pesquisas em \acrshort{slr}, no decorrer das últimas décadas constata-se um progresso pouco expressivo dessa área sobretudo nos aspectos da linguagem e aplicabilidade no mundo real, acerca do qual \citeonline{selvaraj-2022-openhands,yin-2021-sl-in-nlp,cooper-2011-slr} reiteram:


\begin{citacao}
    Quando comparado com a pesquisa de \acrlong{nlp} baseada em texto e fala, o progresso das pesquisas para línguas de sinais está significativamente atrasado. \cite[tradução nossa]{selvaraj-2022-openhands,yin-2021-sl-in-nlp}
\end{citacao}

\begin{citacao}
    Enquanto sistemas de reconhecimento da fala avançaram ao ponto de estarem comercialmente disponíveis, o reconhecimento de sinais ainda está em sua infância.
    Atualmente, todos os serviços comerciais de tradução de sinais são baseados em humanos e requerem que pessoal especializado esteja disponível, o que os tornam caros e pouco acessíveis. \cite[tradução nossa]{cooper-2011-slr}
\end{citacao}


Dessa forma, considerando a discussão desenvolvida até aqui, nesta pesquisa a \acrshort{slr} será posicionada como uma tarefa de \acrshort{nlp}, delimitando-se seu escopo ao âmbito da linguística e representando-se os sinais através de seus fonemas. Além disso, será aplicado o conhecimento disponibilizado pelas ferramentas acima para criar \textit{features} que representem estes fonemas, viabilizando este processo.
Com isso, objetiva-se eliminar o escopo pertinente a outras áreas de pesquisa e concentrar a capacidade dos modelos aplicados ao aprendizado das regras e restrições linguísticas da língua de sinais.

Tal estratégia de abordar a linguagem por meio de suas unidades constituintes mínimas é também observada em outras tarefas de \acrshort{nlp}. \citeonline{jurafsky-2022-speech-lang-processing} afirmam que a ideia da palavra falada ser composta por unidades menores da fala é adotada, por exemplo, por algoritmos utilizados em tarefas de reconhecimento de fala e de conversão de texto em voz.
Observe na \autoref{fig:exemplo-waveform-phone} o exemplo ilustrado pelos autores da forma de onda da fala para a sentença em inglês ``\textit{she just had a baby}'' (ou ``ela acabou de ter um bebê'').
Cada trecho é rotulado na linha inferior com suas respectivas partículas mínimas de som (ou ``fones''), as quais são transcritas utilizando-se o ARPAbet\footnote{
    ARPAbet é um alfabeto fonético simples introduzido por \citeonline{shoup-1980-arpabet} que utiliza símbolos ASCII para representar um subconjunto do \acrshort{ipa} que se refere ao idioma inglês-americano. O \acrfull{ipa}, por sua vez, é a representação fonética padrão para a transcrição das línguas ao redor do mundo \cite{jurafsky-2022-speech-lang-processing}.
}. Esse tipo de partícula é comumente utilizado como \textit{feature} de entrada para tarefas envolvendo o processamento da fala.


\figura[p. 586]
{fig:exemplo-waveform-phone}
{capitulos/metodos/imagens/exemplo_waveform_phone}
{width=0.90\textwidth}
{Formas de onda da fala para a sentença ``\textit{she just had a baby}'' (primeira linha) rotuladas com suas respectivas partículas de som transcritas em ARPAbet (linha inferior)}
{jurafsky-2022-speech-lang-processing}



No caso da abordagem proposta aqui, essas partículas serão substituídas por alguns dos parâmetros fonológicos introduzidos na \autoref{sec:linguistica}. Uma vez que não há \textit{datasets} disponíveis com esse tipo de representação para as línguas de sinais, o primeiro passo consistirá em gerar esse \textit{dataset}. Para isso, o \acrshort{asllvd} será adotado como \textit{dataset} base e suas amostras serão processadas utilizando-se o OpenPose, o qual fornece coordenadas que possibilitam a extração de \textit{features} fonológicas de nível semântico mais elevado.
Em seguida, serão aplicados modelos de \acrfull{ml} comumente utilizados em tarefas de \acrshort{nlp}, com o intuito de avaliar seu desempenho neste contexto e a eficácia da abordagem proposta. 



A \autoref{fig:etapas-abordagem} ilustra essa abordagem, a qual é divida em duas etapas.
No bloco à esquerda, observa-se o processo envolvido na geração do \textit{dataset}, que inicia-se pelas amostras do \acrshort{asllvd}, contempla a obtenção de coordenadas 3D por meio de ferramentas de \acrshort{cv} e finaliza com a geração do ASL-Phono, que contém os respectivos atributos fonológicos.
No bloco à direita, está a etapa de \acrlong{slr}, que engloba a preparação das \textit{features}, o processamento dessas \textit{features} pelos modelos de \acrshort{ml} e a classificação dos sinais.
Todas essas etapas serão discutidas em detalhes nas seções a seguir.

\figura
{fig:etapas-abordagem}
{capitulos/metodos/imagens/etapas_abordagem}
{width=0.95\textwidth}
{Etapas envolvidas na abordagem proposta}
{}

\section{Novos \textit{datasets} da língua de sinais}
\label{sec:metodologia-datasets}

Conforme introduzimos acima, o primeiro passo para que possamos desenvolver e avaliar uma abordagem de aprendizagem de máquina centrada na linguística da língua de sinais consiste em definir um conjunto de dados que viabilize isso. Como a proposta apresentada aqui é nova e, devido a isso, não há \textit{dataset}s diretamente compatíveis com ela, optamos por derivar um novo \textit{dataset} a partir de outro já existente -- o \acrfull{asllvd}.

O \acrshort{asllvd} consiste em um amplo \textit{dataset} público\footnote{Disponível em \url{http://www.bu.edu/asllrp/av/dai-asllvd.html}} da \acrshort{asl} que contém aproximadamente 2.745 sinais representados em cerca de 9.763 sequências de vídeo articuladas por indivíduos Surdos nativos. Suas amostras são capturadas por meio de quatro câmeras sincronizadas: uma visão frontal de alta resolução a meia velocidade, outra visão frontal de resolução total, uma visão lateral e uma visão da face, conforme ilustrado na \autoref{fig:asllvd-example}~\cite{athitsos-2008-asllvd,neidle-2012-asllvd}.

\begin{figure}[ht!]
    \centering
    \caption{\textmd{Exemplo de três perspectivas sincronizadas providas pelo \acrshort{asllvd} para o sinal MERRY-GO-ROUND: 
    vista frontal (\subref{subfig:asllvd-example-front}), 
    vista lateral (\subref{subfig:asllvd-example-side}) e 
    vista da face (\subref{subfig:asllvd-example-close}).}}
    \subcaptionbox{\label{subfig:asllvd-example-front}}{
        \includegraphics[width=0.25\textwidth]{capitulos/metodologia/imagens/asllvd_example_front}
    }%
    \hfill
    \subcaptionbox{\label{subfig:asllvd-example-side}}{
        \includegraphics[width=0.25\textwidth]{capitulos/metodologia/imagens/asllvd_example_side}
    }%
    \hfill
    \subcaptionbox{\label{subfig:asllvd-example-close}}{
        \includegraphics[width=0.25\textwidth]{capitulos/metodologia/imagens/asllvd_example_close}
    }%
    \nomefonte[p. 2]{athitsos-2008-asllvd}
    \label{fig:asllvd-example}
\end{figure}


Para computar parâmetros fonológicos a partir dos frames das amostras do \acrshort{asllvd}, compostos essencialmente de imagens RGB bidimensionais, precisamos realizar um processo de duas etapas: primeiro, estimamos as coordenadas 2D dos esqueletos dos sinalizadores para duas câmeras distintas, frame-a-frame, e as combinamos para projetar um esqueleto no espaço 3D -- isso deu origem ao \textit{dataset} intermediário chamado ASL-Skeleton3D; em seguida, aplicamos um conjunto de operações algébricas sob o esqueleto 3D para calcular os parâmetros fonológicos -- o que gerou assim o nosso \textit{dataset} final chamado ASL-Phono.

Como pode-se imaginar, esse processo envolveu alguns desafios computacionais, entre os quais enumeram-se :

\begin{enumerate}
    \item Definição de abordagens para representar indivíduos no espaço 3D utilizando apenas frames de vídeo simples em 2D do \acrshort{asllvd}, bem como para lidar com as amostras ausentes ou de baixa qualidade encontradas nesse \textit{dataset}.

    \item Estabelecimento de um subconjunto de atributos fonológicos iniciais que pudessem capturar e representar variações significativas no corpo dos indivíduos para o reconhecimento dos sinais, mas que também pudessem ser modelados computacionalmente.
    
    \item Identificação de técnicas matemáticas e medidas antropométricas que suportassem o cálculo e a modelagem dos atributos selecionados.
    
    \item Disponibilização de recursos computacionais significativos para viabilizar o processamento de mais de 9.000 amostras contidas em cada \textit{dataset}, as quais envolveram duas câmeras distintas. Para isso, foram consumidas cerca 40 horas contínuas de processamento em cluster dispondo de CPU e GPU, e gerados mais de 1 TB de dados cada vez que ambos \textit{dataset}s precisaram ser completamente processados.
\end{enumerate}


% Dataset 3d
\input{capitulos/metodologia/datasets_3d}

% Dataset phono
\input{capitulos/metodologia/datasets_phono}

\section{Preparação dos dados}
\label{sec:metodos-preparacao-dataset}

% RESAMPLING DO DATASET -------------------------------------------

Conforme introduzido na seção anterior, o número de amostras disponíveis por sinal não está balanceado homogeneamente no ASL-Phono e, como consequência, isso poderia influenciar de maneira indesejada o desempenho dos modelos utilizados nos experimentos adiante, fazendo com que algumas classes fossem extremamente favorecidas e, outras, severamente penalizadas.

Devido a isso, serão aplicados dois procedimentos para tentar equalizar essa proporção. Primeiro, serão descartados aqueles sinais que apresentam apenas 1 amostra disponível, uma vez que esse número é insuficiente para permitir o modelo aprender e generalizar tais sinais, sobretudo porque o \textit{dataset} será particionado em mais de um subconjunto durante seu treinamento e todas as classes precisam estar igualmente representadas neles.

Em seguida, será realizada uma reamostragem do \textit{dataset} com o intuito de balancear melhor a proporção de amostras.
Será utilizado para isso uma reamostragem \textit{Naive Random Under-Sampling} (ou sub-amostragem aleatória ingênua), que reduz o número de amostras super-representadas selecionando aleatoriamente algumas delas e, em seguida, uma \textit{Naive Random Over-Sampling} (ou sobre-amostragem aleatória ingênua) que, por sua vez, aumenta o número de amostras sub-representadas replicando aleatoriamente algumas existentes \cite{he-2013-imbalanced}.

A \autoref{eqn:resampling-target} define a operação aplicada para definir o número de amostras \(n'\) a ser obtido para cada sinal por processo de reamostragem. Nela, \(\overline{m}\) refere-se à média de amostras por sinal e \(n\) é o número atual de amostras para aquele sinal:

\begin{equation}
    \label{eqn:resampling-target}
    n' = round( \overline{m} + \ln(n) )
\end{equation}

Observe na \autoref{fig:dataset-resampling-depois} a nova distribuição das amostras no \textit{dataset}. De uma forma resumida, ao comparar com a \autoref{fig:dataset-resampling-antes}, percebe-se que a reamostragem homogeneizou a relação de amostras por sinal em torno da nova média do conjunto, trazendo para essa região também os antigos \textit{outliers} (valores atípicos ou pontos fora da curva).

\figura
{fig:dataset-resampling-depois} % Label
{capitulos/metodos/imagens/dataset_resampling_depois} % Path
{height=5.5cm} % Size
{Distribuição do número de amostras por sinal após a reamostragem} % Caption
{} % Citation




% COMPACTAÇÃO DAS FEATURES -------------------------------------------

Por fim, será aplicada uma transformação às amostras do ASL-Phono para tornar a estrutura apresentada no \autoref{cod:sample-json-phono} compatível com a entrada dos modelos que serão aplicados mais adiante.
Para isso, serão compactados os valores informados para os atributos fonológicos de cada \textit{frame} em uma ``palavra'' única, fazendo com que a sequência de \textit{frames} seja então representada como uma sequência dessas palavras.

Por exemplo, considerando-se um \textit{frame} contendo dois atributos com os valores ``\textit{valor\_atributo\_1}'' e ``\textit{valor\_atributo\_2}'', ao compactá-los, eles primeiro seriam abreviados para ``\textit{va1}'' e ``\textit{va2}'' e, em seguida, concatenados para formar uma palavra ``\textit{va1-va2}''.  A sequência de \textit{frames} da amostra tornaria-se, portanto, algo semelhante a uma sequência de palavras \textit{\{``va1-va2'', ``va3-va4'', \dots, ``vaN-vaN''\}}.

Observe na \autoref{tab:codificacao-bloco} um exemplo mais próximo do contexto real para esse processo. Na primeira linha estão os valores originais dos atributos do \textit{frame}; na segunda, os valores abreviados; e, na terceira, a palavra formada através da concatenação deles.

\input{capitulos/metodos/tabelas/codificacao_bloco}

\section{Preparação dos modelos}
\label{sec:metodos-preparacao-modelos}

% SELEÇÃO DOS MODELOS -------------------------------------------

Tomando como referência a discussão introduzida na \autoref{sec:am-ap}, serão adotados nos experimentos deste trabalho três das principais arquiteturas utilizadas em tarefas de \acrfull{nlp}: o \textit{Encoder-Decoder} em uma versão com \acrfull{lstm} e outra com \acrfull{gru}, e também o \textit{Transformer}.

Para estabelecer os parâmetros dessas arquiteturas, as estratégias de otimização e de treinamento, bem como as métricas utilizadas nos experimentos, foram consideradas as discussões apresentadas por \citeonline{goodfellow-2016-deep-learning} e pela \autoref{sec:am}.

Dessa forma, o algoritmo de otimização dos modelos será definido como o \acrfull{sgd} com \textit{momentum} de 0,9 \cite{robbins-2007-stochastic}. Ele será combinado a uma estratégia de redução da taxa de aprendizagem por um fator de 0,2 sempre que o valor do erro médio calculado atingir um platô por 5 épocas seguidas.

A função objetivo (ou função de perda), por sua vez, será a \acrfull{cel} \cite{mitchell-1997-ml}, que é apresentada na \autoref{eqn:cross-entropy-loss}. Nela, \(p\) representa as probabilidades ou pontuações estimadas pelo modelo para as amostras e \(y\) corresponde ao valor correto esperado para essas estimativas:

\begin{equation}
    \label{eqn:cross-entropy-loss}
    L_{\log}(y, p) = -(y \log (p) + (1 - y) \log (1 - p))
\end{equation}


Os dados serão particionados numa proporção de 15\% para o subconjunto de validação, 15\% para o de testes e o restante para o subconjunto treinamento. Os \textit{batches} (ou lotes), por sua vez, possuirão tamanho de 50 amostras.



A seleção dos hiperparâmetros dos modelos foi realizada utilizando-se o algoritmo \textit{Grid Search} (ou busca em grade) com validação cruzada de 5 \textit{folds}. O conjunto de valores de hiperparâmetros utilizados na busca estão apresentados na \autoref{tab:otim-params} e as combinações que melhor reduziram o erro médio para cada modelo foram as seguintes:

\input{capitulos/metodos/tabelas/otim_params}

\begin{itemize}
    \item \textit{Encoder-Decoder} com \acrshort{lstm}: taxa de aprendizagem de 0,1; \textit{dropout} de 0,1; \textit{embeddings} com dimensão de 1024; camadas ocultas com dimensão de 512; e utilização de 2 camadas de \acrshort{lstm} no \textit{encoder} e no \textit{decoder}.

    \item \textit{Encoder-Decoder} com \acrshort{gru}: taxa de aprendizagem de 0,01; \textit{dropout} de 0,1; \textit{embeddings} com dimensão de 1024; camadas ocultas com dimensão de 512; e utilização de 2 camadas de \acrshort{gru} no \textit{encoder} e no \textit{decoder}.

    \item \textit{Transformer}: taxa de aprendizagem de 0,1; \textit{dropout} de 0,5; \textit{embeddings} com dimensão de 512; camadas ocultas com dimensão de 512; utilização de 2 camadas e de 8 cabeças de \textit{attention}.
\end{itemize}



O código-fonte utilizado nos experimentos deste trabalho foi disponibilizado através do endereço indicado abaixo\footnote{
    Disponível em \url{https://www.cin.ufpe.br/~cca5/sl-nlp}.
}.


% Avaliação experimental
\chapter{Avaliação experimental}
\label{cap:avaliacao}

Nos capítulos anteriores, foi introduzida a abordagem proposta e as técnicas utilizadas na preparação e execução dos experimentos deste trabalho.
Neste capítulo, portanto, serão analisados os resultados apresentados por esses experimentos.
Isso será realizado em duas partes: 
primeiro, serão discutidos os resultados coletadas para cada um dos modelos utilizados; em seguida, serão comparados esses resultados com aqueles obtidos por outras pesquisas em \acrfull{slr} que também adotaram o \acrfull{asllvd} como base de análise, mas que seguiram por abordagens distintas.

\section{Análise dos resultados}
\label{sec:avaliacao-resultados}

A \autoref{tab:resultados-modelos} apresenta o desempenho dos modelos utilizados nos experimentos deste trabalho, respectivamente o \textit{Encoder-Decoder} implementado com \acrshort{lstm}, o \textit{Encoder-Decoder} implementado com \acrshort{gru} e o \textit{Transformer}.
Para cada modelo, são listadas as métricas\footnote{
    Uma vez que o reconhecimento de sinais consiste numa classificação multi-classes, os valores das métricas binárias de precisão, \textit{recall} e \textit{F1-score} foram consolidados utilizando-se a média ponderada pelo número de amostras em cada classe.
} de acurácia, precisão, \textit{recall} e \textit{F1-score}, bem como o erro médio calculado.

\input{capitulos/avaliacao/tabelas/resultados-modelos}

Primeiramente, observa-se que os modelos baseados na arquitetura \textit{Encoder-Decoder} apresentaram resultados muito semelhantes entre si de um modo geral.

Dentre eles, percebe-se também que a versão implementada utilizando codificador e decodificador baseados em redes \acrshort{gru} obteve uma pequena vantagem em comparação àquela que utilizou redes \acrshort{lstm} -- ao passo que a primeira alcançou uma acurácia de 88,00\%, a segunda obteve 87,21\%.
O valor das demais métricas e do erro médio computado para ambos os casos replicaram esse mesmo comportamento da acurácia e reforçam tal análise.

Por outro lado, quando analisados os resultados do \textit{Transformer} observa-se um desempenho expressivo com uma acurácia de 100,00\%, a qual excede aquelas apresentadas pelos \textit{Encoder-Decoders} e é reiterada consistentemente pelo valor das demais métricas e também da perda computada.

Isso nos remete à argumentação de \citeonline{wolf-2020-transformers,jurafsky-2022-speech-lang-processing} citada na \autoref{sec:am-ap}, que afirma que essa arquitetura tem se mostrado extremamente bem-sucedida entre tarefas de \acrfull{nlp}, superando arquiteturas como as \acrshortpl{rnn}. Exatamente por isso, elas são atualmente dominantes nessa área.

No contexto deste trabalho, acredita-se que alguns motivos particulares contribuíram para os resultados acima.
Em primeiro lugar, entende-se que a escolha pelo uso da abordagem linguística dos sinais nos permitiu trabalhar num nível semântico muito mais elevado do que seria possível fazer com dados brutos como os pixels de imagens RGB ou coordenadas aleatórias no espaço. Conforme discutido no \autoref{cap:metodos}, essa é uma abordagem comum em tarefas envolvendo linguagem no \acrshort{nlp} e que permite produzir \textit{features} de melhor qualidade para ensinar os modelos acerca da estrutura dessas línguas.

Em segundo lugar, a representação introduzida aqui foi capaz de transformar um conjunto de canais linguísticos complexos dos sinais do \acrshort{asllvd}, em \textit{features} discretas (ou ``palavras'', como aqui as denominamos) mais simples e bem definidas a serem consumidas pelos modelos.
Isso deu origem a um vocabulário que, por sua vez, é muito menos complexo do que aqueles com os quais arquiteturas como o \textit{Transformer} foram originalmente projetadas para lidar. Além disso, nesse processo foi selecionado um número menor de atributos fonológicos da língua de sinais e isso também contribuiu para tornar o vocabulário utilizado aqui ainda mais compacto.

Dessa forma, entende-se que a combinação desse conjunto de \textit{features} semanticamente mais coerentes e simplificadas, com a utilização de modelos robustos no processamento de linguagens como o \textit{Transformer} e os \textit{Encoder-Decoders} foram fatores que conduziram ao desempenho favorável acima.






A \autoref{tab:custo-modelos} apresenta o custo computacional calculado para os modelos adotados neste trabalho ao realizar a inferência de 1.821 amostras durante a etapa de testes. Ela apresenta métricas de tempo e memória para CPU e GPU, bem como o número de \acrfull{flops} para cada um desses modelos.
Ao observar a métrica de tempo de CPU, percebe-se que o \textit{Encoder-Decoder} (\acrshort{gru}) foi aquele que demandou o maior tempo (5,790 segundos), seguido pelo \textit{Transformer} (com 3,378 segundos) e o \textit{Encoder-Decoder} (\acrshort{lstm}) (com 2,912 segundos). 
Com relação à memória de CPU utilizada, os três modelos apresentaram um consumo homogêneo, permanecendo numa faixa entre 335 e 337 MB.

\input{capitulos/avaliacao/tabelas/custo-modelos}

Ao analisar os recursos de GPU, percebe-se que dessa vez o \textit{Encoder-Decoder} (\acrshort{lstm}) demandou o maior tempo (3,101 segundos), seguido pelo \textit{Encoder-Decoder} (\acrshort{gru}) (2,628 segundos) e pelo \textit{Transformer} (com 1,884 segundos). A utilização de memória, por sua vez, manteve-se dentro de uma faixa que oscilou de 24 GB (no caso do \textit{Encoder-Decoder} (\acrshort{gru})) a 30 GB (no caso do \textit{Encoder-Decoder} (\acrshort{lstm})).

Por fim, ao avaliar o número de \acrshort{flops} nota-se que os modelos baseados na arquitetura \textit{Encoder-Decoder} apresentaram o mesmo número de operações. Contudo, no caso do \textit{Transformer} observa-se um número de operações que é aproximadamente cinco vezes maior que os demais modelos.

% considerações importantes
% - são pesquisas que adotam abordagens que utilizam dados brutos (RGB, coordenadas)
% - muitas delas adotam recortes das mãos apenas
% - não consideram a fonologia
% - utilizam um número de sinais limitados, com o intuito de reduzir a complexidade -- consequentemente
%     - isso contribui para que eles obtenham acurácias maiores
%     - contudo, limita o problema perante o contexto real -- uma vez que a afasta da realidade da língua

% insights [ASLLVD tem 2.745 sinais]
% - ao comparar nossos experimentos com todos os trabalhos acima, de uma forma generalizada, vemos que o desempenho dos encoder-decoders (por volta dos 46%) foi mediano com relação aos demais (que em vários deles alcançaram marcas de 70%, 85% e até 93,30%)
% - no caso do transformer, seu desempenho posicionou-se superior aos demais trabalhos
% ------------
% - contudo, considerando que maioria deles modela apenas vocabulários pequenos [que correspondem em média a 2% do vocabulário disponível no ASLLVD (que é o que utilizamos aqui) e não ultrapassam 13% dele], nota-se que 
% - no entanto, se considerarmos apenas aqueles trabalhos que consideraram todos os sinais do ASLLVD (que seria um vocabulário de complexidade equivalente à nossa), percebemos que mesmo os encoder-decores apresentaram um salto importante, ficando em torno dos 46% de acurácia (quando até então observava-se por volta de 16%)
%   - se olharmos o transformer, esse salto é ainda mais expressivo




Para comparar os resultados obtidos neste trabalho, selecionamos algumas pesquisas que também adotaram o \acrshort{asllvd} para realizar o reconhecimento dos sinais.
A maioria deles, no entanto, limita-se aos segmentos das mãos para isso e também utiliza como dados de entrada coordenadas 3D ou imagens RGB dos frames das amostras, as quais são processadas por técnicas de visão computacional.
Além disso, eles comumente selecionam subconjuntos menores de sinais ao invés de considerar o \textit{dataset} completo, como fizemos aqui -- isso contribui para acurácias maiores mas limitá-os perante o contexto real de aplicação da língua.


\begin{itemize}    
    % tipo de dado bruto?
    % recorte de mãos x corpo inteiro?

    \item \citeonline{theodorakis-2014-dynamic-static} adotam técnicas não-supervisionadas combinadas com \acrfull{hmm} para gerar subunidades (denominadas 2-S-U) de movimento e pausa da articulação dos sinais a partir dos frames, que são aplicadas a um subconjunto de 97 sinais. 
    % processa frames de vídeo / foco nas mãos

    \item \citeonline{lim-2016-bhof} introduzem o \textit{Block-based Histogram of Optical Flow} (ou Histograma Baseado em Blocos de Fluxo Óptico) (BHOF), que concentra-se nos segmentos das mãos extraídos a partir dos frames das amostras, para reconhecer os sinais. Eles selecionaram um subconjunto de 20 sinais do \acrshort{asllvd} e apresentaram uma comparação dos resultados com outras técnicas como \acrfull{mei}, \acrfull{mhi}, \acrfull{pca} e \acrfull{hof}.
    % processa frames de vídeo / recorta as mãos (remove outras partes)

    \item \citeonline{metaxas-2018-linguistically} agregam \textit{features} geradas por diferentes técnicas de aprendizagem de máquina para parâmetros dos sinais e aplicam-nas como entrada de um modelo baseado em \textit{Hidden Conditional Ordinal Random Fields} (HCORF) para reconhecer um subconjunto de 350 sinais.
    % extrai features: a) handshapes (start and end); b) number of hands; c) 3D upper body locations, movements of the hands and arms, and distance between the hands; d) facial features (include 66 points from 3D estimates for the forehead, ear, eye, nose, and mouth regions, and their velocities across frames) and head movements; e) contact (extracted from our 3D face and upper body movement estimation, and relate to the possibilities of the hand touching specific parts of the head or body).
    % processa frames de video para extrair features de nivel maior / considera mix de features e coordenadas 3D

    \item \citeonline{lim-2019-isolated-slr-cnn-hei} introduzem a representação \textit{Hand Energy Image} (ou Imagem de Energia da Mão) (HEI) que é utilizada como entrada para uma rede \acrshort{cnn} e aplicada aos 20 sinais definidos por \citeonline{lim-2016-bhof}.
    % processa frames de vídeo / recorta as mãos

    \item Por fim, \citeonline{amorim-2019-stgcn-sl} utilizam grafos para modelar as coordenadas 3D do corpo e a dimensão temporal dos movimentos dos indivíduos, os quais foram processados por uma rede \textit{Spatial-Temporal Graph Convolutional Network} (ou Rede Convolucional de Grafo Espaço-Temporal) (ST-GCN) para os 20 sinais definidos por \citeonline{lim-2016-bhof}, mas também para o \textit{dataset} inteiro.
    % coordenadas 2D para grafos / corpo inteiro
\end{itemize}


\input{capitulos/avaliacao/tabelas/comparacao-resultados}


% Algumas comparações (HEI - Hand Energy Image x MEI, MHI) com ASLLVD
% 2020 - Amorim, Macedo, Zanchettin - Spatial-Temporal Graph Convolutional Networks for Sign Language Recognition
%                         accuracy
%     (tudo)              16.48
%     (20 sinais)         61.04

% 2016 - Lim, Tan, Tan - Block-based histogram of optical flow for isolated sign language recognition
%     (20 sinais)
%         MHI             10.00
%         MEI             25.00
%         PCA             45.00
%         HOF             70.00
%         > BHOF          85.00


% 2019 - Lim et al - Isolated sign language recognition using Convolutional Neural Network hand modelling and Hand Energy Image
%     (20 sinais - duas mãos)
%         > Proposed HEI  31.50
%         MEI             25.00
%         MHI             10.00 
%         MEI + MHI       20.00

% 2014 - Dilsizian et al - A New Framework for Sign Language Recognition based on 3D Handshape Identification and Linguistic Modeling
% https://aclanthology.org/L14-1096/
%     >                  81.76

% 2014 - Theodorakis, Pitsikalis, Maragos - Dynamic–static unsupervised sequentiality, statistical subunits and lexicon for sign language recognition
%     (97 signs)
%     > 2-S-U            63.15

% 2018 - Metaxas, Dilsizian, Neidle - Linguistically-driven Framework for Computationally Efficient and Scalable Sign Recognition
% https://par.nsf.gov/servlets/purl/10065369
%     > (350 signs)      93.3




% Considerações finais
\chapter{Considerações finais}
\label{cap:consideracoes-finais}

Foi introduzida ao longo deste trabalho uma abordagem que concentra-se nos atributos linguísticos da língua de sinais para realizar o seu reconhecimento de forma automática.
Apesar dessa ser uma estratégia comum entre tarefas de \acrfull{nlp}, ela difere-se da abordagem que tem sido utilizado pela maioria das pesquisas em \acrfull{slr} que, por sua vez, concentra-se principalmente no processamento de dados brutos como pixels RGB ou coordenadas.

Dessa maneira, o foco foi trocado do domínio da \acrfull{cv} para o domínio do \acrshort{nlp}, permitindo que os modelos aplicados aprendessem acerca da estrutura linguística em vez das relações entre pixels e coordenadas, por exemplo.
Pelos resultados apresentados no \autoref{cap:avaliacao}, percebe-se que essa abordagem apresentou uma eficácia bastante satisfatória sobretudo quando combinada com o modelo \textit{Transformer} que, por sua vez, é uma arquitetura amplamente utilizada e bem-sucedida ao lidar com tarefas envolvendo linguagens.


Entre as principais contribuições do trabalho, pode-se enumerar que ele:

\begin{itemize}
      \item Aborda a língua de sinais efetivamente como uma língua, ao invés de considerá-la como um conjunto de gestos ou posições estáticas de mãos, conforme também observa-se em muitas pesquisas em \acrshort{slr}.
            Isso posiciona a linguística como o pilar fundamental para essa área e abrange suas complexidades.
 
      \item Conscientiza o leitor acerca da linguística e das particularidades da natureza visual das línguas de sinais.
            Isso permite que pesquisadores apropriem-se desses temas e motivem-se a desenvolver novas pesquisas que contribuam com avanços seguindo essa mesma direção.

      \item Traz clareza acerca das principais lacunas que têm limitado progressos expressivos na área de \acrshort{slr} ao longo das últimas décadas, os quais poderiam viabilizar soluções efetivamente aplicáveis ao contexto real da língua de sinais e do Surdo.

      \item Contribui com um novo \textit{dataset} de atributos linguísticos para a língua de sinais, que viabiliza com que novas pesquisas em \acrshort{slr} sejam desenvolvidas seguindo a mesma abordagem que introduzimos aqui.
            Além disso, este trabalho também contribui com um \textit{dataset} de coordenadas 3D que permite pesquisadores derivar outros tipos de representações linguísticas.
            Ambos baseiam-se no \acrshort{asllvd}, que é um importante \textit{dataset} da \acrshort{asl}.

      \item Introduz uma estratégia para computar atributos linguísticos a partir de dados brutos, como \textit{frames} de vídeos ou coordenadas, a qual foi aplicada na geração do \textit{dataset} acima.
            Nela, as complexidades relacionadas à \acrlong{cv} são delegadas para ferramentas que implementam seus respectivos estados da arte.
            Como consequência, essa estratégia é compatível com diferentes \textit{datasets} que, por sua vez, poderiam ser processados e combinados para produzir volumes ainda maiores de dados, que são atualmente escassos para as línguas de sinais.
 
      \item Contribui com dois artigos para a área de \acrshort{slr}:
            em \citeonline{amorim-2019-stgcn-sl}, além do \acrshort{stgcn}, é introduzido o processamento do \acrshort{asllvd} utilizado como base para a obtenção dos \textit{datasets} apresentados neste trabalho;
            em \citeonline{amorim-2022-asl-datasets} (que foi submetido e encontra-se em revisão), são discutidos os detalhes da criação desses \textit{datasets}.

\end{itemize}

Por outro lado, enxergamos como passos futuros para a abordagem deste trabalho os seguintes itens:

\begin{itemize}
    \item Avaliar a eficácia dessa abordagem para o cenário de sinais contínuos que, por sua vez, é mais complexo por não haver uma segmentação clara entre os sinais -- como no nosso contexto.
          Apesar disso, sinais contínuos são muito mais próximos do contexto real de uso da língua de sinais e, por conta disso, possuem uma grande sinergia com a linha de pesquisa que iniciamos aqui.

    \item Explorar outros atributos linguísticos atualmente não considerados em nosso subconjunto incipiente, para ampliar as \textit{features} do \textit{dataset} introduzido aqui.
          Isso contribuirá para prover mais contexto sobre a estrutura linguística dos sinais aos modelos, os quais precisarão tornar-se gradualmente mais robustos à medida em que nos aprofundamos em cenários mais reais de uso da língua -- como no exemplo dos sinais contínuos.

    \item Produzir um corpus de sinais mais amplo e diversificado para a língua de sinais, utilizando a estratégia de anotação de atributos linguísticos introduzida aqui.
          Conforme discutimos anteriormente, os \textit{datasets} atualmente disponíveis para as línguas de sinais são pequenos e pouco diversificados em comparação àqueles utilizados pelo \acrfull{nlp} para alcançar o estado da arte em outros tipos de linguagens.
          Além disso, técnicas modernas e mais robustas de \acrlong{dl} são orientadas a dados e demandam vocabulários cada vez maiores.

    \item Por fim, avaliar o desempenho da abordagem deste trabalho no cenário online (ou em tempo real) de reconhecimento de sinais.
          Uma vez que é um objetivo de longo prazo contribuir para que a área de \acrshort{slr} consiga prover ferramentas para contextos reais da língua, é importante conhecer sua eficácia também para cenários online afim de assegurar que a direção percorrida pela linha de pesquisa permanece alinhada com esse propósito conforme evolui.
          %   tempo de inferência e potenciais gargalos
\end{itemize}

% próximos passos --------------
% - extender features fonologicas
% - utilizar sinais continuos
% - juntar datasets diferentes para produzir um corpora maior e mais diversificado (a nossa abordagem é genérica e pode ser usada em outros datasets ?)
% - ? pipeline de sistema real / online ?






% TODO: limpar comentários

\bookmarksetup{startatroot}% 


% ----------------------------------------------------------
% ELEMENTOS PÓS-TEXTUAIS
% ----------------------------------------------------------
\postextual


% ----------------------------------------------------------
% BIBLIOGRAFIA
% ----------------------------------------------------------
%\bibliographystyle{abntexalfenglish} %caso seja em inglês, retire o comentário desta linha

% \renewcommand{\bibname}{REFER\^ENCIAS}
%\renewcommand{\bibname}{Bibliography}
% \addbibresource{sample.bib}
\bibliography{referencias}


% ----------------------------------------------------------
% APENDICES
% ----------------------------------------------------------
% força para que não exiba subtítulos em apêndices no sumário
\begin{apendicesenv}
  \addtocontents{toc}{\protect\setcounter{tocdepth}{1}}
  \makeatletter
  \addtocontents{toc}{%
    \begingroup
    \let\protect\l@chapter\protect\l@section
    \let\protect\l@section\protect\l@subsection
  }
  \makeatother

  % Imprime uma página indicando o início dos apêndices
  % \partapendices
  % Apendices

%\input{appendix/apendiceA}
%\input{appendix/apendiceB}
%\input{appendix/apendiceC}



  \addtocontents{toc}{\endgroup}
\end{apendicesenv}



% ----------------------------------------------------------
% ANEXOS
% ----------------------------------------------------------
% força para que não exiba subtítulos em apêndices no sumário
\begin{anexosenv}
  \addtocontents{toc}{\protect\setcounter{tocdepth}{1}}
  \makeatletter
  \addtocontents{toc}{%
    \begingroup
    \let\protect\l@chapter\protect\l@section
    \let\protect\l@section\protect\l@subsection
  }
  \makeatother
  % Imprime uma página indicando o início dos apêndices
  % \partapendices
  % Anexos

%\input{anexos/anexoA}
%\input{anexos/anexoB}

  \addtocontents{toc}{\endgroup}
\end{anexosenv}


% ----------------------------------------------------------
\printindex

\end{document}
