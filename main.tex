%% abtex2-modelo-trabalho-academico.tex, v-1.7.1 laurocesar
%% Copyright 2012-2013 by abnTeX2 group at http://abntex2.googlecode.com/ 
% ------------------------------------------------------------------------

\documentclass[
	% -- opções da classe memoir --
	12pt,				% tamanho da fonte
	openright,			% capítulos começam em pág ímpar (insere página vazia caso preciso)
	oneside,			% para impressão em verso e anverso. Oposto a oneside
	a4paper,			% tamanho do papel. 
	% -- opções da classe abntex2 --
	chapter=TITLE,		% títulos de capítulos convertidos em letras maiúsculas
	section=TITLE,		% títulos de seções convertidos em letras maiúsculas
	%subsection=TITLE,	% títulos de subseções convertidos em letras maiúsculas
	%subsubsection=TITLE,% títulos de subsubseções convertidos em letras maiúsculas
	% -- opções do pacote babel --
	french,				% idioma adicional para hifenização
	spanish,			% idioma adicional para hifenização
	english,
	brazil,
	]{abntex2/abntex2}
	\renewcommand{\baselinestretch}{1.5} %para customizar o espaço entre as linhas do texto
% --
% SETTINGS

\usepackage{abntex2/abntex2-cin-ufpe}

% \usepackage[noframe]{showframe}
% \usepackage{showframe}

%\overfullrule=4mm %para identificar onde existem os alertas de linhas grandes mal formatada pelo LaTex, basta comentar para não aparecer a barra lateral preta na linha em questão.

\renewcommand*\arraystretch{1.2} %para customizar o espaço entre as linhas das tabelas


\usepackage{pdfpages} %para incluir pdf como páginas


% ---
% PACOTES
% ---
\usepackage{float}
\usepackage{cmap}				% Mapear caracteres especiais no PDF
\usepackage{lmodern}			% Usa a fonte Latin Modern			
\usepackage[T1]{fontenc}		% Selecao de codigos de fonte.
\usepackage[utf8]{inputenc}		% Codificacao do documento (conversão automática dos acentos)
\usepackage{lastpage}			% Usado pela Ficha catalográfica
\usepackage{indentfirst}		% Indenta o primeiro parágrafo de cada seção.
%\usepackage{color}				% Controle das cores
\usepackage{graphicx}			% Inclusão de gráficos
\usepackage{lipsum}				% para geração de dummy text
\usepackage[versalete,alf,abnt-and-type=e,abnt-etal-list=0,abnt-etal-cite=3]{abntex2/abntex2cite} 
\usepackage{multirow}
\usepackage[section]{placeins}



% -----------------------------------------------------------
% lista de abreviaturas e siglas
% início
% -----------------------------------------------------------
% \usepackage[noredefwarn,acronym]{glossaries} %GLOSSÁRIO
\usepackage[acronym,nonumberlist,nogroupskip,noredefwarn]{glossaries}
% \usepackage{glossary-superragged}

\newcolumntype{L}[1]{>{\raggedright\let\newline\\\arraybackslash\hspace{0pt}}m{#1}}
\newcolumntype{C}[1]{>{\centering\let\newline\\\arraybackslash\hspace{0pt}}m{#1}}
\newcolumntype{R}[1]{>{\raggedleft\let\newline\\\arraybackslash\hspace{0pt}}m{#1}}

\newglossarystyle{modsuper}{%
  \glossarystyle{super}%
  \renewcommand{\glsgroupskip}{}
  
  % put the glossary in a longtable environment:
 \renewenvironment{theglossary}%
  {
    \begin{longtable}
        {L{0.2\textwidth}L{0.8\textwidth}}}%
    {\end{longtable}
  }%
}

% -----------------------------------------------------------
% lista de abreviaturas e siglas
% fim
% -----------------------------------------------------------
\usepackage{lscape} 
\usepackage{rotating} %rotates the figures, page
\usepackage{tikz}
\usepackage[section]{placeins}
\usepackage{setspace} 



% ----------------------------------------------------------
% PERSONALIZAÇÃO DE CORES
% ----------------------------------------------------------
\definecolor{blue}{RGB}{41,5,195}
\definecolor{gray}{rgb}{.4,.4,.4}
\definecolor{gray}{rgb}{.4,.4,.4}
\definecolor{pblue}{rgb}{0.13,0.13,1}
\definecolor{pgreen}{rgb}{0,0.5,0}
\definecolor{pred}{rgb}{0.9,0,0}
\definecolor{pgrey}{rgb}{0.46,0.45,0.48}
\definecolor{lightgray}{rgb}{0.95, 0.95, 0.96}
\definecolor{whitesmoke}{rgb}{0.96, 0.96, 0.96}
\definecolor{javared}{rgb}{0.6,0,0} % for strings
\definecolor{javagreen}{rgb}{0.25,0.5,0.35} % comments
\definecolor{javapurple}{rgb}{0.5,0,0.35} % keywords
\definecolor{javadocblue}{rgb}{0.25,0.35,0.75} % javadoc
\definecolor{meucinza}{rgb}{0.5, 0.5, 0.5}
%\definecolor{lightgray}{gray}{0.9}


% ----------------------------------------------------------
% PERSONALIZAÇÃO DO USUÁRIO
% ----------------------------------------------------------

% ----------------------------------------------------------
% DADOS DO TRABALHO - CAPA e FOLHA DE ROSTO
% Configure os dados do trabalho aqui
% ----------------------------------------------------------
\titulo{\textbf{Título:} subtítulo}
\autor{CLEISON CORREIA DE AMORIM}
\local{Recife}
\data{\Year}
\areaconcentracao{\textbf{Área de Concentração}: Aprendizagem de Máquina e Mineração}
\orientador{\textbf{Orientador}: Cleber Zanchettin}
%\coorientador{\textbf{Coorientador (a)}: Texto Texto Texto}

\instituicao{UNIVERSIDADE FEDERAL DE PERNAMBUCO \\ CENTRO DE INFORMÁTICA \\PROGRAMA DE PÓS-GRADUAÇÃO EM CIÊNCIA DA COMPUTAÇÃO}
\departamento{Centro de Informática}
\programa{Pós-graduação em Ciência da Computação}
\emailprograma{contato@cin.ufpe.br}
\siteprograma{http://www.cin.ufpe.br}

\tipotrabalho{Dissertação de Mestrado}
% O preambulo deve conter o tipo do trabalho, o objetivo, 
% o nome da instituição e a área de concentração 
%\preambulo{Trabalho apresentado ao Programa de Pós-graduação em Ciência da Computação do Centro de Informática da Universidade Federal de Pernambuco, como requisito parcial para obtenção do grau de Mestre Profissional em Ciência da Computação.}

%\preambuloatadefesa{Dissertação apresentada ao Programa de Pós-Graduação Profissional em Ciência da Computação da Universidade Federal de Pernambuco, como requisito parcial para a obtenção do título de Mestre Profissional em 04 de setembro de 2020.}

\preambulo{Trabalho apresentado ao Programa de Pós-graduação em Ciência da Computação do Centro de Informática da Universidade Federal de Pernambuco, como requisito para obtenção do grau de Mestre em Ciência da Computação.}

\preambuloatadefesa{Dissertação apresentada ao Programa de Pós-Graduação em Ciência da Computação da Universidade Federal de Pernambuco, como requisito para a obtenção do título de Mestre em Ciência da Computação, na data de XX de XXXX de XXXX.}


%\input{userlists}



% ----------------------------------------------------------
% COMPILA O ÍNDICE
% ----------------------------------------------------------
\makeindex
% ---


% ----------------------------------------------------------
% LISTA E ABREVIATURAS E SIGLAS
% ----------------------------------------------------------
% Lista de acrônimos:
\newacronym{oms}{OMS}{Organização Mundial da Saúde}
\newacronym{cfm}{CFM}{Conselho Federal de Medicina}
\newacronym{ibge}{IBGE}{Instituto Brasileiro de Geografia e Estatística}
\newacronym{ines}{INES}{Instituto Nacional de Educação de Surdos}
\newacronym{slr}{RLS}{Reconhecimento de Língua de Sinais}
\newacronym{ia}{IA}{Inteligência Artificial}
\newacronym{nlp}{PLN}{Processamento de Linguagem Natural}
\newacronym{mt}{TA}{Tradução Automática}

% Línguas de sinais
\newacronym{asl}{ASL}{\textit{American Sign Language} (Língua de Sinais Americana)}
\newacronym{csl}{CSL}{\textit{Chinese Sign Language} (Língua de Sinais Chinesa)}
\newacronym{dgs}{DGS}{\textit{Deutsche Gebärdensprache} (Língua de Sinais Alemã)}
\newacronym{bsl}{BSL}{\textit{British Sign Language} (Língua de Sinais Britânica)}
\newacronym{arsl}{ArSL}{\textit{Arabic Sign Language} (Língua de Sinais Britânica)}
\newacronym{jsl}{JSL}{\textit{Japanese Sign Language} (Língua de Sinais Japonesa)}
\newacronym{isl}{ISL}{\textit{Indian Sign Language} (Língua de Sinais Indiana)}
\newacronym{gsl}{GSL}{\textit{Greek Sign Language} (Língua de Sinais Grega)}
\newacronym{tid}{TID}{\textit{Türk İşaret Dili} (Língua de Sinais Turca)}
\newacronym{ngt}{NGT}{\textit{Nederlandse Gebarentaal} (Língua de Sinais Holandesa)}
\newacronym{vgt}{VGT}{\textit{Vlaamse Gebarentaal} (Língua de Sinais Flamenga)}
\newacronym{lis}{LIS}{\textit{Lingua dei Segni Italiana} (Língua de Sinais Italiana)}
\newacronym{auslan}{AUSLAN}{\textit{Australian Sign Language} (Língua de Sinais Austrália)}
\newacronym{lsa}{LSA}{\textit{Lengua de Señas Argentina} (Língua de Sinais Argentina)}
\newacronym{tsl}{TSL}{\textit{Taiwan Sign Language} (Língua de Sinais de Taiwan)}
\newacronym{pjm}{PJM}{\textit{Polski Język Migowy} (Língua de Sinais Polonesa)}
\newacronym{libras}{Libras}{Língua Brasileira de Sinais}
\newacronym{tch}{TCH}{\textit{Teanga Chomharthaíochta na hÉireann} (Língua de Sinais Irlandesa)}
\newacronym{ksl}{KSL}{\textit{Korean Sign Language} (Língua de Sinais Coreana)}
\newacronym{bisindo}{BISINDO}{\textit{Bahasa Isyarat Indonesia} (Língua de Sinais Indonésia)}

\makenoidxglossaries
\renewcommand*{\glsseeformat}[3][\seename]{\textit{#1}  
\glsseelist{#2}}

\renewcommand*{\glspostdescription}{} % remove trailing dot
\renewcommand{\glsnamefont}[1]{\textbf{#1}}

\renewcommand{\familydefault}{\sfdefault}


% ----------------------------------------------------------
% GLOSSÁRIO
% ----------------------------------------------------------
% Glossario:

%\newglossaryentry{libras}
%{
%  name=\textit{Libras},
%  description={Língua Brasileira de Sinais},
%  plural=\textit{Libras}
%}


%\input{packages}


% ----------------------------------------------------------
% INÍCIO DO DOCUMENTO
% ----------------------------------------------------------
\begin{document}

\frenchspacing % Retira espaço extra obsoleto entre as frases.

\imprimircapa
\imprimirfolhaderosto*~
%\input{outros/ficha}
%% Ata de defesa
\includepdf[pages=-]{outros/biblioteca/ata_defesa.pdf}

%\input{outros/folha_aprovacao_original}
% ----------------------------------------------------------
% DEDICATÓRIA
% ----------------------------------------------------------
\begin{dedicatoria}
   \vspace*{\fill}
   % \centering
   % \noindent
   \textit{
      Dedico este trabalho ao meu pai, que partiu ao longo da jornada deste mestrado, mas que deixou um grande exemplo de integridade e simplicidade pelo qual me espelharei em minha caminhada.
   }
\end{dedicatoria}
% ---

% ----------------------------------------------------------
% AGRADECIMENTOS
% ----------------------------------------------------------
\begin{agradecimentos}
    Primeiramente, agradeço a Deus por sua infinita bondade e por me conceber essa oportunidade de completar mais um importante ciclo em minha vida.
    É impossível não recordar como os últimos anos foram difíceis para todos nós, bem como nos trouxeram desafios e perdas que tivemos que superar. 
    No entanto, hoje respirando mais aliviado e com esperanças renovadas, celebro mais essa conquista acadêmica.
    
    Em segundo lugar, agradeço à minha família por todo apoio e compreensão nessa jornada. De um modo especial, sou muito grato aos meus pais por acreditarem, persistirem e ensinarem a cada um de seus filhos que a educação é o legado mais valioso que eles poderiam nos transmitir e que jamais alguém poderá nos tirar.
    Seus ensinamentos são muito sábios e fontes de muitas realizações.

    Por fim, agradeço ao meu orientador por acreditar em meu potencial desde o primeiro momento e por me apoiar compartilhando muito de seu tempo e de seu conhecimento para que eu pudesse concretizar esta pesquisa.
\end{agradecimentos}
\input{outros/epigrafe}
% -----------------------------------------------------------
% PORTUGUÊS
% -----------------------------------------------------------
\begin{resumo}[Resumo]
  \noindent
  A língua de sinais é uma ferramenta essencial na vida do Surdo, capaz de assegurar seu acesso à comunicação, educação e desenvolvimento cognitivo e socio-emocional.
  Na verdade, ela é a principal força que une essa comunidade e o símbolo de identificação entre seus membros.
  Contudo, atualmente o número de indivíduos ouvintes que conseguem se comunicar por meio dessa língua é pequeno e, na prática, isso impõe obstáculos ao cotidiano do Surdo.
  Tarefas simples como utilizar o transporte público, comprar roupas, ir ao cinema ou obter assistência médica acabam se tornando um desafio por conta dessa limitação.
  O Reconhecimento de Língua de Sinais é uma das áreas de pesquisa que objetiva desenvolver tecnologias capazes de reduzir essas barreiras linguísticas e facilitar a comunicação entre ambos indivíduos.
  Apesar disso, ao analisarmos sua evolução ao longo das últimas décadas, percebe-se que seu progresso ainda não é suficiente para disponibilizar soluções efetivamente aplicáveis ao mundo real.
  Isso ocorre principalmente porque várias pesquisas nessa área acabam não apropriando-se ou abordando adequadamente as particularidades linguísticas das línguas de sinais, decorrentes de sua natureza visual.
  % Tendo isso em vista, este trabalho introduz uma abordagem linguística ao reconhecimento computacional da língua de sinais, a qual objetiva estabelecer uma direção capaz de conduzir a avanços mais efetivos para essa área e, consequentemente, contribuir com a superação dos obstáculos hoje enfrentados pelo Surdo.
  Tendo isso em vista, este trabalho apresenta uma abordagem que aplica modelos sequenciais de aprendizagem de máquina para realizar o reconhecimento computacional dos sinais através de seus atributos linguísticos. Além disso, são introduzidos dois novos \textit{datasets} para a língua de sinais, dentre os quais está um \textit{dataset} de atributos linguísticos computados a partir do ASLLVD.
  Com isso, objetiva-se estabelecer uma direção capaz de conduzir a avanços mais efetivos para essa área e, consequentemente, contribuir com a superação dos obstáculos hoje enfrentados pelo Surdo.

  % %FIXME: [cz] acho que faltou dizer o que na pratica vc fez. Falta um paragrafo apresentando seu trabalho, o que o leitor vai encontrar se decidir ler --. [cca5] feito


  % Em meio a tantos desafios, a língua de sinais surge como uma ferramenta poderosa que é capaz de assegurar o desenvolvimento cognitivo, facilitar a comunicação, e possibilitar que esses indivíduos obtenham educação e socio-emocional adequado (World Health Organization, 2021).

  % A língua de sinais é a língua utilizada pela maioria dos Surdos em sua vida diária. Muito mais do que isso, ela é a principal força que une essa comunidade e o símbolo de identificação entre seus membros

  % Apesar disso, Bragg et al. (2019), Agência Senado (2019) observam que atualmente ainda são poucos os ouvintes que conseguem se comunicar por meio dessa língua. Isso traz obstáculos adicionais aos Surdos e transforma muitas de suas atividades corriqueiras num grande desafio. 
  % Por exemplo, no transporte público é difícil solicitar ajuda ou ter acesso às instruções divulgadas nos alto-falantes; em lojas, é raro encontrar vendedores preparados para interagir através dessa língua ou que não os trate com preconceito; no cinema, eles apenas podem consumir filmes estrangeiros, uma vez que os nacionais não dispõem de legenda; no serviço de saúde, não são raros os relatos de pacientes que saem de consultas com prescrições médicas erradas porque o médico não entendeu corretamente seus sintomas; entre outras situações

  % Isso deve-se, de um modo geral, a um conjunto de particularidades que as línguas de sinais apresentam quando comparadas às línguas faladas, bem como à forma com que as pesquisas em RLS têm abordado elas, afirmam Bragg et al. (2019), Cooper, Holt e Bowden (2011). Diferentemente das faladas, as línguas sinalizadas possuem uma natureza visual e transmitem significado através de múltiplos canais ao mesmo tempo, como mãos, corpo, face, entre outros de granularidade ainda menor. Essa natureza faz com que sua linguística seja estruturada de uma forma muito específica, demandando que novas técnicas sejam desenvolvidas para abordar tais particularidades. Contudo, um grande número de pesquisas nessa área não aborda essa linguística ou suas complexidades e, como consequência, acabam não produzindo avanços realmente efetivos.

  % Tendo em vista isso, este trabalho busca contribuir com a área de Reconhecimento de Língua de Sinais (RLS) por meio de uma proposta que aborda a língua de sinais através de sua linguística e aplica técnicas de Processamento de Linguagem Natural (PLN) para estabelecer uma direção que possa conduzir a avanços efetivos na área e, consequentemente, ajude a superar alguns dos desafios cotidianos atualmente encarados pelos Surdos.

  \vspace{\onelineskip}

  \noindent
  \textbf{Palavras-chaves}: Língua de Sinais. Linguística. Processamento de Linguagem Natural.
\end{resumo}



% -----------------------------------------------------------
% INGLÊS
% -----------------------------------------------------------
\begin{resumo}[Abstract]
  \begin{otherlanguage*}{english}
    \noindent
    Sign language is an essential resource to ensure that the Deaf have access to communication, education, as well as to cognitive and socio-emotional development. In fact, it is the main force that unites this community and an identifying trait among its members.
    On the other hand, the number of hearing individuals who are able to communicate through this language is currently small and, in practice, this imposes obstacles to the daily life of the Deaf.
    Simple tasks like using public transportation, shopping for clothes, going to the movies, or getting medical assistance end up becoming challenges due to such limitation.
    The Sign Language Recognition, in turn, is one of the research areas dedicated to developing technologies that aim to reduce such language barriers and facilitate communication between these individuals.
    However, when analyzing its evolution over the last decades, we realize that it has not progressed enough to provide solutions effectively applicable to the real world.
    This is mainly because several researches in this field do not appropriate or address the linguistic particularities presented by the sign languages, which stem from their visual nature.
    % With this in mind, this work introduces a linguistic approach to sign language recognition that aims to establish a direction that can lead to more effective advances in this research area and, consequently, contribute to overcoming the obstacles faced by the Deaf today.
    Considering this problem, the present work introduces an approach that adopts sequential machine learning models to recognize signs through their linguistic attributes. In addition, we introduce two new sign language datasets, among which is a novel dataset of linguistic attributes computed from the ASLLVD.
    Thus, we aim to establish a direction that can lead to more effective advances in this research area and, consequently, contribute to overcoming the obstacles faced by the Deaf today.

    \vspace{\onelineskip}

    \noindent
    \textbf{Keywords}: Sign Language. Linguistics. Natural Language Processing.
  \end{otherlanguage*}
\end{resumo}



% ----------------------------------------------------------
% LISTA DE FIGURAS
% ----------------------------------------------------------
\pdfbookmark[0]{\listfigurename}{lof}
\listoffigures*
\cleardoublepage


% ----------------------------------------------------------
% LISTA DE CÓDIGOS FONTES
% ----------------------------------------------------------
%\pdfbookmark[0]{\lstlistingname}{lol} % caso não tenha quadros, comente esta linha 
\counterwithout{lstlisting}{chapter}



% Altera o nome padrão do rótulo usado no comando \autoref{}
\renewcommand{\lstlistingname}{Código Fonte}

% Altera o rótulo a ser usando no elemento pré-textual "Lista de código"
\renewcommand{\lstlistlistingname}{Lista de códigos}

% Configura a ``Lista de Códigos'' conforme as regras da ABNT (para abnTeX2)
\begingroup\makeatletter
\let\newcounter\@gobble\let\setcounter\@gobbletwo
  \globaldefs\@ne \let\c@loldepth\@ne
  \newlistof{listings}{lol}{\lstlistlistingname}
  \newlistentry{lstlisting}{lol}{0}
\endgroup

\renewcommand{\cftlstlistingaftersnum}{\hfill--\hfil}

\let\oldlstlistoflistings\lstlistoflistings
{
\let\oldnumberline\numberline
\newcommand{\algnumberline}[1]{Código Fonte~#1~\enspace--~\enspace}
\renewcommand{\numberline}{\algnumberline}

\begin{KeepFromToc}
\lstlistoflistings
\end{KeepFromToc}
}
\cleardoublepage



% ----------------------------------------------------------
% LISTA DE QUADROS
% ----------------------------------------------------------
\pdfbookmark[0]{\listofquadrosname}{loq} % caso não tenha quadros, comente esta linha 
\listofquadros* % caso não tenha quadros, comente esta linha 
\cleardoublepage



% ----------------------------------------------------------
% LISTA DE TABELAS
% ----------------------------------------------------------

\pdfbookmark[0]{\listtablename}{lot}
\listoftables*
\cleardoublepage


        
  
% ----------------------------------------------------------
% LISTA E ABREVIATURAS E SIGLAS
% ----------------------------------------------------------
% \printglossary[type=\acronymtype,title={\listadesiglasname},nonumberlist]
% \printglossaries
% compile uma vez com o comando \printglossaries e depois compile novamente com o comando \printglossaries comentado para as páginas glossário e siglas serem ocultadas.

% ----------------------------------------------------------
% LISTA E ABREVIATURAS E SIGLAS
% ----------------------------------------------------------
% \setglossarystyle{modsuper}
\printnoidxglossary[style=modsuper,type=\acronymtype,title={\listadesiglasname},nonumberlist]
% \printglossary[style=super, type=\acronymtype]
\cleardoublepage



% ----------------------------------------------------------
% LISTA DE SIMBOLOS
% ----------------------------------------------------------
%\input{others/list_simbol}

% ----------------------------------------------------------
%\input{others/list_simbol}



% ----------------------------------------------------------
% SUMÁRIO
% ----------------------------------------------------------
\pdfbookmark[0]{\contentsname}{toc}
\tableofcontents*
% \begingroup\intoctrue
% \tableofcontents*
% \endgroup
\cleardoublepage

% \setcounter{page}{13}
\setcounter{tocdepth}{2}
\setcounter{table}{0}




% ----------------------------------------------------------
% ELEMENTOS TEXTUAIS
% ----------------------------------------------------------
\textual


% referencie todos os arquivos de capítulos aqui, fique a vontade para
% fazer a sua organização de diretórios

  \chapter{Introdução}

\section{Motivação}
\label{motivacao}

Lorem ipsum dolor sit...

\begin{figure}
    \centering
    \includegraphics{imagens/institution.png}
    \caption{Lorem ipsum}
    \label{fig:my_label}
\end{figure}

  %\chapter{Fundamentação Teórica}

\section{XXX}
\label{xxx}

Reconhecimento de linguas faladas e texto X linguas de sinais
NLP: discutir evolução no tempo das abordagens utilizadas (para texto e voz) 
Utilização da fonologia + semântica das palavras para reconhecimento?
Linguas faladas e texto -> hoje são acessíveis, oferecidos em escala comercial e em produtos estáveis
Linguas de sinais -> não há ferramentas em escala comercial e há necessidade de contratar profissionais intérpretes para a tarefa
Torna-se cara e pouco acessível
Modelos sequenciais (aprendizagem de maquina)
Transformer: breve introdução (arquitetura e funcionamento)
RNNs (GRU, LSTM, etc)

  
  %% TODO: revisar se essa introdução continua abrangendo o conteúdo do capítulo
Neste capítulo, discutiremos a abordagem proposta por este trabalho, bem como as justificativas para sua adoção, suas contribuições para a área de \acrshort{slr} e os detalhes das técnicas e experimentos envolvidos.

% proposta
% justificativas
% beneficio
%   - desafios (da SLR) que resolve

% técnicas
% experimentos


esta pesquisa possui como proposta adotar uma abordagem centrada na linguística da língua de sinais para realizar o reconhecimento dos sinais
- ao analisar a linguistica, vemos que ela se divide o estudo da língua em três partes distintas, que descrevem níveis crescentes de significado: vão desde seus menores elementos constituintes (na fonologia), passam pela articulação desses elementos para um nível maior de significado -- que seriam as palavras (na morfologia) e alcançam um nível mais complexo, que articula essas palavras para compor sentenças e estruturas mais complexas (na sintaxe).
- como percebe-se, a fonologia é o nível mais fundamental para se abordar uma língua sob uma abordagem linguistica. na língua de sinais, ela descreve os parâmetros mínimos que precisam ser compreendidos e dominados antes de avançar para os próximos níveis de significado. 

- por este motivo, este trabalho adota a hipótese de que este é também o nível mais básico e fundamental para representar e abordar a língua de sinais em técnicas de \acrshort{slr}, o qual deve ser dominado antes de partirmos para explorar níveis de significado mais elevados e complexos. 
- dessa forma, ao invés de utilizar imagens RGB ou coordenadas brutas do corpo dos indivíduos como entrada para as técnicas aplicadas -- como observa-se em muitos estudos na área --, nesta pesquisa representando os sinais segundo o nível fonológico da língua.

imagens RGB e coordenadas do corpo humano não carregam semântica por si só. ao contrário, é necessário um nível de interpretação sobre elas para que seja possível inferir, por exemplo, que um conjunto de coordenadas de dedos e corpo se refere a "uma mão com orientação voltada para o corpo numa trajetória para a esquerda". 
além disso, a quantidade de pixels contidos nessas imagens ou de coordenadas corporais brutas são comumente grandes, mas apenas uma parcela pequena é realmente relevante para os sinais que se deseja interpretar.
de fato, essa necessidade de lidar com dados brutos tem exigido de pesquisadores um esforço adicional para que suas técnicas interpretem-os corretamente e consigam contornar essa complexidade antes de progredir para níveis semânticos maiores da língua de sinais. Por exemplo, é comum observar técnicas como fluxo óptico, 
\acrshort{mei}\footnote{
    \acrlong{mei}: é uma imagem binária, onde a região branca representa onde há movimento ocorrendo e o preto denota a região onde não há movimento \cite{ahad-2012-mhi-for-action}.
}
e
\acrshort{mhi}\footnote{
    \acrlong{mhi}: expressa a sequência de movimento de forma compacta, em uma única imagem, onde pixels em escala de cinza com intensidade menor descrevem frames de movimentos mais antigos nessa sequência \cite{ahad-2012-mhi}.
} 
sendo aplicadas para capturar os movimentos a partir dos pixels dos frames, ou ainda camadas extras sendo adicionadas aos modelos de aprendizagem profunda para lidar com isso, mas que os tornam mais caros e complexos. 


% - reduz a complexidade de aprender o mapeamento coordenadas -> linguistica / foca no aprendizado da linguistica
ao prover parâmetros fonológicos ao invés de dados brutos aos modelos de aprendizagem, retiramos o foco do aprender a interpretar pixels ou coordenadas sem significado e fazemos com que ele passe a aprender as relações e regras linguísticas contidas na língua. conforme discutimos na seção XXX, a linguística da língua já apresenta um conjunto de complexidades e regras que precisam ser abordadas para que a \acrshort{slr} avance de forma consistente. englobar a interpretação desses dados brutos como parte do problema faz com que a complexidade seja redobrada e limita a velocidade desses avanços.

apesar dessa complexidade, muitos desses parâmetros fonológicos poderiam ser computados algebricamente a partir de coordenadas 3d para formatos semanticamente mais elevados de forma mais barata antes de serem ingeridos por algoritmos.








sendo assim, a abordagem deste trabalho adotará parâmetros fonológicos como dados de entrada para o processo de reconhecimento dos sinais. como pode-se imaginar, atualmente não existem \textit{datasets} disponíveis com esse tipo de parâmetros e, portanto, o primeiro desafio nessa direção foi desenvolver um \textit{dataset} que pudesse suportar essa pesquisa. 

dentre as alternativas mais viáveis para isso, optamos por derivar um novo \textit{dataset} a partir do \acrfull{asllvd}, o qual é um dos mais relevantes da \acrshort{asl} e foi desenvolvido na Universidade de Boston por \citeonline{athitsos-2008-asllvd} e \citeonline{neidle-2012-asllvd}. % Ele é composto por um vocabulário de 2.745 sinais representados em cerca de 9.763 sequências de vídeos anotadas que, por sua vez, são articuladas por Surdos nativos nessa língua.

aplicamos um conjunto de processos para computar as representações fonológicas a partir dos frames RGB em 2D das amostras do \acrshort{asllvd}, o que incluiu a criação de uma representação intermediária em 3D sob a qual foi possível aplicar um conjunto de operações algébricas para alcançar nosso objetivo. Discutiremos esse processo em detalhes nas seções a seguir.

% para isso, primeiro foi necessário desenvolver um dataset -- desafios:
% - estabelecer técnicas para computar e representar no nível de parâmetros fonológicos as  coordenadas do corpo humano
%     - mas antes disso: sair de imagens 2D para coordenadas em 3D

% para isso, desafios:
%  - nao ha datasets da fonologia -> construir dataset -> necessário coordenadas 3d
%  - dataset produz imagens 2d -> estimar coordenadas 2d -> reconstruir coordenadas 3d
%  - computar parametros fonologicos a partir das coordenadas 3d
%     - estabelecer funções algebricas


uma vez que os parâmetros de entrada foram estabelecidos, podemos então prosseguir realizando o reconhecimento dos sinais. para isso, aplicaremos algumas arquiteturas clássicas de modelos sequenciais de aprendizagem profunda, além de um \textit{transformer} -- que é comumente adotado em tarefas de \acrshort{nlp} --. para que assim possamos quantificar a eficácia e estabelecer uma linha de base para a abordagem proposta. este processo será discutido em mais detalhes nas seções seguintes.

% TODO: criar imagem detalhando o processo/abordagem proposto
observamos na imagem XXX um diagrama que ilustra a abordagem proposta.


entre os desafios que esta proposta contribui a superar, temos os seguintes:

\begin{enumerate}
    \item aborda o \acrshort{slr} sob uma perspectiva orientada às particularidades da língua de sinais, considerando sua linguística e natureza visual intrínseca como fatores primordiais.
    
    \item por ser centrado na linguística da língua de sinais, este trabalho ajuda a criar consciência e reconhecimento dela como sendo um pilar fundamental para novas técnicas na área de \acrshort{slr}.

    \item é uma abordagem independente de variações nos traços e dimensões corporais, iluminação do ambiente, qualidade das câmeras, etc., bem como separada de complexidades atreladas à detecção partes do corpo ou interpretação de seus movimentos no espaço. isso porque, uma vez que o foco é colocado na fonologia da língua, tais complexidades são terceirizados para algoritmos especializados nesse tipo tarefa. sendo assim, o papel da \acrshort{slr} passa a englobar tão somente a língua de sinais e suas particularidades.
    
    \item as anotações contendo os parâmetros fonológicos são geradas automaticamente por meio de expressões algébricas que analisam o corpo no espaço tridimensional. isso pode ser replicado para outros \textit{datasets} (ou em aplicações do mundo real), o que contribuiria para reduzir a necessidade de produção de anotações manuais e, como consequência, as limitações ao combinar \textit{datasets} distintos com o intuito de elevar a quantidade e diversidade de amostras utilizadas pelas pesquisas em \acrshort{slr}.
    
    \item este trabalho contribui com dos novos \textit{datasets} para a língua de sinais, os quais podem ser utilizados para derivar outros novos \textit{datasets} ou para continuar evoluindo técnicas centradas na linguística. além disso, por se basearem no \acrshort{asllvd}, estes \textit{datasets} são criados a partir de sinais articulados por indivíduos Surdos nativos na língua.

    \item apesar do conjunto específico de parâmetros fonológicos contemplados e de lidar apenas com sinais isolados, entendemos que a abordagem proposta é passível de ser estendida para outros parâmetros e também para sinais contínuos.

\end{enumerate}










% \section{Metodologia}
% \label{sec:metodologia}

% A metodologia aplicada nesta dissertação concentra-se em compreender os desafios atuais da área de pesquisa para assim introduzir uma proposta que suporte avanços futuros coerentes com as necessidades do mundo real.
% As etapas percorridas aqui podem ser sumarizadas como:

% \begin{itemize}
%     \item Revisão do panorama atual da deficiência auditiva e do papel que as línguas de sinais desempenham aqui;
%     \item Revisão do panorama das pesquisas atuais em processamento de língua de sinais e das lacunas que têm limitado progressos mais expressivos na área.
%     \item Desenvolvimento de uma proposta que aborde as lacunas acima, contribuindo para preencher algumas delas e produzindo artefatos que suportem novas pesquisas a evoluir nessa direção;
%     \item Realização de experimentos e análise dos resultados coletados.
% \end{itemize}

% % - análise do panorama atual das línguas de sinais 
% % - análise do panorama atual das pesquisas na área e identificação de principais lacunas para seu avanço
% % - revisão da literatura das línguas de sinais e de sua linguísticas
% % ------
% % - seleção de um conjunto de atributos linguísticos
% % - análise de técnicas algébricas para suportar o processamento de atributos linguísticos selecionados
% % - seleção de modelos sequenciais de aprendizagem de máquina para os experimentos
% % - execução dos experimentos e análise dos resultados coletados






%   Introduzir abordagem (reconhecimento de língua de sinais baseada na linguística)
%       
%   Justificar caminho pelo uso da linguística
%       Abordagens anteriores costumam se limitar à classificação de imagens estáticas ou vídeos
%       Coordenadas 3D não carregam semântica por si só
%           Número de coordenadas é grande e nem todas as mudanças nelas são relevantes ??
%       Linguística considera a dinâmica / semântica da língua (agrupa as entradas em uma granularidade maior, com significado que é compreensível por humanos)
%       Elevar o nível semântico / prover um nível semântico para os algoritmos
%   Benefícios da abordagem (desafios abordados)
%       
%
%   Visão geral do método proposto 
%       Explicar método (reconhecimento de sinais baseado na fonologia)
%           representar informação no nível da fonologia (menor nível da linguística)
%           
%       ASL-Phono
%           Desafios / escassez de datasets diferentes
%           Falar do dataset
%       Modelos sequenciais
%           Transformer: breve introdução (arquitetura e funcionamento)
%           RNNs (GRU, LSTM, etc)
%
%   Experimento
%       Preparação dos dados
%           Transformação das sequências no dataset: frames -> palavras
%           Justificativa??
%       Setup dos modelos (Transformer, LSTM, GRU, etc)
%           Parâmetros
%               Buscar parâmetros (dimensionar os modelos/parâmetros)




  %\input{chapters/outro_capitulo/capitulo_exemplo}

\bookmarksetup{startatroot}% 


% ----------------------------------------------------------
% ELEMENTOS PÓS-TEXTUAIS
% ----------------------------------------------------------
\postextual


% ----------------------------------------------------------
% Referências bibliográficas
% ----------------------------------------------------------
%\bibliographystyle{abntexalfenglish} %caso seja em inglês, retire o comentário desta linha

% \renewcommand{\bibname}{REFER\^ENCIAS}
%\renewcommand{\bibname}{Bibliography}
% \addbibresource{sample.bib}
\bibliography{referencias}


% ----------------------------------------------------------
% Apêndices
% ----------------------------------------------------------
%\include{appendix}


% ----------------------------------------------------------
% Anexos
% ----------------------------------------------------------
%% Anexos

%\input{anexos/anexoA}
%\input{anexos/anexoB}

\printindex


\end{document}
