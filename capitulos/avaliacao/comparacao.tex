
Com o intuito de estabelecer um referencial dos resultados apresentados acima com outras pesquisas dentro da área de \acrfull{slr}, selecionamos alguns estudos que também utilizaram o \acrshort{asllvd}. Esses estudos são enumerados a seguir:

% Para comparar os resultados obtidos neste trabalho, selecionamos algumas pesquisas que também adotaram o \acrshort{asllvd} para realizar o reconhecimento dos sinais.
% A maioria deles, no entanto, limita-se aos segmentos das mãos para isso e também utiliza como dados de entrada coordenadas 3D ou imagens RGB dos frames das amostras, as quais são processadas por técnicas de visão computacional.
% Além disso, eles comumente selecionam subconjuntos menores de sinais ao invés de considerar o \textit{dataset} completo, como fizemos aqui -- isso contribui para acurácias maiores mas limitá-os perante o contexto real de aplicação da língua.


\begin{itemize}    
    % tipo de dado bruto?
    % recorte de mãos x corpo inteiro?

    \item \citeonline{theodorakis-2014-dynamic-static}: aplica técnicas não-supervisionadas e \acrfull{hmm} para gerar subunidades (denominadas 2-S-U) de movimento e pausa a partir dos frames das amostras, as quais são utilizadas para reconhecer os sinais.
    Nesse processo, os autores concentram-se nas mãos dos indivíduos e selecionam um subconjunto de 97 sinais do \acrshort{asllvd}.
    % processa frames de vídeo / foco nas mãos

    \item \citeonline{lim-2016-bhof}: introduz a técnica de \textit{Block-based Histogram of Optical Flow} (ou Histograma Baseado em Blocos de Fluxo Óptico) (BHOF), que gera histogramas do fluxo óptico das mãos dos indivíduos a partir dos frames das amostras.
    Os autores selecionaram um subconjunto pequeno contendo apenas 20 sinais do \acrshort{asllvd}. 
    Além disso, eles compararam seus resultados com aqueles obtidos pelas técnicas \acrfull{mei}, \acrfull{mhi}, \acrfull{pca} e \acrfull{hof} nesse mesmo subconjunto.
    % processa frames de vídeo / recorta as mãos (remove outras partes)

    \item \citeonline{metaxas-2018-linguistically}: utiliza diferentes técnicas para gerar features de tipos distintos referentes à configuração de mão inicial e final; número de mãos; distância entre mãos; coordenadas 3D do corpo, face, mãos e braços; e contato das mãos com o corpo. Elas são utilizadas como entrada para um modelo baseado em \textit{Hidden Conditional Ordinal Random Fields} (HCORF) para reconhecer um subconjunto selecionado de 350 sinais do \acrshort{asllvd}.
    % extrai features: a) handshapes (start and end); b) number of hands; c) 3D upper body locations, movements of the hands and arms, and distance between the hands; d) facial features (include 66 points from 3D estimates for the forehead, ear, eye, nose, and mouth regions, and their velocities across frames) and head movements; e) contact (extracted from our 3D face and upper body movement estimation, and relate to the possibilities of the hand touching specific parts of the head or body).
    % processa frames de video para extrair features de nivel maior / considera mix de features e coordenadas 3D

    \item \citeonline{lim-2019-isolated-slr-cnn-hei}: introduz a representação \textit{Hand Energy Image} (ou Imagem de Energia da Mão) (HEI) que é utilizada como entrada para uma rede \acrfull{cnn}.
    Aqui também os autores concentram-se nas mãos dos indivíduos e utilizam o mesmo subconjunto de 20 sinais definidos por \citeonline{lim-2016-bhof}.
    % processa frames de vídeo / recorta as mãos

    \item \citeonline{amorim-2019-stgcn-sl}: utiliza grafos para modelar a dimensão espacial das coordenadas 2D do corpo e sua relação temporal com os movimentos dos indivíduos, os quais são fornecidos como entrada para uma rede \textit{Spatial-Temporal Graph Convolutional Network} (ou Rede Convolucional de Grafo Espaço-Temporal) (ST-GCN).
    Os autores avaliam os resultados para o mesmo subconjunto de 20 sinais definidos por \citeonline{lim-2016-bhof}, mas também para o \acrshort{asllvd} inteiro.
    % coordenadas 2D para grafos / corpo inteiro
\end{itemize}


% considerações importantes
% - são pesquisas que adotam abordagens que utilizam dados brutos (RGB, coordenadas)
% - muitas delas adotam recortes das mãos apenas
% - não consideram a fonologia
% - utilizam um número de sinais limitados, com o intuito de reduzir a complexidade -- consequentemente
%     - isso contribui para que eles obtenham acurácias maiores
%     - contudo, limita o problema perante o contexto real -- uma vez que a afasta da realidade da língua

% insights [ASLLVD tem 2.745 sinais]
% - ao comparar nossos experimentos com todos os trabalhos acima, de uma forma generalizada, vemos que o desempenho dos encoder-decoders (por volta dos 46%) foi mediano com relação aos demais (que em vários deles alcançaram marcas de 70%, 85% e até 93,30%)
% - no caso do transformer, seu desempenho posicionou-se superior aos demais trabalhos
% ------------
% - contudo, considerando que maioria deles modela apenas vocabulários pequenos [que correspondem em média a 2% do vocabulário disponível no ASLLVD (que é o que utilizamos aqui) e não ultrapassam 13% dele], nota-se que 
% - no entanto, se considerarmos apenas aqueles trabalhos que consideraram todos os sinais do ASLLVD (que seria um vocabulário de complexidade equivalente à nossa), percebemos que mesmo os encoder-decores apresentaram um salto importante, ficando em torno dos 46% de acurácia (quando até então observava-se por volta de 16%)
%   - se olharmos o transformer, esse salto é ainda mais expressivo



% Please add the following required packages to your document preamble:
% \usepackage{multirow}
% \usepackage{graphicx}
% \usepackage[table,xcdraw]{xcolor}
% If you use beamer only pass "xcolor=table" option, i.e. \documentclass[xcolor=table]{beamer}
\begin{table}[ht!]
    \centering
    \caption{Comparação dos resultados de nossos experimentos com outras pesquisas em \acrshort{slr} que também basearam-se no \acrshort{asllvd}}
    \label{tab:comparacao-resultados}
    \resizebox{0.90\textwidth}{!}{%
        \begin{tabular}{l|l|c|c}
            \hline
            \rowcolor[HTML]{EFEFEF}
            \cellcolor[HTML]{EFEFEF}Autor                       & Técnica                         & Nº de Sinais   & Acurácia          \\ \hline
            \citeonline{theodorakis-2014-dynamic-static}        & 2-S-U + \acrshort{hmm}          & 97             & 63,15\%           \\ \hline
                                                                & \acrshort{bhof}                 & 20             & 85,00\%           \\
                                                                & \acrshort{hof}                  & 20             & 70,00\%           \\
                                                                & \acrshort{pca}                  & 20             & 45,00\%           \\
                                                                & \acrshort{mei}                  & 20             & 25,00\%           \\
            \multirow{-5}{*}{\citeonline{lim-2016-bhof}}        & \acrshort{mhi}                  & 20             & 10,00\%           \\ \hline
            \citeonline{metaxas-2018-linguistically}            & \acrshort{hcorf}                & 350            & 93,30\%           \\ \hline
            \citeonline{lim-2019-isolated-slr-cnn-hei}          & \acrshort{hei}                  & 20             & 31,50\%           \\ \hline
                                                                & \acrshort{stgcn}                & 20             & 61,04\%           \\
            \multirow{-2}{*}{\citeonline{amorim-2019-stgcn-sl}} & \acrshort{stgcn}                & 2.745          & 16,48\%           \\ \hline
                                                                & \textbf{Encoder-Decoder (LSTM)} & \textbf{2.650} & \textbf{87,21\%}  \\
                                                                & \textbf{Encoder-Decoder (GRU)}  & \textbf{2.650} & \textbf{88,00\%}  \\
            \multirow{-3}{*}{\textbf{Proposta atual}}           & \textbf{Transformer}            & \textbf{2.650} & \textbf{100,00\%} \\ \hline
        \end{tabular}%
    }
    \nomefonte{}
\end{table}


% Algumas comparações (HEI - Hand Energy Image x MEI, MHI) com ASLLVD
% 2020 - Amorim, Macedo, Zanchettin - Spatial-Temporal Graph Convolutional Networks for Sign Language Recognition
%                         accuracy
%     (tudo)              16.48
%     (20 sinais)         61.04

% 2016 - Lim, Tan, Tan - Block-based histogram of optical flow for isolated sign language recognition
%     (20 sinais)
%         MHI             10.00
%         MEI             25.00
%         PCA             45.00
%         HOF             70.00
%         > BHOF          85.00


% 2019 - Lim et al - Isolated sign language recognition using Convolutional Neural Network hand modelling and Hand Energy Image
%     (20 sinais - duas mãos)
%         > Proposed HEI  31.50
%         MEI             25.00
%         MHI             10.00 
%         MEI + MHI       20.00

% 2014 - Dilsizian et al - A New Framework for Sign Language Recognition based on 3D Handshape Identification and Linguistic Modeling
% https://aclanthology.org/L14-1096/
%     >                  81.76

% 2014 - Theodorakis, Pitsikalis, Maragos - Dynamic–static unsupervised sequentiality, statistical subunits and lexicon for sign language recognition
%     (97 signs)
%     > 2-S-U            63.15

% 2018 - Metaxas, Dilsizian, Neidle - Linguistically-driven Framework for Computationally Efficient and Scalable Sign Recognition
% https://par.nsf.gov/servlets/purl/10065369
%     > (350 signs)      93.3

