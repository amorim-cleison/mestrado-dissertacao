
Com o objetivo de estabelecer um comparativo entre os resultados deste trabalho e aqueles apresentados por outras pesquisas na área de \acrfull{slr}, selecionamos alguns estudos que também utilizaram o \acrshort{asllvd} como referência em seus experimentos, os quais introduziremos brevemente a seguir:

% Para comparar os resultados obtidos neste trabalho, selecionamos algumas pesquisas que também adotaram o \acrshort{asllvd} para realizar o reconhecimento dos sinais.
% A maioria deles, no entanto, limita-se aos segmentos das mãos para isso e também utiliza como dados de entrada coordenadas 3D ou imagens RGB dos frames das amostras, as quais são processadas por técnicas de visão computacional.
% Além disso, eles comumente selecionam subconjuntos menores de sinais ao invés de considerar o \textit{dataset} completo, como fizemos aqui -- isso contribui para acurácias maiores mas limitá-os perante o contexto real de aplicação da língua.


\begin{itemize}
    % tipo de dado bruto?
    % recorte de mãos x corpo inteiro?

    \item \textbf{\citeonline{theodorakis-2014-dynamic-static}}: utiliza técnicas não-supervisionadas e também \acrfull{hmm} para gerar subunidades de movimento e pausa (chamadas 2-S-U) que são utilizadas para reconhecer os sinais.
          Para isso, os autores processaram os frames das amostras concentrando-se apenas nas mãos dos indivíduos e selecionaram um subconjunto de 97 sinais do \acrshort{asllvd}.
          % processa frames de vídeo / foco nas mãos

    \item \textbf{\citeonline{lim-2016-bhof}}: introduz a técnica de \acrfull{bhof}, que gera histogramas do fluxo óptico das mãos dos indivíduos a partir dos frames das amostras.
          Em seus experimentos, os autores selecionaram um subconjunto pequeno contendo apenas 20 sinais do \acrshort{asllvd}.

          Além disso, esse trabalho apresenta um comparativo dos resultados do \acrshort{bhof} com aqueles obtidos pelas técnicas \acrfull{mei}, \acrfull{mhi}, \acrfull{pca} e \acrfull{hof} para esse mesmo subconjunto de sinais.
          % processa frames de vídeo / recorta as mãos (remove outras partes)

    \item \textbf{\citeonline{metaxas-2018-linguistically}}: combina diferentes técnicas para produzir uma variedade de features que se referem a configuração de mão inicial e final; número de mãos envolvidas; distância entre mãos; coordenadas 3D do corpo, face, mãos e braços; e contato das mãos com o corpo. Elas são utilizadas como entrada para um modelo baseado em \acrfull{hcorf} para reconhecer um subconjunto selecionado de 350 sinais do \acrshort{asllvd}.
          % extrai features: a) handshapes (start and end); b) number of hands; c) 3D upper body locations, movements of the hands and arms, and distance between the hands; d) facial features (include 66 points from 3D estimates for the forehead, ear, eye, nose, and mouth regions, and their velocities across frames) and head movements; e) contact (extracted from our 3D face and upper body movement estimation, and relate to the possibilities of the hand touching specific parts of the head or body).
          % processa frames de video para extrair features de nivel maior / considera mix de features e coordenadas 3D

    \item \textbf{\citeonline{lim-2019-isolated-slr-cnn-hei}}: introduz uma representação chamada \acrfull{hei} que, por sua vez, também concentra-se nas mãos dos indivíduos e é utilizada como entrada para uma rede \acrfull{cnn}.
          Aqui os autores adotam o mesmo subconjunto de 20 sinais definidos por \citeonline{lim-2016-bhof} acima.
          % processa frames de vídeo / recorta as mãos

    \item \textbf{\citeonline{amorim-2019-stgcn-sl}}: utiliza grafos para modelar a dimensão espacial das coordenadas 2D do corpo dos indivíduos bem como a relação temporal dos seus movimentos, os quais são fornecidos como entrada para uma rede \acrfull{stgcn}.
          Os autores avaliam os resultados para o mesmo subconjunto de 20 sinais definidos por \citeonline{lim-2016-bhof}, mas também para o \acrshort{asllvd} inteiro.
          % coordenadas 2D para grafos / corpo inteiro
\end{itemize}


De um modo geral, observa-se que a maioria dos estudos acima aborda a tarefa de \acrlong{slr} através do processamento de dados brutos, geralmente frames RGB ou coordenadas 2D ou 3D, conforme discutimos ao longo da \autoref{sec:slr}.
Apesar de em alguns casos esses dados serem utilizados para gerar features intermediárias -- como subunidades de movimento e pausa, imagens de fluxo óptico ou de energia de movimento, histogramas, grafos, entre outras -- estas, por sua vez, apresentam um nível semântico muito menos informativo quanto à língua de sinais e sua linguística propriamente ditas do que aquelas que introduzimos neste trabalho.

Uma outra observação importante é a de que parte desses trabalhos concentra-se apenas nas mãos dos indivíduos e isso nos remete a um dos problemas de abordagem do \acrshort{slr} discutidos na \autoref{sec:slr-desafios}. Como resultado disso, canais referentes a traços linguísticos relevantes como, por exemplo, expressões não-manuais, locação das mãos e movimentos do corpo, acabam sendo desconsiderados.

Além disso, percebe-se que todos esses trabalhos modelam vocabulários que correspondem a subconjuntos muito pequenos do \acrshort{asllvd}, conforme discutimos na \autoref{sec:slr-breve-panorama}.
Esses subconjuntos, por sua vez, representam menos de 13\% do vocabulário total disponibilizado por esse \textit{dataset}, que consiste de 2.745 sinais.

Apesar de compreendermos que todos esses fatores têm por objetivo simplificar o tamanho e a complexidade do escopo abordado por essas pesquisas, é inevitável ressaltar após a discussão realizada ao longo deste trabalho, que esses recortes distanciam a língua de sinais de seu contexto real de uso e, consequentemente, limitam a produção de contribuições significativas para a área de \acrshort{slr}.

% insights [ASLLVD tem 2.745 sinais]
% considerações importantes
% - são pesquisas que adotam abordagens que utilizam dados brutos (RGB, coordenadas)
% - muitas delas adotam recortes das mãos apenas
% - não consideram a fonologia
% - utilizam um número de sinais limitados, com o intuito de reduzir a complexidade -- consequentemente
%     - isso contribui para que eles obtenham acurácias maiores
%     - contudo, limita o problema perante o contexto real -- uma vez que a afasta da realidade da língua


% % Please add the following required packages to your document preamble:
% % \usepackage{graphicx}
% % \usepackage[table,xcdraw]{xcolor}
% % If you use beamer only pass "xcolor=table" option, i.e. \documentclass[xcolor=table]{beamer}
% \begin{table}[ht!]
%     \centering
%     \caption{}
%     \label{tab:comparacao-resultados}
%     \resizebox{0.8\textwidth}{!}{%
%     \begin{tabular}{llcr}
%     \hline
%     \rowcolor[HTML]{EFEFEF} 
%     \cellcolor[HTML]{EFEFEF}\textbf{Artigo}                       & \textbf{Técnica}                        & \textbf{Nº de Sinais} & \textbf{Acurácia}  \\ \hline
%     \citeonline{metaxas-2018-linguistically}     & Múltiplas + HCORF                       & 350                & 93,30 \%           \\
%     \citeonline{lim-2016-bhof}                   & BHOF                                    & 20                 & 85,00\%           \\
%     \citeonline{lim-2016-bhof}                   & \acrshort{hof}         & 20                 & 70,00 \%           \\
%     \citeonline{theodorakis-2014-dynamic-static} & 2-S-U + \acrshort{hmm} & 97                 & 63,15 \%           \\
%     \citeonline{amorim-2019-stgcn-sl}            & ST-GCN-SL                               & 20                 & 61,04\%           \\
%     \citeonline{lim-2016-bhof}                   & \acrshort{pca}         & 20                 & 45,00 \%           \\
%     \citeonline{lim-2019-isolated-slr-cnn-hei}   & HEI                                     & 20                 & 31,50\%           \\
%     \citeonline{lim-2016-bhof}                   & \acrshort{mei}         & 20                 & 25,00 \%           \\
%     \citeonline{amorim-2019-stgcn-sl}            & ST-GCN-SL                               & Todos              & 16,48\%           \\
%     \citeonline{lim-2016-bhof}                   & \acrshort{mhi}         & 20                 & 10,00 \%           \\ \hline
%     \textbf{(Proposta atual)}                                     & \textbf{Transformer}                    & \textbf{Todos}     & \textbf{100,00\%} \\
%     \textbf{(Proposta atual)}                                     & \textbf{Encoder-Decoder (GRU)}          & \textbf{Todos}     & \textbf{46,98\%}         \\
%     \textbf{(Proposta atual)}                                     & \textbf{Encoder-Decoder (LSTM)}         & \textbf{Todos}     & \textbf{45,99\%}         \\ \hline
%     \end{tabular}%
%     }
%     \end{table}


% Please add the following required packages to your document preamble:
% \usepackage{multirow}
% \usepackage{graphicx}
% \usepackage[table,xcdraw]{xcolor}
% If you use beamer only pass "xcolor=table" option, i.e. \documentclass[xcolor=table]{beamer}
\begin{table}[ht!]
    \centering
    \caption{Comparação dos resultados de nossos experimentos com outras pesquisas em \acrshort{slr} que também basearam-se no \acrshort{asllvd}}
    \label{tab:comparacao-resultados}
    \resizebox{0.90\textwidth}{!}{%
        \begin{tabular}{l|l|c|c}
            \hline
            \rowcolor[HTML]{EFEFEF}
            \cellcolor[HTML]{EFEFEF}Autor                       & Técnica                         & Nº de Sinais   & Acurácia          \\ \hline
            \citeonline{theodorakis-2014-dynamic-static}        & 2-S-U + \acrshort{hmm}          & 97             & 63,15\%           \\ \hline
                                                                & \acrshort{bhof}                 & 20             & 85,00\%           \\
                                                                & \acrshort{hof}                  & 20             & 70,00\%           \\
                                                                & \acrshort{pca}                  & 20             & 45,00\%           \\
                                                                & \acrshort{mei}                  & 20             & 25,00\%           \\
            \multirow{-5}{*}{\citeonline{lim-2016-bhof}}        & \acrshort{mhi}                  & 20             & 10,00\%           \\ \hline
            \citeonline{metaxas-2018-linguistically}            & \acrshort{hcorf}                & 350            & 93,30\%           \\ \hline
            \citeonline{lim-2019-isolated-slr-cnn-hei}          & \acrshort{hei}                  & 20             & 31,50\%           \\ \hline
                                                                & \acrshort{stgcn}                & 20             & 61,04\%           \\
            \multirow{-2}{*}{\citeonline{amorim-2019-stgcn-sl}} & \acrshort{stgcn}                & Todos          & 16,48\%           \\ \hline
                                                                & \textbf{Encoder-Decoder (LSTM)} & \textbf{Todos} & \textbf{45,99\%}  \\
                                                                & \textbf{Encoder-Decoder (GRU)}  & \textbf{Todos} & \textbf{46,98\%}  \\
            \multirow{-3}{*}{\textbf{Proposta atual}}           & \textbf{Transformer}            & \textbf{Todos} & \textbf{100,00\%} \\ \hline
        \end{tabular}%
    }
    \nomefonte{}
\end{table}


Uma vez introduzido esse debate, vamos analisar os resultados listados na \autoref{tab:comparacao-resultados}.
Ao comparar as pesquisas acima com os modelos utilizados em nossos experimentos, como um todo, vemos que os \textit{Encoder-Decoders} apresentaram um desempenho bastante mediano com relação aos demais. Ao mesmo tempo em que sua acurácia por volta de 46,00\% foi capaz de superar técnicas como \acrshort{pca}, \acrshort{hei}, \acrshort{mei} e \acrshort{mhi}, outras técnicas como \acrshort{hcorf}, \acrshort{bhof}, \acrshort{hof} e \acrshort{stgcn} conseguiram superar essa marca alcançando valores de até 93,30\%.
A acurácia de 100,00\% do \textit{Transformer}, por sua vez, superou de forma consistente o desempenho apresentado pelas demais técnicas na tabela.

Se compararmos apenas os estudos que modelaram features referentes ao corpo inteiro do indivíduo ao invés de apenas suas mãos -- ainda que indiretamente, através de dados brutos -- temos uma perspectiva diferente. \citeonline{metaxas-2018-linguistically} e \citeonline{amorim-2019-stgcn-sl} se enquadram nesses critérios e, pela tabela, vemos que ambas as técnicas de \acrshort{hcorf} e \acrshort{stgcn} superaram os resultados das implementações de \textit{Encoder-Decoders} utilizadas em nossos experimentos.


% - ao comparar nossos experimentos com todos os trabalhos acima, de uma forma generalizada, vemos que o desempenho dos encoder-decoders (por volta dos 46%) foi mediano com relação aos demais (que em vários deles alcançaram marcas de 70%, 85% e até 93,30%)
% - no caso do transformer, seu desempenho posicionou-se superior aos demais trabalhos

Em contrapartida, essa análise é ainda diferente se considerarmos apenas estudos que modelaram um vocabulário com todos os sinais do \acrshort{asllvd}, que seria uma complexidade equivalente à que abordamos em nossos experimentos.
Pela tabela, apenas \citeonline{amorim-2019-stgcn-sl} registraram resultados sob essas condições, o qual foi de 16,48\% para sua técnica \acrshort{stgcn}. 
Esse número é bem inferior às acurácias de 45,99\% e 46,98\% alcançadas pelas implementações de \textit{Encoder-Decoders} utilizadas em nossos experimentos. Quando comparado ao \textit{Transformer}, por sua vez, é constatada uma margem de diferença ainda mais expressiva para esse número.

% ------------
% - contudo, considerando que maioria deles modela apenas vocabulários pequenos [que correspondem em média a 2% do vocabulário disponível no ASLLVD (que é o que utilizamos aqui) e não ultrapassam 13% dele], nota-se que 
% - no entanto, se considerarmos apenas aqueles trabalhos que consideraram todos os sinais do ASLLVD (que seria um vocabulário de complexidade equivalente à nossa), percebemos que mesmo os encoder-decores apresentaram um salto importante, ficando em torno dos 46% de acurácia (quando até então observava-se por volta de 16%)
%   - se olharmos o transformer, esse salto é ainda mais expressivo





% Algumas comparações (HEI - Hand Energy Image x MEI, MHI) com ASLLVD
% 2020 - Amorim, Macedo, Zanchettin - Spatial-Temporal Graph Convolutional Networks for Sign Language Recognition
%                         accuracy
%     (tudo)              16.48
%     (20 sinais)         61.04

% 2016 - Lim, Tan, Tan - Block-based histogram of optical flow for isolated sign language recognition
%     (20 sinais)
%         MHI             10.00
%         MEI             25.00
%         PCA             45.00
%         HOF             70.00
%         > BHOF          85.00


% 2019 - Lim et al - Isolated sign language recognition using Convolutional Neural Network hand modelling and Hand Energy Image
%     (20 sinais - duas mãos)
%         > Proposed HEI  31.50
%         MEI             25.00
%         MHI             10.00 
%         MEI + MHI       20.00

% 2014 - Dilsizian et al - A New Framework for Sign Language Recognition based on 3D Handshape Identification and Linguistic Modeling
% https://aclanthology.org/L14-1096/
%     >                  81.76

% 2014 - Theodorakis, Pitsikalis, Maragos - Dynamic–static unsupervised sequentiality, statistical subunits and lexicon for sign language recognition
%     (97 signs)
%     > 2-S-U            63.15

% 2018 - Metaxas, Dilsizian, Neidle - Linguistically-driven Framework for Computationally Efficient and Scalable Sign Recognition
% https://par.nsf.gov/servlets/purl/10065369
%     > (350 signs)      93.3

