\section{Análise dos resultados}
\label{sec:avaliacao-resultados}

A \autoref{tab:resultados-modelos} apresenta o desempenho apresentado pelos modelos utilizados nos experimentos deste trabalho, respectivamente o \textit{Encoder-Decoder} implementado com \acrshort{lstm}, o \textit{Encoder-Decoder} implementado com \acrshort{gru} e o \textit{Transformer}.
Para cada modelo, são listadas as métricas\footnote{
    Uma vez que o reconhecimento de sinais consiste numa classificação multi-classes, os valores das métricas binárias de precisão, \textit{recall} e \textit{F1-score} foram consolidados utilizando-se a média ponderada pelo número de amostras em cada classe.
} de acurácia, precisão, \textit{recall} e \textit{F1-score}, bem como a perda resultante.

% precisao, recall, f1 calculados com média ponderada entre classes (weighted average)
% precisao, recall, f1 consistentes com acurácia (justificar)

% Please add the following required packages to your document preamble:
% \usepackage{graphicx}
% \usepackage[table,xcdraw]{xcolor}
% If you use beamer only pass "xcolor=table" option, i.e. \documentclass[xcolor=table]{beamer}
\begin{table}[ht!]
    \centering
    \caption{Resultados dos modelos utilizados neste trabalho}
    \label{tab:resultados-modelos}
    \resizebox{0.85\textwidth}{!}{%
        \begin{tabular}{l|c|c|c|c|c}
            \hline
            \rowcolor[HTML]{EFEFEF}
            Modelo                 & Acurácia & Precisão & Recall   & F1-score & Erro médio/perda   \\ \hline
            Encoder-Decoder (LSTM) & 45,99\%  & 50,19\%  & 45,99\%  & 45,82\%  & 4,88312 \\
            Encoder-Decoder (GRU)  & 46,98\%  & 52,71\%  & 46,98\%  & 47,70\%  & 4,57513 \\
            Transformer            & 100,00\% & 100,00\% & 100,00\% & 100,00\% & 0,00101 \\ \hline
        \end{tabular}%
    }
    \nomefonte{}
\end{table}

% %FIXME: [cz] eu sei que na proxima secao vc vai apresentar a comparação com os trabalhos relacionados, mas aqui senti falta de um baseline que pudesse comparar com a abordagem original qual foi a melhora usando a estratégia linguistica em relacao a usar somente as imagens --> [cca5] um pouco mais a frente eu acabo discutindo essas diferenças (que basicamente giram em torno do nível semântico que aqui é maior e faz mais sentido perante a língua)

Primeiramente, observamos que os modelos baseados na arquitetura \textit{Encoder-Decoder} apresentaram resultados muito semelhantes entre si, apesar de ainda não muito expressivos de um modo geral.
% De um modo geral, percebe-se entre os modelos baseados na arquitetura \textit{Encoder-Decoder} resultados muito semelhantes entre si, mas não muito expressivos.
Dentre eles, percebemos também que a versão implementada utilizando codificador e decodificador baseados em redes \acrshort{gru} obteve uma pequena vantagem em comparação àquela que utilizou redes \acrshort{lstm} -- ao passo que a primeira alcançou uma acurácia de 46,98\%, a segunda obteve 45,99\%.
O valor das demais métricas e da perda computada para ambos os casos replicaram esse mesmo comportamento da acurácia e reforçam tal análise.

Por outro lado, quando analisamos os resultados do \textit{Transformer} vemos um desempenho bastante expressivo com uma acurácia de 100,00\%, a qual excede aquelas apresentadas pelos \textit{Encoder-Decoders} e é reiterada consistentemente pelo valor das demais métricas e também da perda computada.

Isso nos remete à argumentação de \citeonline{wolf-2020-transformers,jurafsky-2022-speech-lang-processing} citada na \autoref{sec:modelos-sequenciais}, que afirma que essa arquitetura tem se mostrado extremamente bem-sucedida entre tarefas de \acrfull{nlp}, superando arquiteturas como as \acrshortpl{rnn}. Exatamente por isso, elas são atualmente dominantes nessa área.

No contexto deste trabalho, acreditamos que alguns motivos particulares contribuíram para os resultados acima.
Em primeiro lugar, entendemos que a escolha pelo uso da abordagem linguística dos sinais nos possibilitou trabalhar num nível semântico muito mais elevado do que seríamos capazes de fazer com dados brutos como os pixels de imagens RGB ou coordenadas aleatórias no espaço. Conforme discutimos no \autoref{cap:metodos}, essa é uma abordagem comum em tarefas envolvendo linguagem no \acrshort{nlp} e que nos permite produzir \textit{features} de melhor qualidade para ensinar os modelos acerca da estrutura dessas línguas.

Em segundo lugar, a representação introduzida aqui foi capaz de transformar um conjunto de canais linguísticos complexos dos sinais do \acrshort{asllvd}, em \textit{features} discretas (ou ``palavras'', como aqui as denominamos) mais simples e bem definidas a serem consumidas pelos modelos.
Isso deu origem a um vocabulário que, por sua vez, é muito menos complexo do que aqueles com os quais arquiteturas como o \textit{Transformer} foram originalmente projetadas para lidar. Além disso, nesse processo selecionamos um número menor de atributos fonológicos da língua de sinais e isso também contribuiu para tornar nosso vocabulário ainda mais compacto.

Dessa forma, entendemos que a combinação desse conjunto de \textit{features} semanticamente mais coerentes e simplificadas, com a utilização de um modelo robusto no processamento de linguagens como o \textit{Transformer}, foram fatores que conduziram ao desempenho favorável acima.
Contudo, para os \textit{Encoder-Decoders} esse desempenho foi mais modesto.

% encoder-decoders com desempenho proximo, mas GRU com uma leve vantagem
% transformer com desempenho muito superior
% - provavelmente pela combinação: 
%   - o transformer é muito robusto e projetado para lidar problemas muito complexos de linguagem. / ele está super-dimensionado para o problema em questão?
%   - essa abordagem super-simplifica o problema da língua de sinais produzindo um conjunto de features predominantemente discretas, como se tivéssemos um vocabulário de texto --> em oposição ao que ocorre quando opta-se por abordagens de visão computacional com dados brutos











% -> Precision is the fraction of detections reported by the model that were correct, while recall is the fraction of true events that were detected.
% In many cases, we wish to summarize the performance of the classifier with a single number rather than a curve. To do so, we can convert precision p and recall r into an F-score (or F-measure) given by:
%     F = 2pr / p + r.
% Another option is to report the total area lying beneath the PR curve.



% averaging techniques applicable to multiclass classification -------------
% (https://towardsdatascience.com/comprehensive-guide-on-multiclass-classification-metrics-af94cfb83fbd)
% weighted: accounts for class imbalance by computing the average of binary metrics weighted by the number of samples of each class in the target. If 3 (precision scores) for 3 classes are: Class 1 (0.85), class 2 (0.80), and class 3 (0.89), the weighted average will be calculated by multiplying each score by the number of occurrences of each class and dividing by the total number of samples.
