A \autoref{tab:resultados-modelos} nos mostra as métricas


% precisao, recall, f1 calculados com média ponderada entre classes (weighted average)
% precisao, recall, f1 consistentes com acurácia (justificar)


% encoder-decoders com desempenho proximo, mas GRU com uma leve vantagem
% transformer com desempenho muito superior
% - provavelmente pela combinação: 
%   - o transformer é muito robusto e projetado para lidar problemas muito complexos de linguagem. / ele está super-dimensionado para o problema em questão?
%   - essa abordagem super-simplifica o problema da língua de sinais produzindo um conjunto de features predominantemente discretas, como se tivéssemos um vocabulário de texto --> em oposição ao que ocorre quando opta-se por abordagens de visão computacional com dados brutos



% Please add the following required packages to your document preamble:
% \usepackage{multirow}
% \usepackage{graphicx}
% \usepackage[table,xcdraw]{xcolor}
% If you use beamer only pass "xcolor=table" option, i.e. \documentclass[xcolor=table]{beamer}
\begin{table}[ht!]
    \centering
    \caption{Resultados dos modelos utilizados neste trabalho}
    \label{tab:resultados-modelos}
    \resizebox{0.95\textwidth}{!}{%
        \begin{tabular}{l|l|rrr}
            \hline
            \rowcolor[HTML]{EFEFEF}
            \cellcolor[HTML]{EFEFEF}                        & \cellcolor[HTML]{EFEFEF}                            & \multicolumn{3}{c}{\cellcolor[HTML]{EFEFEF}Modelo}                                                                                                                                            \\ \cline{3-5}
            \rowcolor[HTML]{EFEFEF}
            \multirow{-2}{*}{\cellcolor[HTML]{EFEFEF}Etapa} & \multirow{-2}{*}{\cellcolor[HTML]{EFEFEF}Métrica}   & \multicolumn{1}{r|}{\cellcolor[HTML]{EFEFEF}\textbf{Encoder-Decoder (LSTM)}} & \multicolumn{1}{r|}{\cellcolor[HTML]{EFEFEF}\textbf{Encoder-Decoder (GRU)}} & \textbf{Transformer}             \\ \hline
                                                            & Acurácia                                            & \multicolumn{1}{r|}{88,93\%}                                                 & \multicolumn{1}{r|}{97,14\%}                                                & 100,00\%                         \\
                                                            & Precisão                                            & \multicolumn{1}{r|}{90,46\%}                                                 & \multicolumn{1}{r|}{97,81\%}                                                & 100,00\%                         \\
                                                            & Recall                                              & \multicolumn{1}{r|}{88,93\%}                                                 & \multicolumn{1}{r|}{97,14\%}                                                & 100,00\%                         \\
                                                            & F1-score                                            & \multicolumn{1}{r|}{87,66\%}                                                 & \multicolumn{1}{r|}{97,05\%}                                                & 100,00\%                         \\
            \multirow{-5}{*}{Treinamento}                   & Erro médio (\acrshort{cel})                         & \multicolumn{1}{r|}{1,268112}                                                & \multicolumn{1}{r|}{0,618142}                                               & 0,001205                         \\ \hline
                                                            & Acurácia                                            & \multicolumn{1}{r|}{42,40\%}                                                 & \multicolumn{1}{r|}{43,56\%}                                                & 100,00\%                         \\
                                                            & Precisão                                            & \multicolumn{1}{r|}{29,03\%}                                                 & \multicolumn{1}{r|}{31,22\%}                                                & 100,00\%                         \\
                                                            & Recall                                              & \multicolumn{1}{r|}{42,40\%}                                                 & \multicolumn{1}{r|}{43,56\%}                                                & 100,00\%                         \\
                                                            & F1-score                                            & \multicolumn{1}{r|}{32,42\%}                                                 & \multicolumn{1}{r|}{34,62\%}                                                & 100,00\%                         \\
            \multirow{-5}{*}{Validação}                     & Erro médio (\acrshort{cel})                         & \multicolumn{1}{r|}{5,184689}                                                & \multicolumn{1}{r|}{4,520252}                                               & 0,000000                         \\ \hline
                                                            & \cellcolor[HTML]{FFF5E1}Acurácia                    & \multicolumn{1}{r|}{\cellcolor[HTML]{FFF5E1}87,21\%}                         & \multicolumn{1}{r|}{\cellcolor[HTML]{FFF5E1}88,00\%}                        & \cellcolor[HTML]{FFF5E1}100,00\% \\
                                                            & Precisão                                            & \multicolumn{1}{r|}{89,69\%}                                                 & \multicolumn{1}{r|}{90,35\%}                                                & 100,00\%                         \\
                                                            & Recall                                              & \multicolumn{1}{r|}{87,21\%}                                                 & \multicolumn{1}{r|}{88,00\%}                                                & 100,00\%                         \\
                                                            & F1-score                                            & \multicolumn{1}{r|}{87,24\%}                                                 & \multicolumn{1}{r|}{88,17\%}                                                & 100,00\%                         \\
            \multirow{-5}{*}{\textbf{Testes}}               & \cellcolor[HTML]{FFF5E1}Erro médio (\acrshort{cel}) & \multicolumn{1}{r|}{\cellcolor[HTML]{FFF5E1}1,343880}                        & \multicolumn{1}{r|}{\cellcolor[HTML]{FFF5E1}1,136651}                       & \cellcolor[HTML]{FFF5E1}0,000000 \\ \hline
        \end{tabular}%
    }
    \nomefonte{}
\end{table}




% -> Precision is the fraction of detections reported by the model that were correct, while recall is the fraction of true events that were detected.
% In many cases, we wish to summarize the performance of the classifier with a single number rather than a curve. To do so, we can convert precision p and recall r into an F-score (or F-measure) given by:
%     F = 2pr / p + r.
% Another option is to report the total area lying beneath the PR curve.



% averaging techniques applicable to multiclass classification -------------
% (https://towardsdatascience.com/comprehensive-guide-on-multiclass-classification-metrics-af94cfb83fbd)
% weighted: accounts for class imbalance by computing the average of binary metrics weighted by the number of samples of each class in the target. If 3 (precision scores) for 3 classes are: Class 1 (0.85), class 2 (0.80), and class 3 (0.89), the weighted average will be calculated by multiplying each score by the number of occurrences of each class and dividing by the total number of samples.
