A \autoref{tab:resultados-modelos} nos mostra as métricas


% precisao, recall, f1 calculados com média ponderada entre classes (weighted average)
% precisao, recall, f1 consistentes com acurácia (justificar)


% encoder-decoders com desempenho proximo, mas GRU com uma leve vantagem
% transformer com desempenho muito superior
% - provavelmente pela combinação: 
%   - o transformer é muito robusto e projetado para lidar problemas muito complexos de linguagem. / ele está super-dimensionado para o problema em questão?
%   - essa abordagem super-simplifica o problema da língua de sinais produzindo um conjunto de features predominantemente discretas, como se tivéssemos um vocabulário de texto --> em oposição ao que ocorre quando opta-se por abordagens de visão computacional com dados brutos



% Please add the following required packages to your document preamble:
% \usepackage{graphicx}
% \usepackage[table,xcdraw]{xcolor}
% If you use beamer only pass "xcolor=table" option, i.e. \documentclass[xcolor=table]{beamer}
\begin{table}[ht!]
    \centering
    \caption{Resultados dos modelos utilizados neste trabalho}
    \label{tab:resultados-modelos}
    \resizebox{0.85\textwidth}{!}{%
        \begin{tabular}{l|c|c|c|c|c}
            \hline
            \rowcolor[HTML]{EFEFEF}
            Modelo                 & Acurácia & Precisão & Recall   & F1-score & Erro médio/perda   \\ \hline
            Encoder-Decoder (LSTM) & 45,99\%  & 50,19\%  & 45,99\%  & 45,82\%  & 4,88312 \\
            Encoder-Decoder (GRU)  & 46,98\%  & 52,71\%  & 46,98\%  & 47,70\%  & 4,57513 \\
            Transformer            & 100,00\% & 100,00\% & 100,00\% & 100,00\% & 0,00101 \\ \hline
        \end{tabular}%
    }
    \nomefonte{}
\end{table}




% -> Precision is the fraction of detections reported by the model that were correct, while recall is the fraction of true events that were detected.
% In many cases, we wish to summarize the performance of the classifier with a single number rather than a curve. To do so, we can convert precision p and recall r into an F-score (or F-measure) given by:
%     F = 2pr / p + r.
% Another option is to report the total area lying beneath the PR curve.



% averaging techniques applicable to multiclass classification -------------
% (https://towardsdatascience.com/comprehensive-guide-on-multiclass-classification-metrics-af94cfb83fbd)
% weighted: accounts for class imbalance by computing the average of binary metrics weighted by the number of samples of each class in the target. If 3 (precision scores) for 3 classes are: Class 1 (0.85), class 2 (0.80), and class 3 (0.89), the weighted average will be calculated by multiplying each score by the number of occurrences of each class and dividing by the total number of samples.
