\subsection{Morfologia}
\label{sec:linguistica-morfologia}

Morfologia é o estudo da estrutura interna das palavras e das regras que determinam sua formação. Um morfema é a menor unidade indivisível de sintaxe que retém significado e, na língua de sinais, é tido como a combinação da configuração de mão, orientação, locação e movimento, afirmam \citeonline{quadros-2004-estudos-linguisticos,jay-2011-dont-just-sign,hill-2019-sign-languages}.


Em línguas faladas, palavras complexas são muitas vezes formadas adicionando-se um prefixo ou sufixo a uma raiz. Por exemplo, o adjetivo ``infeliz'' é constituído de dois morfemas: o prefixo negativo \textit{in-} e o adjetivo \textit{feliz}; o substantivo ``capacidade'', por sua vez, é composto pelo adjetivo \textit{capaz} acrescido do sufixo \textit{-idade}; já o substantivo ``guarda-chuva'' é constituído pelos morfemas \textit{guarda} e \textit{chuva}.



De acordo com \citeonline{klima-1975-wit-poetry-asl,quadros-2004-estudos-linguisticos}, nas línguas de sinais essas formações resultam frequentemente de processos em que uma raiz é enriquecida com movimentos e contornos no espaço de sinalização. Também são utilizadas expressões não-manuais, alterações nos parâmetros fonológicos ou sinais específicos para indicar tempo, grau, intensidade, pluralidade, aspecto, entre outros.


Serão discutidos a seguir os dois principais processos de formação de palavras apresentados pela morfologia tradicional, a derivação e a flexão, sob a perspectiva das línguas de sinais \cite{quadros-2004-estudos-linguisticos,hill-2019-sign-languages,klima-1979-signs-of-language}:

\begin{enumerate}
    \item \textbf{Derivação}: consiste na formação de novas palavras a partir de uma mesma base lexical, como nos exemplos ``infeliz'' e ``capacidade'' introduzidos acima. No contexto das línguas de sinais, esses processos derivacionais podem incluir:

          \begin{enumerate}
              \item \underline{Nominalização}: consiste na derivação de substantivos a partir de verbos, e é um dos processos mais comuns para mudança de classe na morfologia.

                    Na língua de sinais, os substantivos apresentam basicamente os mesmos parâmetros fonológicos que os verbos, mas diferenciam-se pela repetição (ou reduplicação) do seu movimento. A \autoref{fig:sinais-ouvir-ouvinte} ilustra um exemplo de derivação do substantivo OUVINTE a partir do verbo OUVIR.

                    \figura[p. 98]
                    {fig:sinais-ouvir-ouvinte} % Label
                    {capitulos/fundamentacao/imagens/sinais_ouvir_ouvinte} % Path
                    {height=4cm} % Size
                    {O verbo OUVIR (à esquerda) é utilizado para derivar o substantivo OUVINTE (à direita)} % Caption
                    {quadros-2004-estudos-linguisticos} % Citation


              \item \underline{Composição}: consiste na criação de um novo sinal através da junção de duas bases preexistentes.
                    Existem três regras para a composição de sinais: a regra do contato, na qual o contato existente no primeiro ou segundo sinal da composição é mantido; a regra da sequência única, na qual o movimento interno ou repetição dos sinais é eliminada para formar um composto; e a regra da antecipação da mão não-dominante, em que a mão passiva antecipa o segundo sinal no processo de composição.

                    Observe na \autoref{fig:sinal-acidente} o exemplo do sinal ACIDENTE, que é composto a partir dos sinais CARRO e BATER utilizando-se a regra da antecipação da mão não-dominante.


                    \figura[p. 105]
                    {fig:sinal-acidente} % Label
                    {capitulos/fundamentacao/imagens/sinal_acidente} % Path
                    {height=4cm} % Size
                    {Composição do sinal ACIDENTE a partir dos sinais CARRO e BATER} % Caption
                    {quadros-2004-estudos-linguisticos} % Citation



              \item \underline{Incorporação de numeral}: combinação da configuração de mão de numeral a um sinal para especificar variação de quantidade em seu significado. Isso é útil, por exemplo, para representar número de anos, dias, horas, minutos, entre outros.

                    A \autoref{fig:sinais-numero-meses} ilustra o uso desse mecanismo para especificar o número de meses no sinal.

                    \figura[p. 107]
                    {fig:sinais-numero-meses} % Label
                    {capitulos/fundamentacao/imagens/sinais_numero_meses} % Path
                    {height=4cm} % Size
                    {Incorporação de numeral para especificar o número de meses no sinal} % Caption
                    {quadros-2004-estudos-linguisticos} % Citation



              \item \underline{Incorporação de negação}: geração da contraparte negativa de um sinal através da alteração de um de seus parâmetros, que comumente é o seu movimento.

                    A \autoref{fig:sinais-saber-naosaber} ilustra a negação do sinal SABER adicionando-se um movimento e uma expressão não-manual específica de negação.

                    \figura[p. 111]
                    {fig:sinais-saber-naosaber} % Label
                    {capitulos/fundamentacao/imagens/sinais_saber_naosaber} % Path
                    {height=4cm} % Size
                    {Incorporação da negação ao sinal SABER} % Caption
                    {quadros-2004-estudos-linguisticos} % Citation

          \end{enumerate}


    \item \textbf{Flexão}: consiste na adição de informação gramatical a palavras existentes para fazer com que elas se adequem melhor ao contexto em que são utilizadas. Nas línguas de sinais, alguns desses processos flexionais incluem:

          \begin{enumerate}
              \item \underline{Pessoa}: também conhecida como dêixis\footnote{
                        Dêixis: palavra grega que significa apontar ou indicar, e representa uma forma de estabelecer referenciais no espaço que são utilizados para flexionar verbos com concordância.~\cite{quadros-2004-estudos-linguisticos}
                    }, consiste na modificação da referência de pessoa para os verbos.
                    Na prática, isso é feito pelo interlocutor apontando-se para diferentes pontos no espaço à sua frente, os quais serão utilizados como referenciais que representam pessoas (ou objetos) envolvidas no discurso.

                    A concordância do verbo se dará pela articulação do movimento partindo de um desses referenciais para o outro, conforme ilustrado na \autoref{fig:verbo-entregar-deixis}.

                    \figura[p. 114]
                    {fig:verbo-entregar-deixis} % Label
                    {capitulos/fundamentacao/imagens/verbo_entregar_deixis} % Path
                    {height=4cm} % Size
                    {Flexão de pessoa para o verbo ENTREGAR, envolvendo dois referenciais} % Caption
                    {quadros-2004-estudos-linguisticos} % Citation


              \item \underline{Número}: é utilizada para indicar: a forma plural do sinal, a qual é marcada pela sua repetição; ou a existência de múltiplos referentes no discurso, pela articulação da ação na direção dos respectivos referentes no espaço.

                    Veja na \autoref{fig:verbo-entregar-flexao-numero} a flexão do verbo ENTREGAR para um, três e vários referentes.

                    \figura[p. 120]
                    {fig:verbo-entregar-flexao-numero} % Label
                    {capitulos/fundamentacao/imagens/verbo_entregar_flexao_numero} % Path
                    {height=4cm} % Size
                    {Flexão de número do verbo ENTREGAR para um (à esquerda), três (ao centro) e vários referentes (à direita)} % Caption
                    {quadros-2004-estudos-linguisticos} % Citation


              \item \underline{Grau}: adiciona variação de grau ou intensidade ao sinal, a qual geralmente é transmitida utilizando-se expressões não-manuais.
                    A \autoref{fig:sinais-lindo-lindinho-lindissimo} ilustra a flexão do sinal LINDO para os graus de pouco (LINDINHO) e muito (LINDÍSSIMO).

                    \figura[p. 65]
                    {fig:sinais-lindo-lindinho-lindissimo} % Label
                    {capitulos/fundamentacao/imagens/sinais_lindo_lindinho_lindissimo} % Path
                    {height=4cm} % Size
                    {Flexões de grau para o sinal LINDO} % Caption
                    {pereira-2011-conhecimento-alem-sinais} % Citation


              \item \underline{Modo}: especifica a maneira com que uma ação é realizada e também se utiliza de expressões não-manuais.
                    Por exemplo, poderia-se detalhar que uma ação foi realizada ``facilmente'' ou ``com dificuldade''.


              \item \underline{Aspecto temporal}: determina a forma com que uma ação relaciona-se com o tempo, a qual pode ser uma das seguintes: incessante, ininterrupta, habitual (recorrente), contínua, ou duradoura (permanente).

                    Observe na \autoref{fig:verbo-cuidar-flexao-temporal} alguns exemplos de flexões do verbo CUIDAR para os aspectos temporais incessante, ininterrupto e habitual.

                    \figura[p. 123]
                    {fig:verbo-cuidar-flexao-temporal} % Label
                    {capitulos/fundamentacao/imagens/verbo_cuidar_flexao_temporal} % Path
                    {height=4cm} % Size
                    {Flexões temporal do verbo CUIDAR para os aspectos incessante (à esquerda), ininterrupto (ao centro) e habitual (à direita)} % Caption
                    {quadros-2004-estudos-linguisticos} % Citation


              \item \underline{Aspecto distributivo}: determina a forma com que uma ação é distribuída entre diferentes pessoas ou objetos no discurso.
                    Ela pode ser: exaustiva, quando a ação é repetida exaustivamente; específica, quando direciona-se a pessoas ou objetos específicos; não-específica, quando é generalizada ou indeterminada.

                    Exemplos de flexão distributiva podem ser encontrados na \autoref{fig:verbo-entregar-flexao-numero} para os aspectos específico (ao centro) e não-específico (à direita).


              \item \underline{Reciprocidade}: especifica que uma ação (ou relação) ocorre de forma mútua. Ela é representada pela duplicação do sinal, a qual é articulada simultaneamente.
                    Observe na \autoref{fig:sinal-olhar-reciproco} um exemplo para o sinal OLHAR.

                    \figura[p. 122]
                    {fig:sinal-olhar-reciproco} % Label
                    {capitulos/fundamentacao/imagens/sinal_olhar_reciproco} % Path
                    {height=4cm} % Size
                    {Flexão de reciprocidade para o sinal OLHAR} % Caption
                    {quadros-2004-estudos-linguisticos} % Citation

          \end{enumerate}

\end{enumerate}

