\subsubsection{Morfologia}
\label{linguistica-gramatica-morfologia}

A morfologia é o estudo da estrutura interna das palavras e das regras que determinam sua formação. Um morfema é a menor unidade indivisível de sintaxe que retém significado~\cite{quadros-2004-estudos-linguisticos,jay-2011-dont-just-sign}.

% rules for producing a word,
%\cite{hill-2019-sign-languages}

Assim como as palavras em todas as línguas humanas, os sinais pertencem a categorias lexicais ou a classes de palavras tais como substantivo, verbo, pronome, adjetivo, advérbio, numerais, entre outras. A língua de sinais tem um léxico e um sistema de criação de novos sinais em que os morfemas são combinados. Entretanto, essas línguas diferem-se das línguas orais nos tipos de processos combinatórios que frequentemente criam palavras morfologicamente complexas. 

Para as línguas orais, palavras complexas são muitas vezes formadas pela adição de um prefixo ou sufixo a uma raiz, como por exemplo \textit{des-}, \textit{ante-}, \textit{hiper-}, \textit{-ão}, \textit{-dade}, \textit{-ência}, entre outros. Por exemplo, o adjetivo ``infeliz'' é constituído de dois morfemas -- o prefixo negativo \textit{in-} e o adjetivo \textit{feliz} --; também o substantivo ``capacidade'' é composto do adjetivo \textit{capaz} e acrescido do sufixo \textit{-idade}; já o substantivo ``guarda-chuva'' é constituído pelos morfemas \textit{guarda} e \textit{chuva}.

Nas línguas de sinais, essas formas resultam frequentemente de processos em que uma raiz é enriquecida com vários movimentos e contornos no espaço de sinalização. São utilizadas expressões não-manuais, alterações nos parâmetros fonológicos ou sinais específicos para indicar tempo, grau, intensidade, pluralidade, aspecto, entre outros~\cite{klima-1975-wit-poetry-asl,quadros-2004-estudos-linguisticos}.

De acordo com \citeonline{quadros-2004-estudos-linguisticos}, a morfologia tradicional apresenta dois processos de formação de palavras, a derivação e a flexão, os quais exploraremos a seguir:

% A morfologia tradicional apresenta basicamente duas áreas de investiga-  ção: a derivacional e a flexional. A primeira detém-se ao estudo da formação de  diferentes palavras com uma mesma base lexical, por exemplo, no português  ‘sonhador’ é derivado de ‘sonhar’. A segunda envolve o estudo dos processos  que acrescentam informação gramatical à palavra que já existe.


\begin{enumerate}
    \item \textbf{Derivação}: consiste na formação de diferentes palavras a partir de uma mesma base lexical, como nos exemplos ``infeliz'' e ``capacidade'' introduzidos anteriormente. Na língua de sinais, alguns dos processos derivacionais incluem:
    
    % detém-se ao estudo da formação de  diferentes palavras com uma mesma base lexical, por exemplo, no português  ‘sonhador’ é derivado de ‘sonhar’.

    \begin{enumerate}
        \item \underline{Nominalização} (ou derivação de substantivos a partir de verbos): uma das principais funções da morfologia é a mudança de classe, ou seja, a utilização do significado de uma palavra em uma outra classe gramatical.
        Na língua de sinais, forma-se um novo sinal para se utilizar o significado de um sinal existente num contexto que requer uma classe gramatical diferente. 
        
        Um dos processos mais comuns nesse âmbito é a derivação de nomes a partir de verbos (ou vice-versa). Isso ocorre pela mudança no tipo de movimento, ou seja, apesar de apresentarem a mesma locação, configuração e orientação de mão, os nomes simplesmente repetem (ou reduplicam) a estrutura segmental dos verbos (vide \autoref{fig:sinais-pentear-pente} e \autoref{fig:sinais-ouvir-ouvinte}).
        
        \figura[p. 98]
            {fig:sinais-pentear-pente} % Label
            {imagens/fundamentacao/sinais_pentear_pente} % Path
            {height=5cm} % Size
            {Sinal PENTE (à direita), que repete o movimento do verbo PENTEAR (à esquerda)} % Caption
            {quadros-2004-estudos-linguisticos} % Citation
        
        \figura[p. 98]
            {fig:sinais-ouvir-ouvinte} % Label
            {imagens/fundamentacao/sinais_ouvir_ouvinte} % Path
            {height=5cm} % Size
            {Sinal OUVINTE (à direita), que repete o movimento do verbo OUVIR (à esquerda)} % Caption
            {quadros-2004-estudos-linguisticos} % Citation
        

        % Seguindo proposta de Supalla e Newport (1978) para a ASL, observa-se  que a língua de sinais brasileira pode derivar nomes de verbos pela mudança no tipo de movimento. O movimento dos nomes repete e encurta o movimento dos verbos, conforme exemplo da figura a seguir. 

        % Supalla e Newport focalizam a análise do movimento e descrevem as  diferenças entre o movimento do verbo e do nome em detalhes, concluindo  que os pares apresentam a mesma locação, configuração e orientação de mão,  e que o nome simplesmente repete ou reduplica a estrutura segmental do  verbo, conforme exemplos a seguir. 
        % Esse processo de repetição é chamado reduplicação. Semelhante à  nominalização no português, na língua de sinais brasileira repete-se o morfemabase (verbo) e tem-se como produto um nome. O processo de adicionar  morfemas a uma forma base é uma forma de criar novas unidades lexicais. 


        \item \underline{Composição}: processo em que se juntam duas bases preexistentes na língua para criar um novo vocábulo (ou composto). São exemplos de composição no português: trem-de-ferro, aguardente, ciclovia, entre outros.
        
        Na língua de sinais, três regras são utilizadas para criar compostos: a regra do contato (onde o contato existente no primeiro ou segundo sinal utilizado na composição é mantido, como na \autoref{fig:sinal-escola}), a regra da sequência única (onde o movimento interno ou repetição dos sinais é eliminada para formar o composto, como na \autoref{fig:sinal-pais}), e a regra da antecipação da mão não-dominante (onde a mão passiva antecipa o segundo sinal no processo de composição, como na \autoref{fig:sinal-acidente}).

        \figura[p. 103]
            {fig:sinal-escola} % Label
            {imagens/fundamentacao/sinal_escola} % Path
            {height=5cm} % Size
            {Sinal ESCOLA (composição de CASA e ESTUDAR)} % Caption
            {quadros-2004-estudos-linguisticos} % Citation

        \figura[p. 104]
            {fig:sinal-pais} % Label
            {imagens/fundamentacao/sinal_pais} % Path
            {height=5cm} % Size
            {Sinal PAIS (à direita), composto por PAI (à esquerda) e MÃE (à direita)} % Caption
            {quadros-2004-estudos-linguisticos} % Citation

        \figura[p. 105]
            {fig:sinal-acidente} % Label
            {imagens/fundamentacao/sinal_acidente} % Path
            {height=5cm} % Size
            {Sinal ACIDENTE (composição de CARRO e BATER)} % Caption
            {quadros-2004-estudos-linguisticos} % Citation


        % Segundo Rocha (1998, p. 187), a  composição é um processo autônomo em que se juntam duas bases preexistentes  na língua para criar um novo vocábulo, dito composto. São exemplos de composição no português:  Trem-de-ferro  Aguardente  Salário-família  Cadeira de balanço  Lipoaspiração  Ciclovia 

        % Scott Liddel (1984) desenvolveu estudos sobre os compostos na ASL. Ele  observou que dois sinais formam um sinal composto quando mudanças  predicáveis ocorrem como o resultado de aplicação de regras, da mesma forma como acontece com palavras da língua inglesa. Ele apresenta dois tipos de  regras que causam as mudanças – regras morfológicas e regras fonológicas.  
        % Regras morfológicas são aplicadas especificamente para criar novas unidades com significados (compostos). Três regras morfológicas são usadas para  criar compostos na ASL: (1) a regra do contato; (2) a regra da seqüência  única e (3) a regra da antecipação da mão não-dominante. Será observada a  aplicação de tais regras na língua de sinais brasileira. 

        

        \item \underline{Incorporação de numeral}: nesse processo, morfemas são combinados com a configuração de mão de números para prover significado adicional ao sinal. Isso ocorre comumente para indicar variação do número de anos, dias, horas, minutos, entre outros. Por exemplo, a mudança na configuração de mão para 1, 2, 3 ou 4 na \autoref{fig:sinais-numero-meses} altera também o número de meses referido no sinal.
        
        \figura[p. 107]
            {fig:sinais-numero-meses} % Label
            {imagens/fundamentacao/sinais_numero_meses} % Path
            {height=5cm} % Size
            {Sinais UM MÊS, DOIS MESES, TRÊS MESES, e QUATRO MESES} % Caption
            {quadros-2004-estudos-linguisticos} % Citation



        \item \underline{Incorporação de negação}: é possível dar origem à contraparte negativa de um sinal pela alteração de um de seus parâmetros (geralmente o movimento), conforme exemplifica a \autoref{fig:sinais-saber-naosaber}.
        
        \figura[p. 111]
            {fig:sinais-saber-naosaber} % Label
            {imagens/fundamentacao/sinais_saber_naosaber} % Path
            {height=5cm} % Size
            {Sinal SABER e sua negação NÃO SABER} % Caption
            {quadros-2004-estudos-linguisticos} % Citation

        Pode-se também marcar essa negação pela adição de uma expressão facial, sem que haja alteração nos parâmetros do sinal (vide \autoref{fig:sinais-conhecer-naoconhecer}).

        \figura[p. 111]
            {fig:sinais-conhecer-naoconhecer} % Label
            {imagens/fundamentacao/sinais_conhecer_naoconhecer} % Path
            {height=5cm} % Size
            {Sinal CONHECER e sua negação NÃO CONHECER} % Caption
            {quadros-2004-estudos-linguisticos} % Citation

        % Há também outro processo produtivo na língua de sinais brasileira que é a  incorporação da negação. Há alguns sinais que podem incorporar a negação conforme identificado por Ferreira-Brito (1995). A autora menciona que “através de  vários processos, o item a ser negado sofre alteração em um dos parâmetros,  especialmente o parâmetro Movimento, acarretando, assim, o aparecimento de  um item de estrutura ‘fonético-fonológico’ diferente daquele que é a sua base, ou  seja, o aparecimento de sua contraparte negativa” (Ferreira-Brito, 1995, p. 77).  Alguns dos exemplos elencados pela autora foram os seguintes: 

        % Além da incorporação da negação nos sinais, há a negação de forma  marcada através da expressão facial incorporada ao sinal sem alteração de  nenhum dos parâmetros. Este caso é relacionado por Ferreira-Brito (1995)  como negação supra-segmental. 

    \end{enumerate}


    \item \textbf{Flexão}: consiste na adição de informação gramatical a uma palavra existente. Segundo \citeonline{klima-1979-signs-of-language}, entre os processos flexionais na língua de sinais, podem-se enumerar os seguintes:

    % envolve o estudo dos processos que acrescentam informação gramatical à palavra que já existe.


    \begin{enumerate}
        \item \underline{Índice} (ou dêixis\footnote{
            Dêixis: Palavra grega que significa `apontar' ou `indicar'. Descreve uma forma particular de estabelecer nominais no espaço, os quais são utilizados pelos verbos com concordância como parte de sua flexão~\cite{quadros-2004-estudos-linguisticos}.
        }): modificam a referência de pessoa para os verbos. Os referentes são introduzidos apontando-se para diferentes pontos no espaço à frente do sinalizador, os quais são incorporados pelo movimento dos verbos para indicar ação partindo de um desses referentes para o outro. Observe na \autoref{fig:deixis-eu-voce-elea} o estabelecimento de três pessoas no espaço e, na \autoref{fig:verbo-entregar-deixis}, a flexão do verbo para refletir a ação envolvendo pares desses indivíduos.

        \figura[p. 113]
            {fig:deixis-eu-voce-elea} % Label
            {imagens/fundamentacao/deixis_eu_voce_elea} % Path
            {height=5cm} % Size
            {Pontos estabelecidos no espaço para EU, VOCÊ, e ELE/ELA} % Caption
            {quadros-2004-estudos-linguisticos} % Citation

        \figura[p. 114-115]
            {fig:verbo-entregar-deixis} % Label
            {imagens/fundamentacao/verbo_entregar_deixis} % Path
            {height=5cm} % Size
            {Verbo ENTREGAR flexionado entre diferentes pessoas previamente estabelecidas no espaço} % Caption
            {quadros-2004-estudos-linguisticos} % Citation

        
        \item \underline{Reciprocidade}: indicam uma ação ou relação que ocorre de forma mútua. É representada por meio da duplicação do sinal, que é articulada simultaneamente (vide \autoref{fig:sinal-olhar-reciproco}).
        
        \figura[p. 122]
            {fig:sinal-olhar-reciproco} % Label
            {imagens/fundamentacao/sinal_olhar_reciproco} % Path
            {height=4cm} % Size
            {Sinal OLHAR (recíproco)} % Caption
            {quadros-2004-estudos-linguisticos} % Citation
        
        
        \item \underline{Número}: indicam variação entre singular e plural, a qual é marcada através da repetição do sinal (vide \autoref{fig:plural-anos-anteriores}). 
        
        \figura[p. 119]
            {fig:plural-anos-anteriores} % Label
            {imagens/fundamentacao/plural_anos_anteriores} % Path
            {height=5cm} % Size
            {Singular e plural para ANO ANTERIOR} % Caption
            {quadros-2004-estudos-linguisticos} % Citation
            
        Além disso, essa flexão também pode indicar a existência de múltiplos referente no discurso, na qual o verbo direciona-se para dois, três ou mais pontos estabelecidos no espaço ou para uma referência generalizada incluindo todos esses referentes (vide \autoref{fig:verbo-entregar-flexao-numero}).

        \figura[p. 120]
            {fig:verbo-entregar-flexao-numero} % Label
            {imagens/fundamentacao/verbo_entregar_flexao_numero} % Path
            {height=5cm} % Size
            {Verbo ENTREGAR flexionado em número para um, três e múltiplos referentes} % Caption
            {quadros-2004-estudos-linguisticos} % Citation

        \item \underline{Aspecto distributivo}: indicam ações distribuídas entre diferentes refentes envolvidos no discurso. %, como para ``cada indivíduo'', ``determinados indivíduos'', ``indivíduos não especificados'', ``por toda parte'', e ``em volta''. 
        Entre as formas de marcação desse aspecto podemos enumerar a distributiva exaustiva (onde uma ação é repetida exaustivamente), a distributiva específica (onde a ação distribui-se para referentes específicos, como na \autoref{fig:verbo-entregar-flexao-numero} (ao centro)) e distributiva não-específica (onde os refentes são generalizados ou indeterminados, como na \autoref{fig:verbo-entregar-flexao-numero} (à direita)).
        
        \item \underline{Aspecto temporal}: indicam a forma com que uma ação se relaciona com o tempo, %, como ``por um longo período'', ``regularmente'', ``continuamente'', ``incessantemente'', ``repetidamente'', ou ``habitualmente''.
        que pode ser: incessante, ininterrupta, habitual (para ação recorrente), contínua, ou duradoura (para ações de caráter durativo ou permanente) (vide na \autoref{fig:verbo-cuidar-flexao-temporal} exemplos de algumas dessas flexões aplicadas ao verbo CUIDAR).

        \figura[p. 123]
            {fig:verbo-cuidar-flexao-temporal} % Label
            {imagens/fundamentacao/verbo_cuidar_flexao_temporal} % Path
            {height=5cm} % Size
            {Verbo CUIDAR flexionado em aspecto temporal incessante (esquerda), ininterrupto (centro) e habitual (direita)} % Caption
            {quadros-2004-estudos-linguisticos} % Citation

        \item \underline{Foco temporal}: refletem distinções como ``começando a'', ``cada vez mais'', ``gradualmente'', ``progressivamente'', ou ``resultando em''.
        
        \item \underline{Modo}: denotam o modo com que uma ação é realizada, como por exemplo ``com facilidade''.
        
        \item \underline{Grau}: indicam variação de grau, como em ``um pouquinho'', ``aproximadamente'', ``muito'', ou ``demais''.
        Observe a imagem \autoref{fig:sinais-lindo-lindinho-lindissimo}, que mostra o exemplo do sinal LINDO flexionado utilizando expressões faciais para transmitir a ideia de ``pouco'' e ``muito'' lindo.
    
        \figura[p. 65]
            {fig:sinais-lindo-lindinho-lindissimo} % Label
            {imagens/fundamentacao/sinais_lindo_lindinho_lindissimo} % Path
            {height=5.5cm} % Size
            {Sinais LINDO (centro), LINDINHO (esquerda) e LINDÍSSIMO (direita) da \acrshort{libras}} % Caption
            {pereira-2011-conhecimento-alem-sinais} % Citation

    \end{enumerate}


    % Pessoa (deixis): flexão que muda as referências pessoais no verbo.  
    % Número: flexão que indica o singular, o dual, o trial e o múltiplo.  
    % Grau: apresenta distinções para ‘menor’, ‘mais próximo’, ‘muito’, etc. 
    % Modo: apresenta distinções, tais como os graus de facilidade.  
    % Reciprocidade: indica relação ou ação mútua.  
    % Foco temporal: indica aspectos temporais, tais como ‘início’, ‘aumento’,  ‘graduação’, ‘progresso’, ‘consequência’, etc.  
    % Aspecto temporal: indica distinções de tempo, tais como ‘há muito tempo’, ‘por muito tempo’, ‘regularmente’, ‘continuamente’, ‘incessantemente’, ‘repetidamente’, ‘caracteristicamente’, etc.  
    % Aspecto distributivo: indica distinções, tais como ‘cada’, ‘alguns especificados’, ‘alguns não-especificados’, ‘para todos’, etc. 

\end{enumerate}



%Na língua de sinais, não há sinais para afixos como \textit{in-}, \textit{-idade}, entre outros, que alteram o significado das palavras. Em vez disso, são utilizadas expressões não-manuais, alterações nos parâmetros fonológicos ou sinais diferentes para indicar tempo, grau, intensidade, pluralidade, aspecto e muito mais. 

% Observe na \autoref{fig:sinais-sentar-cadeira} os sinais SENTAR e CADEIRA da \acrshort{libras}. Eles possuem a mesma configuração de mãos, locação e orientação das palmas das mãos. No entanto, o movimento é o parâmetro que os diferencia: é mais longo em SENTAR e mais curto e repetido em CADEIRA: 

%Em ASL, não há sinais para afixos como “en”, “ing”, “ly” etc. para alterar o significado das palavras. 
%Em vez disso, o ASL usa marcadores não manuais, alterações nos parâmetros e outros sinais para indicar tempo, grau, intensidade, pluralidade, aspecto e muito mais.

% \figura[p. 70]
%     {fig:sinais-sentar-cadeira} % Label
%     {imagens/fundamentacao/sinais_sentar_cadeira} % Path
%     {width=0.5\linewidth} % Size
%     {Sinais SENTAR e CADEIRA da \acrshort{libras}} % Caption
%     {pereira-2011-conhecimento-alem-sinais} % Citation


% Algo semelhante ocorre com os sinais OUVIR e OUVINTE (vide \autoref{fig:sinais-ouvir-ouvinte}): o movimento de fechar as mãos próximo ao ouvido é mais curto e repetido em OUVINTE, enquanto que no sinal OUVIR, onde a mão se desloca em direção ao ouvido.

% \figura[p. 71]
%     {fig:sinais-ouvir-ouvinte} % Label
%     {imagens/fundamentacao/sinais_ouvir_ouvinte} % Path
%     {width=0.5\linewidth} % Size
%     {Sinais OUVIR e OUVINTE da \acrshort{libras}} % Caption
%     {pereira-2011-conhecimento-alem-sinais} % Citation


% A morfologia é o ramo da linguística que estuda como as palavras são formadas a partir de partes componentes. Um morfema é geralmente descrito como um par consistente de forma (por exemplo, uma sequência de sons ou uma combinação de formato de mão, localização e movimento) e significado, mas existem morfemas que mudam de forma em diferentes contextos, bem como aqueles que não mudam. t parecem ter um significado consistente. \cite{hill-2019-sign-languages}

% \cite{quadros-2004-estudos-linguisticos}
% Assim como as palavras em todas as línguas humanas, mas diferentemente dos gestos, os sinais pertencem a categorias lexicais ou a classes de palavras  tais como nome, verbo, adjetivo, advérbio, etc. As línguas de sinais têm um  léxico e um sistema de criação de novos sinais em que as unidades mínimas  com significado (morfemas) são combinadas. Entretanto, as línguas de sinais  diferem das línguas orais no tipo de processos combinatórios que freqüentemente cria palavras morfologicamente complexas. Para as línguas orais, palavras complexas são muitas vezes formadas pela adição de um prefixo ou sufixo  a uma raiz. Nas línguas de sinais, essas formas resultam freqüentemente de  processos não-concatenativos em que uma raiz é enriquecida com vários movimentos e contornos no espaço de sinalização (Klima e Bellugi, 1979). 


% TODO: revisar este prefácio:
% \citeonline{jay-2011-dont-just-sign} descreve a morfologia da língua de sinais em termos dos seguintes elementos:
 
% \begin{enumerate}
%     \item \textbf{Flexões (advérbios)}: advérbios podem modificar adjetivos, verbos, ou outros advérbios para indicar tempo, local, forma, causa, ou intensidade. Como não há sinais específicos na língua de sinais para os advérbios, eles são criados através da adição de flexões, as quais podem incluir: o uso de expressões faciais mais intensas, o aumento ou diminuição da velocidade de sinalização, o uso de movimentos maiores ou mais largos, a sinalização mais detalhada de um sinal, o aceno mais rápido ou vagaroso com a cabeça, entre outros.
    
%     Sinais diferentes são normalmente flexionados de maneiras diferentes. A imagem \autoref{fig:sinais-lindo-lindinho-lindissimo} mostra o exemplo do sinal LINDO que é flexionado utilizando expressões faciais para transmitir a ideia de ``pouco'' e ``muito'' lindo, respectivamente nos sinais LINDINHO e LINDÍSSIMO.
    
%     \figura[p. 65]
%         {fig:sinais-lindo-lindinho-lindissimo} % Label
%         {imagens/fundamentacao/sinais_lindo_lindinho_lindissimo} % Path
%         {width=0.5\linewidth} % Size
%         {Sinais LINDO, LINDINHO e LINDÍSSIMO da \acrshort{libras}} % Caption
%         {pereira-2011-conhecimento-alem-sinais} % Citation

%     \item \textbf{Pares substantivo-verbo}: pares substantivo-verbo são sinais que utilizam o mesmo formato de mão, locação e orientação, mas usam movimentos distintos para  diferenciar entre substantivo e verbo. Um verbo sinalizado geralmente tem um movimento único e contínuo, enquanto um substantivo geralmente tem um movimento duplo.  
%     Um exemplo disso é ilustrado para os sinais SENTAR e CADEIRA na \autoref{fig:sinais-sentar-cadeira}.
    

%     \item \textbf{Classificadores}: línguas faladas são lineares, ou seja, são expressas uma palavra após a outra. As línguas de sinais, no entanto, são espaciais e expressas no espaço ao redor do sinalizador. Nesse contexto, os classificadores criam profundidade e adicionam clareza, movimento e detalhes às conversas e explicações.

%     Faz mais sentido criar pessoas, animais ou objetos imaginários no espaço de enunciação e mostrar o que acontece com eles em vez de explicar cada palavra de forma linear como você faria oralmente. No entanto, assim como se faz com pronomes na língua falada, não se pode utilizar um classificador em uma frase até explicar o que o ele representa -- ou seja, é necessário primeiro estabelecer o substantivo.

%     A \autoref{fig:sinais-pessoas-andar-cair} ilustra um classificador comumente utilizado para referir-se a DUAS PESSOAS. Combinado ao movimento de ANDAR, é interpretado como DUAS PESSOAS ANDAM; combinado ao movimento de CAIR, é interpretado como DUAS PESSOAS CAEM.

%     \figura[p. 83]
%         {fig:sinais-pessoas-andar-cair} % Label
%         {imagens/fundamentacao/sinais_pessoas_andar_cair} % Path
%         {width=0.5\linewidth} % Size
%         {Classificador utilizado para DUAS PESSOAS, combinado aos movimentos de ANDAR e CAIR} % Caption
%         {pereira-2011-conhecimento-alem-sinais} % Citation

%     Há um número infinito de classificadores e pode-se combinar praticamente qualquer forma de mão com qualquer movimento e locação para criar um classificador.
%     Alguns dos diferentes tipos de classificadores incluem: integrais (que representam um objeto inteiro), de superfícies, de instrumentos, de proporção (tamanho ou profundidade), de quantidade, de forma, de localização, de gestos, de partes do corpo (realizando uma ação), de verbos (que mostram como uma ação é realizada), ou de plural (que representam múltiplos itens).


%     % Línguas faladas como o inglês são lineares — são expressas uma palavra após a outra. ASL, no entanto, é uma linguagem espacial e é expressa no espaço ao seu redor. Os classificadores criam profundidade e adicionam clareza, movimento e detalhes às conversas e explicações.
%     % Em ASL, faz muito mais sentido criar pessoas, animais ou objetos imaginários em seu espaço de sinalização e mostrar o que acontece com eles em vez de explicar cada palavra de forma linear como você faria em inglês.
%     % As formas de mão e os movimentos dos classificadores podem representar pessoas, animais, objetos, etc. e mostrar movimentos, formas, ações e locais. Eles podem mostrar uma pessoa andando, um animal mastigando, alguém dirigindo, um carro passando pelas montanhas — praticamente qualquer coisa!
%     % Em uma frase, um classificador é muito semelhante a um pronome. Você não pode usar um classificador em uma frase até explicar o que o classificador representa. Não são palavras isoladas. Você precisa estabelecer o substantivo antes de aplicar o classificador.
%     % Um exemplo de classificador seria mostrar uma pessoa andando. Você primeiro tem que estabelecer essa pessoa como um referente no contexto e apontar para o referente, e então você pode pegar o dedo indicador de sua mão dominante (o classificador CL: 1) e movê-lo pelo seu espaço de sinalização. O que quer que você faça com este classificador é o que a pessoa está fazendo. Você também pode flexionar o sinal para um significado adicional. Quanto mais rápido ou mais devagar você mover esse classificador, mostra o quão rápido ou lento a pessoa está realizando uma ação. Você também pode usar marcadores não manuais para mostrar como a pessoa está se sentindo ao fazê-lo. E como a maioria das formas de mão do classificador representa uma pessoa ou objeto inteiro, você pode combinar classificadores como CL: 1(pessoa) com outro classificador como CL: 3(carro) para mostrar uma pessoa e um carro e suas localizações em relação um ao outro. Aqui estão dois exemplos de uso de classificadores com as duas mãos ao mesmo tempo:
%     % [...]
%     % Há um número infinito de classificadores que você pode usar. Você pode combinar quase qualquer forma de mão com qualquer movimento e localização para criar um classificador.


%     \item \textbf{Verbos}: há três tipos principais de verbos na língua de sinais: verbos simples, que não se flexionam em pessoa e número, e não incorporam afixos locativos (vide exemplos na \autoref{fig:verbos-simples});
%     verbos direcionais, que flexionam-se em pessoa, número e aspecto, mas não incorporam afixos locativos (vide \autoref{fig:verbos-direcionais}); e
%     verbos espaciais, que incorporam afixos locativos (vide \autoref{fig:verbos-espaciais}).

%     \figura[p. 76]
%         {fig:verbos-simples} % Label
%         {imagens/fundamentacao/verbos_simples} % Path
%         {width=0.5\linewidth} % Size
%         {Verbos simples DIRIGIR, COMER, e PARECER} % Caption
%         {pereira-2011-conhecimento-alem-sinais} % Citation

%     \figura[p. 77]
%         {fig:verbos-direcionais} % Label
%         {imagens/fundamentacao/verbos_direcionais} % Path
%         {width=0.5\linewidth} % Size
%         {Verbos direcionais PERGUNTAR e RESPONDER} % Caption
%         {pereira-2011-conhecimento-alem-sinais} % Citation

%     \figura[p. 77]
%         {fig:verbos-espaciais} % Label
%         {imagens/fundamentacao/verbos_espaciais} % Path
%         {width=0.5\linewidth} % Size
%         {Verbos espaciais IR, CHEGAR, e POR} % Caption
%         {pereira-2011-conhecimento-alem-sinais} % Citation

%     \item \textbf{Tempo}: como não há afixos na língua de sinais para alterar o tempo verbal, ele é comunicado adicionando-se sinais de tempo e flexões desses sinais. Dessa forma, para expressar que ``eu \underline{fui} ao cinema'', seria utilizada a sequência de sinais ``ONTEM EU IR CINEMA''.    
%     Sinas de tempo que indicam futuro movem-se para frente; os que indicam passado movem-se para trás; e os que indicam presente são sinalizados à frente do corpo do sinalizador (vide \autoref{fig:sinais-tempo-semana}).
    
%     \figura[p. 101]
%         {fig:sinais-tempo-semana} % Label
%         {imagens/fundamentacao/sinais_tempo_semana} % Path
%         {width=0.5\linewidth} % Size
%         {Sinais de tempo para SEMANA PASSADA (esquerda), ESTA SEMANA (centro), e PRÓXIMA SEMANA (direita)} % Caption
%         {jay-2011-dont-just-sign} % Citation

%     Numerais podem ser incorporados ao sinais de tempo para especificar número (ou quantidade) de semanas, horas, minutos, anos, meses, dias, entre outros. Observe na \autoref{fig:sinais-cada-tres-semanas} como o sinal SEMANA foi flexionado para expressar TRÊS SEMANAS.

%     Além disso, outras flexões podem ser adicionadas a esses sinais para transmitir significado de duração prolongada (como ``o dia inteiro'' ou ``a noite inteira'', por exemplo) ou de eventos que se repetem numa base regular (como ``a cada três semanas'', ``toda segunda-feira'', ou ``todas as manhãs'', por exemplo). Nesse caso, a flexão pode ocorrer por meio de alterações no movimento ou nas expressões faciais. A \autoref{fig:sinais-dia-inteiro} mostra o sinal DIA sendo flexionado para DIA INTEIRO; a \autoref{fig:sinais-cada-tres-semanas}, por sua vez, ilustra a repetição do sinal TRÊS SEMANAS para expressar a regularidade de A CADA TRÊS SEMANAS.
    
%     \figura[p. 106]
%         {fig:sinais-dia-inteiro} % Label
%         {imagens/fundamentacao/sinais_dia_inteiro} % Path
%         {width=0.50\linewidth} % Size
%         {Sinal de tempo DIA (esquerda) e flexão para expressar duração de DIA INTEIRO (direita)} % Caption
%         {jay-2011-dont-just-sign} % Citation

%     \figura[p. 106]
%         {fig:sinais-cada-tres-semanas} % Label
%         {imagens/fundamentacao/sinais_cada_tres_semanas} % Path
%         {width=0.50\linewidth} % Size
%         {Sinal de tempo TRÊS SEMANAS (esquerda) e flexão para expressar regularidade de A CADA TRÊS SEMANAS (direita)} % Caption
%         {jay-2011-dont-just-sign} % Citation

% \end{enumerate}







% ---------------------------------------------------------------------
% \cite{hill-2019-sign-languages}
% Morphology
% Morphology is the branch of linguistics that studies how words are formed from component parts. A morpheme is generally described as a consistent pairing of form (e.g., a sequence of sounds or a combination of handshape, location, and movement) and meaning, but there are morphemes that change their form in different contexts as well as those that don’t seem to have a consistent meaning.


% ---------------------------------------------------------------------
% \cite{pereira-2011-conhecimento-alem-sinais}
% Aspectos morfológicos (pág 70)
% Como a língua portuguesa, a Libras conta com um léxico e  com recursos que permitem a criação de novos sinais . Contudo, diferentemente das línguas orais, em que palavras complexas  são, muitas vezes, formadas pela adição de um prefixo ou sufixo a  uma raiz, nas línguas de sinais a raiz é frequentemente enriquecida com vários movimentos e contornos no espaço de sinalização  (Klima e Bellugi, 1979).  
% Um processo bastante comum na Libras para a criação de novos sinais é o que deriva nomes de verbos, e vice-versa, por meio  da mudança no movimento . O movimento dos nomes repete e  encurta o movimento dos verbos (Quadros e Karnopp, 2004) . 
% [imagem]
% Processo semelhante é observado em ouvir e ouvinTe . O  movimento de fechar as mãos próximo ao ouvido é mais curto e  repetido em ouvinTe 
% [imagem]
% Outro processo bastante usado na Libras, na criação de novos  sinais, é a composição . Nesse processo, dois sinais se combinam,  dando origem a um novo sinal, como se pode observar em eSCoLA  e igrejA 
% [imagem]
% O sinal de eSCoLA é composto pelos sinais de CASA e eSTuDAr,  enquanto o de igrejA é composto pelo sinais de CASA e Cruz 

% A criação de novos sinais na Libras pode ser obtida, ainda, por  meio da incorporação de um argumento, de um numeral ou de  uma negação.  A incorporação de argumento é muito frequente na Libras por  causa das características visuais e espaciais da língua . O sinal de  LAvAr, por exemplo, varia de acordo com o objeto que está, foi  ou será lavado .
% [imagem]


% A incorporação de um numeral caracteriza-se pela mudança na  configuração de mão do sinal para expressar a quantidade . Assim,  por exemplo, pela mudança na configuração de mão, de 1 para 2  ou para 3, o número de meses, dias ou horas referidos muda . A  localização, a orientação e os traços não manuais permanecem os  mesmos (Quadros e Karnopp, 2004) 
% [imagem]

% A incorporação da negação é outro processo bastante produtivo na Libras e pode dar-se pela alteração do movimento do sinal, caracterizada por mudança de direção, para fora, na maioria  das vezes, com a palma da mão também para fora (Brito, 1995) .  Cabe lembrar que, nos verbos, as formas negativas são acompanhadas de meneio negativo de cabeça e expressão facial de nega-  ção, como se pode observar nos exemplos a seguir.


% Categorias gramaticais (pág 76)
% Como a língua portuguesa, a Libras organiza seus sinais em  classes, como substantivos, verbos, pronomes, advérbios, adjetivos e numerais, entre outras. Serão consideradas aqui as categorias  que apresentam especificidades na Libras decorrentes principalmente do uso do espaço .

% Verbos  
% Os verbos na Libras estão basicamente divididos em três classes (Quadros e Karnopp, 2004, pp . 116-118): simples, direcionais e espaciais .  
% • verbos simples — são verbos que não se flexionam em pessoa e número e não incorporam afixos locativos 
% • verbos direcionais (com concordância) — são verbos  que se flexionam em pessoa, número e aspecto, mas não incorporam afixos locativos 
% • verbos espaciais — são verbos que têm afixos locativos 

% Adjetivos  
% Os adjetivos são sinais que formam uma classe específica na  Libras e estão sempre na forma neutra, não recebendo marcação  para gênero (masculino e feminino) nem para número (singular e plural) . Muitos adjetivos, por serem descritivos e classificadores, expressam a qualidade do objeto, desenhando-a no ar ou  mostrando-a no objeto ou no corpo do emissor (Felipe, 2001) 

% Pronomes  
% • Pessoais - são expressos por meio dos sinais de apontar com o dedo indicador.
% No singular, o sinal para todas as pessoas é o  mesmo; o que difere é a orientação da mão . No  plural, o formato do numeral — dois, três, quatro, até nove — apontando para pessoas ou lugares a quem se faz referência é interpretado como  nós, vocês ou eles dois, três, quatro, até nove.
% • Possessivos - são expressos com a configuração de  mãos em P e seguem os mesmos princípios da expressão dos pronomes pessoais na Libras 

% Classificadores  
% Os classificadores são formas que, substituindo o nome que as  precedem, podem vir junto com o verbo para classificar o sujeito ou  o objeto que está ligado à ação do verbo (Felipe, 2001) . Para Brito  (1995), os classificadores funcionam, em uma sentença, como partes dos verbos de movimento ou de localização . O sistema de classificadores fornece um campo de representações de categoriais que  revelam o tamanho e a forma de um objeto, a animação corporal  de um personagem ou como um instrumento é manipulado (Rayman, 1999) . Morgan (2005) refere que, nas narrativas, um classificador é, muitas vezes, usado para manter a referência a objeto ou  personagem previamente mencionado por meio de um sinal .  
% Em relação às formas dos classificadores, Brito (1995) refere  que a configuração de mão em V pode ser usada para se referir a pessoas, animais ou objetos; em C, para qualquer tipo de objeto  cilíndrico, e em B, para superfícies planas, por exemplo .

% Flexão verbal  
% A flexão de número nos verbos refere-se à distinção para um,  dois, três ou mais referentes . Assim, o verbo que apresenta concordância direciona-se para um, dois ou três pontos estabelecidos  no espaço ou para uma referência generalizada incluindo todos os  referentes integrantes do discurso (Quadros e Karnopp, 2004) 
% A flexão do aspecto está relacionada com as formas e a duração dos movimentos.
% Os aspectos pontual, continuativo, durativo e iterativo são obtidos por meio de alterações do movimento e/ou da configura-  ção da mão (Brito, 1995) 
%---
%A Libras apresenta, ainda, em suas formas verbais, a marca de  tempo de forma diferente de como acontece na língua portuguesa . O tempo é marcado por meio de advérbios de tempo que  indicam se a ação está ocorrendo no presente (hoje, agora), se  ocorreu no passado (ontem, anteontem), ou se ocorrerá no futuro (amanhã, semana que vem) . Para um tempo verbal indefinido,  usam-se os sinais passado e futuro (Felipe, 2001) . Para expressar  a ideia de passado, o sinal de já, antecedendo o verbo, ou o meneio afirmativo com a cabeça, concomitante à realização do sinal,  são muito utilizados 

% Flexão nominal  
% Diferentemente da língua portuguesa na modalidade oral, que  apresenta flexão de gênero modificando os nomes, a indicação de  sexo na Libras é marcada por um sinal que indica marca de gênero feminino ou masculino, antecedendo o nome .  
% Nos substantivos, a flexão de plural é obtida, na maioria das  vezes, pela repetição do sinal, pela anteposição ou posposição de  sinais referentes aos números, ou pelo movimento semicircular, que deve abranger as pessoas ou os objetos envolvidos (Brito, 1995) 


% ---------------------------------------------------------------------
% \cite{quadros-2004-estudos-linguisticos}

% Morfologia das línguas de sinais
% Morfologia é o estudo da estrutura interna das palavras ou dos sinais,  assim como das regras que determinam a formação das palavras. A palavra  morfema deriva do grego morphé, que significa forma. Os morfemas são as  unidades mínimas de significado.  
% Alguns morfemas por si só constituem palavras, outros nunca formam  palavras, apenas constituindo partes de palavras. Desta forma, têm-se os  morfemas presos que, em geral, são os sufixos e os prefixos, uma vez que não  podem ocorrer isoladamente, e os morfemas livres que constituem palavras.  
% Mas, na medida em que se pode formar palavras a partir de outras palavras, é importante reconhecer que as palavras podem ser unidades complexas, constituídas de mais de um elemento. 
% ---
% Assim como as palavras em todas as línguas humanas, mas diferentemente dos gestos, os sinais pertencem a categorias lexicais ou a classes de palavras  tais como nome, verbo, adjetivo, advérbio, etc. As línguas de sinais têm um  léxico e um sistema de criação de novos sinais em que as unidades mínimas  com significado (morfemas) são combinadas. Entretanto, as línguas de sinais  diferem das línguas orais no tipo de processos combinatórios que freqüentemente cria palavras morfologicamente complexas. Para as línguas orais, palavras complexas são muitas vezes formadas pela adição de um prefixo ou sufixo  a uma raiz. Nas línguas de sinais, essas formas resultam freqüentemente de  processos não-concatenativos em que uma raiz é enriquecida com vários movimentos e contornos no espaço de sinalização (Klima e Bellugi, 1979). 

% [...]


% --------------------------------------------------------
% \cite{jay-2011-dont-just-sign}
% Morphology
% In language, morphology is the study of the forms and formations of words. A morpheme is the smallest indivisible unit of syntax that retains meaning.
%For example, in English, the word “threateningly” consists of four morphemes: “threat,” which is a noun; “en,” which changes the noun into a verb; “ing,” which changes it into an adjective; and “ly” which changes it into an adverb. 
% In ASL, there are no signs for affixes like “en,” “ing,” “ly,” etc. to change the meaning of words. Instead, ASL uses non-manual markers, changes in parameters, and other signs to indicate tense, degree, intensity, plurality, aspect, and more.

% • Inflection (Adverbs)
% • Noun-Verb Pairs 
% • Classifiers x
% • Verbs x
% • Time


