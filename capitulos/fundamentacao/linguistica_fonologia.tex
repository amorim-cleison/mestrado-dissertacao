\subsection{Fonologia}
\label{sec:linguistica-fonologia}

Fonologia é o estudo das menores unidades constituintes de uma língua -- denominados fonemas -- e das regras que regem sua produção. Ela objetiva compreender essas unidades, bem como elas são articuladas, para compor unidades maiores com significado, como as palavras, de acordo com \citeonline{quadros-2004-estudos-linguisticos,hill-2019-sign-languages}. 


\citeonline{stokoe-1960-sl-structure} definiu inicialmente três tipos de fonemas (ou parâmetros) para a língua de sinais, os quais são articulados simultaneamente para compor um sinal: configuração de mão, locação e movimento. Em \citeyear{battison-1974-phono-deletion}, \citeauthor{battison-1974-phono-deletion} introduziu um quarto parâmetro, referente à orientação da palma da mão. Posteriormente, estudos como o de \citeonline{baker-padden-1978-nonmanual-components}, adicionaram as expressões não-manuais, como expressões faciais, movimentos da boca e direção do olhar.

Dessa forma, atualmente a fonologia da língua de sinais compreende que os sinais são compostos pelos seguintes parâmetros~\cite{stewart-2021-barrons-asl,jay-2011-dont-just-sign,quadros-2004-estudos-linguisticos}:

\begin{enumerate}
   \item \textbf{Configuração de mão}: configuração assumida pelas mãos ao produzir o sinal, a qual pode permanecer estática ou variar durante a articulação do sinal. É possível que as mãos apresentem configurações distintas nesse processo. A \autoref{fig:config-mao-comuns-asl} ilustra algumas configurações utilizadas na \acrshort{asl}.
   
    \figura[p. 72]
        {fig:config-mao-comuns-asl} % Label
        {capitulos/fundamentacao/imagens/configuracoes_mao_asl} % Path
        {height=8cm} % Size
        {Exemplos de configurações de mãos utilizadas na \acrshort{asl}} % Caption
        {jay-2011-dont-just-sign} % Citation


    \item \textbf{Orientação}: direção apontada pelas palmas das mãos na articulação do sinal. Por exemplo, as palmas podem estar voltadas para o corpo, para fora, para o chão, para cima, entre outras ilustradas na \autoref{fig:orientacoes}. Cada uma das palmas pode também assumir uma orientação distinta.
    
    \figura[p. 59]
        {fig:orientacoes} % Label
        {capitulos/fundamentacao/imagens/orientacoes} % Path
        {height=8cm} % Size
        {Exemplos de orientações que podem ser assumidas pelas palmas das mãos} % Caption
        {quadros-2004-estudos-linguisticos} % Citation
    

    \item \textbf{Movimento}: corresponde à trajetória percorrida pelas mãos em relação ao corpo para articular o sinal. 
    É um parâmetro complexo que pode envolver uma ampla variedade de modos e direções, desde um sutil deslizar entre as mãos ou um movimento interno das mãos e punhos, até uma trajetória complexa desenhada no espaço, por exemplo.
    Além disso, os sinais podem envolver movimentos de uma ou de ambas as mãos.

    
    \item \textbf{Locação}: é o local onde as mãos são posicionadas dentro do espaço de enunciação para articular o sinal. O espaço de enunciação, por sua vez, é uma área que contém todos os pontos possíveis dentro do raio de alcance das mãos, como ilustra a \autoref{fig:espaco-enunciacao}. Nesse espaço, há um número limitado de locações, sendo que algumas são mais exatas -- tais como a ponta do nariz --, e outras são mais abrangentes -- como a frente do tórax. Por fim, as mãos podem permanecer fixas ou se deslocar de uma locação para outro durante a articulação de um sinal.
    
    \figura[p. 57]
        {fig:espaco-enunciacao} % Label
        {capitulos/fundamentacao/imagens/espaco_enunciacao} % Path
        {height=6cm} % Size
        {Espaço de enunciação da língua de sinais} % Caption
        {quadros-2004-estudos-linguisticos} % Citation

        
    \item \textbf{Expressões não-manuais}: consistem nas expressões faciais e movimentos corporais incorporados aos sinais para provê significado adicional. Elas desempenham duas funções essenciais: marcar construções sintáticas (como frases interrogativas, orações relativas, tópicos, concordância e foco) e diferenciar componentes lexicais (como referências específicas, referências pronominais, partículas negativas, advérbios, grau ou aspecto).
    
    De um modo geral, expressões faciais ajudam a prover mais clareza ou alterar o significado de um sinal. 
    Movimentos corporais, por sua vez, são importantes para descrever pessoas em diferentes posições ou locais, ou narrar histórias envolvendo personagens com diferentes papéis, por exemplo.
 
\end{enumerate}
