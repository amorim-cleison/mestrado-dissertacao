\subsection{Fonologia}
\label{sec:linguistica-fonologia}

Fonologia é o estudo das menores unidades constituintes de uma língua -- denominados fonemas -- e das regras que regem sua produção. Ela objetiva compreender o que são essas unidades e como elas são articuladas para compor unidades maiores com significado, como as palavras~\cite{quadros-2004-estudos-linguisticos,hill-2019-sign-languages}. 

% rules for producing a syllable,
%\cite{hill-2019-sign-languages}

% Phonology
% In language, phonology is the study of the smallest part of the language that conveys meaning. In spoken languages, like English, a phoneme is a unit of sound that conveys meaning. 
% In ASL, the smallest parts of the language, the phonemes, are handshape, movement, palm orientation, location, and facial expression. If you change any of these parameters of a sign, then you have changed the meaning of the sign.
% \cite{jay-2011-dont-just-sign}

%It is about the ways in which words are made up of pieces that are not meaningful. It is about what these pieces are and how they work together.
%\cite{hill-2019-sign-languages}

% A fonologia é o ramo da linguística que busca identificar a estrutura e a organização dos constituintes fonológicos, propondo  modelos descritivos e explanatórios. Em outras palavras, ela estuda as maneiras pelas quais as palavras são compostas a partir de unidades menores sem significado -- os fonemas --, bem como o que são essas unidades e de que forma elas funcionam em conjunto~\cite{quadros-2004-estudos-linguisticos,hill-2019-sign-languages}.


\citeonline{stokoe-1960-sl-structure} definiu inicialmente três tipos de fonemas (ou parâmetros) para a língua de sinais, os quais são articulados simultaneamente para compor um sinal: configuração de mão, locação e movimento. Em \citeyear{battison-1974-phono-deletion}, \citeauthor{battison-1974-phono-deletion} introduziu um quarto parâmetro, referente à orientação da palma da mão. Posteriormente, estudos como o de \citeonline{baker-padden-1978-nonmanual-components}, adicionaram as expressões não-manuais, como expressões faciais, movimentos da boca e direção do olhar.

Dessa forma, atualmente a fonologia da língua de sinais compreende que os sinais são compostos pelos seguintes parâmetros~\cite{stewart-2021-barrons-asl,jay-2011-dont-just-sign,quadros-2004-estudos-linguisticos}:

\begin{enumerate}
   \item \textbf{Configuração de mão}: configuração assumida pelas mãos ao produzir o sinal, a qual pode permanecer estática ou variar durante a articulação do sinal. É possível que as mãos apresentem configurações distintas nesse processo. A \autoref{fig:config-mao-comuns-asl} ilustra algumas configurações utilizadas na \acrshort{asl}.
   
    \figura[p. 72]
        {fig:config-mao-comuns-asl} % Label
        {capitulos/fundamentacao/imagens/configuracoes_mao_asl} % Path
        {height=8cm} % Size
        {Exemplos de configurações de mãos utilizadas na \acrshort{asl}.} % Caption
        {jay-2011-dont-just-sign} % Citation

    % is the shape of the hands when the sign is formed. The handshape may remain the same throughout the sign or it can change. If two hands are use to make a sign, both hands can have the same handshape or be different.

    % Esta é a forma da sua mão que é usada para criar o sinal.
    % This is the shape of your hand that is used to create the sign. \cite{jay-2011-dont-just-sign,hill-2019-sign-languages}

    %the configuration assumed by the hand while producing the sign, with one or more selected fingers in a particular position – extended, closed, curved, or bent.

    % Todos os sinais são formados usando um formato de mão específico. Se você alterar a forma de mão de um sinal, poderá alterar o significado do sinal. Por exemplo, se você mudar o formato de mão do sinal CIÊNCIA para um formato de mão B, você estaria assinando o sinal inicializado BIOLOGIA. Portanto, é importante saber como formar com precisão a(s) forma(s) de mão de cada signo.
    % All signs are formed using a specific handshape. If you change the handshape of a sign, you can change the meaning of the sign. For example, if you change the handshape of the sign SCIENCE to a B-handshape, you would be signing the initialized sign BIOLOGY. So, it is important to know how to accurately form the handshape( s) of each sign. \cite{jay-2011-dont-just-sign}



    \item \textbf{Orientação}: direção apontada pelas palmas das mãos na articulação do sinal. Por exemplo, as palmas podem estar voltadas para o corpo, para fora, para o chão, para cima, entre outras ilustradas na \autoref{fig:orientacoes}. Cada uma das palmas pode também assumir uma orientação distinta.
    
    \figura[p. 59]
        {fig:orientacoes} % Label
        {capitulos/fundamentacao/imagens/orientacoes} % Path
        {height=8cm} % Size
        {Exemplos de orientações que podem ser assumidas pelas palmas das mãos} % Caption
        {quadros-2004-estudos-linguisticos} % Citation
    
    % is the position of the hand(s) relative to the body. For example, the palms can be facing the body or away from the body, facing the ground, or facing upward. \cite{stewart-2021-barrons-asl}

    % This is the orientation of your palm when making the sign. \cite{jay-2011-dont-just-sign} 

    % The palm orientation of a sign refers to the position of the palms of your hands and the direction they are facing. If you change the palm orientation of a sign, you can change the meaning of the sign. \cite{jay-2011-dont-just-sign} 

    % the direction the palm points while producing the sign. (cleison)


    \item \textbf{Movimento}: corresponde à trajetória percorrida pelas mãos em relação ao corpo para articular o sinal. 
    É um parâmetro complexo que pode envolver uma ampla variedade de modos e direções, desde um sutil deslizar entre as mãos ou um movimento interno das mãos e punhos, até uma trajetória complexa desenhada no espaço, por exemplo.
    Além disso, os sinais podem envolver movimentos de uma ou de ambas as mãos.

    % \figura[p. 75]
    %     {fig:movimentos-maos} % Label
    %     {capitulos/fundamentacao/imagens/movimentos_maos} % Path
    %     {height=4.5cm} % Size
    %     {Os sinais podem envolver uma mão (esquerda), ambas mãos (centro), ou envolver ambas mas mover apenas uma delas (direita)} % Caption
    %     {jay-2011-dont-just-sign} % Citation

    % it is the direction in which the hand moves relative to the body. There is a variety of movements that range from a simple sliding movement [...] to a complex movement. \cite{stewart-2021-barrons-asl}
 
    % This is the movement of the handshape that makes the sign. \cite{jay-2011-dont-just-sign}

    % The movement of a sign is the action that is used to create the sign. The movement can be in a circle, up and down, forward or backward, etc. If you change the movement of a sign, you can change the meaning of a sign. \cite{jay-2011-dont-just-sign}

    % complex parameter that may involve a wide range of modes and directions, such as the internal  movements of the hands, wrist, and directional movements of the hands in the space (cleison)

    \item \textbf{Locação}: é o local onde as mãos são posicionadas dentro do espaço de enunciação para articular o sinal. O espaço de enunciação, por sua vez, é uma área que contém todos os pontos possíveis dentro do raio de alcance das mãos, como ilustra a \autoref{fig:espaco-enunciacao}. Nesse espaço, há um número limitado de locações, sendo que algumas são mais exatas -- tais como a ponta do nariz --, e outras são mais abrangentes -- como a frente do tórax. Por fim, as mãos podem permanecer fixas ou se deslocar de uma locação para outro durante a articulação de um sinal.
    
    \figura[p. 57]
        {fig:espaco-enunciacao} % Label
        {capitulos/fundamentacao/imagens/espaco_enunciacao} % Path
        {height=6cm} % Size
        {Espaço de enunciação da língua de sinais} % Caption
        {quadros-2004-estudos-linguisticos} % Citation


    % is the place in the signing space where a sign is formed. Signs can be stationary [...] or they can move from one location in the signing space to another [...]. \cite{stewart-2021-barrons-asl}

    % This is the location of the sign on or in front of your body. \cite{jay-2011-dont-just-sign}

    % The location of a sign is where you place and form the sign in your signing area. If you change the location of a sign, you can change the meaning of a sign. Some examples of locations include: • front of your body, • Your head, • Your face (forehead, eyes, temples, ear, nose, cheek, mouth, or chin), • Your neck, • Your shoulder, chest, or stomach, • Your arm or elbow, • Your waist, • Your wrist, and • Your non-dominant hand. In this case, your non-dominant hand will most likely use one of the ABCOS15 handshapes. \cite{jay-2011-dont-just-sign}

    % Signing Area (espaço de enunciação / área de articulação dos sinais)
    % Your signing area is in the shape of a pyramid starting from the top of your head, down past your shoulders, and ending in a horizontal line at your waist. Unless you are signing a formal speech for a large audience, your signs shouldn’t move outside of this area. The space in the center of the chest is called the sightline. The sightline is where you would focus your eyes on a signer. This enables you to use your peripheral vision to see the signer’s hands and facial expressions at the same time. Make sure to try to wear solid colored clothing (without designs) when signing— it is easier on the eyes.  \cite{jay-2011-dont-just-sign}

    % Na língua de sinais brasileira, assim como em outras línguas de sinais até  o momento investigadas, o espaço de enunciação é uma área que contém  todos os pontos dentro do raio de alcance das mãos em que os sinais são  articulados. 
    % Dentro desse espaço de enunciação, pode-se determinar um número finito  (limitado) de locações, sendo que algumas são mais exatas, tais como a ponta  do nariz, e outros são mais abrangentes, como a frente do tórax (Ferreira-Brito  e Langevin, 1995). \cite{quadros-2004-estudos-linguisticos}


    % the area on the body, or in the articulation space, at which or near which the sign is articulated. (cleison)

    \item \textbf{Expressões não-manuais}: consistem nas expressões faciais e movimentos corporais incorporados aos sinais para provê significado adicional. Elas desempenham duas funções essenciais: marcar construções sintáticas (como frases interrogativas, orações relativas, tópicos, concordância e foco) e diferenciar componentes lexicais (como referências específicas, referências pronominais, partículas negativas, advérbios, grau ou aspecto).
    
    De um modo geral, expressões faciais ajudam a prover mais clareza ou alterar o significado de um sinal. 
    Movimentos corporais, por sua vez, são importantes para descrever pessoas em diferentes posições ou locais, ou narrar histórias envolvendo personagens com diferentes papéis, por exemplo.
 
    % \figura[p. 81]
    %     {fig:expressoes-faciais-questoes} % Label
    %     {capitulos/fundamentacao/imagens/expressoes_faciais_questoes} % Path
    %     {height=4.5cm} % Size
    %     {Expressões faciais utilizadas nas perguntas do tipo ``sim/não'' (esquerda) e nos demais tipos (direita) na \acrshort{asl}} % Caption
    %     {jay-2011-dont-just-sign} % Citation

    
    % Body Language/ Role Shifting 
    % Body language is also one of the many non-manual markers. You would use body language for things like role shifting. Role shifting is very much fun in ASL. It’s like acting. 
    % Role shifting is when you take on the “role” of another person and show what that person said, did, or felt. Essentially, you can sign a whole story using role shifting.
    % You need to be sure to include the appropriate facial expressions and other non-manual markers when role playing each person. For example, if you are showing a conversation between a parent and a child, you would take on the characteristics of the parent and gaze downward when you shift one way, and take on the characteristics of the child and gaze upward when you shift the other way. 
    % This is also essential if you are showing people in different positions or locations. You would need to be able to show if someone is standing or sitting, laying down or kneeling, etc. For example, because you can’t turn and sign, you can show that you are talking to someone sitting behind you in a car by using eye gaze and pointing to their location before taking on their characteristics.
    % \cite{jay-2011-dont-just-sign}



    
    % Nonmanual markers add to signs to create meaning. They consist of various facial expressions, head tilting, shoulder movement, and mouth movements. With a non-manual marker, the meaning of the sign can change completely. \cite{stewart-2021-barrons-asl}

    % This is the various facial expressions or body movements that are used to create meaning. \cite{jay-2011-dont-just-sign}

    % Non-Manual Markers are very important in American Sign Language. They consist of the various facial expressions and body movements that are added to signs to create meaning. Non-manual markers can be facial expressions, head shakes, head nods, head tilts, shoulder shrugs, etc. 
    % And not only do non-manual markers have a very important role in ASL grammar, but if you do not use any non-manual markers or facial expressions, your audience may not understand what you are signing, or worse, they may get bored very quickly. In English, this would be like speaking in a monotone voice. \cite{jay-2011-dont-just-sign}

    % Facial Expressions 
    % Facial expressions are the non-manual markers that refer only to the expressions on your face. The meaning of your sign can be affected by the type of facial expression you use while signing it.
    % Facial expressions can also determine what type of question you are asking. If you raise your eyebrows while asking a question, you are asking a yes or no question. If you lower your eyebrows while asking a question, you are asking a question that requires more than a yes or no answer (generally a “wh” word question).
    % Facial expressions also add clarity to what you mean when you are signing. Some signs even require a certain facial expression in order to sign them. For example, the only difference between the signs LATE and NOT-YET is the facial expression. NOT-YET is signed with your tongue hanging out slightly over your bottom teeth. Without this facial expression, the meaning of the sign changes. And there are over 100 mouth movements in ASL that are used to convey an adverb, adjective, or another more descriptive meaning when signing certain ASL words.

    % Body Language/ Role Shifting 
    % Body language is also one of the many non-manual markers. You would use body language for things like role shifting. Role shifting is very much fun in ASL. It’s like acting. 
    % Role shifting is when you take on the “role” of another person and show what that person said, did, or felt. Essentially, you can sign a whole story using role shifting.
    % You need to be sure to include the appropriate facial expressions and other non-manual markers when role playing each person. For example, if you are showing a conversation between a parent and a child, you would take on the characteristics of the parent and gaze downward when you shift one way, and take on the characteristics of the child and gaze upward when you shift the other way. 
    % This is also essential if you are showing people in different positions or locations. You would need to be able to show if someone is standing or sitting, laying down or kneeling, etc. For example, because you can’t turn and sign, you can show that you are talking to someone sitting behind you in a car by using eye gaze and pointing to their location before taking on their characteristics.
    % \cite{jay-2011-dont-just-sign}
    

    % movements performed by the face, eyes, head, and trunk. If we consider a more granular level, this might also include mouth movements, eyebrows, cheeks, shoulders, and others. They perform two essential functions: 1) marking syntactic constructions (interrogative sentences, relative clauses, topicalizations, agreement, and focus) and 2) differentiating lexical components (specific references, pronominal references, negative particles, adverbs, degree, or aspect). (cleison)

\end{enumerate}








% \cite{stewart-2021-barrons-asl} --------------------------------------
% The Physical Dimensions of ASL: The Five Parameters
% ASL is a visual-gestural language. It is visual because we see it and gestural because the signs are formed by the hands. Signing alone, however, is not an accurate picture of ASL. How signs are formed in space is important to understanding what they mean. The critical space is called the signing space and extends from the waist to just above the head and to just beyond the sides of the body. This is also the space in which the hands can move comfortably. As you will learn in this book, the signing space has a role in ASL grammar. Two or more concepts can be simultaneously expressed in ASL. This feat cannot be accomplished in a spoken language because speech is temporal in that one word rolls off the tongue at a time. One further dimension of ASL is the movement of the head and facial expressions, which help shape the meaning of ASL sentences.
% ---
% How Are Signs Formed?
% The five parameters below come together to create a sign.
% 1) Handshape is the shape of the hands when the sign is formed. The handshape may remain the same throughout the sign or it can change. If two hands are use to make a sign, both hands can have the same handshape of be different.
% 2) Orientation is the position of the hand(s) relative to the body. For example, the palms can be facing the body or away from the body, facing the ground, or facing upward.
% 3) Location is the place in the signing space where a sign is formed. Signs can be stationary [...] or they can move from one location in the signing space to another [...].
% 4) Movement of a sign is the direction in which the hand moves relative to the body. There is a variety of movements that range from a simple sliding movement [...] to a complex movement.
% 5) Nonmanual markers add to signs to create meaning. They consist of various facial expressions, head tilting, shoulder movement, and mouth movements. With a non-manual marker, the meaning of the sign can change completely.
% ---


% \cite{jay-2011-dont-just-sign} ---------------------
% Phonology
% In language, phonology is the study of the smallest part of the language that conveys meaning. In spoken languages, like English, a phoneme is a unit of sound that conveys meaning. 
% In ASL, the smallest parts of the language, the phonemes, are handshape, movement, palm orientation, location, and facial expression. If you change any of these parameters of a sign, then you have changed the meaning of the sign.
% ---
% The Five Sign Parameters
% Just like how we see English words as the arrangement of letters, there are five basic sign language elements that make up each sign. If any of these parameters are changed when creating a sign, the meaning of the sign can change. 
% The first four elements are: 
% • Handshape – This is the shape of your hand that is used to create the sign. 
% • Movement – This is the movement of the handshape that makes the sign. 
% • Palm orientation – This is the orientation of your palm when making the sign. 
% • Location – This is the location of the sign on or in front of your body. 
% There is also a fifth element that has recently been included with this list: 
% • Non-manual Markers – This is the various facial expressions or body movements that are used to create meaning. 
% American Sign Language is a very expressive language, and understanding these elements will give you a better understanding of how signs are made and what makes them different.
% ---


% \cite{quadros-2004-estudos-linguisticos} ------------------------------
% Fonologia das línguas de sinais
% Fonologia das línguas de sinais é o ramo da lingüística que objetiva identificar a estrutura e a organização dos constituintes fonológicos, propondo  modelos descritivos e explanatórios. A primeira tarefa da fonologia para línguas de sinais é determinar quais são as unidades mínimas que formam os  sinais. A segunda tarefa é estabelecer quais são os padrões possíveis de combinação entre essas unidades e as variações possíveis no ambiente fonológico. 


% \cite{hill-2019-sign-languages} -------------------------------
% Phonology
% Phonology is often described as the study of the sounds of language and their organization. If that is the way to look at phonology, it would be appropriate to say that sign languages do not have phonology. However, phonology is actually more abstract. It is about the ways in which words are made up of pieces that are not meaningful. It is about what these pieces are and how they work together. In spoken languages, the actual sounds that make up words are part of the study of phonology; yet, phonology is more concerned with the ways we can define these component pieces, how they change in different contexts (e.g., different words), and how they are organized.
% If there are component pieces that make up individual signs, then there is sign phonology. If there are implicit rules about the ways that the pieces can and cannot combine, then there is phonology. Indeed, one of the first linguistic discoveries about American Sign Language (ASL) is that “signs have parts” – [...]

% The Five Sign Parameters
% Just like how we see English words as the arrangement of letters, there are five basic sign language elements that make up each sign. If any of these parameters are changed when creating a sign, the meaning of the sign can change. 
% The first four elements are: 
% • Handshape – This is the shape of your hand that is used to create the sign. 
% • Movement – This is the movement of the handshape that makes the sign. 
% • Palm orientation – This is the orientation of your palm when making the sign. 
% • Location – This is the location of the sign on or in front of your body. 
% There is also a fifth element that has recently been included with this list: 
% • Non-manual Markers – This is the various facial expressions or body movements that are used to create meaning. 
% American Sign Language is a very expressive language, and understanding these elements will give you a better understanding of how signs are made and what makes them different.



