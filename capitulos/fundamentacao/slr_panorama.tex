\subsection{Breve panorama}
\label{sec:slr-breve-panorama}

Nesta seção será discutido brevemente o panorama observado para a \acrshort{slr} ao longo das últimas décadas, tomando como base as revisões literárias apresentadas por \citeonline{koller-2020-quantitative-survey-slr,rastgoo-2021-slr-deep-survey,papastratis-2021-ai-technologies-sl,wadhawan-2019-slr-literature-review}.


% Visão geral -------------

Primeiramente, observa-se na \autoref{tab:slr-visao-geral} uma perspectiva geral do número de estudos publicados nessa área até 2020, conforme identificado por \citeonline{koller-2020-quantitative-survey-slr}.
Percebe-se que esse número praticamente dobrou a cada intervalo de 5 anos e, com o advento de novos dispositivos e algoritmos de \acrshort{ia} por volta dos anos 2010, cresceu de forma ainda mais expressiva. Isso evidencia a ênfase maior recebida pela área recentemente.

Apesar disso, a tabela também nos revela que a maioria desses estudos aborda o \acrshort{slr} utilizando vocabulários muito pequenos, com menos de 200 sinais. Isso corresponde a conjuntos limitados que são geralmente selecionados para simplificar as pesquisas mas que, na prática, acabam limitando também a representatividade da língua de sinais perante seu contexto real de aplicação.
Apenas a partir de 2015 percebe-se um crescimento mais significativo na adoção de vocabulários com mais de 1.000 sinais.

% Please add the following required packages to your document preamble:
% \usepackage{graphicx}
% \usepackage[table,xcdraw]{xcolor}
% If you use beamer only pass "xcolor=table" option, i.e. \documentclass[xcolor=table]{beamer}
\begin{table}[ht!]
    \centering
    \caption{Estudos em \acrshort{slr} publicados até 2020, agrupados por intervalos de cinco anos (na horizontal) e tamanho de vocabulário modelado (na vertical).}
    \label{tab:slr-visao-geral}
    \resizebox{\textwidth}{!}{%
        \begin{tabular}{r|ccccccc|c}
            \hline
            \rowcolor[HTML]{EFEFEF}
            \cellcolor[HTML]{EFEFEF}                              & \multicolumn{7}{c|}{\cellcolor[HTML]{EFEFEF}Ano} & \cellcolor[HTML]{EFEFEF}                                                                                                                                                                                                     \\
            \rowcolor[HTML]{EFEFEF}
            \multirow{-2}{*}{\cellcolor[HTML]{EFEFEF}Vocabulário} & \textless 1990                                   & 1990 - 1995               & 1995 - 2000                & 2000 - 2005                & 2005 - 2010                & 2010 - 2015                & 2015 - 2020                & \multirow{-2}{*}{\cellcolor[HTML]{EFEFEF}Total} \\ \hline
            0 - 50                                                & \cellcolor[HTML]{FFFCF5}2                        & \cellcolor[HTML]{FFF7E6}5 & \cellcolor[HTML]{FFEBC4}12 & \cellcolor[HTML]{FFEBC4}12 & \cellcolor[HTML]{FFBD39}40 & \cellcolor[HTML]{FFC552}35 & \cellcolor[HTML]{FFAD08}50 & 156                                             \\
            50 - 200                                              & 0                                                & \cellcolor[HTML]{FFFDFA}1 & \cellcolor[HTML]{FFF5E1}6  & \cellcolor[HTML]{FFEFCE}10 & \cellcolor[HTML]{FFD37A}27 & \cellcolor[HTML]{FFDB92}22 & \cellcolor[HTML]{FFAB03}51 & 117                                             \\
            200 - 500                                             & 0                                                & \cellcolor[HTML]{FFFDFA}1 & \cellcolor[HTML]{FFFAF0}3  & \cellcolor[HTML]{FFFCF5}2  & \cellcolor[HTML]{FFF3DC}7  & \cellcolor[HTML]{FFE6B5}15 & \cellcolor[HTML]{FFD684}25 & 53                                              \\
            500 - 1000                                            & 0                                                & 0                         & 0                          & 0                          & \cellcolor[HTML]{FFFDFA}1  & \cellcolor[HTML]{FFEABF}13 & \cellcolor[HTML]{FFEABF}13 & 27                                              \\
            \textgreater 1000                                     & 0                                                & 0                         & \cellcolor[HTML]{FFFDFA}1  & \cellcolor[HTML]{FFF5E1}6  & \cellcolor[HTML]{FFFAF0}3  & \cellcolor[HTML]{FFF8EB}4  & \cellcolor[HTML]{FFBD39}40 & 54                                              \\ \hline
            Total                                                 & 2                                                & 7                         & 22                         & 30                         & 78                         & 89                         & 179                        &                                                 \\ \hline
        \end{tabular}%
    }
    \nomefonte[p. 3]{koller-2020-quantitative-survey-slr}
\end{table}



% Sinais isolados x contínuos -------------

A \autoref{tab:slr-isolados-continuos}, por sua vez, divide os estudos acima entre aqueles que abordam a língua utilizando sinais isolados -- os quais são reconhecidos separadamente do contexto do discurso --, e os que utilizam sinais contínuos -- onde sequências completas são utilizadas nesse processo.
Percebe-se que a maior parte desses estudos seguem pela linha dos sinais isolados, uma vez que a abordagem de sinais contínuos é certamente mais complexa e apenas teve seus primeiros \textit{datasets} disponibilizados por volta dos anos 2015.
Os dados também nos mostram que, ao passo em que estudos com sinais isolados comumente modelam vocabulários pequenos, aqueles que optam por sinais contínuos passaram a preferir vocabulários acima de 1.000 sinais.

% Please add the following required packages to your document preamble:
% \usepackage{multirow}
% \usepackage{graphicx}
% \usepackage[table,xcdraw]{xcolor}
% If you use beamer only pass "xcolor=table" option, i.e. \documentclass[xcolor=table]{beamer}
\begin{table}[ht!]
    \centering
    \caption{Comparação dos estudos em \acrshort{slr} que abordaram sinais isolados (acima) ou sinais contínuos (abaixo)}
    \label{tab:slr-isolados-continuos}
    \resizebox{\textwidth}{!}{%
        \begin{tabular}{lrcccccccc}
            \cline{2-10}
                                                                         & \multicolumn{1}{r|}{\cellcolor[HTML]{EFEFEF}}                              & \multicolumn{7}{c|}{\cellcolor[HTML]{EFEFEF}Ano} & \cellcolor[HTML]{EFEFEF}                                                                                                                                                                                                                                                                                 \\
                                                                         & \multicolumn{1}{r|}{\multirow{-2}{*}{\cellcolor[HTML]{EFEFEF}Vocabulário}} & \cellcolor[HTML]{EFEFEF}\textless 1990           & \cellcolor[HTML]{EFEFEF}1990 - 1995 & \cellcolor[HTML]{EFEFEF}1995 - 2000 & \cellcolor[HTML]{EFEFEF}2000 - 2005 & \cellcolor[HTML]{EFEFEF}2005 - 2010 & \cellcolor[HTML]{EFEFEF}2010 - 2015 & \multicolumn{1}{c|}{\cellcolor[HTML]{EFEFEF}2015 - 2020} & \multirow{-2}{*}{\cellcolor[HTML]{EFEFEF}Total} \\ \cline{2-10}
                                                                         & \multicolumn{1}{r|}{0 - 50}                                                & \cellcolor[HTML]{FFFCF5}2                        & \cellcolor[HTML]{FFF8EB}4           & \cellcolor[HTML]{FFF3DC}7           & \cellcolor[HTML]{FFF3DC}7           & \cellcolor[HTML]{FFD37A}27          & \cellcolor[HTML]{FFCF70}29          & \multicolumn{1}{c|}{\cellcolor[HTML]{FFB82B}43}          & 119                                             \\
                                                                         & \multicolumn{1}{r|}{50 - 200}                                              & 0                                                & \cellcolor[HTML]{FFFDFA}1           & \cellcolor[HTML]{FFFCF5}2           & \cellcolor[HTML]{FFF2D7}8           & \cellcolor[HTML]{FFEBC4}12          & \cellcolor[HTML]{FFE3AB}17          & \multicolumn{1}{c|}{\cellcolor[HTML]{FFC757}34}          & 74                                              \\
                                                                         & \multicolumn{1}{r|}{200 - 500}                                             & 0                                                & \cellcolor[HTML]{FFFDFA}1           & \cellcolor[HTML]{FFFCF5}2           & \cellcolor[HTML]{FFFDFA}1           & \cellcolor[HTML]{FFFAF0}3           & \cellcolor[HTML]{FFF5E1}6           & \multicolumn{1}{c|}{\cellcolor[HTML]{FFE0A1}19}          & 32                                              \\
                                                                         & \multicolumn{1}{r|}{500 - 1000}                                            & 0                                                & 0                                   & 0                                   & 0                                   & \cellcolor[HTML]{FFFDFA}1           & \cellcolor[HTML]{FFEDC9}11          & \multicolumn{1}{c|}{\cellcolor[HTML]{FFEBC4}12}          & 24                                              \\
            \multirow{-5}{*}{\rotatebox[origin=c]{90}{Sinais isolados}}  & \multicolumn{1}{r|}{\textgreater 1000}                                     & 0                                                & 0                                   & \cellcolor[HTML]{FFFDFA}1           & \cellcolor[HTML]{FFFAF0}3           & \cellcolor[HTML]{FFFCF5}2           & \cellcolor[HTML]{FFFCF5}2           & \multicolumn{1}{c|}{\cellcolor[HTML]{FFF5E1}6}           & 14                                              \\ \cline{2-10}
                                                                         & \multicolumn{1}{r|}{Total}                                                 & 2                                                & 6                                   & 12                                  & 19                                  & 45                                  & 65                                  & \multicolumn{1}{c|}{114}                                 &                                                 \\ \cline{2-10}
                                                                         &                                                                            &                                                  &                                     &                                     &                                     &                                     &                                     &                                                          &                                                 \\ \cline{2-10}
                                                                         & \multicolumn{1}{r|}{\cellcolor[HTML]{EFEFEF}Vocabulário}                   & \cellcolor[HTML]{EFEFEF}\textless 1990           & \cellcolor[HTML]{EFEFEF}1990 - 1995 & \cellcolor[HTML]{EFEFEF}1995 - 2000 & \cellcolor[HTML]{EFEFEF}2000 - 2005 & \cellcolor[HTML]{EFEFEF}2005 - 2010 & \cellcolor[HTML]{EFEFEF}2010 - 2015 & \multicolumn{1}{c|}{\cellcolor[HTML]{EFEFEF}2015 - 2020} & \cellcolor[HTML]{EFEFEF}Total                   \\ \cline{2-10}
                                                                         & \multicolumn{1}{r|}{0 - 50}                                                & 0                                                & \cellcolor[HTML]{FFFDFA}1           & \cellcolor[HTML]{FFF7E6}5           & \cellcolor[HTML]{FFF7E6}5           & \cellcolor[HTML]{FFEABF}13          & \cellcolor[HTML]{FFF5E1}6           & \multicolumn{1}{c|}{\cellcolor[HTML]{FFF3DC}7}           & 37                                              \\
                                                                         & \multicolumn{1}{r|}{50 - 200}                                              & 0                                                & 0                                   & \cellcolor[HTML]{FFF8EB}4           & \cellcolor[HTML]{FFFCF5}2           & \cellcolor[HTML]{FFE6B5}15          & \cellcolor[HTML]{FFF7E6}5           & \multicolumn{1}{c|}{\cellcolor[HTML]{FFE3AB}17}          & 43                                              \\
                                                                         & \multicolumn{1}{r|}{200 - 500}                                             & 0                                                & 0                                   & \cellcolor[HTML]{FFFDFA}1           & \cellcolor[HTML]{FFFDFA}1           & \cellcolor[HTML]{FFF8EB}4           & \cellcolor[HTML]{FFF0D3}9           & \multicolumn{1}{c|}{\cellcolor[HTML]{FFF5E1}6}           & 21                                              \\
                                                                         & \multicolumn{1}{r|}{500 - 1000}                                            & 0                                                & 0                                   & 0                                   & 0                                   & 0                                   & \cellcolor[HTML]{FFFCF5}2           & \multicolumn{1}{c|}{\cellcolor[HTML]{FFFDFA}1}           & 3                                               \\
            \multirow{-5}{*}{\rotatebox[origin=c]{90}{Sinais contínuos}} & \multicolumn{1}{r|}{\textgreater 1000}                                     & 0                                                & 0                                   & 0                                   & \cellcolor[HTML]{FFFAF0}3           & \cellcolor[HTML]{FFFDFA}1           & \cellcolor[HTML]{FFFCF5}2           & \multicolumn{1}{c|}{\cellcolor[HTML]{FFC757}34}          & 40                                              \\ \cline{2-10}
                                                                         & \multicolumn{1}{r|}{Total}                                                 & 0                                                & 1                                   & 10                                  & 11                                  & 33                                  & 24                                  & \multicolumn{1}{c|}{65}                                  &                                                 \\ \cline{2-10}
        \end{tabular}%
    }
    \nomefonte[p. 3]{koller-2020-quantitative-survey-slr}
\end{table}



% Línguas de sinais  -------------

Ao analisar as línguas de sinais que foram abordadas pelas pesquisas em \acrshort{slr}, conforme \autoref{tab:slr-linguas-sinais}, percebe-se que a \acrshort{asl} foi predominante dentre as demais. Isso provavelmente deve-se ao pioneirismo recebido por ela nos estudos de \citeonline{stokoe-1960-sl-structure} que, além de produzir maior clareza acerca de sua estrutura, viabilizou o desenvolvimento de recursos importantes como \textit{datasets}, que suportaram tais pesquisas.
As línguas \acrfull{csl} e \acrfull{dgs} aparecem em seguida na tabela com uma participação também relevante, a qual desde 2010 apresenta números muito próximos àqueles da \acrshort{asl}. A \acrfull{libras}, por sua vez, aparece de forma mais modesta nesse levantamento realizado por \citeonline{koller-2020-quantitative-survey-slr}.

% Please add the following required packages to your document preamble:
% \usepackage{multirow}
% \usepackage{graphicx}
% \usepackage[table,xcdraw]{xcolor}
% If you use beamer only pass "xcolor=table" option, i.e. \documentclass[xcolor=table]{beamer}
\begin{table}[ht!]
    \centering
    \caption{Línguas de sinais abordadas pelos estudos em \acrshort{slr}}
    \label{tab:slr-linguas-sinais}
    \resizebox{\textwidth}{!}{%
        \begin{tabular}{l|ccccccc|c}
            \hline
            \rowcolor[HTML]{EFEFEF}
            \cellcolor[HTML]{EFEFEF}                                                                              & \multicolumn{7}{c|}{\cellcolor[HTML]{EFEFEF}Ano} & \cellcolor[HTML]{EFEFEF}                                                                                                                                                                                                     \\
            \rowcolor[HTML]{EFEFEF}
            \multirow{-2}{*}{\cellcolor[HTML]{EFEFEF}\begin{tabular}[c]{@{}c@{}}Língua \\ de sinais\end{tabular}} & \textless 1990                                   & 1990 - 1995               & 1995 - 2000                & 2000 - 2005                & 2005 - 2010                & 2010 - 2015                & 2015 - 2020                & \multirow{-2}{*}{\cellcolor[HTML]{EFEFEF}Total} \\ \hline
            \textbf{\acrshort{asl}}                                                                               & \cellcolor[HTML]{FFFEFB}1                        & \cellcolor[HTML]{FFF8EA}5 & \cellcolor[HTML]{FFF8EA}5  & \cellcolor[HTML]{FFEFD0}11 & \cellcolor[HTML]{FFD47F}30 & \cellcolor[HTML]{FFE1A5}21 & \cellcolor[HTML]{FFBD3A}46 & 119                                             \\
            \acrshort{csl}                                                                                        & 0                                                & 0                         & \cellcolor[HTML]{FFFCF6}2  & \cellcolor[HTML]{FFF1D4}10 & \cellcolor[HTML]{FFF9EE}4  & \cellcolor[HTML]{FFE1A5}21 & \cellcolor[HTML]{FFCD69}35 & 72                                              \\
            \acrshort{dgs}                                                                                        & 0                                                & 0                         & \cellcolor[HTML]{FFFEFB}1  & \cellcolor[HTML]{FFFCF6}2  & \cellcolor[HTML]{FFF9EE}4  & \cellcolor[HTML]{FFE7B6}17 & \cellcolor[HTML]{FFC654}40 & 64                                              \\
            \acrshort{bsl}                                                                                        & \cellcolor[HTML]{FFFCF6}2                        & \cellcolor[HTML]{FFFCF6}2 & \cellcolor[HTML]{FFECC7}13 & \cellcolor[HTML]{FFFBF2}3  & 0                          & 0                          & 0                          & 20                                              \\
            \acrshort{arsl}                                                                                       & 0                                                & 0                         & 0                          & 0                          & \cellcolor[HTML]{FFF6E5}6  & \cellcolor[HTML]{FFF5E1}7  & \cellcolor[HTML]{FFF9EE}4  & 17                                              \\
            \acrshort{jsl}                                                                                        & \cellcolor[HTML]{FFFEFB}1                        & \cellcolor[HTML]{FFFEFB}1 & \cellcolor[HTML]{FFF9EE}4  & \cellcolor[HTML]{FFFEFB}1  & \cellcolor[HTML]{FFFCF6}2  & \cellcolor[HTML]{FFFEFB}1  & \cellcolor[HTML]{FFFEFB}1  & 11                                              \\
            \acrshort{isl}                                                                                        & 0                                                & 0                         & 0                          & 0                          & 0                          & \cellcolor[HTML]{FFFCF6}2  & \cellcolor[HTML]{FFF5E1}9  & 11                                              \\
            \acrshort{gsl}                                                                                        & 0                                                & 0                         & 0                          & 0                          & \cellcolor[HTML]{FFFEFB}1  & \cellcolor[HTML]{FFF1D4}10 & \cellcolor[HTML]{FFFCF6}2  & 13                                              \\
            \acrshort{tid}                                                                                        & 0                                                & 0                         & 0                          & 0                          & \cellcolor[HTML]{FFF8EA}5  & 0                          & \cellcolor[HTML]{FFF9EE}4  & 9                                               \\
            \acrshort{ngt}                                                                                        & 0                                                & 0                         & \cellcolor[HTML]{FFFCF6}2  & 0                          & \cellcolor[HTML]{FFFCF6}2  & 0                          & \cellcolor[HTML]{FFF8EA}5  & 9                                               \\
            \acrshort{vgt}                                                                                        & 0                                                & 0                         & 0                          & 0                          & 0                          & 0                          & \cellcolor[HTML]{FFF5E1}7  & 7                                               \\
            \acrshort{lis}                                                                                        & 0                                                & 0                         & 0                          & 0                          & \cellcolor[HTML]{FFFCF6}2  & \cellcolor[HTML]{FFFCF6}2  & \cellcolor[HTML]{FFFEFB}1  & 5                                               \\
            \acrshort{auslan}                                                                                     & 0                                                & \cellcolor[HTML]{FFFEFB}1 & \cellcolor[HTML]{FFFBF2}3  & 0                          & 0                          & 0                          & \cellcolor[HTML]{FFFEFB}1  & 5                                               \\
            \acrshort{lsa}                                                                                        & 0                                                & 0                         & 0                          & 0                          & 0                          & 0                          & \cellcolor[HTML]{FFF8EA}5  & 5                                               \\
            \acrshort{tsl}                                                                                        & 0                                                & 0                         & \cellcolor[HTML]{FFFBF2}3  & \cellcolor[HTML]{FFFEFB}1  & 0                          & 0                          & 0                          & 4                                               \\
            \acrshort{pjm}                                                                                        & 0                                                & 0                         & 0                          & \cellcolor[HTML]{FFFEFB}1  & \cellcolor[HTML]{FFFEFB}1  & \cellcolor[HTML]{FFFCF6}2  & 0                          & 4                                               \\
            \textbf{\acrshort{libras}}                                                                            & 0                                                & 0                         & 0                          & 0                          & \cellcolor[HTML]{FFFEFB}1  & \cellcolor[HTML]{FFFCF6}2  & \cellcolor[HTML]{FFFEFB}1  & 4                                               \\
            ...                                                                                                   & \multicolumn{7}{c|}{...}                         & \multicolumn{1}{c}{...}                                                                                                                                                                                                         \\
            \textit{Outras}                                                                                       & 0                                                & 0                         & \cellcolor[HTML]{FFFCF6}2  & 0                          & \cellcolor[HTML]{FFF5E1}7  & \cellcolor[HTML]{FFFEFB}1  & \cellcolor[HTML]{FFECC7}15 & 15                                              \\ \hline
            Total                                                                                                 & 4                                                & 9                         & 35                         & 29                         & 65                         & 86                         & 176                        &                                                 \\ \hline
        \end{tabular}%
    }
    \nomefonte[p. 9]{koller-2020-quantitative-survey-slr}
\end{table}


% Tipos de dados de entrada -------------

Na \autoref{tab:slr-dados-entrada} observam-se os tipos de dados de entrada utilizados por tais estudos. Imagens ou \textit{frames} de vídeos em RGB aparecem em destaque e vêm sendo adotados desde os anos iniciais até a atualidade. Ao longo da história da \acrshort{slr} eles dividiram espaço com outros tipos de dados mas, a partir de 2005, tornaram-se predominantes provavelmente pela maior adoção de técnicas baseadas em \acrlong{cv} desde então.
Luvas eletrônicas estiveram presentes principalmente nos anos iniciais da \acrshort{slr}, quando técnicas envolvendo dispositivos conectados ao corpo dos indivíduos foram muito utilizadas.
Apesar disso, elas foram gradativamente cedendo espaço a outros tipos como as luvas coloridas e ao \textit{mocap}\footnote{
      \textit{Mocap} (\textit{motion capture} ou captura de movimento): é uma técnica de captura de movimentos que utiliza equipamentos específicos como marcadores ou trajes especiais afixados ao corpo dos indivíduos ou objetos~\cite{kitagawa-2017-mocap}.
} até por volta de 2005.

O surgimento do Kinect em 2010 representou uma revolução para a área de \acrshort{slr}, devido à capacidade que ele introduziu de rastrear os corpos dos indivíduos e de fornecer dados como coordenadas e imagens profundidade além daquelas RGB, o qual não  existia na época. Nos anos posteriores ao seu lançamento, vários outros dispositivos também foram introduzidos com o mesmo propósito.
Na \autoref{tab:slr-dados-entrada}, é possível perceber claramente uma mudança para a adoção de dados RGB combinados aos dados de profundidade, decorrente disso.

% Please add the following required packages to your document preamble:
% \usepackage{multirow}
% \usepackage{graphicx}
% \usepackage[table,xcdraw]{xcolor}
% If you use beamer only pass "xcolor=table" option, i.e. \documentclass[xcolor=table]{beamer}
\begin{table}[]
    \centering
    \caption{Proporção de tipos de dados de entrada utilizados pelos estudos em \acrshort{slr}}
    \label{tab:slr-dados-entrada}
    \resizebox{0.90\textwidth}{!}{%
        \begin{tabular}{l|ccccccc}
            \hline
            \rowcolor[HTML]{EFEFEF}
            \cellcolor[HTML]{EFEFEF}                                                                             & \multicolumn{7}{c}{\cellcolor[HTML]{EFEFEF}Ano}                                                                                                                                                                                                                                                                                                                                                                        \\
            \rowcolor[HTML]{EFEFEF}
            \multirow{-2}{*}{\cellcolor[HTML]{EFEFEF}\begin{tabular}[c]{@{}l@{}}Dados de entrada\end{tabular}} & \multicolumn{1}{r}{\cellcolor[HTML]{EFEFEF}\textless 1990} & \multicolumn{1}{r}{\cellcolor[HTML]{EFEFEF}1990 - 1995} & \multicolumn{1}{r}{\cellcolor[HTML]{EFEFEF}1995 - 2000} & \multicolumn{1}{r}{\cellcolor[HTML]{EFEFEF}2000 - 2005} & \multicolumn{1}{r}{\cellcolor[HTML]{EFEFEF}2005 - 2010} & \multicolumn{1}{r}{\cellcolor[HTML]{EFEFEF}2010 - 2015} & \multicolumn{1}{r}{\cellcolor[HTML]{EFEFEF}2015 - 2020} \\ \hline
            RGB                                                                                                  & \cellcolor[HTML]{FFD47F}50\%                               & \cellcolor[HTML]{FFE6B5}29\%                            & \cellcolor[HTML]{FFE0A3}36\%                            & \cellcolor[HTML]{FFE3AA}33\%                            & \cellcolor[HTML]{FFC146}72\%                            & \cellcolor[HTML]{FFB625}85\%                            & \cellcolor[HTML]{FFB520}87\%                            \\
            Profundidade                                                                                         & 0\%                                                        & 0\%                                                     & 0\%                                                     & 0\%                                                     & \cellcolor[HTML]{FFFEFC}1\%                             & \cellcolor[HTML]{FFDE9D}38\%                            & \cellcolor[HTML]{FFECC7}22\%                            \\
            Luva colorida                                                                                        & 0\%                                                        & 0\%                                                     & \cellcolor[HTML]{FFF0D1}18\%                            & \cellcolor[HTML]{FFF6E5}10\%                            & \cellcolor[HTML]{FFF0D1}18\%                            & \cellcolor[HTML]{FFFCF5}4\%                             & \cellcolor[HTML]{FFFCF7}3\%                             \\
            Luva eletrônica                                                                                      & \cellcolor[HTML]{FFD47F}50\%                               & \cellcolor[HTML]{FFC249}71\%                            & \cellcolor[HTML]{FFDC96}41\%                            & \cellcolor[HTML]{FFD47F}50\%                            & \cellcolor[HTML]{FFF6E5}10\%                            & \cellcolor[HTML]{FFF9ED}7\%                             & \cellcolor[HTML]{FFFCF5}4\%                             \\
            \textit{Mocap}                                                                                       & 0\%                                                        & \cellcolor[HTML]{FFE6B5}29\%                            & \cellcolor[HTML]{FFEBC4}23\%                            & \cellcolor[HTML]{FFCE6D}57\%                            & \cellcolor[HTML]{FFFAF0}6\%                             & \cellcolor[HTML]{FFF9ED}7\%                             & \cellcolor[HTML]{FFFAF0}6\%                             \\ \hline
        \end{tabular}%
    }
    \nomefonte[p. 4]{koller-2020-quantitative-survey-slr}
\end{table}



% Tipos de features modeladas -------------

Por fim, a \autoref{tab:slr-tipos-features} apresenta os tipos de \textit{features} utilizadas pelos estudos acima. Elas estão categorizados entre \textit{features} manuais, que correspondem aos parâmetros de configuração de mão, orientação, movimento e locação introduzidos na \autoref{sec:linguistica-fonologia}; não-manuais, que referem-se às expressões não-manuais introduzidas na mesma seção; e globais, que são aquelas que capturam uma visão completa do corpo dos indivíduos, como coordenadas, imagens RGB, de profundidade, ou de fluxo óptico.


Percebe-se uma predominância da adoção de \textit{features} manuais desde os anos iniciais até por volta de 2015. Isso explica-se principalmente pelo fato de que um grande número dessas pesquisas aborda os sinais através de recortes das mãos dos indivíduos, muitas vezes estáticas e fora do seu contexto original. A esse tipo de abordagem \citeonline{koller-2020-quantitative-survey-slr} enquadrou em seu levantamento como sendo refente à configuração de mão e, consequentemente, uma \textit{feature} manual. \textit{Features} relacionadas à locação e ao movimento das mãos também são encontradas modeladas de diferentes maneiras e contribuem para essa predominância.



Com o uso do Kinect e das técnicas de \acrlong{dl} a partir de 2010, as \textit{features} globais passaram a assumir uma posição bastante expressiva nesses estudos. Isso porque, ao mesmo tempo em que este dispositivo passou a fornecer novos tipos de informações, como coordenadas corporais e dados de profundidade combinados com RGB, essas técnicas introduziram um poder de processamento que possibilitou com que essas informações fossem utilizadas diretamente como \textit{features} de entrada para eles.
Por outro lado, as \textit{features} não-manuais não chegaram a apresentar uma adoção significativa perante os demais tipos, ao analisá-los de um modo geral.

% Please add the following required packages to your document preamble:
% \usepackage{multirow}
% \usepackage{graphicx}
% \usepackage[table,xcdraw]{xcolor}
% If you use beamer only pass "xcolor=table" option, i.e. \documentclass[xcolor=table]{beamer}
\begin{table}[]
    \centering
    \caption{Tipos de features utilizadas pelos estudos em \acrshort{slr}.}
    \label{tab:slr-tipos-features}
    \resizebox{0.95\textwidth}{!}{%
        \begin{tabular}{l|ccccccc}
            \hline
            \rowcolor[HTML]{EFEFEF}
            \cellcolor[HTML]{EFEFEF}                                    & \multicolumn{7}{c}{\cellcolor[HTML]{EFEFEF}Ano}                                                                                                                                                                                                                                                                                                                                                                        \\
            \rowcolor[HTML]{EFEFEF}
            \multirow{-2}{*}{\cellcolor[HTML]{EFEFEF}Tipos de features} & \multicolumn{1}{r}{\cellcolor[HTML]{EFEFEF}\textless 1990} & \multicolumn{1}{r}{\cellcolor[HTML]{EFEFEF}1990 - 1995} & \multicolumn{1}{r}{\cellcolor[HTML]{EFEFEF}1995 - 2000} & \multicolumn{1}{r}{\cellcolor[HTML]{EFEFEF}2000 - 2005} & \multicolumn{1}{r}{\cellcolor[HTML]{EFEFEF}2005 - 2010} & \multicolumn{1}{r}{\cellcolor[HTML]{EFEFEF}2010 - 2015} & \multicolumn{1}{r}{\cellcolor[HTML]{EFEFEF}2015 - 2020} \\ \hline
            Manuais                                                     & \cellcolor[HTML]{FFAA00}100\%                              & \cellcolor[HTML]{FFAA00}100\%                           & \cellcolor[HTML]{FFAA00}100\%                           & \cellcolor[HTML]{FFAA00}100\%                           & \cellcolor[HTML]{FFAA01}99\%                            & \cellcolor[HTML]{FFAA00}100\%                           & \cellcolor[HTML]{FFD786}47\%                            \\
            Não-manuais                                                 & 0\%                                                        & 0\%                                                     & \cellcolor[HTML]{FFFBF2}5\%                             & 0\%                                                     & \cellcolor[HTML]{FFF4DE}13\%                            & \cellcolor[HTML]{FFF0D3}17\%                            & \cellcolor[HTML]{FFF8EA}8\%                             \\
            Globais                                                     & 0\%                                                        & 0\%                                                     & 0\%                                                     & 0\%                                                     & \cellcolor[HTML]{FFDFA0}1\%                             & \cellcolor[HTML]{FFDFA0}37\%                            & \cellcolor[HTML]{FFC656}66\%                            \\ \hline
        \end{tabular}%
    }
    \nomefonte[p. 7]{koller-2020-quantitative-survey-slr}
\end{table}

