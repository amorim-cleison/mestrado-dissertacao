\subsection{Sintaxe}
\label{sec:linguistica-sintaxe}

Sintaxe é o estudo da construção de sentenças a partir das palavras, bem como dos princípios e regras envolvidos nesse processo, afirmam \citeonline{hill-2019-sign-languages,jay-2011-dont-just-sign}.

Nas línguas de sinais, a sintaxe é comunicada através da ordem das palavras e da utilização das expressões não-manuais.
Observemos a seguir alguns dos principais componentes presentes no processo de construção sintática dessas línguas \cite{jay-2011-dont-just-sign,hill-2019-sign-languages,quadros-2004-estudos-linguisticos}:

\begin{enumerate}
    \item \textbf{Ordem das palavras}: existem diferentes possibilidades de ordenação das palavras nas sentenças da língua de sinais. Apesar disso, a ordem \textit{Sujeito-Verbo-Objeto} (SVO) parece ser a mais básica dentre as demais.
          Ordenações como \textit{Objeto-Sujeito-Verbo} (OSV), \textit{Sujeito-Objeto-Verbo} (SOV) e \textit{Verbo-Sujeito-Objeto} (VOS) são derivadas daquela primeira e resultam de operações sintáticas específicas associadas a alguma marcação como a concordância de verbos ou as expressões não-manuais.

          % as sentenças na língua de sinais seguem uma estrutura ``tópico-comentário'' semelhante à estrutura ``sujeito-predicado'' utilizada por línguas faladas. No entanto, o tópico pode ser tanto o subjeito da sentença quanto qualquer objeto a qual o comentário se refira.

          %   De acordo com \citeonline{quadros-2004-estudos-linguisticos}, há várias possibilidades de ordenação das palavras nas sentenças da língua de sinais. Porém, apesar dessa flexibilidade, parece haver uma ordenação mais básica que as demais, que seria a ordem \textit{Sujeito-Verbo-Objeto} (SVO).
          %   Outras ordenações OSV, SOV e VOS são derivadas daquela primeira e resultam de operações sintáticas específicas  associadas a algum tipo de marcação como, por exemplo, a concordância de verbos e as expressões não-manuais \cite{quadros-2004-estudos-linguisticos}.



          % \cite{jay-2011-dont-just-sign}
          % ASL sentences follow a TOPIC-COMMENT structure. This is the same as the English “subject” “predicate” structure. However, instead of the topic always being the subject, the topic in ASL is whatever the comment is referring to. This can either be the subject of the sentence or the object. 
          % The subject of a sentence is the person or object doing the action, the verb of a sentence is the action, and the object of a sentence is what is receiving the action. For example, in the sentence “The boy kicked the ball” the subject is “boy,” the verb is “kicked,” and the object is “ball.” 
          % There are a few different variations of word order in ASL depending on the vocabulary you are using and what you are trying to accomplish.


          % \cite{quadros-2004-estudos-linguisticos}
          % Há dois trabalhos que mencionam a flexibilidade da ordem das frases na  língua de sinais brasileira: Felipe (1989) e Ferreira-Brito (1995). As autoras  observaram que há várias possibilidades de ordenação das palavras nas sentenças, mas que, apesar dessa flexibilidade, parece haver uma ordenação mais  básica que as demais, ou seja, a ordem Sujeito-Verbo-Objeto. Quadros (1999)  apresenta evidências que justificam tal intuição propondo uma representação  para a estrutura da frase nesta língua. As evidências surgem de orações simples, de orações complexas contendo orações subordinadas, da interação com  advérbios, com modais e com auxiliares. As demais ordenações encontradas  na língua de sinais brasileira resultam da interação de outros mecanismos  gramaticais. 
          % [...]
          % Os dados apresentados indicam que a ordem básica na língua de sinais  brasileira é SVO e que OSV, SOV e VOS são ordenações derivadas de SVO.  Assim, as mudanças de ordens resultam de operações sintáticas específicas  associadas a algum tipo de marca, por exemplo, a concordância e as marcas  não-manuais. 



    \item \textbf{Tipos de sentenças}: diferentes tipos de sentenças são produzidos pela utilização de marcadores nas línguas de sinais, como as expressões não-manuais, e podem influenciar a forma com que as palavras são ordenadas nessas sentenças.
          Entre os principais tipos, podemos enumerar:

          % há alguns diferentes tipos de sentenças na língua de sinais, os quais são marcados através da utilização de expressões não-manuais em conjunto com a articulação dos sinais e podem alterar a ordem das palavras na sentença.

          % There are a few different sentence types in ASL. These sentence types are not the same as word order. Word order shows the order in which you can sign your words. Sentence types show how to use word order along with non-manual markers to form certain types of sentences.

          \begin{enumerate}
              \item \underline{Interrogativa}: é geralmente marcada por expressões que combinam movimentos de levantar ou abaixar as sobrancelhas, a inclinação da cabeça, ou a sustentação do último sinal articulado.
                    Observe na \autoref{fig:interrogativa-joao-gostar} (à direita) a ênfase fornecida pela expressão facial ao sinal interrogativo ``QUEM''.

                    \figura[p. 187]
                    {fig:interrogativa-joao-gostar} % Label
                    {capitulos/fundamentacao/imagens/interrogativa_joao_gostar} % Path
                    {height=3cm} % Size
                    {Sentença interrogativa ``JOÃO GOSTAR QUEM?''} % Caption
                    {quadros-2004-estudos-linguisticos} % Citation


              \item \underline{Declarativa}: denota uma declaração afirmativa, negativa, ou neutra, a qual também podem ser marcadas através de expressões faciais.
                    A \autoref{fig:topicalizada-futebol-gostar} (à direita) ilustra uma declaração afirmativa e a \autoref{fig:negacao-futebol-gostar-nao} (à direita) uma negativa.

                    % Declarative sentences are statements. These can be affirmative, negative, or neutral statements and each are recognized by the different non-manual markers that are used.


              \item \underline{Condicional}: construção condicional do tipo ``se \dots então'', que é marcada pela elevação das sobrancelhas seguida de uma expressão afirmativa ou interrogativa

                    A \autoref{fig:condicional-chover-jogo-cancelar} ilustra um exemplo equivalente à sentença ``se chover hoje, então o jogo será cancelado''.

                    \figura[p. 121]
                    {fig:condicional-chover-jogo-cancelar} % Label
                    {capitulos/fundamentacao/imagens/condicional_chover_jogo_cancelar} % Path
                    {height=4cm} % Size
                    {Sentença condicional ``CHOVER HOJE, JOGO CANCELAR''} % Caption
                    {jay-2011-dont-just-sign} % Citation


              \item \underline{Topicalizada}: estabelece uma voz passiva, movimentando o objeto (ou tópico em questão) para o início da sentença e transformando sua ordem para \textit{Objeto-Sujeito-Verbo} (OSV). Com isso, o objeto é geralmente demarcado por uma expressão não-manual diferente do restante da sentença.

                    Observe na \autoref{fig:topicalizada-futebol-gostar} o exemplo da sentença ``FUTEBOL JOÃO GOSTAR'', onde o termo futebol é o tópico central.

                    %   ocorre quando há o movimento do ``objeto'' para o início da sentença, transformando sua ordem para OSV. Isso cria uma voz passiva, diferente da voz ativa utilizada na estrutura SVO.
                    % A expressão facial utilizada para marcar o ``objeto'' difere do restante da sentença (vide \autoref{fig:topicalizada-futebol-gostar}).

                    \figura[p. 147]
                    {fig:topicalizada-futebol-gostar} % Label
                    {capitulos/fundamentacao/imagens/topicalizada_futebol_gostar} % Path
                    {height=3cm} % Size
                    {Frase topicalizada ``FUTEBOL JOÃO GOSTAR''} % Caption
                    {quadros-2004-estudos-linguisticos} % Citation

                    % Topicalization includes the movement of a syntactic element to the front (or beginning) of a sentence and highlighting it as “old or previously discussed” information.

                    % When you use the “object” part of the sentence as the topic of the sentence (OSV word order), this is called topicalization. The facial expression used for the “object” part of the sentence differs from the rest of the sentence. This creates a “passive voice” instead of the “active voice” that is used with SVO structure.
          \end{enumerate}


    \item \textbf{Negação}: existem diversas formas de construir negações de enunciados ou proposições nas línguas de sinais. Podemos enumerar entre as principais: sinalizar NÃO antes ou após outro sinal; balançar a cabeça em negação enquanto sinaliza; utilizar uma orientação oposta para alguns sinais; ou franzir a testa enquanto sinaliza.

          Observe na \autoref{fig:negacao-futebol-gostar-nao} o uso da expressão de balançar a cabeça em negação e a adição do sinal NÃO à parte final da sentença ``FUTEBOL JOÃO GOSTAR''.

          % todas as línguas devem possuir mecanismos para negar enunciados ou proposições expressas em sentido positivo. Na língua de sinais, você pode formar negações de diferentes formas, como sinalizando NÃO antes de uma palavra, balançando a cabeça enquanto sinaliza uma palavra (vide \autoref{fig:negacao-futebol-gostar-nao}), invertendo a orientação da mão para alguns sinais, ou franzindo a testa enquanto sinaliza uma palavra.

          % To form a negative, you can: • Sign NOT before the word. • Shake your head while signing the word. • Use reversal of orientation for some signs. • Frown while signing the word. Non-manual markers are a very important part of negation. For example, if you sign, “ME don’t-LIKE HAMBURGER,” a different facial expression can change the meaning to: “I really dislike hamburgers.”


          \figura[p. 147]
          {fig:negacao-futebol-gostar-nao} % Label
          {capitulos/fundamentacao/imagens/negacao_futebol_gostar_nao} % Path
          {height=3.5cm} % Size
          {Negação da sentença ``FUTEBOL JOÃO GOSTAR''} % Caption
          {quadros-2004-estudos-linguisticos} % Citation

          % All languages must have a way to express negation. For every utterance or proposition that is expressed in a positive sense, there is a mechanism for negating that positive proposition. For instance, “I have three apples” is a positive proposition.

\end{enumerate}




% \cite{hill-2019-sign-languages} ----------------------------
% Syntax
% Syntax is the study of the descriptive rules that are needed to build a sentence in a given language.
% Now we will look at some basic components of ASL syntax and learn how to build a sentence.

% Word order
% Syntatic structures with brow raise
%     Yes/no interrogative
%     Conditionals
%     Topicalization
% Wh-questions
% Negation


% \cite{jay-2011-dont-just-sign} -------------------------------
% Syntax
% Syntax is the study of constructing sentences. Syntax also refers to the rules and principles of sentence structure.
% In ASL, syntax is conveyed through word order and non-manual markers. This section can be confusing, so don’t get discouraged if you don’t understand the first time.

% • Word Order 
%         Word order with plain Verbs
%         Object-subject-verb word order
%         Word order without objects
%         Word order with directional Verbs
%         time-topic-comment
% • Sentence Types
%         Questions
%             Wh-questions
%             Yes/no questions
%             Rethorical questions
%         Declarative sentences
%             Affirmative Declarative ...
%             Negative Declarative ...
%             Neutral Declarative ...
%         Conditional sentences
%         Topicalization
%             Topicalized statements
%             Topicalized "Wh" question
% • Negation
%         Reversal of orientation
% • Pronouns and Indexing
%         Indexing on your non-dominant hand
%         Personal Pronouns
%         Possessive Pronouns
%         Directional Verbs
%         Plural Directional Verbs
% • Nouns
%         Pluralization
% • Adjectives
% • Auxiliary Verbs
% • Prepositions
% • Conjunctions
% • Articles




% \cite{quadros-2004-estudos-linguisticos} ------------------------
% A Sintaxe Espacial
% A língua de sinais brasileira, usada pela comunidade surda brasileira  espalhada por todo o País, é organizada espacialmente de forma tão complexa quanto às línguas orais-auditivas. Analisar alguns aspectos da sintaxe  de uma língua de sinais requer “enxergar” esse sistema que é visuoespacial  e não oral-auditivo. De certa forma, tal desafio apresenta certo grau de dificuldade aos lingüistas; no entanto, abre portas para as investigações no campo  da Teoria da Gramática enquanto manifestação possível da capacidade da  linguagem humana. A organização espacial dessa língua, assim como da  ASL – Língua de Sinais Americana – (Siple, 1978; Lillo-Martin, 1986; Fischer,  1990; Bellugi, Lillo-Martin, O’Grady e van Hoek, 1990), apresenta possibilidades de estabelecimento de relações gramaticais no espaço, através de diferentes formas.
% No espaço em que são realizados os sinais, o estabelecimento nominal e  o uso do sistema pronominal são fundamentais para tais relações sintáticas.  Qualquer referência usada no discurso requer o estabelecimento de um local  no espaço de sinalização (espaço definido na frente do corpo do sinalizador),  observando várias restrições. 










