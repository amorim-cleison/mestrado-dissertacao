\subsubsection{Sintaxe}
\label{linguistica-gramatica-sintaxe}

A sintaxe é o estudo da construção de sentenças numa linguagem, bem como dos princípios e regras descritivas que regem esse processo. Na língua de sinais, a sintaxe é transmitida através da ordem das palavras e das expressões não-manuais.
Observemos a seguir alguns dos componentes básicos da sintaxe dessa língua \cite{jay-2011-dont-just-sign,hill-2019-sign-languages}:

\begin{enumerate}
    \item \textbf{Ordem das palavras}:
    \item \textbf{Tipos de sentenças}:
        Questões
            Wh-questions
            Sim/não
            Retórica
        Condicionais
        Topicalização
    \item \textbf{Negação}:
    \item \textbf{Preposição}:
    \item \textbf{Conjunções}:
\end{enumerate}




% \cite{hill-2019-sign-languages} ----------------------------
% Syntax
% Syntax is the study of the descriptive rules that are needed to build a sentence in a given language.
% Now we will look at some basic components of ASL syntax and learn how to build a sentence.

% Word order
% Syntatic structures with brow raise
%     Yes/no interrogative
%     Conditionals
%     Topicalization
% Wh-questions
% Negation


% \cite{jay-2011-dont-just-sign} -------------------------------
% Syntax
% Syntax is the study of constructing sentences. Syntax also refers to the rules and principles of sentence structure.
% In ASL, syntax is conveyed through word order and non-manual markers. This section can be confusing, so don’t get discouraged if you don’t understand the first time.

% • Word Order 
%         Word order with plain Verbs
%         Object-subject-verb word order
%         Word order without objects
%         Word order with directional Verbs
%         time-topic-comment
% • Sentence Types
%         Questions
%             Wh-questions
%             Yes/no questions
%             Rethorical questions
%         Declarative sentences
%             Affirmative Declarative ...
%             Negative Declarative ...
%             Neutral Declarative ...
%         Conditional sentences
%         Topicalization
%             Topicalized statements
%             Topicalized "Wh" question
% • Negation
%         Reversal of orientation
% • Pronouns and Indexing
%         Indexing on your non-dominant hand
%         Personal Pronouns
%         Possessive Pronouns
%         Directional Verbs
%         Plural Directional Verbs
% • Nouns
%         Pluralization
% • Adjectives
% • Auxiliary Verbs
% • Prepositions
% • Conjunctions
% • Articles




% \cite{quadros-2004-estudos-linguisticos} ------------------------
% A Sintaxe Espacial
% A língua de sinais brasileira, usada pela comunidade surda brasileira  espalhada por todo o País, é organizada espacialmente de forma tão complexa quanto às línguas orais-auditivas. Analisar alguns aspectos da sintaxe  de uma língua de sinais requer “enxergar” esse sistema que é visuoespacial  e não oral-auditivo. De certa forma, tal desafio apresenta certo grau de dificuldade aos lingüistas; no entanto, abre portas para as investigações no campo  da Teoria da Gramática enquanto manifestação possível da capacidade da  linguagem humana. A organização espacial dessa língua, assim como da  ASL – Língua de Sinais Americana – (Siple, 1978; Lillo-Martin, 1986; Fischer,  1990; Bellugi, Lillo-Martin, O’Grady e van Hoek, 1990), apresenta possibilidades de estabelecimento de relações gramaticais no espaço, através de diferentes formas.
% No espaço em que são realizados os sinais, o estabelecimento nominal e  o uso do sistema pronominal são fundamentais para tais relações sintáticas.  Qualquer referência usada no discurso requer o estabelecimento de um local  no espaço de sinalização (espaço definido na frente do corpo do sinalizador),  observando várias restrições. 










