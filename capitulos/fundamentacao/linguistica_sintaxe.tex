\subsection{Sintaxe}
\label{sec:linguistica-sintaxe}

Sintaxe é o estudo da construção de sentenças a partir das palavras, bem como dos princípios e regras envolvidos nesse processo, afirmam \citeonline{hill-2019-sign-languages,jay-2011-dont-just-sign}.

Nas línguas de sinais, a sintaxe é comunicada através da ordem das palavras e da utilização das expressões não-manuais.
Observam-se a seguir alguns dos principais componentes presentes no processo de construção sintática dessas línguas \cite{jay-2011-dont-just-sign,hill-2019-sign-languages,quadros-2004-estudos-linguisticos}:

\begin{enumerate}
    \item \textbf{Ordem das palavras}: existem diferentes possibilidades de ordenação das palavras nas sentenças da língua de sinais. Apesar disso, a ordem \textit{Sujeito-Verbo-Objeto} (SVO) parece ser a mais básica dentre as demais.
          Ordenações como \textit{Objeto-Sujeito-Verbo} (OSV), \textit{Sujeito-Objeto-Verbo} (SOV) e \textit{Verbo-Sujeito-Objeto} (VOS) são derivadas daquela primeira e resultam de operações sintáticas específicas associadas a alguma marcação como a concordância de verbos ou as expressões não-manuais.


    \item \textbf{Tipos de sentenças}: diferentes tipos de sentenças são produzidos pela utilização de marcadores nas línguas de sinais, como as expressões não-manuais, e podem influenciar a forma com que as palavras são ordenadas nessas sentenças.
          Entre os principais tipos, podem-se enumerar:

          \begin{enumerate}
              \item \underline{Interrogativa}: é geralmente marcada por expressões que combinam movimentos de levantar ou abaixar as sobrancelhas, a inclinação da cabeça, ou a sustentação do último sinal articulado.
                    Observe na \autoref{fig:interrogativa-joao-gostar} (à direita) a ênfase fornecida pela expressão facial ao sinal interrogativo ``QUEM''.

                    \figura[p. 187]
                    {fig:interrogativa-joao-gostar} % Label
                    {capitulos/fundamentacao/imagens/interrogativa_joao_gostar} % Path
                    {height=3cm} % Size
                    {Sentença interrogativa ``JOÃO GOSTAR QUEM?''} % Caption
                    {quadros-2004-estudos-linguisticos} % Citation


              \item \underline{Declarativa}: denota uma declaração afirmativa, negativa, ou neutra, a qual também podem ser marcadas através de expressões faciais.
                    A \autoref{fig:topicalizada-futebol-gostar} (à direita) ilustra uma declaração afirmativa e a \autoref{fig:negacao-futebol-gostar-nao} (à direita) uma negativa.


              \item \underline{Condicional}: construção condicional do tipo ``se~\dots~então'', que é marcada pela elevação das sobrancelhas seguida de uma expressão afirmativa ou interrogativa

                    A \autoref{fig:condicional-chover-jogo-cancelar} ilustra um exemplo equivalente à sentença ``se chover hoje, então o jogo será cancelado''.

                    \figura[p. 121]
                    {fig:condicional-chover-jogo-cancelar} % Label
                    {capitulos/fundamentacao/imagens/condicional_chover_jogo_cancelar} % Path
                    {height=4cm} % Size
                    {Sentença condicional ``CHOVER HOJE, JOGO CANCELAR''} % Caption
                    {jay-2011-dont-just-sign} % Citation


              \item \underline{Topicalizada}: estabelece uma voz passiva, movimentando o objeto (ou tópico em questão) para o início da sentença e transformando sua ordem para \textit{Objeto-Sujeito-Verbo} (OSV). Com isso, o objeto é geralmente demarcado por uma expressão não-manual diferente do restante da sentença.

                    Observe na \autoref{fig:topicalizada-futebol-gostar} o exemplo da sentença ``FUTEBOL JOÃO GOSTAR'', onde o termo futebol é o tópico central.

                    \figura[p. 147]
                    {fig:topicalizada-futebol-gostar} % Label
                    {capitulos/fundamentacao/imagens/topicalizada_futebol_gostar} % Path
                    {height=3cm} % Size
                    {Frase topicalizada ``FUTEBOL JOÃO GOSTAR''} % Caption
                    {quadros-2004-estudos-linguisticos} % Citation

          \end{enumerate}


    \item \textbf{Negação}: existem diversas formas de construir negações de enunciados ou proposições nas línguas de sinais. Podem-se enumerar entre as principais: sinalizar NÃO antes ou após outro sinal; balançar a cabeça em negação enquanto sinaliza; utilizar uma orientação oposta para alguns sinais; ou franzir a testa enquanto sinaliza.

          Observe na \autoref{fig:negacao-futebol-gostar-nao} o uso da expressão de balançar a cabeça em negação e a adição do sinal NÃO à parte final da sentença ``FUTEBOL JOÃO GOSTAR''.

          \figura[p. 147]
          {fig:negacao-futebol-gostar-nao} % Label
          {capitulos/fundamentacao/imagens/negacao_futebol_gostar_nao} % Path
          {height=3.5cm} % Size
          {Negação da sentença ``FUTEBOL JOÃO GOSTAR''} % Caption
          {quadros-2004-estudos-linguisticos} % Citation

\end{enumerate}

