\section{Aprendizagem de máquina}
\label{sec:am}

% Esta seção introduzirá conceitos básicos de \acrfull{ml} e desenvolverá uma discussão sobre as principais arquiteturas utilizadas na área de \acrfull{nlp} para lidar com tarefas envolvendo linguagens, como na abordagem proposta por este trabalho.


% \subsection{Breve introdução}
% \label{sec:am-introducao}

\citeonline{quiza-2012-finite-element,russell-2010-modern-approach} definem formalmente a \acrfull{ia} como um ramo da ciência da computação que visa estudar e projetar agentes inteligentes, ou seja, sistemas capazes de perceber seu ambiente e realizar ações que maximizam suas chances de sucesso.
A \acrfull{ml}, por sua vez, é definida por \citeonline{murphy-2012-ml-probabilistic,goodfellow-2016-deep-learning} como uma área que estuda um conjunto de algoritmos de \acrshort{ia} capazes de detectar automaticamente padrões a partir de dados e, em seguida, utilizá-los para prever dados futuros ou realizar tomadas de decisões.
Esses algoritmos comumente adotam modelos estatísticos para realizar análises e inferências sobre esses dados e não necessitam que instruções explícitas sejam programadas para que consigam detectar esses padrões.


% \citeonline{murphy-2012-ml-probabilistic} ressalta que vivemos na era do \textit{big data} (ou do grande volume de dados). 
% Por exemplo, há mais de 1 trilhão de páginas na web mapeadas pelo Google; o YouTube recebe cerca de 1 hora de conteúdo de vídeo a cada segundo; o Walmart processa mais de 1 milhão de transações por hora, totalizando mais de 2,5 petabytes de informações em suas bases de dados; e assim por diante.
% Todo esse volume de dados demanda técnicas automatizadas de análise e extração de informações relevantes e esse é um exemplo de contexto onde a \acrfull{ia} e, mais especificamente a área de \acrfull{ml}, tem desempenhado um papel fundamental atualmente.


% \cite{murphy-2012-ml-probabilistic}
% We are entering the era of big data. For example, there are about 1 trillion web pages1; one hour of video is uploaded to YouTube every second, amounting to 10 years of content every day2; the genomes of 1000s of people, each of which has a length of 3.8 × 109 base pairs, have been sequenced by various labs; Walmart handles more than 1M transactions per hour and has databases containing more than 2.5 petabytes (2.5 × 1015) of information (Cukier 2010); and so on. 
% This deluge of data calls for automated methods of data analysis, which is what machine learning provides. In particular, we define machine learning as a set of methods that can automatically detect patterns in data, and then use the uncovered patterns to predict future data, or to perform other kinds of decision making under uncertainty (such as planning how to collect more data!)

% \cite{goodfellow-2016-deep-learning}
% A machine learning algorithm is an algorithm that is able to learn from data.

% google / oxford
% the use and development of computer systems that are able to learn and adapt without following explicit instructions, by using algorithms and statistical models to analyze and draw inferences from patterns in data.


Segundo \citeonline{mitchell-1997-ml,bishop-2006-pattern-recognition}, os algoritmos de \acrshort{ml} constituem-se de um tipo de estrutura denominado \acrfull{ann}. Essa estrutura é inspirada pela observação dos sistemas de aprendizado biológico, os quais são compostos de redes muito complexas de neurônios interconectados entre si, conforme ilustra a \autoref{fig:exemplo-neuronios}.

\figura[p. 40]
    {fig:exemplo-neuronios} % Label
    {capitulos/fundamentacao/imagens/exemplo_neuronios} % Path
    {height=5cm} % Size
    {Exemplo de neurônios interconectados, que compõem os sistemas biológicos de aprendizado} % Caption
    {quiza-2012-finite-element} % Citation

De uma maneira análoga, a \acrshort{ann} é composta por um conjunto de unidades simples densamente interconectadas e organizadas em camadas, que exercem o papel dos neurônios, conforme ilustra a \autoref{fig:rede-neural}.
Cada unidade recebe como entrada um conjunto de números reais -- geralmente produzidos por unidades da camada anterior -- e produz como saída um outro conjunto de números reais -- que poderão ser utilizados pelas unidades da camada seguinte.
Além disso, cada conexão entre essas unidades recebe um peso, que é um número real que representa a importância daquela conexão. Esses pesos são ajustados à medida em que a \acrshort{ann} adapta-se aos padrões existentes nos dados, ou seja, conforme ela aprende acerca do domínio do problema em questão.

\figura[p. 43]
    {fig:rede-neural} % Label
    {capitulos/fundamentacao/imagens/rede_neural_2} % Path
    {height=7cm} % Size
    {Exemplo de \acrfull{ann} com 4 camadas interconectadas: os dados \(\{X_1, X_2, \dots, X_k\}\) são recebidos pelas unidades da camada de entrada (ou \textit{input layer}), processados pelas camadas ocultas (ou \textit{hidden layers}) e pela camada de saída (ou \textit{output layer}) que, por sua vez, produz as saídas \(\{Y_1, Y_2, \dots, Y_k\}\)} % Caption
    {quiza-2012-finite-element} % Citation

% Network diagram for the two layer neural network corresponding to (5.7). The input, hidden, and output variables are represented by nodes, and the weight parameters are represented by links between the nodes, in which the bias parameters are denoted by links coming from additional input and hidden variables x0 and z0. Arrows denote the direction of information flow through the network during forward propagation.

% \cite{mitchell-1997-ml}
% The study of artificial neural networks (ANNs) has been inspired in part by the observation that biological learning systems are built of very complex webs of interconnected neurons. In rough analogy, artificial neural networks are built out of a densely interconnected set of simple units, where each unit takes a number of real-valued inputs (possibly the outputs of other units) and produces a single real-valued output (which may become the input to many other units).


\citeonline{lecun-2015-deep-learning,russell-2010-modern-approach} comentam que a forma mais utilizada de aprendizado pelas \acrshortpl{ann} é o aprendizado supervisionado, no qual a partir da observação de um conjunto de amostras de pares de entrada-saída, a rede aprende uma função que mapeia daquele tipo de entrada para aquela saída respectiva, ajustando seus pesos de acordo.

Por exemplo, imagine a construção de uma \acrshort{ann} para identificar em imagens a existência de casas, carros, pessoas ou animais de estimação. O primeiro passo consistiria em coletar um grande conjunto de imagens de casas, carros, pessoas e animais de estimação, cada uma identificada quanto à existência de um desses elementos -- formando assim os pares de entrada-saída mencionados acima que, nesse contexto, seriam pares de ``imagem-elemento''.
Esse conjunto de imagens (ou amostras) é dividido em dois subconjuntos: um de treinamento -- utilizado para que a \acrshort{ann} aprenda a identificar esses elementos --, e outro de testes -- utilizado para avaliar o sucesso da rede nesse processo de aprendizado.

O próximo passo consistiria em realizar o treinamento da \acrshort{ann}, para o qual será aplicada a técnica de aprendizado supervisionado e utilizado o subconjunto de imagens de treinamento estabelecido acima.
Nesse processo, a rede visitará cada uma dessas imagens e produzirá uma pontuação com a probabilidade de existência um dos elementos possíveis naquela imagem -- casa, carro, pessoa ou animal de estimação.
Objetiva-se que aquele elemento que está presente na imagem receba a maior pontuação e os demais elementos recebam pontuações menores, mas isso apenas ocorrerá à medida que os pesos da rede forem sendo ajustados para tal.
Esse processo de ajuste de pesos é denominado otimização da \acrshort{ann} e, para que ele ocorra, é necessário que seja introduzida uma função capaz de computar o erro ou a distância entre as pontuações informadas pela rede e aquelas que refletem o padrão correto desejado.
A essa função atribui-se o nome de função objetivo -- a qual também é conhecida como função de perda ou função de custo.

% Para traduzir esse objetivo em termos numéricos e viabilizar o ajuste dos pesos da rede, será introduzida uma ``função objetivo'' -- que também é conhecida como ``função de perda'' ou ``função de custo''. Ela será responsável por computar o erro ou a distância entre as pontuações informadas pela rede e aquelas que refletem o padrão correto desejado.
% Esse erro será então propagado através das camadas e conexões da rede, a qual modificará os pesos respectivos para reduzir esse erro e aproximar seu comportamento do objetivo acima.
% Isso é repetido para muitos pequenos \textit{batches} (ou lotes) de amostras do subconjunto de treinamento até que o erro pare de diminuir. 


% Otimização
\citeonline{lecun-2015-deep-learning,goodfellow-2016-deep-learning} afirmam que um dos algoritmos de otimização mais utilizados é o \acrfull{sgd}.
De uma forma simplificada, ele consiste num processo de apresentar pequenos \textit{batches} (ou lotes) de amostras para a \acrshort{ann}, computar as respectivas pontuações e erros (pela função objetivo), computar os gradientes para cada um dos pesos da rede (os quais indicam o quanto o erro aumentaria ou diminuiria se os pesos fossem ajustados em uma pequena quantidade) e ajustar os pesos na direção oposta à dos gradientes.
Esse processo é repetido para vários \textit{batches} de amostras do subconjunto de treinamento até que o erro médio pare de diminuir.
Uma vez que o menor erro médio é encontrado, assume-se que o treinamento da \acrshort{ann} está finalizado.

% De uma maneira simplificada, ele consiste em apresentar amostras de dados à rede, calcular as saídas e os erros respectivos (fornecidos pela função objetivo), calcular o gradiente médio para essas amostras (obtido a partir dos gradientes individuais calculados para cada peso e que, por sua vez, indicam em que quantidade o erro aumentaria ou diminuiria se aquele peso fosse aumentado em uma pequena quantidade) e ajustar os pesos de acordo com esses gradientes (na direção oposta aos gradientes calculados). 

% To properly adjust the weight vector, the learning algorithm computes a gradient vector that, for each weight, indicates by what amount the error would increase or decrease if the weight were increased by a tiny amount. The weight vector is then adjusted in the opposite direction to the gradient vector.
% The objective function, averaged over all the training examples, can be seen as a  kind of hilly landscape in the high-dimensional space of weight values. The negative gradient vector indicates the direction of steepest descent in this landscape, taking it closer to a minimum, where the output error is low on average.
% In practice, most practitioners use a procedure called stochastic gradient descent (SGD). This consists of showing the input vector for a few examples, computing the outputs and the errors, computing the average gradient for those examples, and adjusting the weights accordingly. The process is repeated for many small sets of examples from the training set until the average of the objective function stops decreasing. It is called stochastic because each small set of examples gives a noisy estimate of the average gradient over all examples. This simple procedure usually finds a good set of weights surprisingly quickly when compared with far more elaborate optimization techniques18. After training, the performance of the system is measured on a different set of examples called a test set. This serves to test the generalization ability of the machine — its ability to produce sensible answers on new inputs that it has never seen during training.


Por fim, o desempenho da \acrshort{ann} será avaliado utilizando-se a mesma função objetivo para calcular o erro médio porém para um subconjunto diferente de amostras -- o subconjunto de testes, estabelecido acima.
Isso permitirá conhecer a capacidade de generalização da rede, ou seja, o quanto ela é bem-sucedida ao tentar produzir respostas sensatas para amostras nunca antes observadas.
% Uma vez que o treinamento da \acrshort{ann} é finalizado, o seu desempenho é então avaliado calculando-se o erro médio para as amostras do subconjunto de testes, o que é feito utilizando-se a mesma função objetivo acima.

% Ao final do treinamento, o desempenho da rede será então avaliado utilizando-se o subconjunto de amostras de testes, que foi separado inicialmente. Isso permitirá testar a capacidade de ``generalização'' da rede que, em outras palavras, significa sua capacidade de produzir respostas sensatas para novas amostras não observadas durante o processo de treinamento.


% Cross-validation

Apesar disso, \citeonline{goodfellow-2016-deep-learning,bishop-2006-pattern-recognition} argumentam que essa divisão das amostras em apenas dois subconjuntos fixos, de treinamento e de testes, pode ser problemático principalmente se isso resultar em um subconjunto de testes pequeno.
Isso implicaria incerteza estatística em torno do erro médio de testes estimado, tornando difícil afirmar que um algoritmo funciona melhor que outro na tarefa em questão.
Para lidar com esse problema, é possível que seja aplicado um procedimento conhecido como validação cruzada, o qual permite com que todas as amostras sejam utilizadas nesse processo de testes ou estimativa de erro médio.

Os algoritmos de validação cruzada baseiam-se na ideia de repetir as etapas de treinamento e testes utilizando subconjuntos diferentes de amostras, escolhidos aleatoriamente a partir do conjunto original, e computando o erro a partir da média dos erros obtidos em cada repetição. 
O mais comum desses algoritmos é o \textit{\(k\)-fold}, que funciona dividindo o conjunto de dados original em \(k\) subconjuntos não sobrepostos e realizando \(k\) repetições de treinamento e testes. 
A cada repetição \(i\), o \(i\)-ésimo subconjunto é utilizado como subconjunto de testes e o restante dos dados é utilizado como o de treinamento, e assim por diante.
O erro médio é então estimado tomando-se a média dos erros obtidos nessas \(k\) repetições.


% Apesar disso, \citeonline{goodfellow-2016-deep-learning,bishop-2006-pattern-recognition} argumentam que dividir o conjunto de dados em um subconjunto fixo de treinamento e outro de testes, conforme realizado acima, pode ser problemático se resultar em um subconjunto de testes pequeno -- principalmente no contexto de redes neurais profundas.
% Um subconjunto pequeno de testes implica incerteza estatística em torno do erro médio estimado de teste, tornando difícil afirmar que um algoritmo funciona melhor que outro na tarefa em questão.
% Se o conjunto de dados possui centenas de milhares de amostras ou mais, isso não é um problema sério; porém, quando esse conjunto é muito pequeno, são necessários procedimentos alternativos que possibilitem utilizar todas as amostras na estimativa do erro médio de teste, ao preço de um aumento do custo computacional.
% Esses procedimentos são geralmente conhecidos como ``validação cruzada'' e baseiam-se na ideia de repetir as etapas de treinamento e testes utilizando diferentes subconjuntos escolhidos aleatoriamente a partir do conjunto de dados original, computando então o erro médio final a partir dos erros obtidos em cada repetição. 
% O mais comum deles é o \textit{\(k\)-fold}, no qual uma partição do conjunto de dados é formada dividindo-o em \(k\) subconjuntos não sobrepostos. O erro de teste pode então ser estimado tomando o erro de teste médio dessas \(k\) tentativas. A cada tentativa \(i\), o \(i\)-ésimo subconjunto dos dados é usado como subconjunto de testes e o restante dos dados é utilizado como subconjunto de treinamento, e assim por diante.


% Busca de parâmetros

Segundo \citeonline{goodfellow-2016-deep-learning}, a maioria das \acrshortpl{ann} também apresenta parâmetros que controlam diferentes aspectos do seu comportamento, os quais são denominados hiperparâmetros.
Exemplos deles incluem o número de camadas da rede, as dimensões dessas camadas, a taxa de aprendizagem e o tamanho dos \textit{batches} adotados no treinamento, entre outros que variam de acordo com o tipo de \acrshort{ann}.
Contudo, diferentemente do que ocorre para os pesos da rede, esses hiperparâmetros não são aprendidos automaticamente durante o processo de treinamento. 
Ao contrário, a escolha dos valores de hiperparâmetros que melhor otimizam a capacidade da rede é geralmente uma tarefa complexa porque demanda uma compreensão mais profunda acerca do papel que eles exercem sobre o desempenho da rede.

% \citeonline{goodfellow-2016-deep-learning} afirmam que a maioria das \acrshortpl{ann} possuem diversos ``hiperparâmetros'', que são configurações utilizadas para controlar diferentes aspectos do comportamento dessas redes como, por exemplo, seu custo de tempo e memória, a qualidade do modelo obtido a partir do treinamento ou ainda a sua capacidade de generalização. 

% Há duas abordagens básicas para a escolha desses hiperparâmetros: selecionando-os manualmente, o que demanda uma compreensão mais aprofundada do papel desempenhado por cada um deles e de como influenciam a capacidade de generalização da rede; ou utilizando algoritmos de seleção automática, que reduzem a demanda por essa compreensão, mas que geralmente introduzem um custo computacional mais elevado.

Por conta disso, é comum que sejam adotadas abordagens automáticas para essa finalidade, a exemplo do algoritmo \textit{grid search} (ou busca de grade) -- que será utilizado mais adiante nos experimentos deste trabalho.
O \textit{grid search} funciona com base num conjunto pequeno e finito de valores a serem explorados para cada um dos hiperparâmetros da rede, os quais são informados pelo usuário ao início da busca.
Com base nisso, o algoritmo gera o produto cartesiano de todas as combinações possíveis desses valores de hiperparâmetros e, em seguida, treina uma rede neural para cada uma dessas combinações.
Ao final desse processo, a rede neural que apresentar o menor erro médio é então considerada aquela que encontrou a melhor combinação de valores de hiperparâmetros.
Obviamente, um problema relevante desse algoritmo é o seu custo computacional, que cresce exponencialmente em função do número de hiperparâmetros e de valores a serem explorados.

Para essa seleção automática de hiperparâmetros é introduzido um novo subconjunto de amostras denominado subconjunto de validação.
Ele é importante para evitar que nesse processo sejam utilizadas as mesmas amostras para treinar e validar o desempenho das redes para cada combinação de hiperparâmetros.
Sendo assim, esse subconjunto é criado a partir de uma divisão do subconjunto de treinamento na qual, tipicamente, 20\% das amostras do subconjunto de treinamento são direcionadas para validação de hiperparâmetros e 80\% permanecem sendo utilizadas para treinamento.
O subconjunto de testes, por sua vez, permanece inalterado e não é envolvido na seleção de hiperparâmetros.

% Nessa otimização, também não é possível que sejam utilizadas as mesmas amostras do subconjunto de treinamento, uma vez que isso tornaria a rede enviesada a esse subconjunto e comprometeria sua capacidade de generalização para novas amostras.
% Para solucionar isso, é estabelecido um novo subconjunto de amostras denominado ``subconjunto de validação'', o qual é criado a partir de uma nova divisão do subconjunto inicial de treinamento -- nesse contexto, tipicamente 20\% das amostras são destinadas para otimização dos hiperparâmetros e 80\% permanecem sendo utilizadas no treinamento da rede.
% Uma vez que a otimização dos hiperparâmetros está completa, o desempenho da rede é avaliado utilizando-se o mesmo subconjunto inicial de testes, conforme discutido anteriormente.

% Um dos algoritmos mais simples para seleção automática de hiperparâmetros é o \textit{grid search} (ou busca de grade), o qual será utilizado mais adiante nos experimentos deste trabalho. Nele, o usuário indica um conjunto pequeno e finito de valores a serem explorados para cada um dos hiperparâmetros em questão. A partir disso, o algoritmo gera o produto cartesiano de todas as combinações possíveis de valores de hiperparâmetros e, em seguida, treina uma rede neural para cada uma dessas combinações. Aquela rede neural que apresentar o menor erro para as amostras do subconjunto de validação é então assumida como sendo a que encontrou a melhor combinação de valores de hiperparâmetros.
% No entanto, um problema relevante desse algoritmo é o seu custo computacional, que cresce exponencialmente em função do número de hiperparâmetros e de valores a serem explorados.

% \citeonline{goodfellow-2016-deep-learning} prosseguem afirmando que a maioria das redes neurais possuem configurações denominadas ``hiperparâmetros'', que controlam muitos aspectos do comportamento dessas redes, como custo o computacional e de memória, a qualidade do modelo recuperado pelo processo de treinamento ou ainda a habilidade de realizar inferências corretas quando expostas a novas entradas. 
% Exemplos dessas configurações incluem a taxa de aprendizagem, a número ou o tamanho das camadas da rede, o tipo de regularização aplicada, entre outros. 
% Diferentemente de parâmetros como os pesos, essas configurações não são aprendidas automaticamente durante o treinamento da rede e comumente são difíceis de serem otimizadas.
% Nessa otimização, também não é possível que sejam utilizadas as mesmas amostras do subconjunto de treinamento, uma vez que isso tornaria a rede enviesada a esse subconjunto e comprometeria sua capacidade de generalização para novas amostras.
% Para solucionar isso, é estabelecido um novo subconjunto de amostras denominado ``subconjunto de validação'', o qual é criado a partir de uma nova divisão do subconjunto inicial de treinamento -- nesse contexto, tipicamente 20\% das amostras são destinadas para otimização dos hiperparâmetros e 80\% permanecem sendo utilizadas no treinamento da rede.
% Uma vez que a otimização dos hiperparâmetros está completa, o desempenho da rede é avaliado utilizando-se o mesmo subconjunto inicial de testes, conforme discutido anteriormente.

% Existem duas abordagens básicas para otimização desses hiperparâmetros: selecionando-os manualmente, o que requer uma compreensão do papel desempenhado por cada um deles e de como as redes neurais alcançam uma boa capacidade de generalização; ou utilizando algoritmos selecioná-los automaticamente, o que reduz a demanda por compreender conceitos avançados, mas que geralmente introduz um custo computacional maior.
% Um dos algoritmos mais simples para seleção automática de hiperparâmetros é o \textit{grid search} (ou busca de grade), o qual será utilizado mais adiante nos experimentos deste trabalho. Nele, o usuário indica um conjunto pequeno e finito de valores a serem explorados para cada um dos hiperparâmetros em questão. A partir disso, o algoritmo gera o produto cartesiano de todas as combinações possíveis de valores de hiperparâmetros e, em seguida, treina uma rede neural para cada uma dessas combinações. Aquela rede neural que apresentar o menor erro para as amostras do subconjunto de validação é então assumida como sendo a que encontrou a melhor combinação de valores de hiperparâmetros.
% No entanto, um problema relevante desse algoritmo é o seu custo computacional, que cresce exponencialmente em função do número de hiperparâmetros e de valores a serem explorados.





% Dentro da área de \acrshort{nlp}, tarefas que lidam com a linguagem num contexto semelhante ao que estamos endereçando neste trabalho comumente adotam arquiteturas conhecidas como \textit{Encoder-Decoder} (Codificador-Decodificador) ou \textit{Sequence-to-Sequence} (Sequência-para-Sequência) \cite{cho-2014-encoder-decoder,sutskever-2014-seq-to-seq}.
% Essas arquiteturas são compatíveis com diferentes tipos de modelagens sequenciais onde a sequência de saída é uma função complexa da sequência completa de entrada e ambas podem possuir comprimentos e ordens distintas \cite{jurafsky-2022-speech-lang-processing,goodfellow-2016-deep-learning}.

% O \textit{Encoder-Decoder} é composto por uma rede codificadora que recebe uma sequência de entrada e gera uma representação contextualizada dela -- que seria o contexto. Essa representação é então passada para um decodificador que produz uma sequência de saída específica para a tarefa em questão, conforme ilustra a \autoref{fig:encoder-decoder-arquitetura}. Uma otimização proposta por \citeonline{bahdanau-2015-mt-align-translate} também adota uma camada de \textit{attention} antes do decodificador para eliminar um gargalo observado ali.
% Por fim, o \textit{Encoder-Decoder} pode ser implementado utilizando-se \acrshortpl{rnn} e os \textit{Transformers}, por sua vez, já possuem uma arquitetura baseada nele \cite{jurafsky-2022-speech-lang-processing}.



% Encoder-decoder or sequence-to-sequence models are used for a different kind of sequence modeling in which the output sequence is a complex function of the entire input sequencer; we must map from a sequence of input words or tokens to a sequence of tags that are not merely direct mappings from individual words. 
% Machine translation is exactly such a task: the words of the target language don’t necessarily agree with the words of the source language in number or order.

% Encoder-decoder networks are very successful at handling these sorts of complicated cases of sequence mappings. Indeed, the encoder-decoder algorithm is not just for MT; it’s the state of the art for many other tasks where complex mappings between two sequences are involved. These include summarization (where we map from a long text to its summary, like a title or an abstract), dialogue (where we map from what the user said to what our dialogue system should respond), semantic parsing (where we map from a string of words to a semantic representation like logic or SQL), and many others.
% Encoder-decoder networks, or sequence-to-sequence networks, are models capable of generating contextually appropriate, arbitrary length, output sequences. Encoder-decoder networks have been applied to a very wide range of applications including machine translation, summarization, question answering, and dialogue.

% \cite{goodfellow-2016-deep-learning}
% This comes up inmany applications, such as speech recognition, machine translation and questionanswering, where the input and output sequences in the training set are generallynot of the same length (although their lengths might be related).








% This chapter introduces two important deep learning architectures designed to address these challenges: recurrent neural networks and transformer networks. Both approaches have mechanisms to deal directly with the sequential nature of language that allow them to capture and exploit the temporal nature of language. The recurrent network offers a new way to represent the prior context, allowing the model’s decision to depend on information from hundreds of words in the past. The transformer offers new mechanisms (self-attention and positional encodings) that help represent time and help focus on how words relate to each other over long distances.


% LSTM
% LSTM networks have been shown to learn long-term dependencies more easilythan the simple recurrent architectures, first on artificial datasets designed fortesting the ability to learn long-term dependencies (Bengio et al., 1994; Hochreiterand Schmidhuber, 1997; Hochreiter et al., 2001), then on challenging sequenceprocessing tasks where state-of-the-art performance was obtained (Graves, 2012;Graves et al., 2013; Sutskever et al., 2014). Variants and alternatives to the LSTMthat have been studied and used are discussed next.
% The LSTM has been found extremely successfulin many applications, such as unconstrained handwriting recognition (Graveset al., 2009), speech recognition (Graves et al., 2013; Graves and Jaitly, 2014),handwriting generation (Graves, 2013), machine translation (Sutskever et al., 2014),image captioning (Kiros et al., 2014b; Vinyals et al., 2014b; Xu et al., 2015), andparsing (Vinyals et al., 2014a).
% \cite{goodfellow-2016-deep-learning}


% GRU
% Which pieces of the LSTM architecture are actually necessary? What other successful architectures could be designed that allow the network to dynamicallycontrol the time scale and forgetting behavior of different units? Some answers to these questions are given with the recent work on gated RNNs, whose units are also known as gated recurrent units, or GRUs (Cho et al., 2014b;Chung et al., 2014, 2015a; Jozefowicz et al., 2015; Chrupala et al., 2015).
% \cite{goodfellow-2016-deep-learning}

% The main difference with the LSTM is that a single gating unit simultaneously controls the forgetting factor and the decision to update the state unit.
% \cite{goodfellow-2016-deep-learning}

% This evolution has recently led to a novel architecture called Gated Recurrent Unit (GRU) [8], that simplifies the complex LSTM cell design.
% [...]
% A noteworthy attempt to simplify LSTMs has recently led to a novel model called Gated Recurrent Unit (GRU) [8], [47], that is based on just two multiplicative gates.
% \cite{ravanelli-2018-li-gru}


% TRANSFORMER
% \cite{jurafsky-2022-speech-lang-processing}
% transformers – an approach to sequence processing that eliminates recurrent connections and returns to architectures reminiscent of the fully connected networks described earlier in Chapter 7.
% Transformers map sequences of input vectors (x1; :::;xn) to sequences of output vectors (y1; :::;yn) of the same length. 
% Transformers are made up of stacks of transformer blocks, which are multilayer networks made by combining simple linear layers, feedforward networks, and self-attention layers, the key innovation of transformers. Self-attention allows a network to directly extract and use information from arbitrarily large contexts without the need to pass it through intermediate recurrent connections as in RNNs. We’ll start by describing how self-attention works and then return to how it fits into larger transformer blocks.


% \cite{vaswani-2017-transformer}
% In this work we propose the Transformer, a model architecture eschewing recurrence and instead relying entirely on an attention mechanism to draw global dependencies between input and output. The Transformer allows for significantly more parallelization and can reach a new state of the art in translation quality after being trained for as little as twelve hours on eight P100 GPUs.


% \cite{wolf-2020-transformers}
% Transformer architectures have facilitated building higher-capacity models and pretraining has made it possible to effectively utilize this capacity for a wide variety of tasks.

% The Transformer (Vaswani et al., 2017) has rapidly become the dominant architecture for natural language processing, surpassing alternative neural models such as convolutional and recurrent neural networks in performance for tasks in both natural language understanding and natural language generation. The architecture scales with training data and model size, facilitates efficient parallel training, and captures long-range sequence features



% 10. MACHINE TRANSLATION AND ENCODER-DECODER MODELS
% \cite{jurafsky-2022-speech-lang-processing}

% This chapter introduces machine translation (MT), the use of computers to translate from one language to another.
% Of course translation, in its full generality, such as the translation of literature, or poetry, is a difficult, fascinating, and intensely human endeavor, as rich as any other area of human creativity.
% Machine translation in its present form therefore focuses on a number of very practical tasks. Perhaps the most common current use of machine translation is information for information access.

% Another common use of machine translation is to aid human translators. MT systems are routinely used to produce a draft translation that is fixed up in a post-editing phase by a human translator. This task is often called computer-aided translation or CAT. CAT is commonly used as part of localization: the task of adapting content or a product to a particular language community.

% Finally, a more recent application of MT is to in-the-moment human communication needs. This includes incremental translation, translating speech on-the-fly before the entire sentence is complete, as is commonly used in simultaneous interpretation. Image-centric translation can be used for example to use OCR of the text on a phone camera image as input to an MT system to translate menus or street signs.

% The standard algorithm for MT is the encoder-decoder network, also called the sequence to sequence network, an architecture that can be implemented with RNNs or with Transformers. We’ve seen in prior chapters that RNN or Transformer architecture can be used to do classification (for example to map a sentence to a positive or negative sentiment tag for sentiment analysis), or can be used to do sequence labeling (for example to assign each word in an input sentence with a part-of-speech, or with a named entity tag). For part-of-speech tagging, recall that the output tag is associated directly with each input word, and so we can just model the tag as output yt for each input word xt .
% Encoder-decoder or sequence-to-sequence models are used for a different kind of sequence modeling in which the output sequence is a complex function of the entire input sequencer; we must map from a sequence of input words or tokens to a sequence of tags that are not merely direct mappings from individual words. 
% Machine translation is exactly such a task: the words of the target language don’t necessarily agree with the words of the source language in number or order.
% [...]
% Encoder-decoder networks are very successful at handling these sorts of complicated cases of sequence mappings. Indeed, the encoder-decoder algorithm is not just for MT; it’s the state of the art for many other tasks where complex mappings between two sequences are involved. These include summarization (where we map from a long text to its summary, like a title or an abstract), dialogue (where we map from what the user said to what our dialogue system should respond), semantic parsing (where we map from a string of words to a semantic representation like logic or SQL), and many others.

% 10.2 The Encoder-Decoder Model
% Encoder-decoder networks, or sequence-to-sequence networks, are models capable of generating contextually appropriate, arbitrary length, output sequences. Encoder-decoder networks have been applied to a very wide range of applications including machine translation, summarization, question answering, and dialogue.
% The key idea underlying these networks is the use of an encoder network that takes an input sequence and creates a contextualized representation of it, often called the context. This representation is then passed to a decoder which generates a task specific output sequence. Fig. 10.3 illustrates the architecture
% [FIGURE]

% Encoder-decoder networks consist of three components:
% 1. An encoder that accepts an input sequence, xn1, and generates a corresponding sequence of contextualized representations, hn1. LSTMs, convolutional networks, and Transformers can all be employed as encoders.
% 2. A context vector, c, which is a function of hn1, and conveys the essence of the input to the decoder.
% 3. A decoder, which accepts c as input and generates an arbitrary length sequence of hidden states hm1, from which a corresponding sequence of output states ym1, can be obtained. Just as with encoders, decoders can be realized by any kind of sequence architecture.

% 10.3 Encoder-Decoder with RNNs
% [...]

% 10.4 Attention
% The simplicity of the encoder-decoder model is its clean separation of the encoder—
% which builds a representation of the source text—from the decoder, which uses this
% context to generate a target text. In the model as we’ve described it so far, this
% context vector is hn, the hidden state of the last (nth) time step of the source text.
% This final hidden state is thus acting as a bottleneck: it must represent absolutely
% everything about the meaning of the source text, since the only thing the decoder
% knows about the source text is what’s in this context vector (Fig. 10.8). Information
% at the beginning of the sentence, especially for long sentences, may not be equally
% well represented in the context vector.
% The attention mechanism is a solution to the bottleneck problem, a way of
% allowing the decoder to get information from all the hidden states of the encoder,
% not just the last hidden state.

% The idea of attention is instead to create the single fixed-length vector c by taking
% a weighted sum of all the encoder hidden states. The weights focus on (‘attend
% to’) a particular part of the source text that is relevant for the token the decoder is
% currently producing. Attention thus replaces the static context vector with one that
% is dynamically derived from the encoder hidden states, different for each token in
% decoding.
% [...]
% The weights Ws, which are then trained during normal end-to-end training, give the
% network the ability to learn which aspects of similarity between the decoder and
% encoder states are important to the current application. This bilinear model also
% allows the encoder and decoder to use different dimensional vectors, whereas the
% simple dot-product attention requires that the encoder and decoder hidden states
% have the same dimensionality.

% 10.5 Beam Search
% [...]

% 10.6 Encoder-Decoder with Transformers
% The encoder-decoder architecture can also be implemented using transformers (rather
% than RNN/LSTMs) as the component modules. At a high-level, the architecture,
% sketched in Fig. 10.15, is quite similar to what we saw for RNNs. It consists of an
% encoder that takes the source language input words X = x1; :::;xT and maps them
% to an output representation Henc = h1; :::;hT ; usually via N = 6 stacked encoder
% blocks. The decoder, just like the encoder-decoder RNN, is essentially a conditional
% language model that attends to the encoder representation and generates the target
% words one by one, at each timestep conditioning on the source sentence and the
% previously generated target language words.
% [IMAGE]
% But the components of the architecture differ somewhat from the RNN and also
% from the transformer block we’ve seen. First, in order to attend to the source language,
% the transformer blocks in the decoder has an extra cross-attention layer.
% Recall that the transformer block of Chapter 9 consists of a self-attention layer that
% attends to the input from the previous layer, followed by layer norm, a feed forward
% layer, and another layer norm. The decoder transformer block includes an
% extra layer with a special kind of attention, cross-attention (also sometimes called
% encoder-decoder attention or source attention). Cross-attention has the same form
% as the multi-headed self-attention in a normal transformer block, except that while
% the queries as usual come from the previous layer of the decoder, the keys and values
% come from the output of the encoder.




% \cite{goodfellow-2016-deep-learning}
% https://www.deeplearningbook.org/contents/rnn.html

% 10.4 Encoder-Decoder Sequence-to-Sequence Architectures
% Here we discuss how an RNN can be trained to map an input sequence to anoutput sequence which is not necessarily of the same length. This comes up inmany applications, such as speech recognition, machine translation and questionanswering, where the input and output sequences in the training set are generallynot of the same length (although their lengths might be related).
% [IMAGE]
% The simplest RNN architecture for mapping a variable-length sequence toanother variable-length sequence was first proposed by Cho et al. (2014a) [https://aclanthology.org/D14-1179/] and shortly after by Sutskever et al. (2014) [https://arxiv.org/abs/1409.3215], who independently developed that architecture and were the first to obtain state-of-the-art translation using this approach.

% The former system is based on scoring proposals generated by another machinetranslation system, while the latter uses a standalone recurrent network to generatethe translations. These authors respectively called this architecture, illustratedin figure 10.12, the encoder-decoder or sequence-to-sequence architecture. Theidea is very simple: (1) AnencoderorreaderorinputRNN processes the inputsequence. The encoder emits the contextC, usually as a simple function of itsfinal hidden state. (2) AdecoderorwriteroroutputRNN is conditioned onthat fixed-length vector (just as in figure 10.9) to generate the output sequenceY= (y(1), . . . , y(ny)). 

% One clear limitation of this architecture is when the contextCoutput by theencoder RNN has a dimension that is too small to properly summarize a longsequence. This phenomenon was observed by Bahdanau et al. (2015) in the contextof machine translation. They proposed to makeCa variable-length sequence ratherthan a fixed-size vector. Additionally, they introduced anattention mechanismthat learns to associate elements of the sequenceCto elements of the outputsequence. See section 12.4.5.1 for more details.







% \cite{jurafsky-2022-speech-lang-processing}
% The most commonly used such extension to RNNs is the Long short-term
% memory (LSTM) network (Hochreiter and Schmidhuber, 1997). LSTMs divide the context management problem into two sub-problems: removing information no longer needed from the context, and adding information likely to be needed for later decision making. The key to solving both problems is to learn how to manage this context rather than hard-coding a strategy into the architecture. LSTMs accomplish this by first adding an explicit context layer to the architecture (in addition to the usual recurrent hidden layer), and through the use of specialized neural units that make use of gates to control the flow of information into and out of the units that comprise the network layers. These gates are implemented through the use of additional weights that operate sequentially on the input, and previous hidden layer, and previous context layers.
% The gates in an LSTM share a common design pattern; each consists of a feedforward layer, followed by a sigmoid activation function, followed by a pointwise multiplication with the layer being gated. The choice of the sigmoid as the activation function arises from its tendency to push its outputs to either 0 or 1. Combining this with a pointwise multiplication has an effect similar to that of a binary mask. Values in the layer being gated that align with values near 1 in the mask are passed through nearly unchanged; values corresponding to lower values are essentially erased.



% ======================================
% \cite{goodfellow-2016-deep-learning}
% 
% Chapter 10
% https://www.deeplearningbook.org/contents/rnn.html
%
% 10.10 The Long Short-Term Memory and Other GatedRNNs (pg 404)
% As of this writing, the most effective sequence models used in practical applications are called gated RNNs. These include the long short-term memory and networks based on the gated recurrent unit.
% Like leaky units, gated RNNs are based on the idea of creating paths through time that have derivatives that neither vanish nor explode. Leaky units did this with connection weights that were either manually chosen constants or were parameters. Gated RNNs generalize this to connection weights that may change at each time step.
% Leaky units allow the network to accumulate information (such as evidence fora particular feature or category) over a long duration. Once that information has been used, however, it might be useful for the neural network to forget the old state. For example, if a sequence is made of subsequences and we want a leaky unit to accumulate evidence inside each sub-subsequence, we need a mechanism to forget the old state by setting it to zero. Instead of manually deciding when to clear the state, we want the neural network to learn to decide when to do it. This is what gated RNNs do.

% 10.10.1 LSTM
% The clever idea of introducing self-loops to produce paths where the gradientcan flow for long durations is a core contribution of the initiallong short-termmemory(LSTM) model (Hochreiter and Schmidhuber, 1997). A crucial additionhas been to make the weight on this self-loop conditioned on the context, rather thanfixed (Gers et al., 2000). By making the weight of this self-loop gated (controlledby another hidden unit), the time scale of integration can be changed dynamically.In this case, we mean that even for an LSTM with fixed parameters, the time scaleof integration can change based on the input sequence, because the time constantsare output by the model itself. 
% The LSTM has been found extremely successfulin many applications, such as unconstrained handwriting recognition (Graveset al., 2009), speech recognition (Graves et al., 2013; Graves and Jaitly, 2014),handwriting generation (Graves, 2013), machine translation (Sutskever et al., 2014),image captioning (Kiros et al., 2014b; Vinyals et al., 2014b; Xu et al., 2015), andparsing (Vinyals et al., 2014a).
% [...]
% Deeper architectures have also been successfully used (Graves et al.,2013; Pascanu et al., 2014a). Instead of a unit that simply applies an element-wisenonlinearity to the affine transformation of inputs and recurrent units, LSTMrecurrent networks have “LSTM cells” that have an internal recurrence (a self-loop),in addition to the outer recurrence of the RNN. Each cell has the same inputs andoutputs as an ordinary recurrent network, but also has more parameters and asystem of gating units that controls the flow of information.
% LSTM networks have been shown to learn long-term dependencies more easilythan the simple recurrent architectures, first on artificial datasets designed fortesting the ability to learn long-term dependencies (Bengio et al., 1994; Hochreiterand Schmidhuber, 1997; Hochreiter et al., 2001), then on challenging sequenceprocessing tasks where state-of-the-art performance was obtained (Graves, 2012;Graves et al., 2013; Sutskever et al., 2014). Variants and alternatives to the LSTMthat have been studied and used are discussed next.

% 10.10.2 Other Gated RNNs
% Which pieces of the LSTM architecture are actually necessary? What other successful architectures could be designed that allow the network to dynamically control the time scale and forgetting behavior of different units? Some answers to these questions are given with the recent work on gated RNNs,whose units are also known as gated recurrent units, or GRUs (Cho et al., 2014b;Chung et al., 2014, 2015a; Jozefowicz et al., 2015; Chrupala et al., 2015). The main difference with the LSTM is that a single gating unit simultaneously controls the forgetting factor and the decision to update the state unit.



% ================================
% \cite{lecun-2015-deep-learning}
% RNNs, once unfolded in time (Fig. 5), can be seen as very deep feedforward networks in which all the layers share the same weights. Although their main purpose is to learn long-term dependencies, theoretical and empirical evidence shows that it is difficult to learn to store information for very long78.
% To correct for that, one idea is to augment the network with an explicit memory. The first proposal of this kind is the long short-term memory (LSTM) networks that use special hidden units, the natural behaviour of which is to remember inputs for a long time79. A special unit called the memory cell acts like an accumulator or a gated leaky neuron: it has a connection to itself at the next time step that has a weight of one, so it copies its own real-valued state and accumulates the external signal, but this self-connection is multiplicatively gated by another unit that learns to decide when to clear the content of the memory.
% LSTM networks have subsequently proved to be more effective than conventional RNNs, especially when they have several layers for each time step87, enabling an entire speech recognition system that goes all the way from acoustics to the sequence of characters in the transcription. LSTM networks or related forms of gated units are also currently used for the encoder and decoder networks that perform so well at machine translation17,72,76
