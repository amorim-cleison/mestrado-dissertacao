Este capítulo apresenta conceitos fundamentais para compreensão do contexto das línguas de sinais e da abordagem proposta por esta pesquisa.
Primeiro, será introduzido o contexto do Surdo e da língua de sinais na \autoref{sec:lingua-sinais}.
Em seguida, a \autoref{sec:linguistica} se aprofundará na linguística e nas particularidades dessa língua.
A \autoref{sec:am}, por sua vez, abordará a \acrlong{ml} e alguns dos algoritmos aplicados ao \acrlong{nlp}.
Por fim, a \autoref{sec:slr} discutirá o panorama atual e os desafios existentes para a área de \acrlong{slr}.

% Este capítulo apresenta os principais conceitos necessários para compreender o trabalho desenvolvido nesta pesquisa.
% Este capítulo aborda o conteúdo teórico necessário para desenvolver a solução para o problema descrito na introdução:

% Primeiro, abordaremos uma visão geral acerca dos Surdos e das línguas 
% Em seguida, ... 
% Por fim, ... 


% - Línguas de sinais
% - Linguística da língua de sinais
% - Estado atual do SLR (survey com avanços atuais)

% - Reconhecimento de línguas faladas e texto x línguas de sinais
% - NLP: discutir evolução no tempo das abordagens utilizadas (para texto e voz) 
% - Utilização da fonologia + semântica das palavras para reconhecimento?

% - Modelos sequenciais (aprendizagem de máquina)
%     - Transformer: breve introdução (arquitetura e funcionamento)
%     - RNNs (GRU, LSTM, etc)

