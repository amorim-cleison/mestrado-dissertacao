\subsection{Desafios da área}
\label{sec:slr-desafios}

Nesta seção, serão discutidos os principais desafios identificados para o \acrfull{slr} segundo perspectivas apresentadas por \citeonline{bragg-2019-slr-interdisciplinary,cooper-2011-slr}.
Apesar de serem constatados vários avanços e um aumento no interesse por essa área ao longo dos últimos anos, muitos desses desafios ainda representam barreiras importantes que limitam a produção de progressos mais expressivos.

Dessa forma, essa discussão ajudará a conscientizar acerca deles e compreender a jornada ainda a ser percorrida pela área, bem como a direção que precisa ser adotar para isso.
Sendo assim, esses desafios podem ser enumerados como:


\begin{enumerate}
      \item \textbf{Abordagem inadequada}:
            de acordo com \citeonline{bragg-2019-slr-interdisciplinary}, muitas pesquisas em \acrshort{slr} ocorrem em silos disciplinares e, por conta disso, acabam por não abordar esse problema de forma abrangente.
            Por exemplo, publicações que limitam-se a rotular imagens de mãos estáticas em determinadas configurações deixam de considerar o contexto em volta delas ou a forma com que aquilo se conecta à linguística.
            Em outros casos, \citeonline{cooper-2011-slr} afirmam que o \acrshort{slr} é tratado como uma tarefa simples de reconhecimento de gestos, na qual os sinais são considerados apenas um conjunto de possibilidades bem definidas a serem rotuladas.
         

            Um outro fator apontado por \citeonline{bragg-2019-slr-interdisciplinary} é de que equipes desenvolvendo algoritmos para as línguas sinais comumente carecem de maior propriedade acerca delas, seja pelo envolvimento de membros Surdos, pela exposição a essa cultura, pelo aprofundamento das complexidades linguísticas que esses algoritmos deveriam considerar, ou pela empatia com os problemas reais que eles deveriam resolver.

            Muito frequentemente também, utilizam-se \textit{datasets} que não refletem contextos reais de uso para treinar tais algoritmos, como no caso daqueles que envolvem indivíduos não-nativos na língua ou que são coletados da internet e cuja procedência não pode ser confirmada.


      \item \textbf{Natureza visual}:
            as línguas de sinais apresentam o desafio de serem multicanais, ou sejam, elas transmitem significado através de vários canais visuais simultâneos que precisam ser considerados -- como os movimentos das mãos, do corpo, expressões da face, entre outros.

            Dessa forma, \citeonline{bragg-2019-slr-interdisciplinary} comentam que não é suficiente simplesmente replicar para o \acrshort{slr} as técnicas utilizadas com sucesso por outras áreas, como o reconhecimento de línguas faladas ou escritas, uma vez que elas assumem a existência de um canal único.
            Além disso, técnicas assim comumente adotam algum tipo de anotação no formato de texto como entrada, no entanto, as línguas de sinais não possuem uma forma escrita ou uma anotação padrão.

            Sendo assim, existe uma necessidade de que técnicas específicas sejam estabelecidas com o intuito de acomodar adequadamente as particularidades dos sinais.



      \item \textbf{Linguística particular}: conforme introduzimos na \autoref{sec:linguistica}, as línguas de sinais possuem um sistema linguístico próprio e complexo desenvolvido em torno do espaço visual, o qual as diferencia das línguas faladas.
            Abordar essa linguística é um fator fundamental para tratá-las efetivamente como línguas e não apenas como sistemas de gestos, o que certamente conduzirá a avanços mais concretos no \acrshort{slr}. Contudo, quando são analisados os estudos nessa área percebe-se que são poucos aqueles que utilizam esse tipo de abordagem.

            \citeonline{cooper-2011-slr} também ressaltam a pouca atenção que é recebida pelos parâmetros não-manuais nessas pesquisas, que comumente se concentram apenas em recortes das mãos dos indivíduos -- conforme discutimos na \autoref{sec:slr-breve-panorama}.
            Apesar disso, a \autoref{sec:linguistica} nos mostra que muito da carga linguística dos sinais é transmitida através desses parâmetros e isso os torna elementos essenciais da língua que precisam ser endereçados.


      \item \textbf{\textit{Datasets} limitados}:
            segundo \citeonline{cooper-2011-slr,bragg-2019-slr-interdisciplinary}, os \textit{datasets} disponíveis publicamente para as línguas de sinais costumam ser limitados em quantidade e qualidade.
            Como consequência, isso acaba também limitando a variedade e o desempenho dos algoritmos que poderiam ser aplicados ao \acrshort{slr}, uma vez que técnicas recentes como a \acrlong{dl} demandam uma grande quantidade de dados para funcionar adequadamente.
            Em áreas como o reconhecimento da fala, por exemplo, o estado da arte foi alcançado graças à utilização de corpus com milhões de amostras de palavras; no caso das línguas de sinais, no entanto, os \textit{datasets} não costumam ultrapassar 10.000 amostras.


            \citeonline{bragg-2019-slr-interdisciplinary} também destacam que há uma representatividade limitada de indivíduos e situações nesses \textit{datasets}, que ocorre porque eles geralmente são criados em ambientes controlados envolvendo consultores ou intérpretes da língua.
            Representatividade nesse contexto refere-se à variedade de características como idade, geografia, tom de pele, deficiência, proficiência, condições do ambiente, qualidade da câmera, ângulos de captura, entre outras.


            Além disso, os autores ressaltam a importância de se desenvolverem mais \textit{datasets} com sinais contínuos, envolvendo sequências completas de diálogos e enunciados mais longos, uma vez que isso reflete melhor contextos reais de utilização da língua.

\end{enumerate}
