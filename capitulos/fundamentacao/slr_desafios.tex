\subsection{Desafios da área}
\label{sec:slr-desafios}

% -----------------------------------------------------
% - Challenges
%Embora a tendência geral dos modelos propostos indique uma melhora significativa na precisão do reconhecimento de linguagem de sinais, ainda existem alguns desafios que precisam ser resolvidos.
%\cite{rastgoo-2021-slr-deep-survey}


% \cite{cooper-2011-slr}
% [...]
% Although SLR and speech recognition are drastically different in many respects,
% they both suffer from similar issues; co-articulation between signs means that a sign
% will be modified by those either side of it. Inter-signer differences are large; every
% signer has their own style, in the same way that everyone has their own accent or
% handwriting. Also similar to handwriting, signers can be either left hand or right
% hand dominant. For a left handed signer, most signs will be mirrored, but time line
% specific ones will be kept consistent with the cultural ‘left to right’ axis. While it
% is not obvious how best to include these higher level linguistic constructs of the
% language, it is obviously essential if true, continuous SLR is to become reality.
% ---
% 5 Research Frontiers
% There are many facets of SLR which have attracted attention in the computer vision
% community. This section serves to outline the areas which are currently generating
% the most interest due to the challenges they propose. While some of these are recent
% topics, others have been challenging computer vision experts for many years. Offered
% here is a brief overview of the seminal work and the current state of the art in
% each area.

% 5.1 Continuous Sign Recognition
% The majority of work on SLR has been focused on recognising isolated instances of
% signs, this is not applicable to a real world sign language recognition system. The
% task of recognising continuous sign language is complicated primarily by the problem
% that in natural sign language, the transition between signs is not clearly marked
% because the hands will be moving to the starting position of the next sign. This
% is referred to as the movement epenthesis or co-articulation (which borrows from
% speech terminology).
% [...]

% 5.2 Signer Independence
% A major problem relating to recognition is that of applying the system to a signer
% on whom the system has not been trained.
% [...]

% 5.3 Fusing Multi-Modal Sign Data
% From the review of SLR by Ong and Ranganath [82], one of their main observations
% is the lack of attention that non-manual features has received in the literature.
% This is still the case several years on. Much of the information in a sign is conveyed
% through this channel, and particularly there are signs that are identical in respect of
% the manual features and only distinguishable by the non-manual features accompanying
% the sign. The difficulty is identifying exactly which elements are important to
% the sign, and which elements are coincidental. For example, does the blink of the
% signer convey information valuable to the sign, or was the signer simply blinking?
% This problem of identifying the parts of the sign that contains information relevant to
% the understanding of the sign makes SLR a complex problem to solve.
% [...]

% 5.4 Using Linguistics
% The task of recognition is often simplified by forcing the possible word sequence
% to conform to a grammar which limits the potential choices and thereby improves
% recognition rates [91, 104, 12, 45].
% [...]

% 5.5 Generalising to More Complex Corpora
% Due to the lack of adequately labelled data sets, research has turned to weakly supervised
% approaches. Several groups have presented work aligning subtitles with
% signed TV broadcasts.
% [...]

