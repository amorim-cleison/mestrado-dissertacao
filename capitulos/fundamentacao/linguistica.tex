\section{Linguística da língua de sinais}
\label{sec:linguistica}

\citeonline{quadros-2004-estudos-linguisticos} afirmam que a linguística é o estudo científico das línguas naturais e humanas. Trata-se de uma ciência que procura desvendar os princípios independentes da lógica e da informação que determinam essa linguagem, bem como todas as formas criativas da comunicação.
Ela busca respostas para problemas essenciais relacionados à linguagem como, por exemplo: ``qual a natureza da linguagem humana?'',  ``como a comunicação se constitui?'', ``quais os princípios que determinam a habilidade dos seres humanos de produzir e compreender a linguagem?''.


Segundo \citeonline {stewart-2021-barrons-asl}, os primeiros estudos linguísticos sobre as línguas de sinais foram realizados pelo professor Dr. William C. Stokoe Jr. da Universidade de Gallaudet, em ~\citeyear{stokoe-1960-sl-structure}. 
Seu primeiro artigo, intitulado ``\textit{Sign Language Structure}'' (ou Estrutura da Língua de Sinais), foi seguido pela publicação do primeiro dicionário da \acrfull{asl} em \citeyear{stokoe-1965-dictionary-asl} -- o ``\textit{Dictionary of American Sign Language on Linguistic Principles}'' (ou Dicionário da Língua de Sinais Americana em Princípios Linguísticos) -- que foi compilado em parceria com dois colegas Surdos. Em 1971, por sua vez, ele estabeleceu o Laboratório de Pesquisa em Linguística da Universidade de Gallaudet.

Por conta disso, Stokoe ficou conhecido como o pai da linguística das línguas de sinais e seu trabalho teve um impacto profundo na conscientização acerca da \acrshort{asl} nos Estados Unidos e no restante do mundo.

Em seus estudos, ele comprovou que a \acrshort{asl} atendia a todos os critérios linguísticos de uma língua genuína -- no léxico, na sintaxe e na capacidade de  gerar uma quantidade infinita de sentenças.
A análise de suas propriedades revelou que a língua de sinais apresenta organização formal nos mesmos níveis encontrados nas línguas faladas, incluindo um nível sublexical de estruturação interna do sinal (análoga ao nível fonológico das línguas orais) e um nível gramatical (morfossintático), que especifica os modos como os sinais devem ser combinados para formarem frases e orações. Dessa forma, comprovou-se que os sinais não são meras imagens, mas símbolos abstratos com uma complexa estrutura interior. Aos estudos de Stokoe seguiram-se outros, que estenderam o escopo de análise às línguas de sinais utilizadas em diferentes países, como França, Itália, Uruguai, Argentina, Suécia e Brasil~\cite{stokoe-1960-sl-structure,quadros-2004-estudos-linguisticos, pereira-2011-conhecimento-alem-sinais}.

Serão abordados nas sessões seguintes aspectos importantes da organização gramatical da língua de sinais, segundo sua fonologia, morfologia e sintaxe.

