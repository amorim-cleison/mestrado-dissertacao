\section{Linguística da língua de sinais}
\label{sec:linguistica}

Segundo \citeauthor{quadros-2004-estudos-linguisticos}, a linguística é o estudo científico das línguas naturais e humanas. É uma ciência que procura desvendar os princípios independentes da lógica e da informação que determinam a linguagem humana, bem como todas as formas criativas da comunicação.
Ela busca respostas para problemas essenciais relacionados à linguagem, como: Qual a natureza da linguagem humana? Como a comunicação se constitui? Quais os princípios que determinam a habilidade dos seres humanos em produzir e compreender a linguagem?~\cite{quadros-2004-estudos-linguisticos}


% ===================================================
% \cite{quadros-2004-estudos-linguisticos}
% ===================================================
% 
% A linguística é o estudo científico das línguas naturais e humanas.
% ---
% A linguística busca desvendar os princípios independentes da lógica e da informação que determinam a linguagem humana. Tais princípios são o que há de comum nos seres  humanos que possibilitam a realização das diferentes línguas. Portanto, nesse  sentido, a teoria linguística extrapola as questões do uso.  
% ---
% A linguística é uma ciência que busca respostas para problemas essenciais relacionados à linguagem  que precisam ser explicados: Qual a natureza da linguagem humana? Como  a comunicação se constitui? Quais os princípios que determinam a habilidade dos seres humanos em produzir e compreender a linguagem? A linguística desmembra tais questões com a finalidade de explicar os problemas e  elaborar uma teoria da linguagem humana e uma teoria da comunicação. 


Os primeiros estudos linguísticos sobre as línguas de sinais foram realizadas pelo professor Dr. William C. Stokoe Jr. da Universidade de Gallaudet há cerca de sessenta anos atrás. Stokoe ficou conhecido como o pai da linguística da língua de sinais e seu trabalho teve um impacto profundo na conscientização acerca da \acrshort{asl} nos Estados Unidos e em todo o mundo, e já percorremos um longo caminho desde então~\cite{stewart-2021-barrons-asl}.

Seu primeiro artigo, publicado em 1960, é intitulado ``\textit{Sign Language Structure}'' (ou Estrutura da Língua de Sinais)~\cite{stokoe-1960-sl-structure}. Ele foi seguido pela publicação do primeiro dicionário de \acrshort{asl} em 1965, intitulado ``\textit{Dictionary of American Sign Language on Linguistic Principles}'' (ou Dicionário da Língua de Sinais Americana sobre Princípios Linguísticos)~\cite{stokoe-1965-dictionary-asl}, o qual foi compilado em parceria com dois colegas Surdos. Em 1971, Stokoe estabeleceu o Laboratório de Pesquisa em Linguística da Universidade de Gallaudet~\cite{stewart-2021-barrons-asl}.


Em seus estudos, Stokoe comprovou que a \acrshort{asl} atendia a todos os critérios linguísticos de uma língua genuína -- no léxico, na sintaxe e na capacidade de  gerar uma quantidade infinita de sentenças.
A análise de suas propriedades revelou que ela apresenta organização formal nos mesmos níveis encontrados nas línguas faladas, incluindo um nível sublexical de estruturação interna do sinal (análoga ao nível fonológico das línguas orais) e um nível gramatical (morfossintático), que especifica os modos como os sinais devem ser combinados para formarem frases e orações. Dessa forma, percebeu-se que os sinais não eram imagens, mas símbolos abstratos com uma complexa estrutura interior~\cite{stokoe-1960-sl-structure,quadros-2004-estudos-linguisticos, pereira-2011-conhecimento-alem-sinais}.


Naturalmente que o trabalho de Stokoe representou o primeiro passo em relação aos estudos das línguas de sinais. Pesquisas posteriores, feitas em grande parte com a \acrshort{asl}, mostraram, entre outras coisas, a riqueza de esquemas e combinações possíveis entre  os elementos formais que servem para ampliar consideravelmente o vocabulário básico~\cite{quadros-2004-estudos-linguisticos}.
Aos estudos de Stokoe se seguiram outros, cujo objeto eram as línguas de sinais usadas pelas comunidades de Surdos em diferentes países, como França, Itália, Uruguai,  Argentina, Suécia, Brasil e muitos outros. Essas línguas são diferentes umas das outras e independem das línguas orais utilizadas nesses países~\cite{pereira-2011-conhecimento-alem-sinais}.


\subsection{Gramática}
\label{linguistica-gramatica}

De uma forma simples, a gramática é um conjunto de regras que regem a utilização de uma língua. Ela é determinada pelo grupo de pessoas que utilizam aquela língua e compreende a sua fonologia, morfologia e sintaxe~\cite{jay-2011-dont-just-sign,quadros-2004-estudos-linguisticos}.

% Ela é determinada pelo grupo de pessoas que a utilizam e, no caso das línguas de sinais, a maneira como a comunidade Surda a utiliza é o que constitui sua gramática.
% American Sign Language tem seu próprio sistema gramatical. Isso significa que a ASL tem suas próprias regras para fonologia, morfologia e sintaxe.

Como pode-se imaginar, a língua de sinais possui um sistema gramatical próprio, que apresenta recursos visuais, espaciais e gestuais que se combinam para criar algumas estruturas sem paralelo no mundo das línguas faladas. Em particular, as qualidades espaciais de sua gramática permitem que um usuário expresse mais de um pensamento simultaneamente – algo que não pode ser duplicado nas línguas orais~\cite{jay-2011-dont-just-sign, stewart-2021-barrons-asl}.


% ASL has visual, spatial, and gestural features that combine to create some grammatical structures that are unparalleled in the world of spoken languages. 
% In particular, the spatial qualities of ASL grammar allow a signer to express more than one thought simultaneously -- a characteristic that cannot be duplicated in English. Facial grammar, or nonmanual signals, is also a significant component of ASL.
% \cite{stewart-2021-barrons-asl}


% Simplificando, a gramática é um conjunto de regras para usar uma linguagem. A gramática de uma língua inclui a fonologia, morfologia e sintaxe dessa língua.
% A gramática de uma língua é determinada pelo grupo de pessoas que a utilizam. Os principais usuários da American Sign Language são os membros da comunidade surda. A maneira como a comunidade surda usa a ASL é o que constitui a gramática da ASL.
% American Sign Language tem seu próprio sistema gramatical. Isso significa que a ASL tem suas próprias regras para fonologia, morfologia e sintaxe.
% \cite{jay-2011-dont-just-sign}

Nas seções a seguir exploraremos um pouco mais essas particularidades e discutiremos a fonologia, morfologia e sintaxe da língua de sinais.

% Fonologia
\subsection{Fonologia}
\label{sec:linguistica-fonologia}

Fonologia é o estudo das menores unidades constituintes de uma língua -- denominados fonemas -- e das regras que regem sua produção. Ela objetiva compreender essas unidades, bem como elas são articuladas, para compor unidades maiores com significado, como as palavras, de acordo com \citeonline{quadros-2004-estudos-linguisticos,hill-2019-sign-languages}. 


\citeonline{stokoe-1960-sl-structure} definiu inicialmente três tipos de fonemas (ou parâmetros) para a língua de sinais, os quais são articulados simultaneamente para compor um sinal: configuração de mão, locação e movimento. Em \citeyear{battison-1974-phono-deletion}, \citeauthor{battison-1974-phono-deletion} introduziu um quarto parâmetro, referente à orientação da palma da mão. Posteriormente, estudos como o de \citeonline{baker-padden-1978-nonmanual-components}, adicionaram as expressões não-manuais, como expressões faciais, movimentos da boca e direção do olhar.

Dessa forma, atualmente a fonologia da língua de sinais compreende que os sinais são compostos pelos seguintes parâmetros~\cite{stewart-2021-barrons-asl,jay-2011-dont-just-sign,quadros-2004-estudos-linguisticos}:

\begin{enumerate}
   \item \textbf{Configuração de mão}: configuração assumida pelas mãos ao produzir o sinal, a qual pode permanecer estática ou variar durante a articulação do sinal. É possível que as mãos apresentem configurações distintas nesse processo. A \autoref{fig:config-mao-comuns-asl} ilustra algumas configurações utilizadas na \acrshort{asl}.
   
    \figura[p. 72]
        {fig:config-mao-comuns-asl} % Label
        {capitulos/fundamentacao/imagens/configuracoes_mao_asl} % Path
        {height=8cm} % Size
        {Exemplos de configurações de mãos utilizadas na \acrshort{asl}} % Caption
        {jay-2011-dont-just-sign} % Citation


    \item \textbf{Orientação}: direção apontada pelas palmas das mãos na articulação do sinal. Por exemplo, as palmas podem estar voltadas para o corpo, para fora, para o chão, para cima, entre outras ilustradas na \autoref{fig:orientacoes}. Cada uma das palmas pode também assumir uma orientação distinta.
    
    \figura[p. 59]
        {fig:orientacoes} % Label
        {capitulos/fundamentacao/imagens/orientacoes} % Path
        {height=8cm} % Size
        {Exemplos de orientações que podem ser assumidas pelas palmas das mãos} % Caption
        {quadros-2004-estudos-linguisticos} % Citation
    

    \item \textbf{Movimento}: corresponde à trajetória percorrida pelas mãos em relação ao corpo para articular o sinal. 
    É um parâmetro complexo que pode envolver uma ampla variedade de modos e direções, desde um sutil deslizar entre as mãos ou um movimento interno das mãos e punhos, até uma trajetória complexa desenhada no espaço, por exemplo.
    Além disso, os sinais podem envolver movimentos de uma ou de ambas as mãos.

    
    \item \textbf{Locação}: é o local onde as mãos são posicionadas dentro do espaço de enunciação para articular o sinal. O espaço de enunciação, por sua vez, é uma área que contém todos os pontos possíveis dentro do raio de alcance das mãos, como ilustra a \autoref{fig:espaco-enunciacao}. Nesse espaço, há um número limitado de locações, sendo que algumas são mais exatas -- tais como a ponta do nariz --, e outras são mais abrangentes -- como a frente do tórax. Por fim, as mãos podem permanecer fixas ou se deslocar de uma locação para outro durante a articulação de um sinal.
    
    \figura[p. 57]
        {fig:espaco-enunciacao} % Label
        {capitulos/fundamentacao/imagens/espaco_enunciacao} % Path
        {height=6cm} % Size
        {Espaço de enunciação da língua de sinais} % Caption
        {quadros-2004-estudos-linguisticos} % Citation

        
    \item \textbf{Expressões não-manuais}: consistem nas expressões faciais e movimentos corporais incorporados aos sinais para provê significado adicional. Elas desempenham duas funções essenciais: marcar construções sintáticas (como frases interrogativas, orações relativas, tópicos, concordância e foco) e diferenciar componentes lexicais (como referências específicas, referências pronominais, partículas negativas, advérbios, grau ou aspecto).
    
    De um modo geral, expressões faciais ajudam a prover mais clareza ou alterar o significado de um sinal. 
    Movimentos corporais, por sua vez, são importantes para descrever pessoas em diferentes posições ou locais, ou narrar histórias envolvendo personagens com diferentes papéis, por exemplo.
 
\end{enumerate}


% Morfologia
\subsubsection{Morfologia}
\label{sec:linguistica-morfologia}

A morfologia é o estudo da estrutura interna das palavras e das regras que determinam sua formação. Um morfema é a menor unidade indivisível de sintaxe que retém significado e, na língua de sinais, é tido como a combinação de configuração de mão, orientação, locação e movimento~\cite{quadros-2004-estudos-linguisticos,jay-2011-dont-just-sign,hill-2019-sign-languages}.

% rules for producing a word,
%\cite{hill-2019-sign-languages}

% Assim como as palavras em todas as línguas humanas, os sinais pertencem a categorias lexicais ou a classes de palavras tais como substantivo, verbo, pronome, adjetivo, advérbio, numerais, entre outras. A língua de sinais tem um léxico e um sistema de criação de novos sinais em que os morfemas são combinados. Entretanto, essas línguas diferem-se das línguas orais nos tipos de processos combinatórios que frequentemente criam palavras morfologicamente complexas. 

Em línguas faladas, palavras complexas são muitas vezes formadas pela adição de um prefixo ou sufixo a uma raiz, como por exemplo \textit{des-}, \textit{ante-}, \textit{hiper-}, \textit{-ão}, \textit{-dade}, \textit{-ência}, entre outros. Por exemplo, o adjetivo ``infeliz'' é constituído de dois morfemas, o prefixo negativo \textit{in-} e o adjetivo \textit{feliz}; o substantivo ``capacidade'' é composto pelo adjetivo \textit{capaz} acrescido do sufixo \textit{-idade}; já o substantivo ``guarda-chuva'' é constituído pelos morfemas \textit{guarda} e \textit{chuva}.

Nas línguas de sinais, no entanto, essas formações resultam frequentemente de processos em que uma raiz é enriquecida com movimentos e contornos no espaço de sinalização. Também são utilizadas expressões não-manuais, alterações nos parâmetros fonológicos ou sinais específicos para indicar tempo, grau, intensidade, pluralidade, aspecto, entre outros~\cite{klima-1975-wit-poetry-asl,quadros-2004-estudos-linguisticos}.

De acordo com \citeonline{quadros-2004-estudos-linguisticos}, a morfologia tradicional apresenta dois processos de formação de palavras: a derivação e a flexão. Exploraremos a seguir um pouco mais sobre eles:

% A morfologia tradicional apresenta basicamente duas áreas de investiga-  ção: a derivacional e a flexional. A primeira detém-se ao estudo da formação de  diferentes palavras com uma mesma base lexical, por exemplo, no português  ‘sonhador’ é derivado de ‘sonhar’. A segunda envolve o estudo dos processos  que acrescentam informação gramatical à palavra que já existe.


\begin{enumerate}
    \item \textbf{Derivação}: consiste na formação de novas palavras a partir de uma mesma base lexical, como nos exemplos ``infeliz'' e ``capacidade'' introduzidos anteriormente. Alguns dos processos derivacionais da língua de sinais incluem:
    
    % detém-se ao estudo da formação de  diferentes palavras com uma mesma base lexical, por exemplo, no português  ‘sonhador’ é derivado de ‘sonhar’.

    \begin{enumerate}
        \item \underline{Nominalização} (ou derivação de substantivos a partir de verbos): é um dos processos mais comuns para mudança de classe na morfologia. Na língua de sinais, os substantivos apresentarem a mesma locação, configuração de mão e orientação dos verbos, mas repetem (ou reduplicam) o seu movimento (vide exemplo na \autoref{fig:sinais-ouvir-ouvinte}).
        
        \figura[p. 98]
            {fig:sinais-ouvir-ouvinte} % Label
            {capitulos/fundamentacao/imagens/sinais_ouvir_ouvinte} % Path
            {height=5cm} % Size
            {Sinal OUVINTE (à direita), que repete o movimento do verbo OUVIR (à esquerda)} % Caption
            {quadros-2004-estudos-linguisticos} % Citation
        

        % Seguindo proposta de Supalla e Newport (1978) para a ASL, observa-se  que a língua de sinais brasileira pode derivar nomes de verbos pela mudança no tipo de movimento. O movimento dos nomes repete e encurta o movimento dos verbos, conforme exemplo da figura a seguir. 

        % Supalla e Newport focalizam a análise do movimento e descrevem as  diferenças entre o movimento do verbo e do nome em detalhes, concluindo  que os pares apresentam a mesma locação, configuração e orientação de mão,  e que o nome simplesmente repete ou reduplica a estrutura segmental do  verbo, conforme exemplos a seguir. 
        % Esse processo de repetição é chamado reduplicação. Semelhante à  nominalização no português, na língua de sinais brasileira repete-se o morfemabase (verbo) e tem-se como produto um nome. O processo de adicionar  morfemas a uma forma base é uma forma de criar novas unidades lexicais. 


        \item \underline{Composição}: consiste na criação de um novo vocábulo (ou composto) através da junção de duas bases preexistentes. Na língua de sinais, três regras são utilizadas para isso: regra do contato, onde o contato existente no primeiro ou segundo sinal da composição é mantido; regra da sequência única, onde o movimento interno ou repetição dos sinais é eliminada para formar um composto; regra da antecipação da mão não-dominante, onde a mão passiva antecipa o segundo sinal no processo de composição (vide \autoref{fig:sinal-acidente}).
        
        %processo em que se juntam duas bases preexistentes na língua para criar um novo vocábulo (ou composto). São exemplos de composição no português: trem-de-ferro, aguardente, ciclovia, entre outros.
        
        % Na língua de sinais, três regras são utilizadas para criar compostos: a regra do contato (onde o contato existente no primeiro ou segundo sinal utilizado na composição é mantido, como na \autoref{fig:sinal-escola}), a regra da sequência única (onde o movimento interno ou repetição dos sinais é eliminada para formar o composto, como na \autoref{fig:sinal-pais}), e a regra da antecipação da mão não-dominante (onde a mão passiva antecipa o segundo sinal no processo de composição, como na \autoref{fig:sinal-acidente}).

        \figura[p. 105]
            {fig:sinal-acidente} % Label
            {capitulos/fundamentacao/imagens/sinal_acidente} % Path
            {height=4.5cm} % Size
            {Sinal ACIDENTE (composição de CARRO e BATER)} % Caption
            {quadros-2004-estudos-linguisticos} % Citation


        % Segundo Rocha (1998, p. 187), a  composição é um processo autônomo em que se juntam duas bases preexistentes  na língua para criar um novo vocábulo, dito composto. São exemplos de composição no português:  Trem-de-ferro  Aguardente  Salário-família  Cadeira de balanço  Lipoaspiração  Ciclovia 

        % Scott Liddel (1984) desenvolveu estudos sobre os compostos na ASL. Ele  observou que dois sinais formam um sinal composto quando mudanças  predicáveis ocorrem como o resultado de aplicação de regras, da mesma forma como acontece com palavras da língua inglesa. Ele apresenta dois tipos de  regras que causam as mudanças – regras morfológicas e regras fonológicas.  
        % Regras morfológicas são aplicadas especificamente para criar novas unidades com significados (compostos). Três regras morfológicas são usadas para  criar compostos na ASL: (1) a regra do contato; (2) a regra da seqüência  única e (3) a regra da antecipação da mão não-dominante. Será observada a  aplicação de tais regras na língua de sinais brasileira. 

        

        \item \underline{Incorporação de numeral}: é a combinação de um sinal com uma configuração de mão de número para indicar, por exemplo, variação de anos, dias, horas, minutos, etc. Observe na \autoref{fig:sinais-numero-meses} como a mudança na configuração de mão altera o número de meses referido no sinal.
        
        % nesse processo, morfemas são combinados com a configuração de mão de números para prover significado adicional ao sinal. Isso ocorre comumente para indicar variação do número de anos, dias, horas, minutos, entre outros. Por exemplo, a mudança na configuração de mão para 1, 2, 3 ou 4 na \autoref{fig:sinais-numero-meses} altera também o número de meses referido no sinal.
        
        \figura[p. 107]
            {fig:sinais-numero-meses} % Label
            {capitulos/fundamentacao/imagens/sinais_numero_meses} % Path
            {height=5cm} % Size
            {Sinais UM MÊS, DOIS MESES, e TRÊS MESES} % Caption
            {quadros-2004-estudos-linguisticos} % Citation



        \item \underline{Incorporação de negação}: consiste na geração da contraparte negativa de um sinal por meio da alteração de um de seus parâmetros (comumente seu movimento, como na \autoref{fig:sinais-saber-naosaber}) ou pela adição de uma expressão não-manual de negativa.
        
        % é possível dar origem à contraparte negativa de um sinal pela alteração de um de seus parâmetros (geralmente o movimento), conforme exemplifica a \autoref{fig:sinais-saber-naosaber}.
        
        \figura[p. 111]
            {fig:sinais-saber-naosaber} % Label
            {capitulos/fundamentacao/imagens/sinais_saber_naosaber} % Path
            {height=4.5cm} % Size
            {Sinal SABER e sua negação NÃO SABER} % Caption
            {quadros-2004-estudos-linguisticos} % Citation

        % Pode-se também marcar essa negação pela adição de uma expressão facial, sem que haja alteração nos parâmetros do sinal (vide \autoref{fig:sinais-conhecer-naoconhecer}).

        % Há também outro processo produtivo na língua de sinais brasileira que é a  incorporação da negação. Há alguns sinais que podem incorporar a negação conforme identificado por Ferreira-Brito (1995). A autora menciona que “através de  vários processos, o item a ser negado sofre alteração em um dos parâmetros,  especialmente o parâmetro Movimento, acarretando, assim, o aparecimento de  um item de estrutura ‘fonético-fonológico’ diferente daquele que é a sua base, ou  seja, o aparecimento de sua contraparte negativa” (Ferreira-Brito, 1995, p. 77).  Alguns dos exemplos elencados pela autora foram os seguintes: 

        % Além da incorporação da negação nos sinais, há a negação de forma  marcada através da expressão facial incorporada ao sinal sem alteração de  nenhum dos parâmetros. Este caso é relacionado por Ferreira-Brito (1995)  como negação supra-segmental. 

    \end{enumerate}


    \item \textbf{Flexão}: é um processo que modifica a forma das palavras para fazer com que elas se adequem melhor ao contexto gramatical \cite{hill-2019-sign-languages}. \citeonline{klima-1979-signs-of-language} enumeram os seguintes principais tipos de flexão:
    
    % consiste na adição de informação gramatical a uma palavra existente. Segundo \citeonline{klima-1979-signs-of-language}, entre os processos flexionais na língua de sinais, podem-se enumerar os seguintes:

    % envolve o estudo dos processos que acrescentam informação gramatical à palavra que já existe.

    % \cite{hill-2019-sign-languages}
    % One type, inflectional morphology, adjusts words so that they fit better into the grammatical context in which they are used. [...]
    % As mentioned earlier, inflection is a process that changes the form of a word so that it better fits the grammatical context.


    \begin{enumerate}
        \item \underline{Pessoa} (ou dêixis\footnote{
            Dêixis: Palavra grega que significa `apontar' ou `indicar'. Descreve uma forma particular de estabelecer nominais no espaço, os quais são utilizados pelos verbos com concordância como parte de sua flexão~\cite{quadros-2004-estudos-linguisticos}.
        }): modifica a referência de pessoa para os verbos, o que é realizado apontando-se para diferentes pontos no espaço à frente do sinalizador que representam as diferentes pessoas (ou objetos) envolvidos no discurso. O movimento do verbo, por sua vez, partirá de um desses pontos na direção do outro para indicar a direção da ação (vide \autoref{fig:verbo-entregar-deixis}).
        
        % Os referentes são introduzidos apontando-se para diferentes pontos no espaço à frente do sinalizador, os quais são incorporados pelo movimento dos verbos para indicar ação partindo de um desses referentes para o outro. Observe na \autoref{fig:deixis-eu-voce-elea} o estabelecimento de três pessoas no espaço e, na \autoref{fig:verbo-entregar-deixis}, a flexão do verbo para refletir a ação envolvendo pares desses indivíduos.

        \figura[p. 114-115]
            {fig:verbo-entregar-deixis} % Label
            {capitulos/fundamentacao/imagens/verbo_entregar_deixis} % Path
            {height=5cm} % Size
            {Verbo ENTREGAR flexionado entre as pessoas EU e VOCÊ (esquerda) e ELE 1 e ELE 2 (direita)} % Caption
            {quadros-2004-estudos-linguisticos} % Citation

        
        \item \underline{Número}: flexiona o sinal para sua forma no plural (que é marcada pela simples repetição do sinal) ou pode indicar a existência de múltiplos referentes no discurso (na qual o verbo direciona-se a ação para dois, três ou mais pontos estabelecidos no espaço ou para uma referência generalizada que inclui todos eles, conforme ilustra a \autoref{fig:verbo-entregar-flexao-numero}).

        \figura[p. 120]
            {fig:verbo-entregar-flexao-numero} % Label
            {capitulos/fundamentacao/imagens/verbo_entregar_flexao_numero} % Path
            {height=5cm} % Size
            {Verbo ENTREGAR flexionado em número para um (esquerda), três (centro) e múltiplos referentes (direita)} % Caption
            {quadros-2004-estudos-linguisticos} % Citation


        \item \underline{Grau}: indica variação de grau ou intensidade (como em ``um pouquinho'', ``aproximadamente'', ``muito'', ou ``demais''), a qual é geralmente transmitida por meio de expressões não-manuais (vide \autoref{fig:sinais-lindo-lindinho-lindissimo}).
    
        \figura[p. 65]
            {fig:sinais-lindo-lindinho-lindissimo} % Label
            {capitulos/fundamentacao/imagens/sinais_lindo_lindinho_lindissimo} % Path
            {height=4.0cm} % Size
            {Sinais LINDO (esquerda), LINDINHO (centro) e LINDÍSSIMO (direita)} % Caption
            {pereira-2011-conhecimento-alem-sinais} % Citation


        \item \underline{Modo}: detalha a maneira com que uma ação é realizada (como por exemplo ``facilmente'' ou ``com dificuldade'') e também se utiliza de expressões não-manuais.

        
        \item \underline{Aspecto temporal}: representa a relação de um ação com o tempo, que pode ser: incessante, ininterrupta, habitual (ação recorrente), contínua ou duradoura (ação permanente) (vide exemplos na \autoref{fig:verbo-cuidar-flexao-temporal}).

        \figura[p. 123]
            {fig:verbo-cuidar-flexao-temporal} % Label
            {capitulos/fundamentacao/imagens/verbo_cuidar_flexao_temporal} % Path
            {height=5cm} % Size
            {Verbo CUIDAR flexionado em aspecto temporal incessante (esquerda), ininterrupto (centro) e habitual (direita)} % Caption
            {quadros-2004-estudos-linguisticos} % Citation


        \item \underline{Foco temporal}: indica distinções do tipo ``começando a'', ``cada vez mais'', ``gradualmente'', ``progressivamente'' ou ``resultando em''.


        \item \underline{Aspecto distributivo}: representa uma ação que é distribuída entre diferentes pessoas ou objetos envolvidos no discurso, a qual pode ser: exaustiva, quando a ação é repetida exaustivamente; específica, quando direciona-se a pessoas ou objetos específicos (vide \autoref{fig:verbo-entregar-flexao-numero}, ao centro); não-específica, quando é generalizada ou indeterminada (vide \autoref{fig:verbo-entregar-flexao-numero}, à direita).
        
        
        \item \underline{Reciprocidade}: indica uma ação ou relação mútua, a qual é representada pela duplicação do sinal articulada de forma simultânea (vide \autoref{fig:sinal-olhar-reciproco}).
        
        \figura[p. 122]
            {fig:sinal-olhar-reciproco} % Label
            {capitulos/fundamentacao/imagens/sinal_olhar_reciproco} % Path
            {height=4cm} % Size
            {Sinal OLHAR (recíproco)} % Caption
            {quadros-2004-estudos-linguisticos} % Citation
                
    \end{enumerate}


    % Pessoa (deixis): flexão que muda as referências pessoais no verbo.  
    % Número: flexão que indica o singular, o dual, o trial e o múltiplo.  
    % Grau: apresenta distinções para ‘menor’, ‘mais próximo’, ‘muito’, etc. 
    % Modo: apresenta distinções, tais como os graus de facilidade.  
    % Reciprocidade: indica relação ou ação mútua.  
    % Foco temporal: indica aspectos temporais, tais como ‘início’, ‘aumento’,  ‘graduação’, ‘progresso’, ‘consequência’, etc.  
    % Aspecto temporal: indica distinções de tempo, tais como ‘há muito tempo’, ‘por muito tempo’, ‘regularmente’, ‘continuamente’, ‘incessantemente’, ‘repetidamente’, ‘caracteristicamente’, etc.  
    % Aspecto distributivo: indica distinções, tais como ‘cada’, ‘alguns especificados’, ‘alguns não-especificados’, ‘para todos’, etc. 

\end{enumerate}



%Na língua de sinais, não há sinais para afixos como \textit{in-}, \textit{-idade}, entre outros, que alteram o significado das palavras. Em vez disso, são utilizadas expressões não-manuais, alterações nos parâmetros fonológicos ou sinais diferentes para indicar tempo, grau, intensidade, pluralidade, aspecto e muito mais. 

% Observe na \autoref{fig:sinais-sentar-cadeira} os sinais SENTAR e CADEIRA da \acrshort{libras}. Eles possuem a mesma configuração de mãos, locação e orientação das palmas das mãos. No entanto, o movimento é o parâmetro que os diferencia: é mais longo em SENTAR e mais curto e repetido em CADEIRA: 

%Em ASL, não há sinais para afixos como “en”, “ing”, “ly” etc. para alterar o significado das palavras. 
%Em vez disso, o ASL usa marcadores não manuais, alterações nos parâmetros e outros sinais para indicar tempo, grau, intensidade, pluralidade, aspecto e muito mais.

% \figura[p. 70]
%     {fig:sinais-sentar-cadeira} % Label
%     {capitulos/fundamentacao/imagens/sinais_sentar_cadeira} % Path
%     {width=0.5\linewidth} % Size
%     {Sinais SENTAR e CADEIRA da \acrshort{libras}} % Caption
%     {pereira-2011-conhecimento-alem-sinais} % Citation


% Algo semelhante ocorre com os sinais OUVIR e OUVINTE (vide \autoref{fig:sinais-ouvir-ouvinte}): o movimento de fechar as mãos próximo ao ouvido é mais curto e repetido em OUVINTE, enquanto que no sinal OUVIR, onde a mão se desloca em direção ao ouvido.

% \figura[p. 71]
%     {fig:sinais-ouvir-ouvinte} % Label
%     {capitulos/fundamentacao/imagens/sinais_ouvir_ouvinte} % Path
%     {width=0.5\linewidth} % Size
%     {Sinais OUVIR e OUVINTE da \acrshort{libras}} % Caption
%     {pereira-2011-conhecimento-alem-sinais} % Citation


% A morfologia é o ramo da linguística que estuda como as palavras são formadas a partir de partes componentes. Um morfema é geralmente descrito como um par consistente de forma (por exemplo, uma sequência de sons ou uma combinação de formato de mão, localização e movimento) e significado, mas existem morfemas que mudam de forma em diferentes contextos, bem como aqueles que não mudam. t parecem ter um significado consistente. \cite{hill-2019-sign-languages}

% \cite{quadros-2004-estudos-linguisticos}
% Assim como as palavras em todas as línguas humanas, mas diferentemente dos gestos, os sinais pertencem a categorias lexicais ou a classes de palavras  tais como nome, verbo, adjetivo, advérbio, etc. As línguas de sinais têm um  léxico e um sistema de criação de novos sinais em que as unidades mínimas  com significado (morfemas) são combinadas. Entretanto, as línguas de sinais  diferem das línguas orais no tipo de processos combinatórios que freqüentemente cria palavras morfologicamente complexas. Para as línguas orais, palavras complexas são muitas vezes formadas pela adição de um prefixo ou sufixo  a uma raiz. Nas línguas de sinais, essas formas resultam freqüentemente de  processos não-concatenativos em que uma raiz é enriquecida com vários movimentos e contornos no espaço de sinalização (Klima e Bellugi, 1979). 









% \citeonline{jay-2011-dont-just-sign} descreve a morfologia da língua de sinais em termos dos seguintes elementos:
 
% \begin{enumerate}
%     \item \textbf{Flexões (advérbios)}: advérbios podem modificar adjetivos, verbos, ou outros advérbios para indicar tempo, local, forma, causa, ou intensidade. Como não há sinais específicos na língua de sinais para os advérbios, eles são criados através da adição de flexões, as quais podem incluir: o uso de expressões faciais mais intensas, o aumento ou diminuição da velocidade de sinalização, o uso de movimentos maiores ou mais largos, a sinalização mais detalhada de um sinal, o aceno mais rápido ou vagaroso com a cabeça, entre outros.
    
%     Sinais diferentes são normalmente flexionados de maneiras diferentes. A imagem \autoref{fig:sinais-lindo-lindinho-lindissimo} mostra o exemplo do sinal LINDO que é flexionado utilizando expressões faciais para transmitir a ideia de ``pouco'' e ``muito'' lindo, respectivamente nos sinais LINDINHO e LINDÍSSIMO.
    
%     \figura[p. 65]
%         {fig:sinais-lindo-lindinho-lindissimo} % Label
%         {capitulos/fundamentacao/imagens/sinais_lindo_lindinho_lindissimo} % Path
%         {width=0.5\linewidth} % Size
%         {Sinais LINDO, LINDINHO e LINDÍSSIMO da \acrshort{libras}} % Caption
%         {pereira-2011-conhecimento-alem-sinais} % Citation

%     \item \textbf{Pares substantivo-verbo}: pares substantivo-verbo são sinais que utilizam o mesmo formato de mão, locação e orientação, mas usam movimentos distintos para  diferenciar entre substantivo e verbo. Um verbo sinalizado geralmente tem um movimento único e contínuo, enquanto um substantivo geralmente tem um movimento duplo.  
%     Um exemplo disso é ilustrado para os sinais SENTAR e CADEIRA na \autoref{fig:sinais-sentar-cadeira}.
    

%     \item \textbf{Classificadores}: línguas faladas são lineares, ou seja, são expressas uma palavra após a outra. As línguas de sinais, no entanto, são espaciais e expressas no espaço ao redor do sinalizador. Nesse contexto, os classificadores criam profundidade e adicionam clareza, movimento e detalhes às conversas e explicações.

%     Faz mais sentido criar pessoas, animais ou objetos imaginários no espaço de enunciação e mostrar o que acontece com eles em vez de explicar cada palavra de forma linear como você faria oralmente. No entanto, assim como se faz com pronomes na língua falada, não se pode utilizar um classificador em uma frase até explicar o que o ele representa -- ou seja, é necessário primeiro estabelecer o substantivo.

%     A \autoref{fig:sinais-pessoas-andar-cair} ilustra um classificador comumente utilizado para referir-se a DUAS PESSOAS. Combinado ao movimento de ANDAR, é interpretado como DUAS PESSOAS ANDAM; combinado ao movimento de CAIR, é interpretado como DUAS PESSOAS CAEM.

%     \figura[p. 83]
%         {fig:sinais-pessoas-andar-cair} % Label
%         {capitulos/fundamentacao/imagens/sinais_pessoas_andar_cair} % Path
%         {width=0.5\linewidth} % Size
%         {Classificador utilizado para DUAS PESSOAS, combinado aos movimentos de ANDAR e CAIR} % Caption
%         {pereira-2011-conhecimento-alem-sinais} % Citation

%     Há um número infinito de classificadores e pode-se combinar praticamente qualquer forma de mão com qualquer movimento e locação para criar um classificador.
%     Alguns dos diferentes tipos de classificadores incluem: integrais (que representam um objeto inteiro), de superfícies, de instrumentos, de proporção (tamanho ou profundidade), de quantidade, de forma, de localização, de gestos, de partes do corpo (realizando uma ação), de verbos (que mostram como uma ação é realizada), ou de plural (que representam múltiplos itens).


%     % Línguas faladas como o inglês são lineares — são expressas uma palavra após a outra. ASL, no entanto, é uma linguagem espacial e é expressa no espaço ao seu redor. Os classificadores criam profundidade e adicionam clareza, movimento e detalhes às conversas e explicações.
%     % Em ASL, faz muito mais sentido criar pessoas, animais ou objetos imaginários em seu espaço de sinalização e mostrar o que acontece com eles em vez de explicar cada palavra de forma linear como você faria em inglês.
%     % As formas de mão e os movimentos dos classificadores podem representar pessoas, animais, objetos, etc. e mostrar movimentos, formas, ações e locais. Eles podem mostrar uma pessoa andando, um animal mastigando, alguém dirigindo, um carro passando pelas montanhas — praticamente qualquer coisa!
%     % Em uma frase, um classificador é muito semelhante a um pronome. Você não pode usar um classificador em uma frase até explicar o que o classificador representa. Não são palavras isoladas. Você precisa estabelecer o substantivo antes de aplicar o classificador.
%     % Um exemplo de classificador seria mostrar uma pessoa andando. Você primeiro tem que estabelecer essa pessoa como um referente no contexto e apontar para o referente, e então você pode pegar o dedo indicador de sua mão dominante (o classificador CL: 1) e movê-lo pelo seu espaço de sinalização. O que quer que você faça com este classificador é o que a pessoa está fazendo. Você também pode flexionar o sinal para um significado adicional. Quanto mais rápido ou mais devagar você mover esse classificador, mostra o quão rápido ou lento a pessoa está realizando uma ação. Você também pode usar marcadores não manuais para mostrar como a pessoa está se sentindo ao fazê-lo. E como a maioria das formas de mão do classificador representa uma pessoa ou objeto inteiro, você pode combinar classificadores como CL: 1(pessoa) com outro classificador como CL: 3(carro) para mostrar uma pessoa e um carro e suas localizações em relação um ao outro. Aqui estão dois exemplos de uso de classificadores com as duas mãos ao mesmo tempo:
%     % [...]
%     % Há um número infinito de classificadores que você pode usar. Você pode combinar quase qualquer forma de mão com qualquer movimento e localização para criar um classificador.


%     \item \textbf{Verbos}: há três tipos principais de verbos na língua de sinais: verbos simples, que não se flexionam em pessoa e número, e não incorporam afixos locativos (vide exemplos na \autoref{fig:verbos-simples});
%     verbos direcionais, que flexionam-se em pessoa, número e aspecto, mas não incorporam afixos locativos (vide \autoref{fig:verbos-direcionais}); e
%     verbos espaciais, que incorporam afixos locativos (vide \autoref{fig:verbos-espaciais}).

%     \figura[p. 76]
%         {fig:verbos-simples} % Label
%         {capitulos/fundamentacao/imagens/verbos_simples} % Path
%         {width=0.5\linewidth} % Size
%         {Verbos simples DIRIGIR, COMER, e PARECER} % Caption
%         {pereira-2011-conhecimento-alem-sinais} % Citation

%     \figura[p. 77]
%         {fig:verbos-direcionais} % Label
%         {capitulos/fundamentacao/imagens/verbos_direcionais} % Path
%         {width=0.5\linewidth} % Size
%         {Verbos direcionais PERGUNTAR e RESPONDER} % Caption
%         {pereira-2011-conhecimento-alem-sinais} % Citation

%     \figura[p. 77]
%         {fig:verbos-espaciais} % Label
%         {capitulos/fundamentacao/imagens/verbos_espaciais} % Path
%         {width=0.5\linewidth} % Size
%         {Verbos espaciais IR, CHEGAR, e POR} % Caption
%         {pereira-2011-conhecimento-alem-sinais} % Citation

%     \item \textbf{Tempo}: como não há afixos na língua de sinais para alterar o tempo verbal, ele é comunicado adicionando-se sinais de tempo e flexões desses sinais. Dessa forma, para expressar que ``eu \underline{fui} ao cinema'', seria utilizada a sequência de sinais ``ONTEM EU IR CINEMA''.    
%     Sinas de tempo que indicam futuro movem-se para frente; os que indicam passado movem-se para trás; e os que indicam presente são sinalizados à frente do corpo do sinalizador (vide \autoref{fig:sinais-tempo-semana}).
    
%     \figura[p. 101]
%         {fig:sinais-tempo-semana} % Label
%         {capitulos/fundamentacao/imagens/sinais_tempo_semana} % Path
%         {width=0.5\linewidth} % Size
%         {Sinais de tempo para SEMANA PASSADA (esquerda), ESTA SEMANA (centro), e PRÓXIMA SEMANA (direita)} % Caption
%         {jay-2011-dont-just-sign} % Citation

%     Numerais podem ser incorporados ao sinais de tempo para especificar número (ou quantidade) de semanas, horas, minutos, anos, meses, dias, entre outros. Observe na \autoref{fig:sinais-cada-tres-semanas} como o sinal SEMANA foi flexionado para expressar TRÊS SEMANAS.

%     Além disso, outras flexões podem ser adicionadas a esses sinais para transmitir significado de duração prolongada (como ``o dia inteiro'' ou ``a noite inteira'', por exemplo) ou de eventos que se repetem numa base regular (como ``a cada três semanas'', ``toda segunda-feira'', ou ``todas as manhãs'', por exemplo). Nesse caso, a flexão pode ocorrer por meio de alterações no movimento ou nas expressões faciais. A \autoref{fig:sinais-dia-inteiro} mostra o sinal DIA sendo flexionado para DIA INTEIRO; a \autoref{fig:sinais-cada-tres-semanas}, por sua vez, ilustra a repetição do sinal TRÊS SEMANAS para expressar a regularidade de A CADA TRÊS SEMANAS.
    
%     \figura[p. 106]
%         {fig:sinais-dia-inteiro} % Label
%         {capitulos/fundamentacao/imagens/sinais_dia_inteiro} % Path
%         {width=0.50\linewidth} % Size
%         {Sinal de tempo DIA (esquerda) e flexão para expressar duração de DIA INTEIRO (direita)} % Caption
%         {jay-2011-dont-just-sign} % Citation

%     \figura[p. 106]
%         {fig:sinais-cada-tres-semanas} % Label
%         {capitulos/fundamentacao/imagens/sinais_cada_tres_semanas} % Path
%         {width=0.50\linewidth} % Size
%         {Sinal de tempo TRÊS SEMANAS (esquerda) e flexão para expressar regularidade de A CADA TRÊS SEMANAS (direita)} % Caption
%         {jay-2011-dont-just-sign} % Citation

% \end{enumerate}







% ---------------------------------------------------------------------
% \cite{hill-2019-sign-languages}
% Morphology
% Morphology is the branch of linguistics that studies how words are formed from component parts. A morpheme is generally described as a consistent pairing of form (e.g., a sequence of sounds or a combination of handshape, location, and movement) and meaning, but there are morphemes that change their form in different contexts as well as those that don’t seem to have a consistent meaning.


% ---------------------------------------------------------------------
% \cite{pereira-2011-conhecimento-alem-sinais}
% Aspectos morfológicos (pág 70)
% Como a língua portuguesa, a Libras conta com um léxico e  com recursos que permitem a criação de novos sinais . Contudo, diferentemente das línguas orais, em que palavras complexas  são, muitas vezes, formadas pela adição de um prefixo ou sufixo a  uma raiz, nas línguas de sinais a raiz é frequentemente enriquecida com vários movimentos e contornos no espaço de sinalização  (Klima e Bellugi, 1979).  
% Um processo bastante comum na Libras para a criação de novos sinais é o que deriva nomes de verbos, e vice-versa, por meio  da mudança no movimento . O movimento dos nomes repete e  encurta o movimento dos verbos (Quadros e Karnopp, 2004) . 
% [imagem]
% Processo semelhante é observado em ouvir e ouvinTe . O  movimento de fechar as mãos próximo ao ouvido é mais curto e  repetido em ouvinTe 
% [imagem]
% Outro processo bastante usado na Libras, na criação de novos  sinais, é a composição . Nesse processo, dois sinais se combinam,  dando origem a um novo sinal, como se pode observar em eSCoLA  e igrejA 
% [imagem]
% O sinal de eSCoLA é composto pelos sinais de CASA e eSTuDAr,  enquanto o de igrejA é composto pelo sinais de CASA e Cruz 

% A criação de novos sinais na Libras pode ser obtida, ainda, por  meio da incorporação de um argumento, de um numeral ou de  uma negação.  A incorporação de argumento é muito frequente na Libras por  causa das características visuais e espaciais da língua . O sinal de  LAvAr, por exemplo, varia de acordo com o objeto que está, foi  ou será lavado .
% [imagem]


% A incorporação de um numeral caracteriza-se pela mudança na  configuração de mão do sinal para expressar a quantidade . Assim,  por exemplo, pela mudança na configuração de mão, de 1 para 2  ou para 3, o número de meses, dias ou horas referidos muda . A  localização, a orientação e os traços não manuais permanecem os  mesmos (Quadros e Karnopp, 2004) 
% [imagem]

% A incorporação da negação é outro processo bastante produtivo na Libras e pode dar-se pela alteração do movimento do sinal, caracterizada por mudança de direção, para fora, na maioria  das vezes, com a palma da mão também para fora (Brito, 1995) .  Cabe lembrar que, nos verbos, as formas negativas são acompanhadas de meneio negativo de cabeça e expressão facial de nega-  ção, como se pode observar nos exemplos a seguir.


% Categorias gramaticais (pág 76)
% Como a língua portuguesa, a Libras organiza seus sinais em  classes, como substantivos, verbos, pronomes, advérbios, adjetivos e numerais, entre outras. Serão consideradas aqui as categorias  que apresentam especificidades na Libras decorrentes principalmente do uso do espaço .

% Verbos  
% Os verbos na Libras estão basicamente divididos em três classes (Quadros e Karnopp, 2004, pp . 116-118): simples, direcionais e espaciais .  
% • verbos simples — são verbos que não se flexionam em pessoa e número e não incorporam afixos locativos 
% • verbos direcionais (com concordância) — são verbos  que se flexionam em pessoa, número e aspecto, mas não incorporam afixos locativos 
% • verbos espaciais — são verbos que têm afixos locativos 

% Adjetivos  
% Os adjetivos são sinais que formam uma classe específica na  Libras e estão sempre na forma neutra, não recebendo marcação  para gênero (masculino e feminino) nem para número (singular e plural) . Muitos adjetivos, por serem descritivos e classificadores, expressam a qualidade do objeto, desenhando-a no ar ou  mostrando-a no objeto ou no corpo do emissor (Felipe, 2001) 

% Pronomes  
% • Pessoais - são expressos por meio dos sinais de apontar com o dedo indicador.
% No singular, o sinal para todas as pessoas é o  mesmo; o que difere é a orientação da mão . No  plural, o formato do numeral — dois, três, quatro, até nove — apontando para pessoas ou lugares a quem se faz referência é interpretado como  nós, vocês ou eles dois, três, quatro, até nove.
% • Possessivos - são expressos com a configuração de  mãos em P e seguem os mesmos princípios da expressão dos pronomes pessoais na Libras 

% Classificadores  
% Os classificadores são formas que, substituindo o nome que as  precedem, podem vir junto com o verbo para classificar o sujeito ou  o objeto que está ligado à ação do verbo (Felipe, 2001) . Para Brito  (1995), os classificadores funcionam, em uma sentença, como partes dos verbos de movimento ou de localização . O sistema de classificadores fornece um campo de representações de categoriais que  revelam o tamanho e a forma de um objeto, a animação corporal  de um personagem ou como um instrumento é manipulado (Rayman, 1999) . Morgan (2005) refere que, nas narrativas, um classificador é, muitas vezes, usado para manter a referência a objeto ou  personagem previamente mencionado por meio de um sinal .  
% Em relação às formas dos classificadores, Brito (1995) refere  que a configuração de mão em V pode ser usada para se referir a pessoas, animais ou objetos; em C, para qualquer tipo de objeto  cilíndrico, e em B, para superfícies planas, por exemplo .

% Flexão verbal  
% A flexão de número nos verbos refere-se à distinção para um,  dois, três ou mais referentes . Assim, o verbo que apresenta concordância direciona-se para um, dois ou três pontos estabelecidos  no espaço ou para uma referência generalizada incluindo todos os  referentes integrantes do discurso (Quadros e Karnopp, 2004) 
% A flexão do aspecto está relacionada com as formas e a duração dos movimentos.
% Os aspectos pontual, continuativo, durativo e iterativo são obtidos por meio de alterações do movimento e/ou da configura-  ção da mão (Brito, 1995) 
%---
%A Libras apresenta, ainda, em suas formas verbais, a marca de  tempo de forma diferente de como acontece na língua portuguesa . O tempo é marcado por meio de advérbios de tempo que  indicam se a ação está ocorrendo no presente (hoje, agora), se  ocorreu no passado (ontem, anteontem), ou se ocorrerá no futuro (amanhã, semana que vem) . Para um tempo verbal indefinido,  usam-se os sinais passado e futuro (Felipe, 2001) . Para expressar  a ideia de passado, o sinal de já, antecedendo o verbo, ou o meneio afirmativo com a cabeça, concomitante à realização do sinal,  são muito utilizados 

% Flexão nominal  
% Diferentemente da língua portuguesa na modalidade oral, que  apresenta flexão de gênero modificando os nomes, a indicação de  sexo na Libras é marcada por um sinal que indica marca de gênero feminino ou masculino, antecedendo o nome .  
% Nos substantivos, a flexão de plural é obtida, na maioria das  vezes, pela repetição do sinal, pela anteposição ou posposição de  sinais referentes aos números, ou pelo movimento semicircular, que deve abranger as pessoas ou os objetos envolvidos (Brito, 1995) 


% ---------------------------------------------------------------------
% \cite{quadros-2004-estudos-linguisticos}

% Morfologia das línguas de sinais
% Morfologia é o estudo da estrutura interna das palavras ou dos sinais,  assim como das regras que determinam a formação das palavras. A palavra  morfema deriva do grego morphé, que significa forma. Os morfemas são as  unidades mínimas de significado.  
% Alguns morfemas por si só constituem palavras, outros nunca formam  palavras, apenas constituindo partes de palavras. Desta forma, têm-se os  morfemas presos que, em geral, são os sufixos e os prefixos, uma vez que não  podem ocorrer isoladamente, e os morfemas livres que constituem palavras.  
% Mas, na medida em que se pode formar palavras a partir de outras palavras, é importante reconhecer que as palavras podem ser unidades complexas, constituídas de mais de um elemento. 
% ---
% Assim como as palavras em todas as línguas humanas, mas diferentemente dos gestos, os sinais pertencem a categorias lexicais ou a classes de palavras  tais como nome, verbo, adjetivo, advérbio, etc. As línguas de sinais têm um  léxico e um sistema de criação de novos sinais em que as unidades mínimas  com significado (morfemas) são combinadas. Entretanto, as línguas de sinais  diferem das línguas orais no tipo de processos combinatórios que freqüentemente cria palavras morfologicamente complexas. Para as línguas orais, palavras complexas são muitas vezes formadas pela adição de um prefixo ou sufixo  a uma raiz. Nas línguas de sinais, essas formas resultam freqüentemente de  processos não-concatenativos em que uma raiz é enriquecida com vários movimentos e contornos no espaço de sinalização (Klima e Bellugi, 1979). 

% [...]


% --------------------------------------------------------
% \cite{jay-2011-dont-just-sign}
% Morphology
% In language, morphology is the study of the forms and formations of words. A morpheme is the smallest indivisible unit of syntax that retains meaning.
%For example, in English, the word “threateningly” consists of four morphemes: “threat,” which is a noun; “en,” which changes the noun into a verb; “ing,” which changes it into an adjective; and “ly” which changes it into an adverb. 
% In ASL, there are no signs for affixes like “en,” “ing,” “ly,” etc. to change the meaning of words. Instead, ASL uses non-manual markers, changes in parameters, and other signs to indicate tense, degree, intensity, plurality, aspect, and more.

% • Inflection (Adverbs)
% • Noun-Verb Pairs 
% • Classifiers x
% • Verbs x
% • Time




% Sintaxe
\subsection{Sintaxe}
\label{sec:linguistica-sintaxe}

A sintaxe é o estudo da construção de sentenças, bem como dos princípios e regras que regem esse processo. Na língua de sinais, a sintaxe é transmitida principalmente através da ordem das palavras e das expressões não-manuais.
Observemos a seguir alguns dos componentes básicos de sua sintaxe \cite{jay-2011-dont-just-sign,hill-2019-sign-languages,quadros-2004-estudos-linguisticos}:

\begin{enumerate}
    \item \textbf{Ordem das palavras}: as sentenças na língua de sinais seguem uma estrutura ``\textit{tópico-comentário}'' parecida com a estrutura ``\textit{sujeito-predicado}'' utilizada nas línguas faladas. No entanto, ao invés de ser sempre o sujeito, o tópico na língua de sinais é qualquer coisa a que o comentário se refira -- que pode ser o sujeito ou o objeto da sentença \cite{jay-2011-dont-just-sign}. 
    
    De acordo com \citeonline{quadros-2004-estudos-linguisticos}, há várias possibilidades de ordenação das palavras nas sentenças da língua de sinais. Porém, apesar dessa flexibilidade, parece haver uma ordenação mais básica que as demais, que seria a ordem \textit{Sujeito-Verbo-Objeto} (SVO). 
    Outras ordenações OSV, SOV e VOS são derivadas daquela primeira e resultam de operações sintáticas específicas  associadas a algum tipo de marcação como, por exemplo, a concordância de verbos e as expressões não-manuais \cite{quadros-2004-estudos-linguisticos}. 
    
    

    % \cite{jay-2011-dont-just-sign}
    % ASL sentences follow a TOPIC-COMMENT structure. This is the same as the English “subject” “predicate” structure. However, instead of the topic always being the subject, the topic in ASL is whatever the comment is referring to. This can either be the subject of the sentence or the object. 
    % The subject of a sentence is the person or object doing the action, the verb of a sentence is the action, and the object of a sentence is what is receiving the action. For example, in the sentence “The boy kicked the ball” the subject is “boy,” the verb is “kicked,” and the object is “ball.” 
    % There are a few different variations of word order in ASL depending on the vocabulary you are using and what you are trying to accomplish.


    % \cite{quadros-2004-estudos-linguisticos}
    % Há dois trabalhos que mencionam a flexibilidade da ordem das frases na  língua de sinais brasileira: Felipe (1989) e Ferreira-Brito (1995). As autoras  observaram que há várias possibilidades de ordenação das palavras nas sentenças, mas que, apesar dessa flexibilidade, parece haver uma ordenação mais  básica que as demais, ou seja, a ordem Sujeito-Verbo-Objeto. Quadros (1999)  apresenta evidências que justificam tal intuição propondo uma representação  para a estrutura da frase nesta língua. As evidências surgem de orações simples, de orações complexas contendo orações subordinadas, da interação com  advérbios, com modais e com auxiliares. As demais ordenações encontradas  na língua de sinais brasileira resultam da interação de outros mecanismos  gramaticais. 
    % [...]
    % Os dados apresentados indicam que a ordem básica na língua de sinais  brasileira é SVO e que OSV, SOV e VOS são ordenações derivadas de SVO.  Assim, as mudanças de ordens resultam de operações sintáticas específicas  associadas a algum tipo de marca, por exemplo, a concordância e as marcas  não-manuais. 



    \item \textbf{Tipos de sentenças}: há alguns diferentes tipos de sentenças na língua de sinais, os quais são marcados através da utilização de expressões não-manuais em conjunto com a articulação dos sinais e podem alterar a ordem das palavras na sentença.
    
    % There are a few different sentence types in ASL. These sentence types are not the same as word order. Word order shows the order in which you can sign your words. Sentence types show how to use word order along with non-manual markers to form certain types of sentences.

    \begin{enumerate}
        \item \underline{Interrogativa}: as perguntas são geralmente marcadas por expressões que combinam movimentos de levantar ou abaixar as sobrancelhas e a inclinação da cabeça, além da sustentação do último sinal articulado na sentença. Além disso, a ordem da sentença pode ser sofrer alterações. Observe a marcação não-manual e ênfase no sinal ``QUEM'' na \autoref{fig:interrogativa-joao-gostar}.
        
        \figura[p. 187]
            {fig:interrogativa-joao-gostar} % Label
            {capitulos/fundamentacao/imagens/interrogativa_joao_gostar} % Path
            {height=4cm} % Size
            {Interrogativa ``JOÃO GOSTAR QUEM?''} % Caption
            {quadros-2004-estudos-linguisticos} % Citation
    

        \item \underline{Declarativa}: consistem de declarações, que podem ser afirmativas, negativas, ou neutras. As expressões não-manuais auxiliam a denotar a existência e o grau de afirmação (vide \autoref{fig:topicalizada-futebol-gostar}) ou negação (vide \autoref{fig:negacao-futebol-gostar-nao}).

        % Declarative sentences are statements. These can be affirmative, negative, or neutral statements and each are recognized by the different non-manual markers that are used.


        \item \underline{Condicional}: seguem uma estrutura do tipo ``se \dots então'' onde a expressão de erguer as sobrancelhas marca a parte do ``se'' e uma afirmação ou interrogação marca a parte do ``então'' (vide exemplo na \autoref{fig:condicional-chover-jogo-cancelar}).
        
        \figura[p. 121]
            {fig:condicional-chover-jogo-cancelar} % Label
            {capitulos/fundamentacao/imagens/condicional_chover_jogo_cancelar} % Path
            {height=5cm} % Size
            {Condicional ``CHOVER HOJE, JOGO CANCELAR''} % Caption
            {jay-2011-dont-just-sign} % Citation

        \item \underline{Topicalizada}: ocorre quando há o movimento do ``objeto'' para o início da sentença, transformando sua ordem para OSV. Isso cria uma voz passiva, diferente da voz ativa utilizada na estrutura SVO.
        A expressão facial utilizada para marcar o ``objeto'' difere do restante da sentença (vide \autoref{fig:topicalizada-futebol-gostar}).
        
        \figura[p. 147]
            {fig:topicalizada-futebol-gostar} % Label
            {capitulos/fundamentacao/imagens/topicalizada_futebol_gostar} % Path
            {height=4cm} % Size
            {Frase topicalizada ``FUTEBOL JOÃO GOSTAR''} % Caption
            {quadros-2004-estudos-linguisticos} % Citation

        % Topicalization includes the movement of a syntactic element to the front (or beginning) of a sentence and highlighting it as “old or previously discussed” information.

        % When you use the “object” part of the sentence as the topic of the sentence (OSV word order), this is called topicalization. The facial expression used for the “object” part of the sentence differs from the rest of the sentence. This creates a “passive voice” instead of the “active voice” that is used with SVO structure.

    \end{enumerate}
    
    \item \textbf{Negação}: todas as línguas devem possuir mecanismos para negar enunciados ou proposições expressas em sentido positivo. Na língua de sinais, você pode formar negações de diferentes formas, como sinalizando NÃO antes de uma palavra, balançando a cabeça enquanto sinaliza uma palavra (vide \autoref{fig:negacao-futebol-gostar-nao}), invertendo a orientação da mão para alguns sinais, ou franzindo a testa enquanto sinaliza uma palavra.
    
    % To form a negative, you can: • Sign NOT before the word. • Shake your head while signing the word. • Use reversal of orientation for some signs. • Frown while signing the word. Non-manual markers are a very important part of negation. For example, if you sign, “ME don’t-LIKE HAMBURGER,” a different facial expression can change the meaning to: “I really dislike hamburgers.”

    
    \figura[p. 147]
        {fig:negacao-futebol-gostar-nao} % Label
        {capitulos/fundamentacao/imagens/negacao_futebol_gostar_nao} % Path
        {height=4cm} % Size
        {Negação ``FUTEBOL JOÃO GOSTAR NÃO''} % Caption
        {quadros-2004-estudos-linguisticos} % Citation

    % All languages must have a way to express negation. For every utterance or proposition that is expressed in a positive sense, there is a mechanism for negating that positive proposition. For instance, “I have three apples” is a positive proposition.

\end{enumerate}




% \cite{hill-2019-sign-languages} ----------------------------
% Syntax
% Syntax is the study of the descriptive rules that are needed to build a sentence in a given language.
% Now we will look at some basic components of ASL syntax and learn how to build a sentence.

% Word order
% Syntatic structures with brow raise
%     Yes/no interrogative
%     Conditionals
%     Topicalization
% Wh-questions
% Negation


% \cite{jay-2011-dont-just-sign} -------------------------------
% Syntax
% Syntax is the study of constructing sentences. Syntax also refers to the rules and principles of sentence structure.
% In ASL, syntax is conveyed through word order and non-manual markers. This section can be confusing, so don’t get discouraged if you don’t understand the first time.

% • Word Order 
%         Word order with plain Verbs
%         Object-subject-verb word order
%         Word order without objects
%         Word order with directional Verbs
%         time-topic-comment
% • Sentence Types
%         Questions
%             Wh-questions
%             Yes/no questions
%             Rethorical questions
%         Declarative sentences
%             Affirmative Declarative ...
%             Negative Declarative ...
%             Neutral Declarative ...
%         Conditional sentences
%         Topicalization
%             Topicalized statements
%             Topicalized "Wh" question
% • Negation
%         Reversal of orientation
% • Pronouns and Indexing
%         Indexing on your non-dominant hand
%         Personal Pronouns
%         Possessive Pronouns
%         Directional Verbs
%         Plural Directional Verbs
% • Nouns
%         Pluralization
% • Adjectives
% • Auxiliary Verbs
% • Prepositions
% • Conjunctions
% • Articles




% \cite{quadros-2004-estudos-linguisticos} ------------------------
% A Sintaxe Espacial
% A língua de sinais brasileira, usada pela comunidade surda brasileira  espalhada por todo o País, é organizada espacialmente de forma tão complexa quanto às línguas orais-auditivas. Analisar alguns aspectos da sintaxe  de uma língua de sinais requer “enxergar” esse sistema que é visuoespacial  e não oral-auditivo. De certa forma, tal desafio apresenta certo grau de dificuldade aos lingüistas; no entanto, abre portas para as investigações no campo  da Teoria da Gramática enquanto manifestação possível da capacidade da  linguagem humana. A organização espacial dessa língua, assim como da  ASL – Língua de Sinais Americana – (Siple, 1978; Lillo-Martin, 1986; Fischer,  1990; Bellugi, Lillo-Martin, O’Grady e van Hoek, 1990), apresenta possibilidades de estabelecimento de relações gramaticais no espaço, através de diferentes formas.
% No espaço em que são realizados os sinais, o estabelecimento nominal e  o uso do sistema pronominal são fundamentais para tais relações sintáticas.  Qualquer referência usada no discurso requer o estabelecimento de um local  no espaço de sinalização (espaço definido na frente do corpo do sinalizador),  observando várias restrições. 





























% ########################################################################################

% ===================================================
% \cite{pereira-2011-conhecimento-alem-sinais}
% ===================================================
%
% * Seguindo convenção proposta por James Woodward (1982), neste livro será  usado o termo “surdo” para se referir à condição audiológica de não ouvir, e o termo “Surdo” para se referir a um grupo particular de pessoas surdas que partilham  uma língua e uma cultura. 
% ---
% A língua de sinais é a língua usada pela maioria dos Surdos, na  vida diária . É a principal força que une a comunidade Surda, o  símbolo de identificação entre seus membros 
% ---
% Cada país tem sua língua de sinais, como tem sua língua na  modalidade oral. As línguas de sinais são línguas naturais, ou  seja, nasceram “naturalmente” nas comunidades Surdas. Uma vez  que não se pode falar em comunidade universal, tampouco está  correto falar em língua universal.  
% Outro aspecto a considerar é a relação estreita que existe entre  língua e cultura . As línguas de sinais refletem a cultura dos diferentes países onde são usadas, e esse é mais um argumento contra  a ideia de uma língua de sinais universal 
%---
% As línguas de sinais distinguem-se das línguas orais porque utilizam o canal visual-espacial em vez do oral-auditivo . Por esse motivo, são denominadas línguas de modalidade gestual-visual  (ou visual-espacial), uma vez que a informação linguística é recebida pelos olhos e produzida no espaço, pelas mãos, pelo movimento do corpo e pela expressão facial .  
% Apesar da diferença existente entre línguas de sinais e línguas  orais, ambas seguem os mesmos princípios com relação ao fato  de que têm um léxico, isto é, um conjunto de símbolos convencionais, e uma gramática, ou seja, um sistema de regras que rege  o uso e a combinação desses símbolos em unidades maiores .  
% ---
% Primeiros estudos (intro a linguística):
% As primeiras pesquisas linguísticas sobre as línguas de sinais,  mais especificamente sobre a língua de sinais americana, foram realizadas por William Stokoe, no início dos anos 1960, e tiveram como objetivo mostrar que os sinais poderiam ser vistos  como mais do que gestos holísticos aos quais faltava uma estrutura interna (Stokoe, 1960) . Ao contrário do que se poderia pensar à primeira vista, eles poderiam ser descritos em termos de um  conjunto limitado de elementos formacionais que se combinavam para formar os sinais .  
% A análise das propriedades formais da língua de sinais americana revelou que ela apresenta organização formal nos mesmos níveis encontrados nas línguas faladas, incluindo um nível  sublexical de estruturação interna do sinal (análoga ao nível fonológico das línguas orais) e um nível gramatical, que especifica os  modos como os sinais devem ser combinados para formarem frases e orações (Klima e Bellugi, 1979) .  
% Aos estudos sobre a língua de sinais americana se seguiram outros, cujo objeto eram as línguas de sinais usadas pelas comunidades de surdos em diferentes países, como França, Itália, Uruguai,  Argentina, Suécia, Brasil e muitos outros .  
% Essas línguas são diferentes umas das outras e independem  das línguas orais utilizadas nesses países 
% -----------------
% Concepções de surdez e de surdos 
% - Concepção clínico-patológica
% - Concepção socioantropológica (tentar adotar essa aqui)

% [Pág 30]:
% Martins (2004, p . 204-205) afirma que: “Sem língua não existem nem os surdos nem o modo de ser, cultural, surdo . Existiriam  apenas deficientes auditivos .” E segue com uma boa afirmação  em defesa da língua: “[ . . .] não é simplesmente o nível de audição  que vai definir quem é surdo ou deficiente auditivo” (op . cit .) . 

% [Pág 33]:
% Na história, constata-se que os Surdos sofreram perseguições  pelas pessoas ouvintes, que não aceitavam as diferenças e exigiam  uma cultura única por meio do modelo  ouvintista ou ouvintismo . São muitas  as lutas e histórias nas comunidades Surdas, em que o povo Surdo se une contra  as práticas dos ouvintes que não respeitam a cultura Surda (Strobel, 2008) .  
% Ainda hoje, muitos ouvintes tentam  diminuir os Surdos para que vivam isolados e tendo de assumir a  cultura ouvinte, como se esta fosse uma cultura única; ser “normal” para a sociedade significa ouvir e falar oralmente . Os ouvintes não prestam atenção aos Surdos que se comunicam por  meio da Libras . Consequentemente, não acreditam que os Surdos sejam capazes de estudar em faculdade ou realizar mestrado e  doutorado, por exemplo . “Os sujeitos ouvintes veem os sujeitos  surdos com curiosidade e, às vezes, zombam por eles serem diferentes” (Strobel, 2008, p . 22) .  
% A luta dos Surdos tem conduzido a várias vitórias, como o reconhecimento da Libras, o direito a tradutores e intérpretes da  língua brasileira de sinais–língua portuguesa e a uma educação  bilíngue para as crianças Surdas, que contemple a Libras e o português, este na modalidade escrita, entre muitas outras conquistas 

% [Pág 34] Cultura Surda
% Os Surdos constituem uma comunidade linguística minoritária, cujos elementos identificatórios são a língua de sinais e uma  cultura própria eminentemente visual . Têm um espírito gregário  muito importante que se manifesta em vários espaços . Esses espa-  ços “dos Surdos” são associações e clubes de Surdos onde desenvolvem suas próprias atividades . Constituem refúgios naturais da  língua de sinais e da identidade Surda (Strobel, 2008, p . 45) .  
% Diante da comunidade majoritariamente ouvinte, as comunidades Surdas apresentam suas próprias condutas linguísticas e seus  valores culturais . A comunidade Surda tem uma atitude diferente diante do déficit auditivo, já que não leva em conta o grau de  perda auditiva de seus membros . Pertencer à comunidade Surda pode ser definido pelo domínio da língua de sinais e pelos sentimentos de identidade grupal, fatores que consideram a surdez  como uma diferença, e não como uma deficiência .  
% Como ocorre com qualquer outra cultura, os membros das comunidades de Surdos compartilham valores, crenças, comportamentos e, o mais importante, uma língua diferente da utilizada  pelo restante da sociedade .  A língua de sinais, uma língua visual-espacial com gramática  própria, é uma das maiores produções culturais dos Surdos (Perlin, 2006) . Lane, Hoffmeister e Bahan (1996) referem que a língua de sinais tem basicamente três papéis para os Surdos: ela é símbolo da identidade social, é um meio de interação social e é  um depositário de conhecimento cultural.

% [Pág 97]
% A pessoa surda é definida como aquela que, por ter perda  auditiva, compreende o mundo e interage com ele por meio de  experiências visuais, manifestando sua cultura principalmente  pelo uso da Libras .




% Aspectos morfológicos (pág 70)
% Como a língua portuguesa, a Libras conta com um léxico e  com recursos que permitem a criação de novos sinais . Contudo, diferentemente das línguas orais, em que palavras complexas  são, muitas vezes, formadas pela adição de um prefixo ou sufixo a  uma raiz, nas línguas de sinais a raiz é frequentemente enriquecida com vários movimentos e contornos no espaço de sinalização  (Klima e Bellugi, 1979).  
% Um processo bastante comum na Libras para a criação de novos sinais é o que deriva nomes de verbos, e vice-versa, por meio  da mudança no movimento . O movimento dos nomes repete e  encurta o movimento dos verbos (Quadros e Karnopp, 2004) . 
% [imagem]
% Processo semelhante é observado em ouvir e ouvinTe . O  movimento de fechar as mãos próximo ao ouvido é mais curto e  repetido em ouvinTe 
% [imagem]
% Outro processo bastante usado na Libras, na criação de novos  sinais, é a composição . Nesse processo, dois sinais se combinam,  dando origem a um novo sinal, como se pode observar em eSCoLA  e igrejA 
% [imagem]
% O sinal de eSCoLA é composto pelos sinais de CASA e eSTuDAr,  enquanto o de igrejA é composto pelo sinais de CASA e Cruz 

% A criação de novos sinais na Libras pode ser obtida, ainda, por  meio da incorporação de um argumento, de um numeral ou de  uma negação.  A incorporação de argumento é muito frequente na Libras por  causa das características visuais e espaciais da língua . O sinal de  LAvAr, por exemplo, varia de acordo com o objeto que está, foi  ou será lavado .
% [imagem]


% A incorporação de um numeral caracteriza-se pela mudança na  configuração de mão do sinal para expressar a quantidade . Assim,  por exemplo, pela mudança na configuração de mão, de 1 para 2  ou para 3, o número de meses, dias ou horas referidos muda . A  localização, a orientação e os traços não manuais permanecem os  mesmos (Quadros e Karnopp, 2004) 
% [imagem]

% A incorporação da negação é outro processo bastante produtivo na Libras e pode dar-se pela alteração do movimento do sinal, caracterizada por mudança de direção, para fora, na maioria  das vezes, com a palma da mão também para fora (Brito, 1995) .  Cabe lembrar que, nos verbos, as formas negativas são acompanhadas de meneio negativo de cabeça e expressão facial de nega-  ção, como se pode observar nos exemplos a seguir.


% Categorias gramaticais (pág 76)
% Como a língua portuguesa, a Libras organiza seus sinais em  classes, como substantivos, verbos, pronomes, advérbios, adjetivos e numerais, entre outras. Serão consideradas aqui as categorias  que apresentam especificidades na Libras decorrentes principalmente do uso do espaço .

% Verbos  
% Os verbos na Libras estão basicamente divididos em três classes (Quadros e Karnopp, 2004, pp . 116-118): simples, direcionais e espaciais .  
% • verbos simples — são verbos que não se flexionam em pessoa e número e não incorporam afixos locativos 
% • verbos direcionais (com concordância) — são verbos  que se flexionam em pessoa, número e aspecto, mas não incorporam afixos locativos 
% • verbos espaciais — são verbos que têm afixos locativos 

% Adjetivos  
% Os adjetivos são sinais que formam uma classe específica na  Libras e estão sempre na forma neutra, não recebendo marcação  para gênero (masculino e feminino) nem para número (singular e plural) . Muitos adjetivos, por serem descritivos e classificadores, expressam a qualidade do objeto, desenhando-a no ar ou  mostrando-a no objeto ou no corpo do emissor (Felipe, 2001) 

% Pronomes  
% • Pessoais - são expressos por meio dos sinais de apontar com o dedo indicador.
% No singular, o sinal para todas as pessoas é o  mesmo; o que difere é a orientação da mão . No  plural, o formato do numeral — dois, três, quatro, até nove — apontando para pessoas ou lugares a quem se faz referência é interpretado como  nós, vocês ou eles dois, três, quatro, até nove.
% • Possessivos - são expressos com a configuração de  mãos em P e seguem os mesmos princípios da expressão dos pronomes pessoais na Libras 


% Classificadores  
% Os classificadores são formas que, substituindo o nome que as  precedem, podem vir junto com o verbo para classificar o sujeito ou  o objeto que está ligado à ação do verbo (Felipe, 2001) . Para Brito  (1995), os classificadores funcionam, em uma sentença, como partes dos verbos de movimento ou de localização . O sistema de classificadores fornece um campo de representações de categoriais que  revelam o tamanho e a forma de um objeto, a animação corporal  de um personagem ou como um instrumento é manipulado (Rayman, 1999) . Morgan (2005) refere que, nas narrativas, um classificador é, muitas vezes, usado para manter a referência a objeto ou  personagem previamente mencionado por meio de um sinal .  
% Em relação às formas dos classificadores, Brito (1995) refere  que a configuração de mão em V pode ser usada para se referir a pessoas, animais ou objetos; em C, para qualquer tipo de objeto  cilíndrico, e em B, para superfícies planas, por exemplo .

% Flexão verbal  
% A flexão de número nos verbos refere-se à distinção para um,  dois, três ou mais referentes . Assim, o verbo que apresenta concordância direciona-se para um, dois ou três pontos estabelecidos  no espaço ou para uma referência generalizada incluindo todos os  referentes integrantes do discurso (Quadros e Karnopp, 2004) 
% A flexão do aspecto está relacionada com as formas e a duração dos movimentos.
% Os aspectos pontual, continuativo, durativo e iterativo são obtidos por meio de alterações do movimento e/ou da configura-  ção da mão (Brito, 1995) 
%---
%A Libras apresenta, ainda, em suas formas verbais, a marca de  tempo de forma diferente de como acontece na língua portuguesa . O tempo é marcado por meio de advérbios de tempo que  indicam se a ação está ocorrendo no presente (hoje, agora), se  ocorreu no passado (ontem, anteontem), ou se ocorrerá no futuro (amanhã, semana que vem) . Para um tempo verbal indefinido,  usam-se os sinais passado e futuro (Felipe, 2001) . Para expressar  a ideia de passado, o sinal de já, antecedendo o verbo, ou o meneio afirmativo com a cabeça, concomitante à realização do sinal,  são muito utilizados 

% Flexão nominal  
% Diferentemente da língua portuguesa na modalidade oral, que  apresenta flexão de gênero modificando os nomes, a indicação de  sexo na Libras é marcada por um sinal que indica marca de gênero feminino ou masculino, antecedendo o nome .  
% Nos substantivos, a flexão de plural é obtida, na maioria das  vezes, pela repetição do sinal, pela anteposição ou posposição de  sinais referentes aos números, ou pelo movimento semicircular, que deve abranger as pessoas ou os objetos envolvidos (Brito, 1995) 


% Aspectos sintáticos
% Embora pesquisas sobre a ordem dos sinais na Libras refiram  S-V-O como predominante (Quadros, 1999), a ordem tópico-comentário parece ser a mais utilizada, principalmente pelos surdos  menos oralizados, como se pode observar nos exemplos a seguir 
% [IMAGENS/EXEMPLOS]






% ===================================================
% \cite{stewart-2021-barrons-asl}
% ===================================================
% What is ASL?
% American Sign Language, or ASL, is the language of the American Deaf Community. It is used in North America, and it is the only complete and natural sign language recognized by te Deaf community. This is a simple definition. Once it is understood that ASL maintains grammatical structure, syntax, and rules entirely separate from English grammar, we can begin to understand how to use it properly -- the way the American Deaf community intents.
% As you venture through this book, you will notice that ASL grammar is not only comprised of signed words or concepts, but it heavily involves the use of facial grammar to give information or meaning to signs. Facial grammar, including eye contact, facial expressions, eye gazing, and head movements are part of the unique grammatical structure of ASL. Because of this, and many other reasons, the visual language of ASL is fascinating for nonsigners to observe and quickly becomes a desired language to learn.

% English gloss
% ASL is an expressive and receptive language only. Because of the spatial and gestural qualities of ASL, there can be no convenient written form of ASL. What we can do is write English glosses of ASL signs. An English gloss is the best approximation of the meaning of a sign. It gives us a way of laying out ASL so that it can be studied and discussed, but is not a written form of ASL.

% Signing as a Choice of Communication
% [...] why, until recently, did so many hearing people know so little about signing? There are at least three reasons for this. First, Deaf people make up just a small fraction of the population in any area. Therefore, many hearing people never encounter a Deaf person in their through life. Second, speech is the dominant form of communication in society and gets the most attention. Third, Deaf people tend to socialize with one another and with hearing people who know how to sign.

% ASL Awareness
% Awareness of ASL has been growing since Professor William C. Stokoe, Jr., of Gallaudet University, known as the father of ASL linguistics, published his research on the linguistics of ASL about sixty years ago. His first paper, published in 1960, is titled "Sign Language Structure". This was followed by the first dictionary of ASL in 1965, Dictionary of American Sign Language on Linguistic Principles. Stokoe compiled the dictionary with two Deaf colleagues at Gallaudet, Carl Croneberg and Dorothy Casterline. In 1971, Stokoe established the Linguistic Research Laboratory at Gallaudet. Stokoe's work had a profound impact on ASL awareness in the United States and throughout the world, and we've even come a long way since then.
% ASL courses in high schools and colleges are booming. The television and movies industry has discovered the value of including Deaf actors and actresses in films. [...]

% The Physical Dimensions of ASL: The Five Parameters
% ASL is a visual-gestural language. It is visual because we see it and gestural because the signs are formed by the hands. Signing alone, however, is not an accurate picture of ASL. How signs are formed in space is important to understanding what they mean. The critical space is called the signing space and extends from the waist to just above the head and to just beyond the sides of the body. This is also the space in which the hands can move comfortably. As you will learn in this book, the signing space has a role in ASL grammar. Two or more concepts can be simultaneously expressed in ASL. This feat cannot be accomplished in a spoken language because speech is temporal in that one word rolls off the tongue at a time. One further dimension of ASL is the movement of the head and facial expressions, which help shape the meaning of ASL sentences.

% How Are Signs Formed?
% The five parameters below come together to create a sign.
% 1) Handshape is the shape of the hands when the sign is formed. The handshape may remain the same throughout the sign or it can change. If two hands are use to make a sign, both hands can have the same handshape of be different.
% 2) Orientation is the position of the hand(s) relative to the body. For example, the palms can be facing the body or away from the body, facing the ground, or facing upward.
% 3) Location is the place in the signing space where a sign is formed. Signs can be stationary [...] or they can move from one location in the signing space to another [...].
% 4) Movement of a sign is the direction in which the hand moves relative to the body. There is a variety of movements that range from a simple sliding movement [...] to a complex movement.
% 5) Nonmanual markers add to signs to create meaning. They consist of various facial expressions, head tilting, shoulder movement, and mouth movements. With a non-manual marker, the meaning of the sign can change completely.

% Cultural Importance
% ASL gives us access to Deaf culture. Learning ASL is not simply about learning another language. It is also about access. Even though we can learn something about any culture from reading about it, we acquire a deeper understanding when we can experience the culture or hear firsthand accounts from the people who are a part of the culture.
% ASL is one of the defining characteristics of the Deaf community. Although groups exist withing the community, such as the Black Deaf community and LGBTQAI+ Deaf communities. Deaf community members are bound instead by their language: ASL. To learn more about Deaf culture and tap into the resources of the Deaf community, you need a solid grasp of ASL.

% -----
% The Deaf Community
% Many hearing people view the world of the Deaf as a place where people don't hear -- where silence is a loud reminder of the difference between the two groups of people. But for Deaf people, silence is not the focus. What is important to us is that we obtain a lot of pleasure by being with other Deaf people. We relish the tales about other Deaf people's experiences in the Deaf community [...]

% Self-identification
% Deaf ("big D"):
% - identify as a culturally Deaf and part of the Deaf community
% - take pride in Deaf identity
% - may have an auditory device, such as cochlear implant, hearing aid, or FM system
% - may have a more severe hearing loss
% - use sign language as their primary source of communication
% - most likely attend a Deaf school/program
% - feel more comfortable in the Deaf world
%
% deaf ("little d"):
% - do not typically associate as members of the Deaf community
% - may have an auditory device, such as a cochlear implant, hearing aid, or FM system
% - may refer to hearing loss as a medical condition
% - may not use sign language as their primary choice of communication
% - may attend a mainstream school
% - may feel more comfortable in the hearing world
%
% Hard of hearing (HoH):
% - do not associate as members of the Deaf community
% - have hearing loss but may have residual hearing
% - possibly use an auditory device, such as a hearing or FM system, to access sounds
% - refer to hearing loss as a medical condition
% - may use sign language as their primary choice of communication
% - may attend a mainstream school
% - feel more comfortable in the hearing world

% Who Are Deaf People?
% Deaf people are a group of people who have a hearing loss, use a sign language as their primary means of communication, and have shared experiences associated with the hearing loss and the use of sign language.
% There is no way that you can point to a person sitting and reading a magazine in a lobby whom you have never met before and say, "that person is Deaf". Even if the person is wearing hearing aids, we don't know which community the person identifies with. Similarly, it's not important whether the person is European, African-American, Asian, or of some other ethnic origin. Age is not relevant, and neither is the social class or gender of the person. 
% The Deaf community is not shaped by any of these characteristics. In fact, having a hearing loss does not mean that a person is a member of the Deaf community, although it is certainly an important requirement.

% The pivotal mark of a Deaf person is how this person communicates. A Deaf person uses sign language [...]. This does not mean that the person cannot use other forms of communication, such as writing and speaking. Rather, ASL is the linguistic trademark that sets Deaf people apart from the communication behavior of all other groups of people. It is the reason we say that Deaf people represent a linguistic minority. It is also why some people who are deaf do not see themselves as belonging to the Deaf community. 
% We use the lowercase spelling of deaf to refer to a person or a group of people who have a substantial degree of hearing loss. Having a hearing loss does not mean that a person automatically knows how to sign. If a deaf person does not know sign language, then that person will not be able to access the varied cultural experiences associated with the Deaf community. Communication is basic, and ASL is the communication of the Deaf community.

% Can a nondeaf person who is fluent in ASL be a member of the Deaf community? No. Deaf people do not view nondeaf people as members of their community because nondeaf people lack a third critical characteristic, which is shared experiences. If you have normal hearing, then you will never have the experiences of a life that is centered on seeing. 

% Let's put all of this in perspective. Let's say you have a young friend who acquired a hearing loss and was fitted with hearing aids. WOuld we say that he was Deaf? No, we would say that he is hard of hearing because speaking is still his main means of communication. If his hearing continues to deteriorate, then we might say that he is becoming deaf; that is, he is acquiring a substantial degree of hearing loss. What if his difficulty with hearing leads him to learn to sign ASL, which becomes his primary language of communication, and he begins participating in some activities in the Deaf community? Would we say then that he is Deaf? We would probably say, "friend, welcome to the club".


% Technology and Other Adaptations
% "Many of the technological advances for the majority in our society have penalized Deaf people. This irony emerges most clearly in telecommunications. The invention of the telephone made it difficult for Deaf people to compete in the labor market. Radio became an important means of broadcasting information, whether commercial, political, governmental, or whatever, further cutting off Deaf people from the larger society surrounding them. Television did little to improve the situation, though it embraced the technology that could have (and to some extent does) include Deaf people. Talking pictures were a blow to the entertainment and education of Deaf people; they could enjoy the "silents" on par with the rest of the audience. But Deaf people and their supporters have not passively accepted the status quo. They have taken steps to reduce the handicap the new technologies have imposed". -- Jerome Schein (At Home Among Strangers)

% -----
% ASL Grammar
% ASL has visual, spatial, and gestural features that combine to create some grammatical structures that are unparalleled in the world of spoken languages. As you learn about ASL grammar, look for similarities in the world of spoken languages. [...]
% In particular, the spatial qualities of ASL grammar allow a signer to express more than one thought simultaneously -- a characteristic that cannot be duplicated in English. Facial grammar, or nonmanual signals, is also a significant component of ASL.

% Facial Grammar
% What's in a sign may not be what's in the mind. To capture the sense of what a signer is signing, you must read the signer's face and body. When you listen to someone speak, you listen not only to the words but also to how the words are spoken. The tone of the voice, the rise and fall of the pitch, the length of the pause, and the steadiness of voice are all features that you latch onto with little effort in your spoken communication.
% These traits are nonexistent in signing, but they do have parallel traits that are crucial to ASL's grammar. The raised eyebrow, the tilted head, the open mouth, hunching of the shoulders, and a sign held slightly longer than others shape the meaning of the signs that are made by the hands. We call these nonmanual signals (NMS) facial grammar, which allows you to use facial expressions, your body, and gestures to add meaning and additional information to your signing. Mastering ASL cannot occur without a mastery of facial grammar.


% ===================================================
% \cite{hill-2019-sign-languages}
% ===================================================
%
% [Pág 2]
% Sign languages are produced by the hands, face, and body and perceived primarily visually, in contrast to spoken languages, which are produced by the mouth and vocal tract and perceived primarily auditorily (although manual gestures and visual perception of gestures and mouth movements are also important for spoken languages). Natural sign languages emerge (are not invented) when Deaf people form a community, often through educational systems. Sign languages are, therefore, primarily the languages of Deaf people, who cherish them for their cultural and community-building value. 
% It is important to recognize the connection between sign languages and Deaf communities. Until relatively recently, Deaf communities have been told (explicitly and implicitly) that their “sign communication” was inferior, broken, unimportant, or insufficient. Educational systems and the broader hearing majority community would stress the value of learning the spoken language, even at the expense of the sign language. In fact, such attitudes persist, both in areas where the national sign language has not been deeply studied linguistically and in areas where it has been studied but the focus for economic advancement is on the spoken language. However, the natural sign languages of Deaf communities are completely linguistic, rule-governed, capable of expressing anything, and fully worthwhile. We unreservedly endorse such affirmations of the value of sign languages and promote their use in all aspects of the lives of Deaf people. 

% Who belongs to the Deaf community? The “d” is capitalized to reinforce the view that Deaf communities form cultural groups with practices and values that are in some cases distinct from those of non-Deaf communities. These cultural effects are passed down within the community, from parents to children in some cases, but more often through interactions of Deaf people from different families. The leaders of Deaf communities are usually Deaf adults who were raised with Deaf parents or within the community from a very early age. Generally, members of the Deaf community are audiologically deaf or hard-of-hearing (and they shun the label “hearing impaired”). The hearing children born to Deaf parents are often known as Codas (from the name of an organization, CODA, ‘children of Deaf adults’), and they are sometimes part of the Deaf community. 

% It is important to note that people have many identities with intersectional effects, and in this respect, not all Deaf people have the same experiences, values, and life view. A Deaf person’s identity as Deaf will be affected by their identity in other ways, including race, ethnicity, gender identity, etc. Almost all research on the American Deaf community has focused only on a subset of Deaf people, so it is important to bear in mind that others might share some but not all of the characteristics described here. 
% Sign languages are, then, Deaf languages. Just as with the languages of other minority groups who have experienced oppression, hearing researchers who benefit from the study of sign languages (both in personal satisfaction and in economic, career, and other means) must acknowledge the primacy of Deaf signers and treat their language with the utmost respect.

%---
% Phonology
% Phonology is often described as the study of the sounds of language and their organization. If that is the way to look at phonology, it would be appropriate to say that sign languages do not have phonology. However, phonology is actually more abstract. It is about the ways in which words are made up of pieces that are not meaningful. It is about what these pieces are and how they work together. In spoken languages, the actual sounds that make up words are part of the study of phonology; yet, phonology is more concerned with the ways we can define these component pieces, how they change in different contexts (e.g., different words), and how they are organized.
% If there are component pieces that make up individual signs, then there is sign phonology. If there are implicit rules about the ways that the pieces can and cannot combine, then there is phonology. Indeed, one of the first linguistic discoveries about American Sign Language (ASL) is that “signs have parts” – [...]

% Morphology
% Morphology is the branch of linguistics that studies how words are formed from component parts. A morpheme is generally described as a consistent pairing of form (e.g., a sequence of sounds or a combination of handshape, location, and movement) and meaning, but there are morphemes that change their form in different contexts as well as those that don’t seem to have a consistent meaning.


% Syntax
% Syntax is the study of the descriptive rules that are needed to build a sentence in a given language.
% Now we will look at some basic components of ASL syntax and learn how to build a sentence.

% Word order
% Syntatic structures with brow raise
%     Yes/no interrogative
%     Conditionals
%     Topicalization
% Wh-questions
% Negation





% ===================================================
% \cite{quadros-2004-estudos-linguisticos}
% ===================================================
% 
% A linguística é o estudo científico das línguas naturais e humanas.
% ---
% A linguística busca desvendar os princípios independentes da lógica e da informação que determinam a linguagem humana. Tais princípios são o que há de comum nos seres  humanos que possibilitam a realização das diferentes línguas. Portanto, nesse  sentido, a teoria linguística extrapola as questões do uso.  A área da linguística está crescendo como área de estudo,
% ---
% A linguística é uma ciência que busca respostas para problemas essenciais relacionados à linguagem  que precisam ser explicados: Qual a natureza da linguagem humana? Como  a comunicação se constitui? Quais os princípios que determinam a habilidade dos seres humanos em produzir e compreender a linguagem? A linguística desmembra tais questões com a finalidade de explicar os problemas e  elaborar uma teoria da linguagem humana e uma teoria da comunicação. 
%
% ---
% Stokoe, em 1960,  percebeu e comprovou que a língua dos sinais atendia a todos os critérios  linguísticos de uma língua genuína, no léxico, na sintaxe e na capacidade de  gerar uma quantidade infinita de sentenças.  Stokoe observou que os sinais não eram imagens, mas símbolos abstratos complexos, com uma complexa estrutura interior. Ele foi o primeiro, portanto, a procurar uma estrutura, a analisar os sinais, dissecá-los e a pesquisar  suas partes constituintes. Comprovou, inicialmente, que cada sinal apresentava pelo menos três partes independentes (em analogia com os fonemas da  fala) – a localização, a configuração de mãos e o movimento – e que cada parte possuía um número limitado de combinações.
% ---
% Naturalmente que o trabalho de Stokoe (1960) representou o primeiro passo em relação aos estudos das línguas de sinais. Pesquisas posteriores, feitas em grande parte com a língua de sinais americana, mostraram,  entre outras coisas, a riqueza de esquemas e combinações possíveis entre  os elementos formais que servem para ampliar consideravelmente o vocabulário básico. 

% Fonologia das línguas de sinais
% Fonologia das línguas de sinais é o ramo da lingüística que objetiva identificar a estrutura e a organização dos constituintes fonológicos, propondo  modelos descritivos e explanatórios. A primeira tarefa da fonologia para línguas de sinais é determinar quais são as unidades mínimas que formam os  sinais. A segunda tarefa é estabelecer quais são os padrões possíveis de combinação entre essas unidades e as variações possíveis no ambiente fonológico. 


% Morfologia das línguas de sinais
% Morfologia é o estudo da estrutura interna das palavras ou dos sinais,  assim como das regras que determinam a formação das palavras. A palavra  morfema deriva do grego morphé, que significa forma. Os morfemas são as  unidades mínimas de significado.  
% Alguns morfemas por si só constituem palavras, outros nunca formam  palavras, apenas constituindo partes de palavras. Desta forma, têm-se os  morfemas presos que, em geral, são os sufixos e os prefixos, uma vez que não  podem ocorrer isoladamente, e os morfemas livres que constituem palavras.  
% Mas, na medida em que se pode formar palavras a partir de outras palavras, é importante reconhecer que as palavras podem ser unidades complexas, constituídas de mais de um elemento. 
% ---
% Assim como as palavras em todas as línguas humanas, mas diferentemente dos gestos, os sinais pertencem a categorias lexicais ou a classes de palavras  tais como nome, verbo, adjetivo, advérbio, etc. As línguas de sinais têm um  léxico e um sistema de criação de novos sinais em que as unidades mínimas  com significado (morfemas) são combinadas. Entretanto, as línguas de sinais  diferem das línguas orais no tipo de processos combinatórios que freqüentemente cria palavras morfologicamente complexas. Para as línguas orais, palavras complexas são muitas vezes formadas pela adição de um prefixo ou sufixo  a uma raiz. Nas línguas de sinais, essas formas resultam freqüentemente de  processos não-concatenativos em que uma raiz é enriquecida com vários movimentos e contornos no espaço de sinalização (Klima e Bellugi, 1979). 

% [...]


% A Sintaxe Espacial
% A língua de sinais brasileira, usada pela comunidade surda brasileira  espalhada por todo o País, é organizada espacialmente de forma tão complexa quanto às línguas orais-auditivas. Analisar alguns aspectos da sintaxe  de uma língua de sinais requer “enxergar” esse sistema que é visuoespacial  e não oral-auditivo. De certa forma, tal desafio apresenta certo grau de dificuldade aos lingüistas; no entanto, abre portas para as investigações no campo  da Teoria da Gramática enquanto manifestação possível da capacidade da  linguagem humana. A organização espacial dessa língua, assim como da  ASL – Língua de Sinais Americana – (Siple, 1978; Lillo-Martin, 1986; Fischer,  1990; Bellugi, Lillo-Martin, O’Grady e van Hoek, 1990), apresenta possibilidades de estabelecimento de relações gramaticais no espaço, através de diferentes formas.
% No espaço em que são realizados os sinais, o estabelecimento nominal e  o uso do sistema pronominal são fundamentais para tais relações sintáticas.  Qualquer referência usada no discurso requer o estabelecimento de um local  no espaço de sinalização (espaço definido na frente do corpo do sinalizador),  observando várias restrições. 





% ===================================================
% \cite{jay-2011-dont-just-sign}
% ===================================================
% To put it simply, grammar is a set of rules for using a language. The grammar of a language includes the phonology, morphology, and syntax of that language. 
% A language’s grammar is determined by the group of people who use that language. The main users of American Sign Language are the members of the Deaf community. The way that the Deaf community uses ASL is what constitutes ASL grammar. 
% American Sign Language has its own grammar system. This means that ASL has its own rules for phonology, morphology, and syntax.

% Phonology
% In language, phonology is the study of the smallest part of the language that conveys meaning. In spoken languages, like English, a phoneme is a unit of sound that conveys meaning. 
% In ASL, the smallest parts of the language, the phonemes, are handshape, movement, palm orientation, location, and facial expression. If you change any of these parameters of a sign, then you have changed the meaning of the sign.

% The Five Sign Parameters
% Just like how we see English words as the arrangement of letters, there are five basic sign language elements that make up each sign. If any of these parameters are changed when creating a sign, the meaning of the sign can change. 
% The first four elements are: 
% • Handshape – This is the shape of your hand that is used to create the sign. 
% • Movement – This is the movement of the handshape that makes the sign. 
% • Palm orientation – This is the orientation of your palm when making the sign. 
% • Location – This is the location of the sign on or in front of your body. 
% There is also a fifth element that has recently been included with this list: 
% • Non-manual Markers – This is the various facial expressions or body movements that are used to create meaning. 
% American Sign Language is a very expressive language, and understanding these elements will give you a better understanding of how signs are made and what makes them different.


% Morphology
% In language, morphology is the study of the forms and formations of words. A morpheme is the smallest indivisible unit of syntax that retains meaning.
%For example, in English, the word “threateningly” consists of four morphemes: “threat,” which is a noun; “en,” which changes the noun into a verb; “ing,” which changes it into an adjective; and “ly” which changes it into an adverb. 
% In ASL, there are no signs for affixes like “en,” “ing,” “ly,” etc. to change the meaning of words. Instead, ASL uses non-manual markers, changes in parameters, and other signs to indicate tense, degree, intensity, plurality, aspect, and more.

% • Inflection (Adverbs)
% • Noun-Verb Pairs
% • Classifiers
% • Verbs
% • Time



% Syntax
% Syntax is the study of constructing sentences. Syntax also refers to the rules and principles of sentence structure.
% In ASL, syntax is conveyed through word order and non-manual markers. This section can be confusing, so don’t get discouraged if you don’t understand the first time.

% • Word Order 
%         Word order with plain Verbs
%         Object-subject-verb word order
%         Word order without objects
%         Word order with directional Verbs
%         time-topic-comment
% • Sentence Types
%         Questions
%             Wh-questions
%             Yes/no questions
%             Rethorical questions
%         Declarative sentences
%             Affirmative Declarative ...
%             Negative Declarative ...
%             Neutral Declarative ...
%         Conditional sentences
%         Topicalization
%             Topicalized statements
%             Topicalized "Wh" question
% • Negation
%         Reversal of orientation
% • Pronouns and Indexing
%         Indexing on your non-dominant hand
%         Personal Pronouns
%         Possessive Pronouns
%         Directional Verbs
%         Plural Directional Verbs
% • Nouns
%         Pluralization
% • Adjectives
% • Auxiliary Verbs
% • Prepositions
% • Conjunctions
% • Articles
