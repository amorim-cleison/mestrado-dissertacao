\section{O Surdo e a língua de sinais}
\label{sec:lingua-sinais}

Segundo \citeonline{pereira-2011-conhecimento-alem-sinais}, a língua de sinais é utilizada pela maioria dos Surdos em sua vida diária e é, na verdade, muito mais do que uma língua: ela é a principal força que une a comunidade Surda e o símbolo de identificação entre seus membros. 

% A língua de sinais é a língua utilizada pela maioria dos Surdos em sua vida diária. É a principal força que une a comunidade Surda, o  símbolo de identificação entre seus membros. 
%\cite{pereira-2011-conhecimento-alem-sinais}


Quando utilizamos os termos ``Surdo'' ou ``comunidade Surda'' (com ``s'' capitalizado) não nos referimos apenas a uma condição clínica, mas a um grupo de indivíduos que, além de possuírem perda auditiva, utilizam a língua de sinais como principal meio de comunicação e compartilham experiências culturais associadas à surdez e ao uso dessa língua. Esses elementos estão interconectados e não pode-se definir comunidade Surda sem considerá-los como um todo.

De fato, ter uma perda auditiva não significa que uma pessoa seja membro da comunidade Surda ou que saiba automaticamente sinalizar, embora certamente esses sejam requisitos importantes. \citeonline{stewart-2021-barrons-asl} prosseguem:

% Having a hearing loss does not mean that a person automatically knows how to sign. If a deaf person does not know sign language, then that person will not be able to access the varied cultural experiences associated with the Deaf community. Communication is basic, and ASL is the communication of the Deaf community
% In fact, having a hearing loss does not mean that a person is a member of the Deaf community, although it is certainly an important requirement.


\begin{citacao}
    Não há como apontar uma pessoa sentada num saguão lendo uma revista e dizer: ``esta pessoa é Surda''. Ainda que ela esteja utilizando aparelhos auditivos, não há como saber com qual comunidade ela se identifica.
    
    De maneira semelhante, não importa se ela é europeia, afro-americana, asiática ou de outra etnia; sua idade não é relevante, tampouco sua classe social ou gênero; a comunidade Surda não é moldada por nenhuma dessas características.~\cite[tradução nossa]{stewart-2021-barrons-asl}
    
    % There is no way that you can point to a person sitting and reading a magazine in a lobby whom you have never met before and say, "that person is Deaf". Even if the person is wearing hearing aids, we don't know which community the person identifies with. Similarly, it's not important whether the person is European, African-American, Asian, or of some other ethnic origin. Age is not relevant, and neither is the social class or gender of the person. 
    % The Deaf community is not shaped by any of these characteristics. In fact, having a hearing loss does not mean that a person is a member of the Deaf community, although it is certainly an important requirement.
\end{citacao}


Há um aspecto cultural essencial acompanhado de um forte sentimento de identidade grupal, que fazem com que a surdez seja percebida como uma diferença e não como uma deficiência \cite{pereira-2011-conhecimento-alem-sinais}.


% Existe um aspecto cultural fundamental, acompanhado de um forte sentimento de identidade grupal com a comunidade Surda, que fazem com que a surdez seja percebida como uma diferença e não mais como uma deficiência, tornando o grau de perda auditiva um fator irrelevante aqui. Como ocorre em qualquer cultura, esses indivíduos compartilham valores, crenças, comportamentos e, além disso, uma língua própria que nesse contexto é diferente daquela utilizada majoritariamente em sociedade~\cite{pereira-2011-conhecimento-alem-sinais}.

% Pertencer à comunidade Surda pode ser definido pelo domínio da língua de sinais e pelos sentimentos de identidade grupal, fatores que consideram a surdez  como uma diferença, e não como uma deficiência.

% Self-identification
% Deaf ("big D"):
% - identify as a culturally Deaf and part of the Deaf community
% - take pride in Deaf identity
% - may have an auditory device, such as cochlear implant, hearing aid, or FM system
% - may have a more severe hearing loss
% - use sign language as their primary source of communication
% - most likely attend a Deaf school/program
% - feel more comfortable in the Deaf world


% \citeonline{stewart-2021-barrons-asl} consideram que a língua de sinais é a chave para acessar a cultura Surda. Aprender uma língua de sinais não é simplesmente aprender uma nova língua; é também sobre receber acesso. Embora seja possível aprender sobre qualquer cultura lendo sobre ela, adquirimos uma compreensão mais profunda quando somos capazes de experimentá-la ou ouvir relatos em primeira mão das pessoas inseridas nela.


Apesar disso, no decorrer da história os Surdos tiveram sua identidade estigmatizada e desvaloriza pela sociedade ouvinte, que não aceitava a língua de sinais. Segundo \citeonline{hill-2019-sign-languages}, há até relativamente pouco tempo esses indivíduos foram ditos que sua forma de comunicação era inferior, quebrada, sem importância ou insuficiente. Os sistemas educacionais e a comunidade majoritariamente ouvinte fizeram prevalecer a utilização da língua falada, mesmo às custas da língua de sinais. 

De fato, atitudes como essas ainda persistem atualmente. \citeonline{pereira-2011-conhecimento-alem-sinais} comentam que muitos ouvintes tentam  diminuir os Surdos para que eles vivam isolados, tendo de assumir a cultura ouvinte como se ela fosse a única existente. Ser ``normal'' significaria, portanto, poder ouvir e falar oralmente: 

% Segundo as autoras, os ouvintes não prestam atenção aos Surdos e, consequentemente, não acreditam que eles sejam capazes de estudar em faculdades ou realizar mestrado e doutorado, por exemplo. Os indivíduos ouvintes veem os Surdos com curiosidade e, às vezes, zombam por eles serem diferentes~\cite{pereira-2011-conhecimento-alem-sinais, hill-2019-sign-languages}. 

\begin{citacao}
    Os ouvintes não prestam atenção aos Surdos que se comunicam por meio da língua de sinais. Consequentemente, não acreditam que eles sejam capazes de estudar em faculdade ou realizar mestrado e  doutorado, por exemplo. \cite{pereira-2011-conhecimento-alem-sinais}
\end{citacao}

\begin{citacao}
    Os sujeitos ouvintes veem os sujeitos Surdos com curiosidade e, às vezes, zombam por eles serem diferentes. \cite{strobel-2016-cultura-surda}
\end{citacao}


São muitas as lutas pelas quais os Surdos têm atravessado mas que, sobretudo nos últimos anos, têm conduzido a vitórias importantes como o reconhecimento das línguas de sinais como línguas oficiais em seus países, o direito a tradutores e intérpretes em eventos e canais públicos de comunicação e o acesso a uma educação bilíngue para as crianças Surdas, entre outras conquistas. No Brasil, por exemplo, a \acrshort{libras} foi reconhecida como língua de sinais em 2002 \cite{brasil-2002-lei10436} e, nos Estados Unidos; a \acrshort{asl} foi reconhecida em 1989 \cite{pereira-2011-conhecimento-alem-sinais, jay-2011-dont-just-sign}..

% A luta dos Surdos tem conduzido a várias vitórias, como o reconhecimento da Libras, o direito a tradutores e intérpretes da  língua brasileira de sinais–língua portuguesa e a uma educação  bilíngue para as crianças Surdas, que contemple a Libras e o português, este na modalidade escrita, entre muitas outras conquistas 

% São muitas as lutas e histórias nas comunidades Surdas, nas quais esses indivíduos tem se unido contra práticas que não respeitam sua cultura. Essas lutas tem conduzido a várias vitórias sobretudo nos últimos anos, como o reconhecimento das línguas de sinais como línguas oficiais em seus países (no Brasil, por exemplo, a \acrfull{libras} foi reconhecida em 2002; nos Estados Unidos, a \acrfull{asl} foi reconhecida em 1989), o direito a tradutores e intérpretes em eventos e canais públicos de comunicação e o acesso a uma educação bilíngue para as crianças Surdas, entre outras conquistas~\cite{pereira-2011-conhecimento-alem-sinais, brasil-2002-lei10436, jay-2011-dont-just-sign}.



% No entanto, as línguas de sinais naturais das comunidades surdas são completamente linguísticas, governadas por regras, capazes de expressar qualquer coisa e totalmente valiosas. \cite{pereira-2011-conhecimento-alem-sinais, hill-2019-sign-languages}.

% \cite{pereira-2011-conhecimento-alem-sinais}
% os Surdos tiveram, historicamente,  sua identidade estigmatizada e se sentiram desvalorizados pela  sociedade ouvinte, que não aceitava a língua de sinais, considerada apenas mímica e gesto .
% Na história, constata-se que os Surdos sofreram perseguições  pelas pessoas ouvintes, que não aceitavam as diferenças e exigiam  uma cultura única por meio do modelo  ouvintista ou ouvintismo . São muitas  as lutas e histórias nas comunidades Surdas, em que o povo Surdo se une contra  as práticas dos ouvintes que não respeitam a cultura Surda (Strobel, 2008) .  
% Ainda hoje, muitos ouvintes tentam  diminuir os Surdos para que vivam isolados e tendo de assumir a  cultura ouvinte, como se esta fosse uma cultura única; ser “normal” para a sociedade significa ouvir e falar oralmente . Os ouvintes não prestam atenção aos Surdos que se comunicam por  meio da Libras . Consequentemente, não acreditam que os Surdos sejam capazes de estudar em faculdade ou realizar mestrado e  doutorado, por exemplo . “Os sujeitos ouvintes veem os sujeitos  surdos com curiosidade e, às vezes, zombam por eles serem diferentes” (Strobel, 2008, p . 22) .  
% A luta dos Surdos tem conduzido a várias vitórias, como o reconhecimento da Libras, o direito a tradutores e intérpretes da  língua brasileira de sinais–língua portuguesa e a uma educação  bilíngue para as crianças Surdas, que contemple a Libras e o português, este na modalidade escrita, entre muitas outras conquistas 


%\cite{hill-2019-sign-languages}
% It is important to recognize the connection between sign languages and Deaf communities. Until relatively recently, Deaf communities have been told (explicitly and implicitly) that their “sign communication” was inferior, broken, unimportant, or insufficient. Educational systems and the broader hearing majority community would stress the value of learning the spoken language, even at the expense of the sign language. In fact, such attitudes persist, both in areas where the national sign language has not been deeply studied linguistically and in areas where it has been studied but the focus for economic advancement is on the spoken language. However, the natural sign languages of Deaf communities are completely linguistic, rule-governed, capable of expressing anything, and fully worthwhile. We unreservedly endorse such affirmations of the value of sign languages and promote their use in all aspects of the lives of Deaf people. 


% \cite{pereira-2011-conhecimento-alem-sinais}
% [...] os Surdos tiveram, historicamente,  sua identidade estigmatizada e se sentiram desvalorizados pela  sociedade ouvinte, que não aceitava a língua de sinais, considerada apenas mímica e gesto . O uso ou não da língua de sinais  seria, portanto, o que definiria basicamente a identidade do sujeito, que só seria adquirida quando em contato com outro Surdo .  O que ocorre, segundo Santana e Bérgamo (2005), é que, nesse  contato com outro Surdo que também use a língua de sinais, surgem novas possibilidades interativas, de compreensão, de diálogo  e de aprendizagem que não são possíveis por meio apenas da linguagem oral . A aquisição de uma língua — e de todos os mecanismos afeitos a ela — faz creditar à língua de sinais a capacidade  de ser a única que pode oferecer uma identidade ao Surdo. 
%
% [Pág 33]:
% Na história, constata-se que os Surdos sofreram perseguições  pelas pessoas ouvintes, que não aceitavam as diferenças e exigiam  uma cultura única por meio do modelo  ouvintista ou ouvintismo . São muitas  as lutas e histórias nas comunidades Surdas, em que o povo Surdo se une contra  as práticas dos ouvintes que não respeitam a cultura Surda (Strobel, 2008) .  
% Ainda hoje, muitos ouvintes tentam  diminuir os Surdos para que vivam isolados e tendo de assumir a  cultura ouvinte, como se esta fosse uma cultura única; ser “normal” para a sociedade significa ouvir e falar oralmente . Os ouvintes não prestam atenção aos Surdos que se comunicam por  meio da Libras . Consequentemente, não acreditam que os Surdos sejam capazes de estudar em faculdade ou realizar mestrado e  doutorado, por exemplo . “Os sujeitos ouvintes veem os sujeitos  surdos com curiosidade e, às vezes, zombam por eles serem diferentes” (Strobel, 2008, p . 22) .  
% A luta dos Surdos tem conduzido a várias vitórias, como o reconhecimento da Libras, o direito a tradutores e intérpretes da  língua brasileira de sinais–língua portuguesa e a uma educação  bilíngue para as crianças Surdas, que contemple a Libras e o português, este na modalidade escrita, entre muitas outras conquistas 


% \cite{stewart-2021-barrons-asl}
% Signing as a Choice of Communication
% [...] why, until recently, did so many hearing people know so little about signing? There are at least three reasons for this. First, Deaf people make up just a small fraction of the population in any area. Therefore, many hearing people never encounter a Deaf person in their through life. Second, speech is the dominant form of communication in society and gets the most attention. Third, Deaf people tend to socialize with one another and with hearing people who know how to sign.




De acordo com \citeonline{hill-2019-sign-languages,pereira-2011-conhecimento-alem-sinais}, as línguas de sinais distinguem-se das línguas orais porque utilizam o canal visual-espacial em vez do oral-auditivo. Por esse motivo, são denominadas línguas de modalidade gestual-visual (ou visual-espacial), uma vez que a informação linguística é recebida pelos olhos e produzida no espaço pelas mãos, pelo movimento do corpo e pela expressão facial. 
Devido a disso, acrescentam \citeonline{stewart-2021-barrons-asl}, não existe uma forma escrita conveniente dessas línguas, mas apenas glosas que representam uma aproximação do significado de seus sinais.

Elas são consideradas como línguas naturais, ou seja, aquelas que emergem naturalmente quando indivíduos se reúnem formando uma comunidade. Assim como ocorre para as línguas orais, não existe uma universalidade: cada país tem sua própria língua sinalizada e elas, por sua vez, refletem a cultura dos diferentes países em que são utilizadas.


% Línguas de sinais são línguas naturais, ou seja, que emergem (e não são inventadas) ``naturalmente'' quando os Surdos formam uma comunidade, muitas vezes por meio de sistemas educacionais. Assim como ocorre com as línguas na modalidade oral, cada país tem sua língua de sinais e não há uma universalidade. Devido à estreita relação entre língua e cultura, essas línguas acabam por refletir a cultura dos diferentes países onde são utilizadas~\cite{pereira-2011-conhecimento-alem-sinais, hill-2019-sign-languages}.

% Elas são produzidas no espaço pelas mãos, rosto e corpo e percebidas principalmente visualmente, em contraste com as línguas faladas, que são produzidas pela boca e trato vocal e percebidas principalmente auditivamente (embora gestos manuais e percepção visual de gestos e movimentos da boca sejam também importante para as línguas faladas). 
% Devido a isso, elas são denominadas línguas de modalidade gestual-visual (ou visual-espacial). Também por esse motivo, não existe uma forma escrita conveniente da língua de sinais, mas apenas glosas que representam uma aproximação do significado dos sinais~\cite{hill-2019-sign-languages, pereira-2011-conhecimento-alem-sinais,stewart-2021-barrons-asl}.

Apesar das particularidades acima, línguas orais e línguas sinalizadas compartilham os mesmos princípios quanto ao fato de que possuem um léxico e uma gramática. Ou seja, ambas apresentam um conjunto de símbolos convencionais bem como um sistema de regras que rege a combinação desses símbolos em unidades maiores de significado. 
Portanto, abordaremos na seção seguinte um pouco mais sobre esses elementos linguísticos presentes nas línguas de sinais.

% No entanto, línguas faladas e línguas de sinais seguem os mesmos princípios com relação ao fato de que têm um léxico e uma gramática, ou seja, elas apresentam um conjunto de símbolos convencionais e um sistema de regras que rege a combinação desses símbolos em unidades maiores~\cite{pereira-2011-conhecimento-alem-sinais}. Na seção seguinte discutiremos um pouco mais sobre esses elementos e a linguística da língua de sinais.


%\cite{hill-2019-sign-languages}
% Sign languages are produced by the hands, face, and body and perceived primarily visually, in contrast to spoken languages, which are produced by the mouth and vocal tract and perceived primarily auditorily (although manual gestures and visual perception of gestures and mouth movements are also important for spoken languages). Natural sign languages emerge (are not invented) when Deaf people form a community, often through educational systems. Sign languages are, therefore, primarily the languages of Deaf people, who cherish them for their cultural and community-building value. 

%\cite{pereira-2011-conhecimento-alem-sinais}
% ---
% As línguas de sinais distinguem-se das línguas orais porque utilizam o canal visual-espacial em vez do oral-auditivo . Por esse motivo, são denominadas línguas de modalidade gestual-visual  (ou visual-espacial), uma vez que a informação linguística é recebida pelos olhos e produzida no espaço, pelas mãos, pelo movimento do corpo e pela expressão facial .  
% Apesar da diferença existente entre línguas de sinais e línguas  orais, ambas seguem os mesmos princípios com relação ao fato  de que têm um léxico, isto é, um conjunto de símbolos convencionais, e uma gramática, ou seja, um sistema de regras que rege  o uso e a combinação desses símbolos em unidades maiores .












% ########################################################################################

% ===================================================
% \cite{pereira-2011-conhecimento-alem-sinais}
% ===================================================
%
% * Seguindo convenção proposta por James Woodward (1982), neste livro será  usado o termo “surdo” para se referir à condição audiológica de não ouvir, e o termo “Surdo” para se referir a um grupo particular de pessoas surdas que partilham  uma língua e uma cultura. 
% ---
% A língua de sinais é a língua usada pela maioria dos Surdos, na  vida diária . É a principal força que une a comunidade Surda, o  símbolo de identificação entre seus membros 
% ---
% Cada país tem sua língua de sinais, como tem sua língua na  modalidade oral. As línguas de sinais são línguas naturais, ou  seja, nasceram “naturalmente” nas comunidades Surdas. Uma vez  que não se pode falar em comunidade universal, tampouco está  correto falar em língua universal.  
% Outro aspecto a considerar é a relação estreita que existe entre  língua e cultura . As línguas de sinais refletem a cultura dos diferentes países onde são usadas, e esse é mais um argumento contra  a ideia de uma língua de sinais universal 
%---
% As línguas de sinais distinguem-se das línguas orais porque utilizam o canal visual-espacial em vez do oral-auditivo . Por esse motivo, são denominadas línguas de modalidade gestual-visual  (ou visual-espacial), uma vez que a informação linguística é recebida pelos olhos e produzida no espaço, pelas mãos, pelo movimento do corpo e pela expressão facial .  
% Apesar da diferença existente entre línguas de sinais e línguas  orais, ambas seguem os mesmos princípios com relação ao fato  de que têm um léxico, isto é, um conjunto de símbolos convencionais, e uma gramática, ou seja, um sistema de regras que rege  o uso e a combinação desses símbolos em unidades maiores .  
% ---
% Primeiros estudos (intro a linguística):
% As primeiras pesquisas linguísticas sobre as línguas de sinais,  mais especificamente sobre a língua de sinais americana, foram realizadas por William Stokoe, no início dos anos 1960, e tiveram como objetivo mostrar que os sinais poderiam ser vistos  como mais do que gestos holísticos aos quais faltava uma estrutura interna (Stokoe, 1960) . Ao contrário do que se poderia pensar à primeira vista, eles poderiam ser descritos em termos de um  conjunto limitado de elementos formacionais que se combinavam para formar os sinais .  
% A análise das propriedades formais da língua de sinais americana revelou que ela apresenta organização formal nos mesmos níveis encontrados nas línguas faladas, incluindo um nível  sublexical de estruturação interna do sinal (análoga ao nível fonológico das línguas orais) e um nível gramatical, que especifica os  modos como os sinais devem ser combinados para formarem frases e orações (Klima e Bellugi, 1979) .  
% Aos estudos sobre a língua de sinais americana se seguiram outros, cujo objeto eram as línguas de sinais usadas pelas comunidades de surdos em diferentes países, como França, Itália, Uruguai,  Argentina, Suécia, Brasil e muitos outros .  
% Essas línguas são diferentes umas das outras e independem  das línguas orais utilizadas nesses países 
% -----------------
% Concepções de surdez e de surdos 
% - Concepção clínico-patológica
% - Concepção socioantropológica (tentar adotar essa aqui)

% [Pág 30]:
% Martins (2004, p . 204-205) afirma que: “Sem língua não existem nem os surdos nem o modo de ser, cultural, surdo . Existiriam  apenas deficientes auditivos .” E segue com uma boa afirmação  em defesa da língua: “[ . . .] não é simplesmente o nível de audição  que vai definir quem é surdo ou deficiente auditivo” (op . cit .) . 

% [Pág 33]:
% Na história, constata-se que os Surdos sofreram perseguições  pelas pessoas ouvintes, que não aceitavam as diferenças e exigiam  uma cultura única por meio do modelo  ouvintista ou ouvintismo . São muitas  as lutas e histórias nas comunidades Surdas, em que o povo Surdo se une contra  as práticas dos ouvintes que não respeitam a cultura Surda (Strobel, 2008) .  
% Ainda hoje, muitos ouvintes tentam  diminuir os Surdos para que vivam isolados e tendo de assumir a  cultura ouvinte, como se esta fosse uma cultura única; ser “normal” para a sociedade significa ouvir e falar oralmente . Os ouvintes não prestam atenção aos Surdos que se comunicam por  meio da Libras . Consequentemente, não acreditam que os Surdos sejam capazes de estudar em faculdade ou realizar mestrado e  doutorado, por exemplo . “Os sujeitos ouvintes veem os sujeitos  surdos com curiosidade e, às vezes, zombam por eles serem diferentes” (Strobel, 2008, p . 22) .  
% A luta dos Surdos tem conduzido a várias vitórias, como o reconhecimento da Libras, o direito a tradutores e intérpretes da  língua brasileira de sinais–língua portuguesa e a uma educação  bilíngue para as crianças Surdas, que contemple a Libras e o português, este na modalidade escrita, entre muitas outras conquistas 

% [Pág 34] Cultura Surda
% Os Surdos constituem uma comunidade linguística minoritária, cujos elementos identificatórios são a língua de sinais e uma  cultura própria eminentemente visual . Têm um espírito gregário  muito importante que se manifesta em vários espaços . Esses espa-  ços “dos Surdos” são associações e clubes de Surdos onde desenvolvem suas próprias atividades . Constituem refúgios naturais da  língua de sinais e da identidade Surda (Strobel, 2008, p . 45) .  
% Diante da comunidade majoritariamente ouvinte, as comunidades Surdas apresentam suas próprias condutas linguísticas e seus  valores culturais . A comunidade Surda tem uma atitude diferente diante do déficit auditivo, já que não leva em conta o grau de  perda auditiva de seus membros . Pertencer à comunidade Surda pode ser definido pelo domínio da língua de sinais e pelos sentimentos de identidade grupal, fatores que consideram a surdez  como uma diferença, e não como uma deficiência .  
% Como ocorre com qualquer outra cultura, os membros das comunidades de Surdos compartilham valores, crenças, comportamentos e, o mais importante, uma língua diferente da utilizada  pelo restante da sociedade .  A língua de sinais, uma língua visual-espacial com gramática  própria, é uma das maiores produções culturais dos Surdos (Perlin, 2006) . Lane, Hoffmeister e Bahan (1996) referem que a língua de sinais tem basicamente três papéis para os Surdos: ela é símbolo da identidade social, é um meio de interação social e é  um depositário de conhecimento cultural.

% [Pág 97]
% A pessoa surda é definida como aquela que, por ter perda  auditiva, compreende o mundo e interage com ele por meio de  experiências visuais, manifestando sua cultura principalmente  pelo uso da Libras .



% ===================================================
% \cite{stewart-2021-barrons-asl}
% ===================================================
% What is ASL?
% American Sign Language, or ASL, is the language of the American Deaf Community. It is used in North America, and it is the only complete and natural sign language recognized by te Deaf community. This is a simple definition. Once it is understood that ASL maintains grammatical structure, syntax, and rules entirely separate from English grammar, we can begin to understand how to use it properly -- the way the American Deaf community intents.
% As you venture through this book, you will notice that ASL grammar is not only comprised of signed words or concepts, but it heavily involves the use of facial grammar to give information or meaning to signs. Facial grammar, including eye contact, facial expressions, eye gazing, and head movements are part of the unique grammatical structure of ASL. Because of this, and many other reasons, the visual language of ASL is fascinating for nonsigners to observe and quickly becomes a desired language to learn.

% English gloss
% ASL is an expressive and receptive language only. Because of the spatial and gestural qualities of ASL, there can be no convenient written form of ASL. What we can do is write English glosses of ASL signs. An English gloss is the best approximation of the meaning of a sign. It gives us a way of laying out ASL so that it can be studied and discussed, but is not a written form of ASL.

% Signing as a Choice of Communication
% [...] why, until recently, did so many hearing people know so little about signing? There are at least three reasons for this. First, Deaf people make up just a small fraction of the population in any area. Therefore, many hearing people never encounter a Deaf person in their through life. Second, speech is the dominant form of communication in society and gets the most attention. Third, Deaf people tend to socialize with one another and with hearing people who know how to sign.

% ASL Awareness
% Awareness of ASL has been growing since Professor William C. Stokoe, Jr., of Gallaudet University, known as the father of ASL linguistics, published his research on the linguistics of ASL about sixty years ago. His first paper, published in 1960, is titled "Sign Language Structure". This was followed by the first dictionary of ASL in 1965, Dictionary of American Sign Language on Linguistic Principles. Stokoe compiled the dictionary with two Deaf colleagues at Gallaudet, Carl Croneberg and Dorothy Casterline. In 1971, Stokoe established the Linguistic Research Laboratory at Gallaudet. Stokoe's work had a profound impact on ASL awareness in the United States and throughout the world, and we've even come a long way since then.
% ASL courses in high schools and colleges are booming. The television and movies industry has discovered the value of including Deaf actors and actresses in films. [...]

% The Physical Dimensions of ASL: The Five Parameters
% ASL is a visual-gestural language. It is visual because we see it and gestural because the signs are formed by the hands. Signing alone, however, is not an accurate picture of ASL. How signs are formed in space is important to understanding what they mean. The critical space is called the signing space and extends from the waist to just above the head and to just beyond the sides of the body. This is also the space in which the hands can move comfortably. As you will learn in this book, the signing space has a role in ASL grammar. Two or more concepts can be simultaneously expressed in ASL. This feat cannot be accomplished in a spoken language because speech is temporal in that one word rolls off the tongue at a time. One further dimension of ASL is the movement of the head and facial expressions, which help shape the meaning of ASL sentences.

% How Are Signs Formed?
% The five parameters below come together to create a sign.
% 1) Handshape is the shape of the hands when the sign is formed. The handshape may remain the same throughout the sign or it can change. If two hands are use to make a sign, both hands can have the same handshape of be different.
% 2) Orientation is the position of the hand(s) relative to the body. For example, the palms can be facing the body or away from the body, facing the ground, or facing upward.
% 3) Location is the place in the signing space where a sign is formed. Signs can be stationary [...] or they can move from one location in the signing space to another [...].
% 4) Movement of a sign is the direction in which the hand moves relative to the body. There is a variety of movements that range from a simple sliding movement [...] to a complex movement.
% 5) Nonmanual markers add to signs to create meaning. They consist of various facial expressions, head tilting, shoulder movement, and mouth movements. With a non-manual marker, the meaning of the sign can change completely.

% Cultural Importance
% ASL gives us access to Deaf culture. Learning ASL is not simply about learning another language. It is also about access. Even though we can learn something about any culture from reading about it, we acquire a deeper understanding when we can experience the culture or hear firsthand accounts from the people who are a part of the culture.
% ASL is one of the defining characteristics of the Deaf community. Although groups exist withing the community, such as the Black Deaf community and LGBTQAI+ Deaf communities. Deaf community members are bound instead by their language: ASL. To learn more about Deaf culture and tap into the resources of the Deaf community, you need a solid grasp of ASL.

% -----
% The Deaf Community
% Many hearing people view the world of the Deaf as a place where people don't hear -- where silence is a loud reminder of the difference between the two groups of people. But for Deaf people, silence is not the focus. What is important to us is that we obtain a lot of pleasure by being with other Deaf people. We relish the tales about other Deaf people's experiences in the Deaf community [...]

% Self-identification
% Deaf ("big D"):
% - identify as a culturally Deaf and part of the Deaf community
% - take pride in Deaf identity
% - may have an auditory device, such as cochlear implant, hearing aid, or FM system
% - may have a more severe hearing loss
% - use sign language as their primary source of communication
% - most likely attend a Deaf school/program
% - feel more comfortable in the Deaf world
%
% deaf ("little d"):
% - do not typically associate as members of the Deaf community
% - may have an auditory device, such as a cochlear implant, hearing aid, or FM system
% - may refer to hearing loss as a medical condition
% - may not use sign language as their primary choice of communication
% - may attend a mainstream school
% - may feel more comfortable in the hearing world
%
% Hard of hearing (HoH):
% - do not associate as members of the Deaf community
% - have hearing loss but may have residual hearing
% - possibly use an auditory device, such as a hearing or FM system, to access sounds
% - refer to hearing loss as a medical condition
% - may use sign language as their primary choice of communication
% - may attend a mainstream school
% - feel more comfortable in the hearing world

% Who Are Deaf People?
% Deaf people are a group of people who have a hearing loss, use a sign language as their primary means of communication, and have shared experiences associated with the hearing loss and the use of sign language.
% There is no way that you can point to a person sitting and reading a magazine in a lobby whom you have never met before and say, "that person is Deaf". Even if the person is wearing hearing aids, we don't know which community the person identifies with. Similarly, it's not important whether the person is European, African-American, Asian, or of some other ethnic origin. Age is not relevant, and neither is the social class or gender of the person. 
% The Deaf community is not shaped by any of these characteristics. In fact, having a hearing loss does not mean that a person is a member of the Deaf community, although it is certainly an important requirement.

% The pivotal mark of a Deaf person is how this person communicates. A Deaf person uses sign language [...]. This does not mean that the person cannot use other forms of communication, such as writing and speaking. Rather, ASL is the linguistic trademark that sets Deaf people apart from the communication behavior of all other groups of people. It is the reason we say that Deaf people represent a linguistic minority. It is also why some people who are deaf do not see themselves as belonging to the Deaf community. 
% We use the lowercase spelling of deaf to refer to a person or a group of people who have a substantial degree of hearing loss. Having a hearing loss does not mean that a person automatically knows how to sign. If a deaf person does not know sign language, then that person will not be able to access the varied cultural experiences associated with the Deaf community. Communication is basic, and ASL is the communication of the Deaf community.

% Can a nondeaf person who is fluent in ASL be a member of the Deaf community? No. Deaf people do not view nondeaf people as members of their community because nondeaf people lack a third critical characteristic, which is shared experiences. If you have normal hearing, then you will never have the experiences of a life that is centered on seeing. 

% Let's put all of this in perspective. Let's say you have a young friend who acquired a hearing loss and was fitted with hearing aids. WOuld we say that he was Deaf? No, we would say that he is hard of hearing because speaking is still his main means of communication. If his hearing continues to deteriorate, then we might say that he is becoming deaf; that is, he is acquiring a substantial degree of hearing loss. What if his difficulty with hearing leads him to learn to sign ASL, which becomes his primary language of communication, and he begins participating in some activities in the Deaf community? Would we say then that he is Deaf? We would probably say, "friend, welcome to the club".


% Technology and Other Adaptations
% "Many of the technological advances for the majority in our society have penalized Deaf people. This irony emerges most clearly in telecommunications. The invention of the telephone made it difficult for Deaf people to compete in the labor market. Radio became an important means of broadcasting information, whether commercial, political, governmental, or whatever, further cutting off Deaf people from the larger society surrounding them. Television did little to improve the situation, though it embraced the technology that could have (and to some extent does) include Deaf people. Talking pictures were a blow to the entertainment and education of Deaf people; they could enjoy the "silents" on par with the rest of the audience. But Deaf people and their supporters have not passively accepted the status quo. They have taken steps to reduce the handicap the new technologies have imposed". -- Jerome Schein (At Home Among Strangers)

% -----
% ASL Grammar
% ASL has visual, spatial, and gestural features that combine to create some grammatical structures that are unparalleled in the world of spoken languages. As you learn about ASL grammar, look for similarities in the world of spoken languages. [...]
% In particular, the spatial qualities of ASL grammar allow a signer to express more than one thought simultaneously -- a characteristic that cannot be duplicated in English. Facial grammar, or nonmanual signals, is also a significant component of ASL.

% Facial Grammar
% What's in a sign may not be what's in the mind. To capture the sense of what a signer is signing, you must read the signer's face and body. When you listen to someone speak, you listen not only to the words but also to how the words are spoken. The tone of the voice, the rise and fall of the pitch, the length of the pause, and the steadiness of voice are all features that you latch onto with little effort in your spoken communication.
% These traits are nonexistent in signing, but they do have parallel traits that are crucial to ASL's grammar. The raised eyebrow, the tilted head, the open mouth, hunching of the shoulders, and a sign held slightly longer than others shape the meaning of the signs that are made by the hands. We call these nonmanual signals (NMS) facial grammar, which allows you to use facial expressions, your body, and gestures to add meaning and additional information to your signing. Mastering ASL cannot occur without a mastery of facial grammar.


% ===================================================
% \cite{hill-2019-sign-languages}
% ===================================================
%
% [Pág 2]
% Sign languages are produced by the hands, face, and body and perceived primarily visually, in contrast to spoken languages, which are produced by the mouth and vocal tract and perceived primarily auditorily (although manual gestures and visual perception of gestures and mouth movements are also important for spoken languages). Natural sign languages emerge (are not invented) when Deaf people form a community, often through educational systems. Sign languages are, therefore, primarily the languages of Deaf people, who cherish them for their cultural and community-building value. 
% It is important to recognize the connection between sign languages and Deaf communities. Until relatively recently, Deaf communities have been told (explicitly and implicitly) that their “sign communication” was inferior, broken, unimportant, or insufficient. Educational systems and the broader hearing majority community would stress the value of learning the spoken language, even at the expense of the sign language. In fact, such attitudes persist, both in areas where the national sign language has not been deeply studied linguistically and in areas where it has been studied but the focus for economic advancement is on the spoken language. However, the natural sign languages of Deaf communities are completely linguistic, rule-governed, capable of expressing anything, and fully worthwhile. We unreservedly endorse such affirmations of the value of sign languages and promote their use in all aspects of the lives of Deaf people. 

% Who belongs to the Deaf community? The “d” is capitalized to reinforce the view that Deaf communities form cultural groups with practices and values that are in some cases distinct from those of non-Deaf communities. These cultural effects are passed down within the community, from parents to children in some cases, but more often through interactions of Deaf people from different families. The leaders of Deaf communities are usually Deaf adults who were raised with Deaf parents or within the community from a very early age. Generally, members of the Deaf community are audiologically deaf or hard-of-hearing (and they shun the label “hearing impaired”). The hearing children born to Deaf parents are often known as Codas (from the name of an organization, CODA, ‘children of Deaf adults’), and they are sometimes part of the Deaf community. 

% It is important to note that people have many identities with intersectional effects, and in this respect, not all Deaf people have the same experiences, values, and life view. A Deaf person’s identity as Deaf will be affected by their identity in other ways, including race, ethnicity, gender identity, etc. Almost all research on the American Deaf community has focused only on a subset of Deaf people, so it is important to bear in mind that others might share some but not all of the characteristics described here. 
% Sign languages are, then, Deaf languages. Just as with the languages of other minority groups who have experienced oppression, hearing researchers who benefit from the study of sign languages (both in personal satisfaction and in economic, career, and other means) must acknowledge the primacy of Deaf signers and treat their language with the utmost respect.
