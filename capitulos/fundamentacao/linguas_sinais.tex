\section{O Surdo e a língua de sinais}
\label{sec:lingua-sinais}

Segundo \citeonline{pereira-2011-conhecimento-alem-sinais}, a língua de sinais é a língua utilizada pela maioria dos Surdos em sua vida diária. Muito mais do que isso, ela é a principal força que une essa comunidade e o símbolo de identificação entre seus membros.


Quando utilizados os termos ``Surdo'' ou ``comunidade Surda'' (com ``s'' capitalizado) não nos referimos apenas a uma condição clínica, mas a um grupo de indivíduos que, além de possuírem perda auditiva, utilizam a língua de sinais como principal meio de comunicação e compartilham experiências culturais associadas à surdez e ao uso dessa língua. Esses elementos estão interconectados e não pode-se definir comunidade Surda sem considerá-los como um todo.

De fato, ter uma perda auditiva não significa que uma pessoa seja membro da comunidade Surda ou que automaticamente saiba sinalizar, embora certamente esses sejam requisitos importantes:


\begin{citacao}
    Não há como apontar uma pessoa sentada num saguão lendo uma revista e dizer: ``esta pessoa é Surda''. Ainda que ela esteja utilizando aparelhos auditivos, não há como saber com qual comunidade ela se identifica.

    De maneira semelhante, não importa se ela é europeia, afro-americana, asiática ou de outra etnia; sua idade não é relevante, tampouco sua classe social ou gênero; a comunidade Surda não é moldada por nenhuma dessas características.~\cite[tradução nossa]{stewart-2021-barrons-asl}

\end{citacao}


Há um aspecto cultural essencial acompanhado de um forte sentimento de identidade grupal, que fazem com que a surdez seja percebida como uma diferença e não como uma deficiência, reiteram \citeonline{pereira-2011-conhecimento-alem-sinais}.


Apesar disso, no decorrer da história os Surdos tiveram sua identidade estigmatizada e desvalorizada pela sociedade ouvinte, que não aceitava a língua de sinais. Segundo \citeonline{hill-2019-sign-languages}, há até relativamente pouco tempo dizia-se a esses indivíduos que sua forma de comunicação era inferior, quebrada, sem importância ou insuficiente. Os sistemas educacionais e a comunidade majoritariamente ouvinte fizeram prevalecer a utilização da língua falada, mesmo às custas da língua de sinais.

De fato, atitudes como essas ainda persistem. \citeonline{pereira-2011-conhecimento-alem-sinais} comentam que muitos ouvintes tentam  diminuir os Surdos para que eles vivam isolados, tendo de assumir a cultura ouvinte como se ela fosse a única existente. Ser ``normal'' significaria, portanto, poder ouvir e falar oralmente:


\begin{citacao}
    Os ouvintes não prestam atenção aos Surdos que se comunicam por meio da língua de sinais. Consequentemente, não acreditam que eles sejam capazes de estudar em faculdade ou realizar mestrado e  doutorado, por exemplo. \cite{pereira-2011-conhecimento-alem-sinais}
\end{citacao}

\begin{citacao}
    Os ouvintes veem os Surdos com curiosidade e, às vezes, zombam deles por serem diferentes. \cite{strobel-2016-cultura-surda}
\end{citacao}


São muitas as lutas pelas quais os Surdos têm atravessado mas que, sobretudo nos últimos anos, têm conduzido a vitórias importantes como o reconhecimento das línguas de sinais oficialmente como línguas em diversos países, o direito a tradutores e intérpretes em eventos e canais públicos de comunicação e o acesso a uma educação bilíngue para as crianças Surdas, entre outras conquistas. No Brasil, por exemplo, a \acrfull{libras} foi reconhecida como uma língua em 2002 e, nos Estados Unidos, a \acrfull{asl} foi reconhecida ainda em 1989 \cite{brasil-2002-lei10436,pereira-2011-conhecimento-alem-sinais, jay-2011-dont-just-sign}.



De acordo com \citeonline{hill-2019-sign-languages,pereira-2011-conhecimento-alem-sinais}, as línguas de sinais distinguem-se das línguas orais porque utilizam o canal visual-espacial em vez do oral-auditivo. Por esse motivo, são denominadas línguas de modalidade gestual-visual, onde a informação linguística é recebida pelos olhos e produzida no espaço pelas mãos, pelo movimento do corpo e pela expressão facial.
Devido a disso, acrescentam \citeonline{stewart-2021-barrons-asl}, não existe uma forma escrita conveniente para essas línguas, mas apenas glosas que representam uma aproximação do significado de seus sinais.

Elas são consideradas como línguas naturais, ou seja, aquelas que emergem naturalmente quando indivíduos se reúnem formando uma comunidade. Assim como ocorre para as línguas orais, não existe uma universalidade: cada país tem sua própria língua sinalizada e elas, por sua vez, refletem a cultura dos diferentes países em que são utilizadas.


Apesar das particularidades acima, línguas orais e línguas sinalizadas compartilham os mesmos princípios quanto ao fato de que possuem um léxico e uma gramática. Ou seja, ambas apresentam um conjunto de símbolos convencionais bem como um sistema de regras que rege a combinação desses símbolos em unidades maiores de significado.
Dessa forma, a seção seguinte explorará em maior profundidade esses elementos linguísticos presentes nas línguas de sinais.
