\section{Reconhecimento de língua de sinais}
\label{sec:slr}

De acordo com \citeonline{wadhawan-2019-slr-literature-review,cooper-2011-slr}, o \acrfull{slr} é uma área de pesquisa colaborativa e multidisciplinar que tem por objetivo elaborar métodos e algoritmos para identificar os sinais articulados pelos usuários dessa língua e compreender seu significado.

É uma área capaz de contribuir com a quebra de barreiras existentes para os usuários dessa língua e facilitar a comunicação cotidiana entre Surdos e ouvintes, segundo \citeonline{rastgoo-2021-slr-deep-survey,papastratis-2021-ai-technologies-sl}.
Isso é importante porque, além de promover a inclusão dos Surdos em sociedade, aborda o problema atual de que as tecnologias de comunicação são, em sua maioria, desenvolvidas para suportar línguas faladas ou escritas, mas excluem as línguas sinalizadas. Por exemplo, o WhatsApp, Telegram e iMessage tornaram-se ferramentas imprescindíveis em nossas vidas, porém, a comunidade Surda enfrenta diversos desafios para utilizá-las.

Ainda segundo os autores, apesar dessas necessidades terem sido identificadas há muito tempo pela comunidade acadêmica, apenas recentemente a área de \acrshort{slr} passou a receber mais atenção.
Isso deve-se principalmente aos avanços ocorridos nas tecnologias de sensoriamento e dos algoritmos de \acrshort{ia}, que abriram caminho para o desenvolvimento de aplicações capazes de abordar tais demanda de maneira mais efetiva. 
Além disso, o advento das arquiteturas de \acrshort{dl} proporcionou uma melhora significativa no desempenho dos algoritmos utilizados nesta área.

\citeonline{koller-2020-quantitative-survey-slr} realizou uma análise baseada nos estudos mais relevantes em \acrshort{slr} publicados desde 1983, a qual possibilita delinear melhor essa evolução recente e o estado da arte atual.
Além disso, as revisões apresentadas por \citeonline{rastgoo-2021-slr-deep-survey,papastratis-2021-ai-technologies-sl,wadhawan-2019-slr-literature-review} também contribuem para estender essa análise. Esse panorama será discutido na seção a seguir.

