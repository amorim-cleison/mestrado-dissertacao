\section{Objetivos}
\label{sec:introducao-objetivos}

O objetivo geral deste trabalho consiste em propor uma abordagem de \acrlong{slr} baseada na linguística, a qual considere as complexidades de sua natureza visual e preencha algumas das lacunas deixadas em aberto por pesquisas na área, afim de contribuir com avanços mais efetivos.

% O objetivo geral deste trabalho consiste em propor uma abordagem de processamento de línguas de sinais baseada na sua linguística, que seja coerente com as complexidades de sua natureza visual e as necessidades do mundo real para, dessa forma, contribuir com o preenchimento das lacunas apresentadas acima que limitam avanços significativos nessa área.

% Objetivo geral:
% - abordagem que contribua para o avanço do processamento das línguas de sinais 
%     - mas que preencha muitas das lacunas deixadas em aberto por pesquisas na área
%     - explorando as complexidades dos sinais apresentadas acima

Como objetivos específicos, este trabalho busca alcançar:

\begin{itemize}
    \item Propor uma estratégia para computar atributos linguísticos a partir de \textit{datasets} existentes da língua de sinais;

    \item Disponibilizar um \textit{dataset} de atributos linguísticos, o qual atualmente é inexistente, para suportar o desenvolvimento de novas técnicas para as línguas de sinais;

          %FIXME: vc disse anteriormente que era importante considerar a lingua como um todo, visto que ela era visual, mas seu ultimo objetivo especifico fala somente de NLP. nao precisa contextualizar melhor q vc considera mais coisa?
          % \item Aplicar algoritmos de \acrlong{nlp} para reconhecer sinais através desses atributos linguísticos e avaliar os resultados obtidos a partir disso.
    \item Identificar, aplicar e avaliar algoritmos que possibilitem abordar o reconhecimento da língua de sinais através de sua linguística, acomodando as complexidades de sua natureza visual.

\end{itemize}


% Objetivos específicos:
% - propor uma abordagem de processamento de língua de sinais 
%     - que considere as particularidades de sua linguística
%     - dê um passo na direção de superar as lacunas deixadas abertas por pesquisas na área 

% - introduzir um dataset de features baseadas na fonologia da língua de sinais

% - introduzir um dataset de coordenadas de sinalizadores reais 
%     - simplifique novas pesquisas derivar features sintéticas 
%     - computadas a partir das coordenadas 
