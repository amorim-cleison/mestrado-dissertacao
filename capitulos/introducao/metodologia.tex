\section{Metodologia}
\label{sec:introducao-metodologia}

A metodologia aplicada neste trabalho consistiu em primeiro compreender os desafios atuais do Surdo, da língua de sinais e da área de pesquisa para, em seguida, estabelecer uma proposta capaz de abordar algumas das lacunas encontradas e avaliar seus resultados.
As etapas percorridas para isso incluem:

% A metodologia aplicada nesta dissertação concentra-se em compreender os desafios atuais da área de pesquisa para assim introduzir uma proposta que suporte avanços futuros coerentes com as necessidades do mundo real.
% As etapas percorridas aqui podem ser sumarizadas como:

\begin{itemize}
    \item Revisão do panorama do Surdo, das línguas de sinais e de sua linguística;
    \item Revisão da área de \acrlong{slr} e das lacunas existentes;
    \item Elaboração de uma proposta que aborde as lacunas acima e produza artefatos que suportem novas pesquisas nessa direção;
    \item Realização de experimentos e análise dos resultados.
\end{itemize}

% \begin{itemize}
%     \item Revisão do panorama atual da deficiência auditiva e do papel que as línguas de sinais desempenham aqui;
%     \item Revisão do panorama das pesquisas atuais em processamento de língua de sinais e das lacunas que têm limitado progressos mais expressivos na área.
%     \item Desenvolvimento de uma proposta que aborde as lacunas acima, contribuindo para preencher algumas delas e produzindo artefatos que suportem novas pesquisas a evoluir nessa direção;
%     \item Realização de experimentos e análise dos resultados coletados.
% \end{itemize}

% - análise do panorama atual das línguas de sinais 
% - análise do panorama atual das pesquisas na área e identificação de principais lacunas para seu avanço
% - revisão da literatura das línguas de sinais e de sua linguísticas
% - seleção de um conjunto de atributos linguísticos
% - análise de técnicas algébricas para suportar o processamento de atributos linguísticos selecionados
% - seleção de um modelos sequenciais de aprendizagem de máquina para os experimentos
% - execução dos experimentos e análise dos resultados coletados
