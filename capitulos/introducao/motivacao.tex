% TODO: corrigir 'deficiente auditivo' ??? -- de acordo com PNS (pg 40), é apenas aquele que tem: (1) Muita dificuldade de ouvir ou não consegue de modo algum ouvir
Segundo a \acrfull{oms}, 19\% da população mundial apresenta algum grau de perda auditiva -- ou seja, em meio a uma população de 7,7 bilhões de habitantes, 1,5 bilhões de indivíduos sofrem com perda auditiva. Dentre esses, 450 milhões referem-se a perda de grau moderado ou severo e que necessitam de acesso a cuidados auditivos e outros serviços de reabilitação -- é o que nos revela o \textit{World Report on Hearing} (Relatório Mundial sobre Audição) publicado em 2021~\cite{who-2021-report-hearing}.

% Conforme o mundo alcança 10 bilhões de habitantes até 2050, a \acrshort{oms} estima que esse 2,5 bilhões de indivíduos (ou 25\% da população mundial) viverão com perda auditiva, dentre os quais 700 milhões serão perda de grau moderado ou severo~\cite{who-2021-report-hearing, opas-2021-oms-estima}.

% O mundo possui hoje 7,7 bilhões de habitantes, dentre os quais 1,5 bilhões de indivíduos possuem perda auditiva (o que equivale a 19\% da população mundial), sendo 450 milhões perda auditiva de moderada ou severa -- é o que aponta a \acrfull{oms} através do seu \textit{World Report on Hearing} (Relatório Mundial sobre Audição) publicado em 2021~\cite{who-2021-report-hearing}.

% Conforme a população mundial cresce para cerca de 10 bilhões de habitantes em 2050, estima-se que 2,5 bilhões de pessoas terão perda auditiva, dentre os quais 700 milhões serão de grau moderado ou elevado e necessitarão de acesso a cuidados auditivos e outros serviços de reabilitação~\cite{who-2021-report-hearing, opas-2021-oms-estima}

No Brasil, o panorama é de 10,7 milhões de habitantes com perda auditiva (ou 5\% da população), dos quais 2,3 milhões correspondem a perda moderada ou severa~\cite{ebc-2019-10-milhoes-pessoas, ibge-2021-pns, ibge-2021-projecao-populacao}. De acordo com o Instituto Locomotiva e o \citeauthor{ibge-2021-pns}, há uma homogeneidade entre homens e mulheres nessa população (54\% homens e 46\% mulheres), mas uma concentração maior nas regiões Sudeste (42\%) e Nordeste (26\%), seguidas pelo Sul (19\%), Norte (7\%) e Centro-Oeste (6\%). Além disso, como a perda auditiva é adquirida ao longo da vida em 91\% dos casos e agrava-se com o passar dos anos, nota-se uma predominância de indivíduos na faixa dos 60 anos de idade ou mais (57\%)~\cite{ebc-2019-10-milhoes-pessoas, ibge-2021-pns}.

Ainda de acordo com a \acrshort{oms}, as últimas décadas testemunharam avanços revolucionários no diagnóstico e reabilitação de problemas auditivos, como no campo da tecnologia auditiva, diagnóstico e telemedicina, com inovações que permitem que perdas e doenças relacionadas à audição sejam identificadas em qualquer idade e ambiente. Gestão médica e cirúrgica, aparelhos auditivos, implantes cocleares, terapia de reabilitação, linguagem de sinais e legendagem são exemplos de soluções que podem garantir que pessoas com doenças de ouvido ou com perda auditiva tenham acesso à educação e comunicação e, assim, tenham a oportunidade de cumprir seu potencial~\cite{who-2021-report-hearing}.

% Ao analisar dispositivos e intervenções capazes de auxiliar no diagnóstico e reabilitação dessas pessoas, a \acrshort{oms} afirma que as últimas décadas testemunharam avanços revolucionários, como no campo da tecnologia auditiva, diagnóstico e telemedicina, com inovações que permitem que perdas e doenças relacionadas à audição sejam identificadas em qualquer idade e ambiente. Gestão médica e cirúrgica, aparelhos auditivos, implantes cocleares, terapia de reabilitação, linguagem de sinais e legendagem são exemplos de soluções que podem garantir que pessoas com doenças de ouvido ou com perda auditiva tenham acesso à educação e comunicação e, assim, tenham a oportunidade de cumprir seu potencial~\cite{who-2021-report-hearing}.


No entanto, há uma disparidade no acesso a esses recursos, uma vez que a grande maioria das pessoas com perda auditiva não consegue acessá-los por viver em locais de baixa renda, onde profissionais especializados e serviços para cuidados auditivos não estão comumente disponíveis~\cite{who-2021-report-hearing}. Segundo a \citeauthor{opas-2021-oms-estima}, 78\% desses países têm menos de um otorrinolaringologista, 93\% têm menos de um audiologista, 83\% têm menos de um fonoaudiólogo, e 50\% têm menos de um professor para deficientes auditivos por milhão de habitantes. Mesmo em países com proporções relativamente altas desses profissionais, há uma distribuição desigual e isso não só representa desafios para as pessoas que precisam de atenção, mas também impõe demandas excessivas aos quadros que prestam esses serviços~\cite{opas-2021-oms-estima}.

%\begin{quote}
%    Entre os países de baixa renda, cerca de 78\% têm menos de um especialista em ouvido, nariz e garganta por milhão de habitantes; 93\% têm menos de um audiologista por milhão; apenas 17\% têm um ou mais fonoaudiólogos por milhão; e 50\% têm um ou mais professores para pessoas com deficiência auditiva por milhão. Mesmo em países com proporções relativamente altas de profissionais de saúde auditiva e de ouvido, há distribuição desigual de especialistas. Isso não só representa desafios para as pessoas que precisam de atenção, mas também impõe demandas excessivas aos quadros que prestam esses serviços~\cite{opas-2021-oms-estima}.
%\end{quote}

Marcos Sarvat, membro da Câmara Técnica de Cirurgia de Cabeça e Pescoço e Otorrinolaringologia do \acrfull{cfm}, ratifica que esse problema na assistência à saúde para deficientes auditivos é acentuado em países subdesenvolvidos, entre os quais está o Brasil. Segundo ele, é difícil o acesso à prevenção por meio de exames periódicos, bem como às consultas, aos medicamentos, às cirurgias e às próteses~\cite{ebc-2021-oms-estima}. Ele destaca ainda que essa falta de cuidados adequados gera deficiências estruturais, que resultam numa pior qualidade de vida que se estende a todas as gerações:

% Marcos Sarvat, membro da Câmara Técnica de Cirurgia de Cabeça e Pescoço e Otorrinolaringologia do \acrfull{cfm}, ratifica que o problema na assistência à saúde para as pessoas com perda de audição é acentuado nos países subdesenvolvidos, entre os quais está o Brasil. Segundo ele, é difícil o acesso à prevenção por meio de exames periódicos, bem como às consultas, aos medicamentos, às cirurgias e às próteses~\cite{ebc-2021-oms-estima}. Ele destaca ainda que essa falta de cuidados adequados à saúde auditiva gera deficiências estruturais, que resultam numa pior qualidade de vida que se estende a todas as gerações:

\begin{citacao}
    {[...]} isso reflete na educação das crianças. Uma criança que ouve mal aprende mal e se torna um adulto menos capaz do que seria, e assim por diante. Temos uma cascata de efeitos do idoso abandonado, do idoso solitário, da depressão, da perda de motivação, do desvínculo familiar, da perda de condições de trabalho. Tudo isso vem em cascata e é ignorado~\cite{ebc-2021-oms-estima}.
\end{citacao}

% TODO: revisar 'surdo' aqui:
Reflexos dessas deficiências estruturais são também encontrados na dificuldade de acesso a oportunidades básicas por esses indivíduos, como educação (dados de 2019 mostram que apenas 6\% concluíram o ensino superior; 14\% o ensino médio, 8\% o ensino fundamental e 71\% não possuem grau de instrução) e emprego (apenas 25\% estão no mercado de trabalho)~\cite{ibge-2021-pns}. Além disso, a grande maioria (87\%) não utiliza aparelhos auditivos porque são caros e inacessíveis. Renato Meirelles, que é presidente do Instituto Locomotiva, comenta:

\begin{citacao}
    {[...]} como a população surda teve menos oportunidade de estudar do que a população ouvinte; como tem mais dificuldade no mercado de trabalho do que a população ouvinte; o dinheiro para conseguir o aparelho é ainda mais difícil. Esse conjunto de preconceitos que existe na sociedade acaba criando um círculo vicioso que não possibilita que os surdos e os ouvintes tenham as mesmas oportunidades de se dar bem na vida~\cite{ebc-2019-10-milhoes-pessoas}.
\end{citacao}


Em meio a esses desafios que limitam o acesso aos cuidados e tecnologias adequados para reabilitação desses indivíduos, a língua de sinais acaba sobressaindo-se como uma alternativa que é mais inclusiva -- isso porque impõe menos barreiras ao não demandar investimento financeiro elevado ou a disponibilidade de profissionais especializados no serviço público de saúde pelo país. Apesar disso, ela é capaz de assegurar o desenvolvimento cognitivo, facilitar a comunicação, e possibilitar que crianças com deficiência auditiva obtenham educação e desenvolvimento socio-emocional adequado~\cite{who-2021-report-hearing}. % TODO: adicionar mais aqui
É possível aprender línguas de sinais por meio da própria família, da comunidade surda ou de recursos como livros, cursos, vídeos e materiais disponibilizados muitas vezes gratuitamente pela internet ou em escolas da rede pública de educação e institutos de surdos -- entre os quais o \acrfull{ines} é talvez o mais antigo, fundado em 1857~\cite{pereira-2011-conhecimento-alem-sinais}. 


% é capaz de assegurar o desenvolvimento cognitivo, facilitar a comunicação, e possibilitar que crianças obtenham educação e desenvolvimento socio-emocional adequado~\cite{who-2021-report-hearing}. Ela torna-se mais acessível porque não demanda investimento financeiro elevado ou que profissionais especializados estejam disponíveis de forma igualitária sobretudo no serviço público de saúde pelo país -- é possível aprender a língua de sinais através da própria família, da comunidade surda ou por meio de cursos, livros e outros recursos disponibilizados de forma gratuita na internet, em escolas públicas ou em institutos de surdos -- como o \acrfull{ines}, em funcionamento desde 1857~\cite{pereira-2011-conhecimento-alem-sinais}.

No Brasil, a língua sinalizada oficial é a \acrfull{libras}, a que foi reconhecida como meio legal de comunicação e expressão pela Lei n° 10.436, de 24 de abril de 2002~\cite{brasil-2002-lei10436}. Através dessa lei, a língua passou a receber maior atenção por educadores e pesquisadores, fazendo crescer significativamente o número de adeptos e defensores de sua utilização~\cite{pereira-2011-conhecimento-alem-sinais}.

% TODO: revisar se há contradição aqui:
% No entanto, apesar do papel fundamental que essa língua é capaz de exercer na comunicação do deficiente auditivo, dados do \acrshort{ibge} revelam que nem todos esses indivíduos a conhecem -- no Brasil, apenas 13\% dos que possuem perda auditiva moderada e 61\% daqueles com perda severa conhecem a \acrshort{libras}~\cite{ibge-2021-pns}. Apesar de não haver um análise concreta das razões que conduzem a isso, pode-se enumerar alguns motivos que contribuem para tal: uma vez que a perda auditiva é adquirida ao longo da vida em 91\% dos casos e, muitos desses indivíduos já são oralizados e usuários do português, eles preferem continuar utilizando a própria fala (auxiliados por dispositivos auditivos ou implantes cocleares) ou adotar a leitura labial; em outra parcela, há resistência de muitos pais de crianças que nascem surdas ou perdem a audição muito cedo, os quais rejeitam a língua de sinais e impõem a oralização de seus filhos; por fim, pode haver ainda um deficit ao incorporar e preparar os professores para o ensino da \acrshort{libras} como parte da currículo regular, fazendo com que muitas indivíduos concluam seus estudos como analfabetos funcionais~\cite{lobato-2021-surdez-nao-sinonimo-libras, ebc-2019-10-milhoes-pessoas, senado-2019-baixo-alcance-lingua-sinais}.


Se por um lado a língua de sinais é uma poderosa ferramenta que permite a comunicação e o desenvolvimento cognitivo do deficiente auditivo; por outro lado ainda há um obstáculo relevante que os distancia de uma plena inclusão em sociedade: a escassez de ouvintes que conhecem a língua de sinais e conseguem se comunicam por meio dela~\cite{bragg-2019-slr-interdisciplinary}. De acordo com a \citeauthor{senado-2019-baixo-alcance-lingua-sinais}, a língua de sinais está restrita à comunidade surda e isso transforma atividades corriqueiras num grande desafio, levando eles ao isolamento. No transporte público, por exemplo, esses indivíduos não conseguem pedir ajuda, requisitar informações sobre as rotas ou ter acesso aos anúncios nos alto-falantes sobre mudanças nelas; no cinema, eles não podem assistir filmes nacionais, pois apenas filmes estrangeiros dispõem de legendas; em lojas, é raro encontrar vendedores preparados para interagir em língua de sinais; no serviço de saúde, os surdos perdem sua vez se não estiverem atentos à enfermeira que fala o nome do próximo paciente, além de estarem suscetíveis a situações ainda mais trágicas: ONGs dedicadas aos surdos afirmam que não são raros os casos em que pacientes saem de consultas com uma prescrição errada de medicamento porque o médico não entendeu quais eram os sintomas; no mercado de trabalho, é comum empresas contratarem esses indivíduos para cumprir as cotas exigidas em lei, mas não se preocuparem em saber como eles são -- eles são muitas vezes vistos como incapazes e são designados a tarefas secundárias~\cite{senado-2019-baixo-alcance-lingua-sinais}.


% Apesar disso, os surdos ainda estão longe de uma plena inclusão na sociedade: o grande obstáculo é a escassez de ouvintes que se comuniquem na língua de sinais. De acordo com a \citeauthor{senado-2019-baixo-alcance-lingua-sinais}, a \acrshort{libras} está restrita à comunidade surda e isso transforma atividades corriqueiras num grande desafio, levando o surdo ao isolamento. No transporte público, por exemplo, esses indivíduos não conseguem pedir ajuda, requisitar informações sobre as rotas ou ter acesso aos anúncios nos alto-falantes sobre mudanças nelas; no cinema, eles não podem assistir filmes nacionais, pois apenas filmes estrangeiros dispõem de legendas; em lojas, não é difícil encontrar vendedores despreparados para interagir em língua de sinais ou que demonstrem preconceito com esses indivíduos; no serviço de saúde, os surdos perdem vez se não estiverem atentos à enfermeira que chama o nome do próximo paciente e estão suscetíveis a situações que podem ser ainda mais trágicas: ONGs dedicadas aos surdos afirmam que não são raros os casos em que pacientes com problemas sérios de saúde saem de consultas com uma prescrição errada de remédio, porque o médico não entendeu quais eram os sintomas; no mercado de trabalho, é comum empresas contratarem os surdos para cumprir as cotas exigidas em lei, mas não se preocuparem em saber como eles são -- elas encaram os surdos como incapazes e os atribuem tarefas secundárias~\cite{senado-2019-baixo-alcance-lingua-sinais}.

Um passo importante para contribuir com a quebra dessas barreiras consiste na pesquisa e desenvolvimento de tecnologias de processamento de língua de sinais -- que inclui o reconhecimento, geração e tradução dessa língua. 
Por meio delas, é possível conectar usuários e não-usuários da língua sinalizada, provendo assistência à sua comunicação e permitindo que a mensagem seja compreendida pelos interlocutores independentemente da barreira linguística -- semelhante ao que se observa para diálogos envolvendo línguas não-visuais de nacionalidades distintas.
Segundo \citeauthor{bragg-2019-slr-interdisciplinary}, essas tecnologias também tornariam acessíveis aos usuários dessa língua recursos úteis ao nosso cotidiano e que ainda não estão disponíveis para eles, como os assistentes pessoais, que são limitados ao uso de comandos de voz e que poderiam passar a responder pessoas articulando sinais. Além disso, a tradução dos sinais para texto poderia permitir o uso de mecanismos de busca ou outros sistemas baseados em texto de uma forma mais amigável; o processo inverso, por sua vez, poderia substituir automaticamente conteúdo textual por animações em língua de sinais. Outras possibilidades incluiriam a transcrição automática de vídeos produzidos em língua de sinais -- o que permitiria, por exemplo, uma indexação e busca desse conteúdo; a interpretação em tempo real quando não há intérpretes humanos disponíveis; ou outras ferramentas e aplicativos educacionais~\cite{bragg-2019-slr-interdisciplinary}.


% These technologies would make voice-activated services newly accessible to deaf sign language users – for example, enabling the use of personal assistants (e.g., Siri and Alexa) by training them to respond to people signing. They would also enable the use of text-based systems – for example by translating signed content into written queries for a search engine, or automatically replacing displayed text with sign language videos. Other possibilities include automatic transcription of signed content, which would enable indexing and search of sign language videos, real-time interpreting when human interpreters are not available, and many educational tools and applications.

% reconhecimento, tradução e transcrição dessa língua para outros idiomas (e de outros idiomas para ela). Em sua essência, essas tecnologias visam prover assistência nessa comunicação permitindo que esses indivíduos compreendam-se mutuamente mesmo utilizando línguas distintas. É uma tarefa desafiadora que tem sido abordada de forma multidisciplinar no decorrer das últimas décadas por diferentes subáreas da Ciência da Computação. \cite{bragg-2019-slr-interdisciplinary}

% Um passo importante para contribuir com a quebra dessas barreiras consiste na pesquisa e desenvolvimento de tecnologias para o reconhecimento, tradução e transcrição dessa língua para outros idiomas (e de outros idiomas para ela). Em sua essência, essas tecnologias visam prover assistência nessa comunicação permitindo que esses indivíduos compreendam-se mutuamente mesmo utilizando línguas distintas. É uma tarefa desafiadora que tem sido abordada de forma multidisciplinar no decorrer das últimas décadas por diferentes subáreas da Ciência da Computação. \cite{bragg-2019-slr-interdisciplinary}



% Um passo importante para redução dessas barreiras na comunicação entre usuários e não-usuários da língua sinalizada compreende a pesquisa e desenvolvimento de tecnologias capazes de realizar o reconhecimento, tradução e transcrição dessa língua para outros idiomas e vice-versa. Essa é uma tarefa desafiadora que vem sendo abordada de forma multidisciplinar no decorrer das últimas décadas por subáreas da Ciência da Computação -- a exemplo do \acrfull{slr}, que aplica reconhecimento de padrões, visão computacional, processamento de linguagem natural e linguística para construir métodos e algoritmos objetivam identificar sinais produzidos e compreender seu significado~\cite{wadhawan-2021-slr-systems-review}.

% TODO: revisar se SLR deveria ser especificado aqui (ou falar de forma mais abrangente das pesquisas)
O \acrfull{slr} é uma das áreas de pesquisa empenhadas em contribuir para que tecnologias como essas sejam desenvolvidas. Trata-se de uma área colaborativa que envolve reconhecimento de padrões, visão computacional, processamento de linguagem natural e linguística para construir métodos e algoritmos capazes de identificar sinais produzidos e compreender seu significado~\cite{wadhawan-2021-slr-systems-review}. 
Apesar disso, \citeauthor{cooper-2011-slr} acreditam que o progresso apresentado por ela nas últimas décadas ainda foi pouco expressivo:

\begin{citacao}
    Enquanto sistemas de reconhecimento automático da fala avançaram ao ponto de estarem comercialmente disponíveis, o reconhecimento de sinais ainda está em sua infância. Atualmente, todos os serviços comerciais de tradução de sinais são baseados em humanos e requerem pessoal especializado -- o que os tornam caros~\cite[tradução nossa]{cooper-2011-slr}.
\end{citacao}

% \citeauthor{bragg-2019-slr-interdisciplinary} argumentam que 
%Current research in sign language processing occurs in disciplinary silos, and as a result does not address the problem comprehensively. For example, there are many computer science publications presenting algorithms for recognizing (and less frequently translating) signed content. The teams creating these algorithms often lack Deaf members with lived experience of the problems the technology could or should solve, and lack knowledge of the linguistic complexities of the language for which their algorithms must account. The algorithms are also often trained on datasets that do not reflect real-world use cases. As a result, such single-disciplinary approaches to sign language processing have limited real-world value [39].
%To overcome these problems, we argue for an interdisciplinary approach to sign language processing. Deaf studies must be included in order to understand the community that the technology is built to serve. Linguistics is essential for identifying the structures of sign languages that algorithms must handle. Natural Language Processing (NLP) and machine translation (MT) provide powerful methods for modeling, analyzing, and translating. Computer vision is required for detecting signed content, and computer graphics are required for generating signed content. Finally, Human-Computer Interaction (HCI) and design are essential for creating end-to-end systems that meet the community's needs and integrate into people's lives.

% Os autores acreditam que parte disso deve-se ao fato de que muitas pesquisas em \acrshort{slr} abordam o problema de forma incorreta: eles encaram o reconhecimento da língua de sinais como uma tarefa de reconhecimento de gestos, concentrando-se na identificação de características e métodos para rotular um sinal dentre um conjunto de possibilidades bem definidas -- porém, essas línguas vão muito além disso.


\citeauthor{bragg-2019-slr-interdisciplinary} e \citeauthor{cooper-2011-slr} acreditam que, de uma forma geral, isso está relacionado a fatores como os desafios particulares a essas línguas e a forma com que as pesquisas desenvolvidas na área vêm abordando o problema no decorrer das últimas décadas.
Pode-se sumarizar esses fatores nos seguintes itens principais:

\begin{enumerate}
    % TODO: revisar esse contraponto ao reconhecimento da lingua de sinais
    \item Abordagem inadequada do problema: pesquisas atuais em processamento de língua de sinais ocorrem em silos disciplinares e, como resultado, não abordam o problema de uma forma holística. Por exemplo, muitas delas abordam o reconhecimento, mas poucas contemplam a tradução dos sinais em contextos reais ou consideram elementos como a linguística nesse processo.
    Há ainda pesquisas que abordam o problema como uma tarefa de reconhecimento de gestos e buscam identificar métodos para rotular um sinal dentre um conjunto de possibilidades bem definidas -- no entanto, essas línguas vão mais além do que uma coleção de gestos bem definidos.

    \item Natureza multicanal: línguas de sinais transmitem significado através de múltiplos canais ao mesmo tempo -- como por exemplo, a interação entre as mãos, a interação delas para com o corpo do interlocutor, suas orientações e configurações enquanto se movem, as expressões da face, olhos, cabeça e ombros, a intensidade com eles são articulados, entre outros que são abordados na linguística e que ocorrem simultaneamente.
    Isso significa que observar um canal isolado -- como observa-se em muitas pesquisas que contemplam apenas mãos estáticas ou fora do contexto -- é insuficiente para abordar corretamente a língua.
    
    \item Linguística particular: a linguística das línguas sinalizadas nos evidencia que elas possuem diversas particularidades atreladas à sua natureza visual, e isso torna muitas das técnicas hoje utilizadas no processamento das línguas faladas inadequadas a esse contexto.
    Sendo assim, ao invés de reproduzir técnicas empregadas às línguas faladas, é necessário que pesquisas em processamento de línguas sinalizadas desenvolvam técnicas próprias que explorem tais particularidades linguísticas e, com isso, que sejam capazes de contribuir avanços relevantes à área.
    %\item Linguística particular: já é evidente através dos estudos da linguística das línguas sinalizadas que suas particularidades enquanto línguas visuais tornam muitas das técnicas utilizadas para reconhecimento da fala inadequadas para o \acrshort{slr}. Dessa forma, não é correto simplesmente reproduzir abordagens de processamento de fala; mas é preciso compreender e desenvolver técnicas que levem em consideração tais particularidades.
    
    \item \textit{Datasets} limitados: os \textit{datasets} de língua de sinais disponíveis publicamente são limitados em quantidade e qualidade, e isso impõe um desafio relevante para as pesquisas na área. Técnicas modernas de aprendizagem de máquina demandam uma quantidade crescente de dados que representem o contexto de aplicação no mundo real. A escassez desses dados faz que os algoritmos desenvolvidos não sejam capazes de aprender corretamente sobre o domínio do problema e, consequentemente, não sejam capazes de generalizar soluções úteis para o mundo real.
    
\end{enumerate}


Dessa forma, tendo em face os desafios apresentados até aqui, a importância das línguas de sinais para o desenvolvimento social e cognitivo do deficiente auditivo, e o papel fundamental da ciência para produzir avanços que auxiliem esses indivíduos a superar muitas de suas barreiras cotidianas, esse trabalho traz uma nova abordagem para o processamento de línguas de sinais baseada na sua linguística, bem como dois novos \textit{datasets} que visam contribuir com diversas outras pesquisas na área.

É certo que os artefatos aqui produzidos podem não representar o estado da arte para a área de pesquisa em questão. No entanto, o principal legado consiste em estabelecer pontes para que pesquisas atuais possam se beneficiar desses artefatos e convergir na direção de técnicas centradas na linguística das línguas de sinais -- as quais são capazes de abordar as reais complexidades e produzir avanços significativos.




% Many approaches to SLR incorrectly treat the problem as Gesture Recognition (GR). So research has thus far focused on identifying optimal features and classification methods to correctly label a given sign from a set of possible signs. However, sign language is far more than just a collection of well specified gestures.
% Sign languages pose the challenge that they are multi-channel; conveying meaning through many modes at once. While the studies of sign language linguistics are still in their early stages, it is already apparent that this makes many of the techniques used by speech recognition unsuitable for SLR. In addition, publicly available data sets are limited both in quantity and quality, rendering many traditional computer vision learning algorithms inadequate for the task of building classifiers.
% However, even given the lack of translation tools, most public services are not translated into sign. There is no commonly-used, written form of sign language, so all written communication is in the local spoken language



% The use of computational approaches to help the learning, communication, translation, and interaction using sign language is essential to include this population in society. In the machine learning research area, Sign Language Recognition (SLR) refers to a collaborative field that involves pattern matching, computer vision, natural language processing, and linguistics [2] to enable sign language users to communicate with spoken languages users. However, while speech recognition systems have advanced to the point of being commercially available, SLR systems are still in very early stages. Therefore, commercial translation services for sign languages are human-based, require specialized personnel, and are often expensive [3].
% Among the main challenges involved, there is the scarcity of approaches structured around the linguistics of the sign language, the limitation in quantity and quality of public datasets that could lead to significant advancements of the SLR techniques, and the multi-channel nature of the language – which convey meaning through many modes at once [3].





% \cite{cooper-2011-slr}:
% While automatic speech recognition has now advanced to the point of being commercially available, automatic SLR is still in its infancy. Currently all commercial translation services are human based, and therefore expensive, due to the experienced personnel required.
% SLR aims to develop algorithms and methods to correctly identify a sequence of produced signs and to understand their meaning. Many approaches to SLR incorrectly treat the problem as Gesture Recognition (GR). So research has thus far focused on identifying optimal features and classification methods to correctly label a given sign from a set of possible signs. However, sign language is far more than just a collection of well specified gestures.
% Sign languages pose the challenge that they are multi-channel; conveying meaning through many modes at once. While the studies of sign language linguistics are still in their early stages, it is already apparent that this makes many of the techniques used by speech recognition unsuitable for SLR. In addition, publicly available data sets are limited both in quantity and quality, rendering many traditional computer vision learning algorithms inadequate for the task of building classifiers.
% However, even given the lack of translation tools, most public services are not translated into sign. There is no commonly-used, written form of sign language, so all written communication is in the local spoken language.


% \cite{wadhawan-2021-slr-systems-review}:
% Sign language recognition is a collaborative research area which involves pattern matching, computer vision, natural language processing, and linguistics. Its objective is to build various methods and algorithms in order to identify already produced signs and to perceive their meaning. Sign language recognition systems are Human Computer Interaction (HCI) based systems that are designed to enable effective and engaging interaction. These system follows a multidisciplinary approach of data acquisition, SL technology, SL testing and SL linguistics. Such a system can be deployed in public services like hotels, railways, resorts, banks, offices etc. to enable hearing impaired people learn new concepts and facts and to control emotional behavior [1].







% =======
% Baixo alcance da língua de sinais leva surdos ao isolamento
% https://www12.senado.leg.br/noticias/especiais/especial-cidadania/baixo-alcance-da-lingua-de-sinais-leva-surdos-ao-isolamento
% Fonte: Agência Senado
% -------
% Em 2002, a Lei 10.436 deu à Libras o status de meio legal de comunicação e expressão. Desde então, escolas, faculdades, repartições do governo e empresas concessionárias de serviços públicos estão obrigadas a providenciar intérpretes para atender aos surdos. A lei faz aniversário em 24 de abril, que, por isso, transformou-se no Dia Nacional da Língua Brasileira de Sinais.
% ------
% Apesar da lei de 2002, os surdos ainda estão longe da plena inclusão na sociedade. Como o curta Libras É Merda? denuncia (às avessas), o grande obstáculo é a escassez de ouvintes que se comuniquem na língua de sinais. A Libras está restrita à comunidade surda. Isso pode transformar atividades corriqueiras num inferno. No ônibus, os surdos não conseguem saber do cobrador qual é a parada em que devem descer. Se o alto-falante do aeroporto anuncia troca de portão, eles correm o risco de perder o avião caso não estejam com os olhos grudados nos telões de voos.
% No cinema, não podem ver filmes nacionais, pois só os estrangeiros são legendados. Na loja, o vendedor menos paciente e esclarecido pode confundir os gestos da língua de sinais com brincadeira ou deficiência mental e simplesmente virar as costas para os clientes surdos.
% Nos serviços de saúde, os surdos perdem a vez quando não estão atentos à enfermeira que grita o nome do próximo paciente. Essa dificuldade, aliás, levou à criação de um projeto-piloto no Sistema Único de Saúde (SUS): o Hospital Federal de Ipanema, no Rio de Janeiro, abriu uma central de atendimento e consulta mediada por intérprete de Libras.
% -----
% As situações podem inclusive ser trágicas. ONGs dedicadas aos surdos dizem que não são raros os casos em que pacientes com problemas sérios de saúde saem de consultas com uma prescrição errada de remédio, porque o médico não entendeu quais eram os sintomas, e situações em que inocentes são mortos porque não ouviram a ordem de parar e o policial atirou por não perceber que eram surdos.
% "Deficiente não é o surdo, mas a sociedade que não sabe se comunicar com ele. Se o surdo encontrasse no dia a dia pessoas que soubessem a língua de sinais, ele não enfrentaria tantas barreiras e, por isso, nem perceberia a surdez como deficiência" afirma a coordenadora do Laboratório de Educação de Surdos e Libras, da Universidade de Brasília (UnB), Edeilce Buzar.
% Segundo o Censo mais recente, viviam em 2010 no Brasil 2,1 milhões de pessoas que escutavam muito pouco ou nada — o equivalente à população de Manaus. A pesquisa do IBGE não apontou quantas faziam uso da língua de sinais.

% CENSO - IBGE 2018:
% - 5% (9,6 milhões) da população brasileira têm deficiência auditiva
% - 2,1 milhões são surdos ou escutam muito pouco
% - 7,5 milhões apresentam alguma dificuldade para ouvir
% - 360 milhões de pessoas sofrem algum tipo de surdez no mundo
% - 32 milhões são crianças
% - 31 milhões vivem em países em desenvolvimento

% As primeiras barreiras linguísticas por vezes são impostas pela própria família. Quando a criança nasce surda ou perde a audição ainda pequena, muitos pais rejeitam a língua de sinais e impõem a oralização. Sem ouvir a própria voz, o treinamento da fala e da leitura labial costuma ser lento e penoso. O aprendizado da língua de sinais, ao contrário, é natural para quem, compensando a lacuna da audição, tem na visão o sentido mais apurado.
% -----
% Raras escolas estão adaptadas para receber alunos surdos. A mera presença de um intérprete da língua de sinais ao lado do professor não é suficiente. Por um lado, muitas crianças surdas chegam ao colégio sem saber língua nenhuma e vão ter que aprender a Libras do zero. Por outro, as que já sabem a língua de sinais não encontram professores preparados para ensinar-lhes o português escrito. Nessa situação, como a Libras é a primeira língua do estudante, o português precisa ser apresentado como segundo idioma, com uma metodologia completamente diferente, tal como uma língua estrangeira. O professor precisa ser bilíngue e ter uma formação específica.
% Em consequência do despreparo das escolas, muitos surdos chegam ao fim dos estudos como analfabetos funcionais. É por isso, aliás, que tentar se comunicar por escrito com um surdo nem sempre dá certo.

% — Os surdos acabam sendo forçados a viver encapsulados em seus próprios mundos. São como almas que passam por nós sem que nos preocupemos em enxergá-los ou interagir com eles — compara o intérprete de Libras e ex-presidente da Associação de Pais e Amigos dos Deficientes Auditivos do Distrito Federal (Apada-DF) Marcos de Brito.
% -----
% Desde 1991, a Lei 8.213 obriga as empresas a reservar uma parte de suas vagas para funcionários com algum tipo de deficiência. Para firmas que tenham entre 100 e 200 trabalhadores, por exemplo, a cota é de 2%. A inclusão efetiva nem sempre ocorre. Tarcísio Barroso, de 31 anos, ficou surdo ainda bebê, também por causa da meningite. Ele é oralizado, mas tem a Libras como primeira língua. Mesmo pós-graduado na área da tecnologia da informação, acabou sendo relegado a tarefas secundárias em muitas das empresas onde trabalhou, em Brasília.
% — A comunicação com meus chefes sempre foi falha. Alguns não se preocupavam em articular bem as palavras na hora de falar, para eu fazer a leitura labial. Outros preferiam se comunicar por escrito, mas usavam palavras difíceis ou frases pouco objetivas, o que dificultava a minha compreensão. De tanto eu pedir que explicassem novamente cada orientação, acabavam concluindo que eu tinha deficiência intelectual e passavam a me deixar de lado. Já chorei muito por causa disso.
% A mãe dele, Vanilda Barroso, lembra que a inadequação que ele agora sente no trabalho é muito parecida com a que sentia no colégio, quando os professores não exploravam as suas potencialidades e só apontavam as dificuldades.
% — As empresas contratam um surdo e nem se preocupam em saber como os surdos são. Só querem cumprir a cota exigida pela lei. Não estão interessadas na inclusão — ela afirma. — E a inclusão não é difícil. Bastaria que as empresas deixassem de encarar o surdo como um incapaz e passassem a vê-lo como um estrangeiro que não compreende plenamente o português e precisa de explicações numa linguagem mais clara.



% ==============
% IBGE confirma: surdez não é sinônimo de Libras
% https://desculpenaoouvi.com.br/ibge-confirma-surdez-nao-e-sinonimo-de-libras/?fbclid=IwAR1Z5mcQeTlaaVCuEtDXDQoLW9cwT17yBplyQUgFF7NXxM7qT5saMTjFaWk
% 
% IBGE - Pesquisa Nacional de Saúde (PNS) - 2019:
% - 8,4% (17,3 milhões) da população com 2+ anos com algum grau de deficiência
%
% Com deficiência auditiva:
% - 1,3% da população em idade de trabalhar (14+ anos)
% - 2,6% da população fora da força de trabalho
%
% Nível de ocupação: 
% - 25,4% das pessoas de 14+ anos
% 
% Nível de instrução (pessoas com 18+ anos): 
% - 67,6% sem instrução e fundamental incompleto
% - 10,8% fundamental completo e médio incompleto
% - 16,6% médio completo e superior incompleto
% - 5% superior completo
%
% Uso de tecnologias auditivas:
% - 0,8% da população 2+ anos
% - 3,1% da população 60+ anos
%
% Uso da Libras (população 5+ anos) por grau de dificuldade para ouvir
% - Alguma dificuldade: 1,8% sabe usar
% - Muita dificuldade: 3,0% sabe usar
% - Não consegue de modo algum: 35,8% sabe usar
% --> "com deficiência auditiva" (grande dificuldade + não consegue de modo algum): 22,4%

% “Além disso, a informação [PNS] é importante para não generalizar as pessoas com deficiência auditiva: ao analisar os dados, é possível perceber que nem todos sabem se comunicar em Libras, pois fazem uso da fala, se comunicam oralmente, muitos deles com o apoio da leitura labial. Portanto, precisam de outros recursos de acessibilidade, como por exemplo as legendas, além de necessitarem de propostas educacionais distintas”, diz a pesquisadora.
% Comprovadamente, acessibilidade para surdos não se resume à interpretação em Libras e, como mostrado pelo IBGE, não atende a necessidade da maioria.

% -----
% Esses dados são essenciais para que se desmistifique que surdez é sinônimo de ser usuário exclusivo da Libras. A Libras tem função fundamental na comunicação de uma parcela de pessoas Surdas, assim como a Língua Portuguesa também tem função fundamental para a grande maioria das pessoas com deficiência auditiva.
% -----
% Não podemos continuar com políticas como as que vêm sendo implementadas nos últimos 10 anos, onde a iniciativa pública e privada decidem erroneamente que basta incluir Libras e todos os surdos serão contemplados com ela.
% -----
% Graças à Pesquisa Nacional de Saúde de 2019 realizada pelo IBGE, podemos afirmar categoricamente que A SURDEZ É PLURAL e as formas de acessibilidade também devem ser.



% =====
% Painel de Indicadores de Saúde – Pesquisa Nacional de Saúde
% https://www.pns.icict.fiocruz.br/painel-de-indicadores-mobile-desktop/


% =====
% País tem 10,7 milhões de pessoas com deficiência auditiva, diz estudo
% https://agenciabrasil.ebc.com.br/geral/noticia/2019-10/brasil-tem-107-milhoes-de-deficientes-auditivos-diz-estudo
% ------

% Instituto Locomotiva + Semana da Acessibilidade Surda:
% - 10,7 milhões de pessoas com deficiência auditiva (Brasil)
%    - Dessas, 2,3 milhões têm deficiência severa
%       - 15% já nasceram surdos
%
% - A surdez atinge 54% de homens e 46% de mulheres
% - A predominância (57%) é na faixa de 60+ anos
% 
% Condição nascida x adquirida:
% - 9% das pessoas nasceram 
% - 91% adquiriram ao longo da vida (metade foi antes dos 50 anos)
% 
% Do total pesquisado:
% - 87% não usam aparelhos auditivos (13% utilizam)

% ------
% “A deficiência auditiva é uma deficiência que se agrava com o passar dos anos. E como o Brasil está passando por um processo de envelhecimento da população, hoje já temos 59 milhões de brasileiros com mais de 50 anos e, em 2050, vamos chegar com mais de 98 milhões de brasileiros com mais de 50 anos de idade, essa é uma tendência que só vai crescer”, disse Renato Meirelles, presidente do Instituto Locomotiva. Completou que a “sociedade, claramente, não está preparada para isso”.

% ------
% Dois em cada três brasileiros 
% - 66,6% relataram enfrentar dificuldades nas atividades do cotidiano
% --> “Com isso, eles se divertem menos, têm menos chance no mercado de trabalho, não têm as mesmas oportunidades educacionais que os ouvintes têm”. 
% -----
% A falta de acolhimento e inclusão limitam o acesso dos surdos às oportunidades básicas, como educação
% - somente 7% têm ensino superior completo
% - 15% frequentaram até o ensino médio
% - 46% até o fundamental
% - 32% não possuem grau de instrução
% -------
% - 20% das pessoas com deficiência auditiva idosos não conseguem sair sozinhas
% - 37% estão no mercado de trabalho
% - 87% não usam aparelhos auditivos
% --> “Porque é muito caro e inacessível para a maioria dessa população”, disse Meirelles. “E como a população surda teve menos oportunidade de estudar do que a população ouvinte, como tem mais dificuldade no mercado de trabalho do que a população ouvinte, o dinheiro para conseguir o aparelho é ainda mais difícil. Esse conjunto de preconceitos que existe na sociedade acaba criando um círculo vicioso que não possibilita que os surdos e os ouvintes tenham as mesmas oportunidades de se dar bem na vida.”
% --> “Quando comecei no meu trabalho, as pessoas pensavam que eu não era capaz de fazer as coisas. Demorou demais para que elas acreditassem que eu tinha capacidades, mas às vezes ainda me olham com discriminação e desconfiança por eu ser quem sou”, afirmou uma mulher com deficiência auditiva de 30 anos, entrevistada em São Paulo.

% ------
% Entre os tipos de ocupação desempenhada pelas pessoas com deficiência auditiva com 18+ anos:
% - 43% empregado no setor privado 
% - 37% trabalhador por conta própria
%
% Segundo Renato Meirelles, “essas pessoas desistiram de arrumar emprego e passaram a empreender para garantir o seu sustento”.
% A pesquisa foi realizada entre os dias 1º e 5 de setembro passado, com 1,5 mil brasileiros surdos e ouvintes. No total, o Brasil possui 50,3 milhões de pessoas com deficiência. 

% -----
% A pesquisa mostra que a maior parcela de pessoas com deficiência auditiva está:
% - 42%: Região Sudeste
% - 26%: Nordeste 
% - 19%: Sul
% - 6%: Centro-Oeste 
% - 7%: Norte

% Das pessoas com deficiência auditiva:
% - 28% declararam ter também algum tipo de deficiência visual
% - 2% deficiência intelectual.

% Dos brasileiros com problemas auditivos:
% 14% não se sentem à vontade e poder falar sobre quase tudo com a família (11% da população de forma geral)
% 40% sentem isso em relação a amigos (34% da população de forma geral)
% A sondagem revela, ainda, que pessoas com deficiência auditiva severa têm três vezes mais chance de sofrerem discriminação em serviços de saúde do que pessoas ouvintes.

% -----
% Segundo a Organização Mundial da Saúde (OMS) existem 500 milhões de surdos no mundo e, até 2050, haverá pelo menos 1 bilhão em todo o globo. 





% =====
% Livro lingua de sinais
% https://books.google.com.br/books?hl=pt-BR&lr=&id=JVVYWP6btj4C&oi=fnd&pg=PA9&dq=lingua+de+sinais&ots=paEl9LqTn7&sig=-xCW5BpO6r8eNs7R8OtSymDrdjU#v=onepage&q=lingua%20de%20sinais&f=false


% https://www.scielo.br/j/es/a/LScdWL65Vmp8xsdkJ9rNyNk/?format=pdf&lang=pt
