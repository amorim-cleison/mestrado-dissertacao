Segundo a \citeonline{who-2021-report-hearing}, o mundo possui hoje cerca de 1,5 bilhões de pessoas com algum grau de perda auditiva, o que corresponde a 19\% da população mundial.
Desse número, 450 milhões se referem a perda de grau moderado a total\footnote{
    O grau de perda refere-se à intensidade mínima de som que um ouvido pode detectar.
    Na perda leve, essa intensidade está entre 20 e 34 dB; na moderada, entre 35 e 49 dB; na moderadamente severa, entre 50 e 64 dB; na severa, entre 65 e 79 dB; na profunda, entre 80 e 94 dB; e na total (ou surdez), 95 dB ou mais.
    \cite[p. 38]{who-2021-report-hearing}
}, as quais necessitarão de acesso a cuidados auditivos e outros serviços de reabilitação.
No Brasil, o número dos que têm perda auditiva é de 10,7 milhões e o dos que apresentam perda moderada a total é de 2,3 milhões, de acordo com \citeonline{ebc-2019-10-milhoes-pessoas, ibge-2021-pns, ibge-2021-projecao-populacao}.


Ao analisar os dispositivos e intervenções que são capazes de auxiliar no diagnóstico e reabilitação desses indivíduos, a \citeonline{who-2021-report-hearing} afirma que as últimas décadas testemunharam avanços revolucionários, como nos campos da tecnologia auditiva, diagnóstico e telemedicina, com inovações que possibilitam problemas relacionadas à audição serem identificadas em qualquer idade e ambiente.
Contudo, uma vez que a grande maioria daqueles com perda auditiva vive em locais de baixa renda, onde profissionais especializados e serviços para cuidados auditivos não estão comumente disponíveis, existe uma disparidade no acesso a tais recursos:

\begin{citacao}
    Cerca de 78\% dos países de baixa renda têm menos de um otorrinolaringologista por milhão de habitantes; 93\% têm menos de um audiologista por milhão; 17\% têm um ou mais fonoaudiólogos por milhão; e 50\% têm um ou mais professores para pessoas com perda auditiva por milhão.

    Mesmo em países com proporções relativamente altas desses especialistas, há uma distribuição desigual que, além de trazer desafios para essa população, impõe uma sobrecarga excessiva aos quadros que prestam esses serviços.~\cite{opas-2021-oms-estima}
\end{citacao}

Essa carência de cuidados adequados gera deficiências estruturais que dificultam o acesso a oportunidades básicas como educação e emprego, e resultam numa pior qualidade de vida para eles.
Isso estende-se a todas as gerações, afirmam \citeonline{ebc-2021-oms-estima,ebc-2019-10-milhoes-pessoas}:

\begin{citacao}
    Essas deficiências estruturais refletem-se na educação das crianças. Uma criança que ouve mal, aprende mal e torna-se um adulto menos capaz do que poderia ser, e assim por diante.~\cite{ebc-2021-oms-estima}
\end{citacao}

\begin{citacao}
    Uma vez que esses indivíduos têm menos oportunidades de estudar e acessar o mercado de trabalho do que a população ouvinte, o dinheiro para conseguir o aparelho auditivo é ainda mais difícil. Esse conjunto de preconceitos acaba criando um círculo vicioso que não possibilita que eles tenham as mesmas oportunidades de se dar bem na vida.~\cite{ebc-2019-10-milhoes-pessoas}
\end{citacao}


Em meio a tantos desafios, a língua de sinais surge como uma ferramenta poderosa que é capaz de assegurar o desenvolvimento cognitivo, facilitar a comunicação, e possibilitar que esses indivíduos obtenham educação e desenvolvimento socio-emocional adequado \cite{who-2021-report-hearing}.
Ela pode ser aprendida através de membros da própria família, da comunidade Surda, de conteúdos geralmente gratuitos na internet (como livros, cursos e vídeos) disponibilizados por instituições como o \acrshort{ines}\footnote{
    O \acrfull{ines}, fundado em 1857, é o centro de referência nacional que subsidia a formulação de políticas públicas para o Surdo. Ele atende a estudantes da educação infantil até o ensino superior e também apoia a pesquisa de novas metodologias de ensino nesse contexto. \cite{mec-2021-conheca-ines}
} ou em escolas públicas que ofertam seu ensino.
Por conta disso, ela torna-se uma alternativa acessível para a inclusão desses indivíduos, uma vez que muitas das barreiras como a demanda por recursos financeiros ou profissionais especializados são removidas.


\citeonline{stewart-2021-barrons-asl} afirmam que a língua de sinais é a chave para acessar a cultura Surda.
O termo Surdo (escrito com ``s'' capitalizado), por sua vez, não refere-se apenas a uma condição clínica, mas a um grupo de indivíduos que, além de possuírem perda auditiva, utilizam a língua de sinais como principal meio de comunicação e compartilham experiências culturais associadas à surdez e ao uso dessa língua.
Há um aspecto cultural fundamental, reiteram \citeonline{pereira-2011-conhecimento-alem-sinais}, acompanhado de um forte sentimento de identidade grupal que faz com que esses indivíduos compartilhem valores, crenças, comportamentos e uma língua própria.


Apesar disso, \citeonline{bragg-2019-slr-interdisciplinary,senado-2019-baixo-alcance-lingua-sinais} observam que atualmente ainda são poucos os ouvintes que conseguem se comunicar por meio dessa língua.
Isso traz obstáculos adicionais aos Surdos e transforma muitas de suas atividades corriqueiras num grande desafio.
Por exemplo, no transporte público é difícil solicitar ajuda ou ter acesso às instruções divulgadas nos alto-falantes;
em lojas, é raro encontrar vendedores preparados para interagir através dessa língua ou que não os trate com preconceito;
no cinema, eles apenas podem consumir filmes estrangeiros, uma vez que os nacionais não dispõem de legenda;
no serviço de saúde, não são raros os relatos de pacientes que saem de consultas com prescrições médicas erradas porque o médico não entendeu corretamente seus sintomas; entre outras situações.


Para contribuir com a superação desses desafios é importante, entre outros fatores, que a comunidade acadêmica esteja mobilizada para impulsionar o desenvolvimento de alternativas e tecnologias.
O \acrfull{slr} é um dos campos de pesquisa que se dedica a desenvolver algumas delas. Segundo \citeonline{wadhawan-2019-slr-literature-review}, trata-se de uma área colaborativa e multidisciplinar que envolve \acrlong{cv} \cite{szeliski-2022-computer-vision}, \acrlong{nlp} \cite{jurafsky-2022-speech-lang-processing}, Reconhecimento de Padrões \cite{bishop-2006-pattern-recognition} e Linguística \cite{quadros-2004-estudos-linguisticos} para construir métodos e algoritmos capazes de identificar sinais produzidos pelo articulador e compreender seu significado.
Por meio deles, seria possível reduzir a barreira linguística entre Surdos e ouvintes permitindo que mensagens transmitidas utilizando-se a língua de sinais fossem transcritas automaticamente e compreendidas por aqueles que não a conhecem.


No entanto, apesar do potencial que o \acrshort{slr} possui, \citeonline{selvaraj-2022-openhands,yin-2021-sl-in-nlp,cooper-2011-slr} acreditam que o progresso apresentado por essa área ao longo das últimas décadas foi insuficiente para conduzir a avanços expressivos:

\begin{citacao}
    Quando comparado com a pesquisa de \acrlong{nlp} baseada em texto e fala, o progresso das pesquisas para línguas de sinais está significativamente atrasado. \cite[tradução nossa]{selvaraj-2022-openhands,yin-2021-sl-in-nlp}
\end{citacao}


\begin{citacao}
    Enquanto sistemas de reconhecimento da fala avançaram ao ponto de estarem comercialmente disponíveis, o reconhecimento de sinais ainda está em sua infância.
    Atualmente, todos os serviços comerciais de tradução de sinais são baseados em humanos e requerem que pessoal especializado esteja disponível, o que os tornam caros e pouco acessíveis. \cite[tradução nossa]{cooper-2011-slr}
\end{citacao}



Isso deve-se, de um modo geral, a um conjunto de particularidades que as línguas de sinais apresentam quando comparadas às línguas faladas, bem como à forma com que as pesquisas em \acrshort{slr} têm abordado elas, afirmam \citeonline{bragg-2019-slr-interdisciplinary,cooper-2011-slr}.
Diferentemente das faladas, as línguas sinalizadas possuem uma natureza visual e transmitem significado através de múltiplos canais ao mesmo tempo, como mãos, corpo, face, entre outros de granularidade ainda menor.
Essa natureza faz com que sua linguística seja estruturada de uma forma muito específica, demandando que novas técnicas sejam desenvolvidas para abordar tais particularidades.


Contudo, segundo \citeonline{cooper-2011-slr,yin-2021-sl-in-nlp}, um grande número de pesquisas nessa área trata o \acrshort{slr} como uma tarefa de reconhecimento de gestos não-estruturados ou poses de mãos estáticas, que são mapeados a partir de imagens RGB, dados de luvas eletrônicas ou coordenadas dos corpos dos indivíduos.
Isso faz com que elas deixem de abordar aspectos essenciais da língua de sinais -- como sua linguística e suas particularidades -- e desviem o foco para um conjunto de desafios pertinentes à área de \acrfull{cv}, como a detecção, segmentação e rastreamento de partes do corpo; a interação entre mãos e oclusões decorrentes disso; variações de tom de pele; entre outros que comumente já são abordados ou solucionados por outras subáreas da \acrshort{cv}.
Como consequência, essas pesquisas acabam não produzindo avanços realmente efetivos para a \acrshort{slr}.


Tendo em vista isso, este trabalho busca contribuir com a área de \acrfull{slr} por meio de uma proposta que aborda a língua de sinais através de sua linguística, pela introdução de um novo \textit{dataset} de atributos linguísticos e pela adoção de técnicas de \acrfull{nlp} nesse contexto afim de estabelecer uma direção que possa conduzir a avanços efetivos na área e, consequentemente, ajude a superar alguns dos desafios cotidianos atualmente encarados pelos Surdos.

