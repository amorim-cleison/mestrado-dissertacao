\section{Novos \textit{datasets} da língua de sinais}
\label{sec:metodologia-datasets}

Conforme introduzimos acima, o primeiro passo para que possamos desenvolver e avaliar uma abordagem de aprendizagem de máquina centrada na linguística da língua de sinais consiste em definir um conjunto de dados que viabilize isso. Como a proposta apresentada aqui é nova e, devido a isso, não há \textit{dataset}s diretamente compatíveis com ela, optamos por derivar um novo \textit{dataset} a partir de outro já existente -- o \acrfull{asllvd}.

O \acrshort{asllvd} consiste em um amplo \textit{dataset} público\footnote{Disponível em \url{http://www.bu.edu/asllrp/av/dai-asllvd.html}} da \acrshort{asl} que contém aproximadamente 2.745 sinais representados em cerca de 9.763 sequências de vídeo articuladas por indivíduos Surdos nativos. Suas amostras são capturadas por meio de quatro câmeras sincronizadas: uma visão frontal de alta resolução a meia velocidade, outra visão frontal de resolução total, uma visão lateral e uma visão da face, conforme ilustrado na \autoref{fig:asllvd-example}~\cite{athitsos-2008-asllvd,neidle-2012-asllvd}.

\begin{figure}[ht!]
    \centering
    \caption{\textmd{Exemplo de três perspectivas sincronizadas providas pelo \acrshort{asllvd} para o sinal MERRY-GO-ROUND: 
    vista frontal (\subref{subfig:asllvd-example-front}), 
    vista lateral (\subref{subfig:asllvd-example-side}) e 
    vista da face (\subref{subfig:asllvd-example-close}).}}
    \subcaptionbox{\label{subfig:asllvd-example-front}}{
        \includegraphics[width=0.25\textwidth]{capitulos/metodologia/imagens/asllvd_example_front}
    }%
    \hfill
    \subcaptionbox{\label{subfig:asllvd-example-side}}{
        \includegraphics[width=0.25\textwidth]{capitulos/metodologia/imagens/asllvd_example_side}
    }%
    \hfill
    \subcaptionbox{\label{subfig:asllvd-example-close}}{
        \includegraphics[width=0.25\textwidth]{capitulos/metodologia/imagens/asllvd_example_close}
    }%
    \nomefonte[p. 2]{athitsos-2008-asllvd}
    \label{fig:asllvd-example}
\end{figure}


Para computar parâmetros fonológicos a partir dos frames das amostras do \acrshort{asllvd}, compostos essencialmente de imagens RGB bidimensionais, precisamos realizar um processo de duas etapas: primeiro, estimamos as coordenadas 2D dos esqueletos dos sinalizadores para duas câmeras distintas, frame-a-frame, e as combinamos para projetar um esqueleto no espaço 3D -- isso deu origem ao \textit{dataset} intermediário chamado ASL-Skeleton3D; em seguida, aplicamos um conjunto de operações algébricas sob o esqueleto 3D para calcular os parâmetros fonológicos -- o que gerou assim o nosso \textit{dataset} final chamado ASL-Phono.

Como pode-se imaginar, esse processo envolveu alguns desafios computacionais, entre os quais enumeram-se :

\begin{enumerate}
    \item Definição de abordagens para representar indivíduos no espaço 3D utilizando apenas frames de vídeo simples em 2D do \acrshort{asllvd}, bem como para lidar com as amostras ausentes ou de baixa qualidade encontradas nesse \textit{dataset}.

    \item Estabelecimento de um subconjunto de atributos fonológicos iniciais que pudessem capturar e representar variações significativas no corpo dos indivíduos para o reconhecimento dos sinais, mas que também pudessem ser modelados computacionalmente.
    
    \item Identificação de técnicas matemáticas e medidas antropométricas que suportassem o cálculo e a modelagem dos atributos selecionados.
    
    \item Disponibilização de recursos computacionais significativos para viabilizar o processamento de mais de 9.000 amostras contidas em cada \textit{dataset}, as quais envolveram duas câmeras distintas. Para isso, foram consumidas cerca 40 horas contínuas de processamento em cluster dispondo de CPU e GPU, e gerados mais de 1 TB de dados cada vez que ambos \textit{dataset}s precisaram ser completamente processados.
\end{enumerate}


% Dataset 3d
\subsection{ASL-Skeleton3D}
\label{sec:metodologia-datasets-3d}

O ASL-Skeleton3D é um \textit{dataset} intermediário oriundo deste trabalho que introduz a representação em coordenadas 3D das amostras do \acrshort{asllvd}. Esse tipo de representação possibilita uma observação mais precisa do corpo dos indivíduos enquanto articulam os sinais, permitindo a pesquisadores em \acrshort{slr} explorar novas técnicas ou derivar outros \textit{dataset}s computados a partir dessas coordenadas.

A estratégia adotada para projetar os sinais do \acrshort{asllvd} dentro do espaço 3D consistiu em combinar duas de suas perspectivas 2D capturadas por câmeras posicionadas perpendicularmente entre si: a vista frontal e a vista lateral. Com isso, a vista frontal fornecerá a observação dos eixos \(x\) e \(y\), e a vista lateral nos fornecerá a dimensão de profundidade (ou eixo \( z\)), conforme ilustra a \autoref{fig:our-strategy-3d}.


% TODO: criar componente para sub-figuras (lembrar de formatar bordas e espaçamento horizontal, conforme padrão ABNT)
\begin{figure}[ht!]
    \centering
    \caption{\textmd{Estratégia adotada para representar os sinais no espaço 3D: as visões 2D para a frente~(\subref{subfig:our-strategy-3d-front}) e lateral~(\subref{subfig:our-strategy-3d-side}) são posicionadas perpendicularmente para reconstruir a visão 3D do objeto~(\subref{subfig:our-strategy-3d-persp}).}}
    \subcaptionbox{\label{subfig:our-strategy-3d-front}}{
        \includegraphics[height=4cm]{capitulos/metodologia/imagens/chair_front}
    }%
    \hfill
    \subcaptionbox{\label{subfig:our-strategy-3d-side}}{
        \includegraphics[height=4cm]{capitulos/metodologia/imagens/chair_side}
    }%
    \hfill
    \subcaptionbox{\label{subfig:our-strategy-3d-persp}}{
        \includegraphics[height=4cm]{capitulos/metodologia/imagens/chair_perspective}
    }%
    \nomefonte{}
    \label{fig:our-strategy-3d}
\end{figure}


Os passos utilizados para gerar o ASL-Skeleton3D basearam-se no processamento realizado por \citeonline{amorim-2019-stgcn-sl} na construção de um \textit{dataset} de esqueletos 2D da \acrshort{asl}. De um modo geral, eles envolvem a obtenção das amostras, a segmentação dos sinais, a estimativa e a normalização dos esqueletos. Apesar disso, várias adaptações foram necessárias a cada passo para acomodar a estratégia descrita acima e lidar com os desafios encontrados aqui, conforme discutiremos a seguir:

\begin{enumerate}
    \item \textbf{Obtenção das amostras}: essa etapa consiste em recuperar as amostras de vídeos do \acrshort{asllvd}, porém considerando duas câmeras -- frontal e lateral -- para aplicar a estratégia descrita acima. Existem dois formatos originalmente disponíveis para as câmeras: \textit{mov} (vídeos compactados, geralmente menores e mais fáceis de processar) e \textit{vid} (vídeos brutos, mais longos para baixar e processar).

          Ao analisar as amostras, identificamos que parte delas possuía as duas câmeras disponíveis em ambos os formatos; para outras, havia ambas câmeras, mas cada uma estava em um formato distinto; nos piores casos, contudo, faltava uma das câmeras ou ela estava corrompida, fazendo com que essas amostras precisassem ser descartadas. Lidar com isso acrescentou uma complexidade extra, mas que foi contornada com uma perda de apenas 0,16\% das amostras originais -- resultando num total de 9.747 amostras.

          % -----
    \item \textbf{Segmentação dos sinais}: consiste em segmentar as sequências de vídeos do \acrshort{asllvd}, que são extensas e contemplam múltiplos sinais, em pedaços menores com apenas um sinal. Nesse processo, reduzimos a taxa de quadros de 60 para 3 FPS uma vez que, dadas as condições em que os indivíduos foram filmados, assumimos que eles não conseguiriam realizar mais de três movimentos em um único segundo -- consequentemente, muitos dos frames originais conteriam informação duplicada. Isso também nos ajudou a reduzir em cerca de 20 vezes o número de frames a serem processados nas etapas posteriores.

          % -----
    \item \textbf{Estimativa dos esqueletos 3D}: aqui realizamos a estimativa dos esqueletos dos sinalizadores utilizando o OpenPose~\cite{cao-2017-openpose,simon-2017-openpose-hand-face}. Isso foi feito para ambas as câmeras frontal e lateral, o que nos rendeu dois esqueletos 2D:

          \begin{enumerate}
              \item \textit{Esqueleto frontal}, contendo as coordenadas plotadas sobre o eixos \(x\) e \(y\) (vide \autoref{subfig:front-side-persp-skeletons-front}).
                    Essas coordenadas descrevem a mesma visão frontal que esperamos obter ao imaginar o nosso esqueleto 3D final observado de frente. Por conta disso, utilizamos essas coordenadas \(x\) e \(y\) da forma como são fornecidas para ancorar o indivíduo no espaço tridimensional. Precisaremos apenas adicionar a dimensão de profundidade.

              \item \textit{Esqueleto lateral}, que também contém um par de coordenadas \(x\) e \(y\) (vide \autoref{subfig:front-side-persp-skeletons-side}).
                    No entanto, vamos aplicar a estratégia descrita acima e posicioná-lo perpendicularmente ao esqueleto frontal (como no ~\autoref{subfig:front-side-persp-skeletons-persp}). Nesse caso, observamos que embora o eixo \(y\) fornecido aqui contenha as mesmas coordenadas que no esqueleto frontal, o eixo \(x\) descreverá coordenadas equivalentes às de profundidade. Dessa forma, tomaremos o eixo \(x\) do esqueleto lateral como sendo o eixo \(z\) (eixo de profundidade) para o nosso esqueleto 3D final.
          \end{enumerate}

          \begin{figure}[ht!]
              \centering
              \caption{\textmd{As coordenadas \(x\) e \(y\) do esqueleto 3D final são obtidas das coordenadas estimadas para a perspectiva frontal~(\subref{subfig:front-side-persp-skeletons-front}). A coordenada \(z\) é originada do eixo \(x\) do esqueleto lateral~(\subref{subfig:front-side-persp-skeletons-side}). Quando colocados juntos, eles são visualizados como em (\subref{subfig:front-side-persp-skeletons-persp}).}}
              \subcaptionbox{\label{subfig:front-side-persp-skeletons-front}}{
                  \includegraphics[height=3cm]{capitulos/metodologia/imagens/asllvd_example_front_skeleton}
              }%
              \hfill
              \subcaptionbox{\label{subfig:front-side-persp-skeletons-side}}{
                  \includegraphics[height=3cm]{capitulos/metodologia/imagens/asllvd_example_side_skeleton}
              }%
              \hfill
              \subcaptionbox{\label{subfig:front-side-persp-skeletons-persp}}{
                  \includegraphics[height=4.5cm]{capitulos/metodologia/imagens/asllvd_front_side_perspective_skeleton}
              }%
              \nomefonte{}
              \label{fig:front-side-persp-skeletons}
          \end{figure}

          % -----
    \item \textbf{Normalização dos esqueletos 3D}: no último passo, normalizamos os esqueletos 3D para remover variações decorrentes do posicionamento das câmeras e diferenças nos corpos dos indivíduos. Isso é importante porque o \acrshort{asllvd} foi capturado em diferentes seções e envolvendo diferentes sinalizadores.

          \figura
          {fig:shoulders-width} % Label
          {capitulos/metodologia/imagens/shoulders_width} % Path
          {height=3cm} % Size
          {Largura entre ombros, que foi utilizada como referência para normalizar as coordenadas nos esqueletos 3D.} % Caption
          {} % Citation

          Para isso, adotamos como referência a largura entre os ombros dos sinalizadores (vide \autoref{fig:shoulders-width}), a qual foi inspirada pela medida antropométrica \textit{diâmetro biacromial} \cite{stoudt-1970-skinfolds}. Dessa forma, a largura dos ombros \(W_{shoulders}\) é calculada através da distância euclidiana \(d\) \cite{anton-2013-algebra} entre as coordenadas do ombro esquerdo \(S_{l}\) e direito \(S_ {r}\), como na \autoref{eqn:shoulders-width}:

          \begin{equation}
              \label{eqn:shoulders-width}
              W_{shoulders} = d\left(S_{l}, S_{r}\right)
          \end{equation}

          Uma vez que \(W_{shoulders}\) foi calculado, podemos então transformar cada coordenada \(K\) em sua respectiva versão normalizada \(K_{norm}\), conforme \autoref{eqn:normalized-keypoint}:

          % Normalização de pontos-chave:
          \begin{equation}
              \label{eqn:normalized-keypoint}
              K_{norm} = \frac{K}{W_{shoulders}}
          \end{equation}

\end{enumerate}


A \autoref{fig:sample-json-datasetTD} exemplifica uma amostra do ASL-Skeleton3D, bem como as propriedades fornecidas para ela. Observam-se no início do arquivo algumas propriedades com informações básicas extraídas do \acrshort{asllvd}, como rótulo, consultor, sessão, cena, frames de início e fim, entre outras. Além disso, temos a propriedade ``frames'', que contém a lista de frames da amostras e os esqueletos estimados para cada um deles. Cada frame contém grupos que correspondem às partes do corpo e, dentro desses grupos, estão as coordenadas contendo seu respectivo nome, um \textit{score} que identifica a acurácia da estimativa daquela coordenada, e os eixos \(x\), \(y\) e \(z\).
Por exemplo, se observarmos o segundo índice de cada um dos arrays contidos no grupo \textit{body} (do corpo do sinalizador), notaremos que ele corresponde à coordenada \textit{neck} (do pescoço), a qual possui um \textit{score} de 62\% e coordenadas \(x\), \(y\) e \(z\) localizadas em 4,469, 2,871 e 1,898, aproximadamente.

\figura
    {fig:sample-json-datasetTD} % Label
    {capitulos/metodologia/imagens/code_3d} % Path
    {width=0.7\linewidth} % Size
    {Exemplo de amostra do ASL-Skeleton3D e propriedades fornecidas para ela.} % Caption
    {} % Citation

Por fim, o \textit{dataset} resultante do processamento acima, bem como o código-fonte utilizado para isso, está disponível publicamente na URL listada abaixo\footnote{Disponível em \url{http://www.cin.ufpe.br/~cca5/asl-skeleton3d}}.


% Dataset phono
\subsection{ASL-Phono: O \textit{dataset} da fonologia da língua de sinais}
\label{sec:metodologia-datasets-phono}

O ASL-Phono é um \textit{dataset} que introduz uma representação baseada na linguística da língua de sinais e a descreve em termos de seus atributos fonológicos. Ele é produzido a partir dos esqueletos fornecidos pelo \textit{dataset} ASL-Skeleton3D e, assim como este, também apresenta 9.747 amostras correspondentes a 2.650 sinais distintos.

Por se tratar de uma versão inicial da representação proposta, neste trabalho selecionaremos apenas um subconjunto dos parâmetros fonológicos introduzidos na \autoref{sec:linguistica} para que, dessa forma, possamos validar sua efetividade e traçar uma direção para iterações futuras.
Sendo assim, descreveremos a seguir esses parâmetros, bem como a estratégia utilizada para calculá-los e representá-los computacionalmente:

% Como trata-se de uma versão inicial da representação proposta, optamos por selecionar um subconjunto de quatro parâmetros fonológicos dentre aqueles discutidos na \autoref{sec:linguistica-fonologia}, os quais podem ser expandidos em incrementos futuros desta pesquisa. Esses parâmetros incluem a configuração de mão, o movimento da mão, a orientação da palma e uma expressão não-manual refente à abertura da boca, conforme discutiremos a seguir:

\begin{enumerate}
    \item \textbf{Configuração de mão}: é a configuração de mão utilizada pelo sinalizador na articulação do sinal.

          O \acrshort{asllvd} fornece originalmente as configurações de mão inicial e final para cada sinal, descrita de acordo com as 88 opções apresentadas por \citeonline{neidle-2020-asllrp} no \acrfull{asllrp}\footnote{
              Disponível em \url{http://www.bu.edu/asllrp}
          }.
          Utilizamos essa mesma informação para o ASL-Phono, porém adicionamos um passo extra para distribuir essas configurações entre todos os frame, e não apenas o inicial e final.
          Dessa forma, dividimos a sequência de frames em duas metades e associamos à primeira delas a configuração inicial provida pelo \acrshort{asllvd} e, à segunda, a configuração final.

    \item \textbf{Orientação}: é a direção apontada pelas palmas das mãos na articulação dos sinais.

          Para calculá-la, utilizamos um pouco de álgebra linear e exploramos a relação das mãos com o espaço tridimensional em que suas coordenadas estão inseridas.
          Iniciamos assumindo que cada palma é um plano cartesiano que atravessa as coordenadas estimadas para as mãos (vide \autoref{subfig:palm-orientation}). Selecionamos então três dessas coordenadas para descrever o plano: \(W\), que corresponde à coordenada do pulso; \(L\), localizada na base do dedo mínimo; e \(I\), localizada na base do dedo indicador.

          \begin{figure}[ht!]
              \centering
              \caption{
                  \textmd{
                      A orientação da palma \(O_{palm}\) é calculada a partir da normal \(\protect \overrightarrow{n}\) do plano cartesiano que representa a palma da mão.
                      %   As coordenadas \(W\), \(L\) e \(I\) e os vetores auxiliares são utilizados para obter a normal \(\protect \overrightarrow{n}\) da palma da mão~(\subref{subfig:palm-orientation}).
                      %   A direção da palma \(O_{palm}\) é então calculada a partir de \(\protect \overrightarrow{n}\) e descrita dentre um conjunto de direções possíveis~(\subref{ subfig:palm-directions}).
                  }
              }
              \subcaptionbox{\label{subfig:palm-orientation}}{
                  \includegraphics[height=4cm]{capitulos/metodologia/imagens/palm_orientation_algebra}
              }%
              \hspace{1cm}
              \subcaptionbox{\label{subfig:palm-directions}}{
                  \includegraphics[height=4cm]{capitulos/metodologia/imagens/hand_orientations}
              }%
              \nomefonte{}
              \label{fig:palm-orientation-directions}
          \end{figure}

          A partir dessas coordenadas, \citeonline{anton-2013-algebra} afirmam que pode-se estabelecer dois vetores auxiliares em termos dos quais esse mesmo plano cartesiano também é descrito (vide \autoref{subfig:palm-orientation}): \(\overrightarrow{WL}\), indicado pela seta azul e \(\overrightarrow{WI}\), indicado pela seta verde.
          Por meio deles, calculamos o vetor normal \(\overrightarrow{n}\) utilizando a \autoref{eqn:normal-palm-left} (para a mão esquerda) e \autoref{eqn:normal-palm-right} (para a mão direita). \(\overrightarrow{n}\), que é perpendicular à palma da mão, é ilustrado na \autoref{subfig:palm-orientation} como uma seta tracejada vermelha.

          %   Por meio desses vetores deles, é possível estabelecer o vetor normal \(\overrightarrow{n}\) perpendicular ao plano da palma, que é indicado pela seta vermelha tracejada, e calculado conforme \autoref{eqn:normal-palm-left} (para a mão esquerda) e \autoref{eqn:normal-palm-right} (para a mão direita):

          % Palm orientation:
          \begin{equation}
              \label{eqn:normal-palm-left}
              \overrightarrow{n}_{left} = \overrightarrow{WI} \times \overrightarrow{WL}
          \end{equation}

          \begin{equation}
              \label{eqn:normal-palm-right}
              \overrightarrow{n}_{right} = \overrightarrow{WL} \times \overrightarrow{WI}
          \end{equation}

          Por fim, utilizaremos os valores das coordenadas de \(\overrightarrow{n}\) para definir a orientação da palma \(O_{palm}\), conforme \autoref{eqn:palm-orientation-directions}.
          Essa orientação consiste na combinação de até três das seguintes direções: \textit{right} (direita), \textit{left} (esquerda), \textit{up} (para cima), \textit{down} (para baixo), \textit{body} (voltada para o corpo) ou \textit{front} (para frente).
          Por exemplo, ``\textit{right\_front\_down}'' seria uma orientação válida indicando que a palma da mão está inclinada, conforme ilustrado na \autoref{subfig:palm-directions}.

          %   Finalmente, ao avaliar os valores dos eixos \(x\), \(y\) e \(z\) da normal \(\overrightarrow{n}\), é possível definir a orientação da palma \(O_{palm}\) como sendo a combinação de até três direções, cujas opções são: \textit{right} (direita), \textit{left} (esquerda), \textit{up} (para cima), \textit{down} (para baixo), \textit{body} (para o corpo) ou \textit{front} (para frente).
          %   Por exemplo, ``\textit{right\_down}'' e ``\textit{left\_up\_body}'' seriam orientações válidas. Essa avaliação é realizada conforme \autoref{eqn:palm-orientation-directions}:

          % Directions
          \begin{equation}
              \label{eqn:palm-orientation-directions}
              O_{palm} =
              \begin{cases}
                  right & \text{if $\overrightarrow{n}_x < {-k}$ } \\
                  left  & \text{if $\overrightarrow{n}_x > {k}$ }  \\
                  up    & \text{if $\overrightarrow{n}_y < {-k}$ } \\
                  down  & \text{if $\overrightarrow{n}_y > {k}$ }  \\
                  body  & \text{if $\overrightarrow{n}_z < {-k}$ } \\
                  front & \text{if $\overrightarrow{n}_z > {k}$ }  \\
              \end{cases}
          \end{equation}

          Na \autoref{eqn:palm-orientation-directions}, \(k\) é definido empiricamente como 0,30 para filtrar variações pouco significativas em \(\overrightarrow{n}\).


    \item \textbf{Movimento}: é o deslocamento realizado pelas mãos na articulação do sinal.

          Para calculá-lo, primeiro selecionaremos a coordenada \(M\) localizada na base do dedo médio como referência (vide \autoref{subfig:hand-movement}).
          Em seguida, obteremos seu deslocamento entre os frames anterior (tempo \(t-1\)) e atual (tempo \(t\)) utilizando a \autoref{eqn:hand-movement} que, por sua vez, nos fornecerá o vetor de movimento \(\overrightarrow{m}\) (indicado pela seta tracejada vermelha na \autoref{subfig:hand-movement}).

          \begin{figure}[ht!]
              \centering
              \caption{
                  \textmd{
                      O movimento da mão \(V_{hand}\) é calculado a partir da trajetória da coordenada \(M\) entre os frames anterior (\(t-1\)) e atual (\(t\)).
                      %   O vetor de movimento \(\protect \overrightarrow{m}\) é calculado pela trajetória da coordenada \(M\) entre os frames anterior (\(t-1\)) e atual (\(t\))~(\subref{subfig:hand-movement}).
                      %   O movimento da mão \(V_{hand}\) é então obtido a partir de \(\protect \overrightarrow{m}\) e descrito dentre um conjunto de direções possíveis~(\subref{subfig:hand-directions}).
                  }
              }
              \subcaptionbox{\label{subfig:hand-movement}}{
                  \includegraphics[height=3.5cm]{capitulos/metodologia/imagens/hand_movement}
              }%
              \hspace{1cm}
              \subcaptionbox{\label{subfig:hand-directions}}{
                  \includegraphics[height=4cm]{capitulos/metodologia/imagens/hand_movement_2}
              }%
              \nomefonte{}
              \label{fig:hand-movement-directions}
          \end{figure}

          % Movement of the hands:
          \begin{equation}
              \label{eqn:hand-movement}
              \overrightarrow{m} = M_{t} - M_{t-1}
          \end{equation}

          A partir de \(\overrightarrow{m}\), podemos então calcular o movimento da mão \(V_{hand}\) através da \autoref{eqn:hand-movement-directions}, que consiste numa operação semelhante àquela utilizada para o movimento da mão.
          Com isso, \(V_{hand}\) também consistirá na combinação de até três direções: \textit{right} (direita), \textit{left} (esquerda), \textit{up} (para cima), \textit{down} (para baixo), \textit{body} (para o corpo) ou \textit{front} (para frente).
          A figura \autoref{subfig:hand-directions} ilustra um movimento na categorizado com a direção ``\textit{front}''.

          %   uma operação semelhante àquela utilizada para a orientação da mão, para definir o movimento da mão \(V_{hand}\) como a combinação de até três direções, cujas opções são: \textit{right} (direita), \textit{left} (esquerda), \textit{up} (para cima), \textit{down} (para baixo), \textit{body} (para o corpo) ou \textit{front} (para frente). Essa operação é detalhada na \autoref{eqn:hand-movement-directions} e a \autoref{subfig:hand-directions} ilustra ela segundo a perspectiva do sinalizador:

          % TODO: substituir V por M e atualizar a imagem
          % Directions
          \begin{equation}
              \label{eqn:hand-movement-directions}
              V_{hand} =
              \begin{cases}
                  right & \text{if $\overrightarrow{m}_x < {-k}$ } \\
                  left  & \text{if $\overrightarrow{m}_x > {k}$ }  \\
                  up    & \text{if $\overrightarrow{m}_y < {-k}$ } \\
                  down  & \text{if $\overrightarrow{m}_y > {k}$ }  \\
                  body  & \text{if $\overrightarrow{m}_z < {-k}$ } \\
                  front & \text{if $\overrightarrow{m}_z > {k}$ }  \\
              \end{cases}
          \end{equation}

          Na \autoref{eqn:hand-movement-directions} o limiar \(k\) foi também estabelecido empiricamente como 0,30, para filtrar movimentos com baixa relevância.


    \item \textbf{Expressão não-manual (abertura da boca)}: captura o grau de abertura da boca para cada frame na articulação do sinal que, por sua vez, denota a existência de expressão não-manual envolvendo ela.

          Para computar essa \textit{feature}, utilizamos como referência o trabalho de \citeonline{ferrario-2000-study-lips}, que analisa e propõe medidas antropométricas para os lábios.
          Dentre elas, selecionamos a \textit{vermilion height to mouth width} (ou altura dos lábios com relação à largura da boca) para estabelecer a abertura da boca \(P_{mouth}\), uma vez que essa medida é capaz de capturar a proporção de abertura dos lábios em termos de um único número.
          A \autoref{eqn:mouth-openness} define formalmente o cálculo de \(P_{mouth}\), que consiste basicamente na proporção entre a altura e a largura dos lábios:

          % Abertura da boca:
          \begin{equation}
              \label{eqn:mouth-openness}
              P_{mouth} = \frac{d(LS, LI)}{d(CH_r, CH_l)}
          \end{equation}

          A altura dos lábios é dada pela distância \(d\) entre as coordenadas do \textit{labiale superius} \(LS\) e do \textit{labiale inferius} \(LI\), que são os pontos mais externos aos lábios superior e inferior, respectivamente. A largura, por sua vez, consiste na distância entre as coordenadas do \textit{cheilion} direito \(CH_r\) e do \textit{cheilion} esquerdo \(CH_l\), que são os pontos situados nos cantos direito e esquerdo dos lábios, conforme ilustra a \autoref{fig:mouth-openness}.

          \figura
          {fig:mouth-openness} % Label
          {capitulos/metodologia/imagens/mouth_openness} % Path
          {height=4cm} % Size
          {A abertura da boca \(P_{mouth}\) é obtida a partir da medida antropométrica \textit{vermilion height to mouth width} que utiliza quatro coordenadas dos lábios para calcular uma única proporção.} % Caption
          {ferrario-2000-study-lips} % Citation

\end{enumerate}


Essas são as \textit{features} extraídas para o ASL-Phono. 
A \autoref{fig:sample-json-phono} exemplifica uma amostra resultante desse processo, bem como a disposição dos atributos fonológicos em suas propriedades.
Sua estrutura é muito semelhante àquela apresentada para o ASL-Skeleton3D, com exceção apenas da propriedade ``frames'' que, ao invés de coordenadas 3D, aqui apresenta os atributos fonológicos computados acima. Além do respectivo valor computado, cada atributo pode conter também um \textit{score}, que é obtido a partir da precisão estimada para as coordenadas envolvidas naquele cálculo.

\figura
{fig:sample-json-phono} % Label
{capitulos/metodologia/imagens/code_phono} % Path
{width=0.65\linewidth} % Size
{Exemplo de amostra do ASL-Phono.} % Caption
{} % Citation



% TODO: continuar daqui

% Estatísticas:
Na \autoref{tab:dataset-phono-stats}, podemos observar algumas estatísticas calculadas a partir do ASL-Phono, que nos permitem compreender como as amostras ficaram organizadas no \textit{dataset} após o processamento acima.
Nela, os números são listados segundo três perspectivas:

\begin{itemize}
    \item \textit{Dataset} inteiro: contabiliza o número total de amostras, sinais, movimentos distintos, entre outros, para todo o \textit{dataset}. Por exemplo, pode-se ler que o \textit{dataset} possui 9.747 amostras, 2.650 sinais e 26 movimentos para a mão dominante.

    \item Por amostra: contém estatísticas de números mínimos, máximos, médios e desvio padrão para os parâmetros computados acima. Por exemplo, lê-se que, em média, as amostras possuem 3,02 frames, sendo que a mais curta contém apenas 1 frame e a mais longa possui 12 frames. De maneira semelhante, as amostras têm em média de 1,94 movimentos e 2,20 orientações de mão distintas para a mão dominante.

    \item Por sinal: contém estatísticas de números mínimos, máximos, médios e desvio padrão para os sinais contidos no \textit{dataset}. Por exemplo, lê-se que cada sinal possui em média de 3,68 amostras, sendo que alguns deles possuem apenas 1 e outros possuem valores atípicos de 59 amostras (a considerar pelo desvio padrão apresentado). Além disso, cada sinal apresenta em média 6,04 movimentos e 5,37 orientações de mão distintas para a mão dominante.
\end{itemize}

% Please add the following required packages to your document preamble:
% \usepackage{multirow}
% \usepackage{graphicx}
% \usepackage[table,xcdraw]{xcolor}
% If you use beamer only pass "xcolor=table" option, i.e. \documentclass[xcolor=table]{beamer}
\begin{table}[]
    \centering
    \caption{Estatísticas calculadas a partir do ASL-Phono, as quais são visualizadas para todo o \textit{dataset} e segundo agrupamentos por amostra e por sinal. (D) refere-se à mão dominante e (ND) refere-se à mão não-dominante.}
    \label{tab:dataset-phono-stats}
    \resizebox{0.9\textwidth}{!}{%
        \begin{tabular}{cr|ccc|ccccccc}
            \hline
            \rowcolor[HTML]{EFEFEF}
                                                                                     &        & \cellcolor[HTML]{EFEFEF}                           & \cellcolor[HTML]{EFEFEF}                         & \cellcolor[HTML]{EFEFEF}                         & \multicolumn{2}{c}{\cellcolor[HTML]{EFEFEF}Movimento} & \multicolumn{2}{c}{\cellcolor[HTML]{EFEFEF}Orientação} & \multicolumn{2}{c}{\cellcolor[HTML]{EFEFEF}Config. mão} & \cellcolor[HTML]{EFEFEF}                                                                                                                       \\
            \rowcolor[HTML]{EFEFEF}
                                                                                     &        & \multirow{-2}{*}{\cellcolor[HTML]{EFEFEF}Amostras} & \multirow{-2}{*}{\cellcolor[HTML]{EFEFEF}Sinais} & \multirow{-2}{*}{\cellcolor[HTML]{EFEFEF}Frames} & (D)                                                   & (ND)                                                   & (D)                                                     & (ND)                     & (D)  & (ND) & \multirow{-2}{*}{\cellcolor[HTML]{EFEFEF}\begin{tabular}[c]{@{}c@{}}Abertura \\ da boca\end{tabular}} \\ \hline
            \begin{tabular}[c]{@{}c@{}}Dataset \\ inteiro\end{tabular}               &        & 9.747                                              & 2.650                                            & -                                                & 26                                                    & 26                                                     & 26                                                      & 26                       & 85   & 78   & -                                                                                                     \\ \hline
                                                                                     & Mín    & -                                                  & -                                                & 1                                                & 0                                                     & 0                                                      & 0                                                       & 0                        & 0    & 0    & 0,01                                                                                                  \\
                                                                                     & Máx    & -                                                  & -                                                & 12                                               & 10                                                    & 8                                                      & 6                                                       & 5                        & 2    & 2    & 2,19                                                                                                  \\
                                                                                     & Média  & -                                                  & -                                                & 3,02                                             & 1,94                                                  & 1,26                                                   & 2,20                                                    & 1,18                     & 1,17 & 0,72 & 0,13                                                                                                  \\
            \multirow{-4}{*}{\begin{tabular}[c]{@{}c@{}}Por \\ amostra\end{tabular}} & Desvio & -                                                  & -                                                & 0,87                                             & 0,83                                                  & 1,15                                                   & 0,79                                                    & 1,07                     & 0,38 & 0,58 & 0,12                                                                                                  \\ \hline
                                                                                     & Mín    & 1                                                  & -                                                & -                                                & 1                                                     & 0                                                      & 1                                                       & 0                        & 1    & 0    & 0,02                                                                                                  \\
                                                                                     & Máx    & 59                                                 & -                                                & -                                                & 24                                                    & 16                                                     & 22                                                      & 14                       & 8    & 8    & 0,99                                                                                                  \\
                                                                                     & Média  & 3,68                                               & -                                                & -                                                & 6,04                                                  & 3,66                                                   & 5,37                                                    & 2,63                     & 1,81 & 1,17 & 0,13                                                                                                  \\
            \multirow{-4}{*}{\begin{tabular}[c]{@{}c@{}}Por \\ sinal\end{tabular}}   & Desvio & 2,67                                               & -                                                & -                                                & 2,93                                                  & 3,18                                                   & 2,29                                                    & 2,35                     & 0,99 & 1,14 & 0,07                                                                                                  \\ \hline
        \end{tabular}%
    }
    \nomefonte{}
\end{table}


O \textit{dataset} resultante do processamento acima e o código-fonte utilizado para isso estão disponíveis publicamente na URL listada abaixo\footnote{Disponível em \url{http://www.cin.ufpe.br/~cca5/asl-phono}}.
