\section{Reconhecendo os sinais}
\label{sec:metodologia-reconhecimento}

Uma vez que temos os atributos processados no ASL-Phono, prosseguiremos com a preparação das features e dos modelos para reconhecimentos dos sinais utilizando a abordagem proposta.

Para preparar as features de entrada dos modelos, aplicamos uma codificação às propriedades das amostras para representar os atributos fonológicos num formato mais compacto, como único bloco ou ``palavra''. Dessa forma, ao invés de frames que contêm propriedades associadas, agora as amostras são representadas como sequências de blocos compactos que codificam a mesma informação.

Observe na \autoref{tab:codificacao-bloco} \dots

% Please add the following required packages to your document preamble:
% \usepackage{graphicx}
% \usepackage[table,xcdraw]{xcolor}
% If you use beamer only pass "xcolor=table" option, i.e. \documentclass[xcolor=table]{beamer}
\begin{table}[ht!]
    \centering
    \caption{Exemplo de compactação dos atributos fonológicos do frame de uma amostra do ASL-Phono em uma ``palavra''.}
    \label{tab:codificacao-bloco}
    \resizebox{0.95\textwidth}{!}{%
        \begin{tabular}{r|cccccc}
            \hline
            \rowcolor[HTML]{EFEFEF}
            \multicolumn{1}{l|}{\cellcolor[HTML]{EFEFEF}} & \multicolumn{6}{c}{\cellcolor[HTML]{EFEFEF}Atributos}                                                                                                                                                                                                                          \\ \cline{2-7}
            \rowcolor[HTML]{EFEFEF}
                                                          & \multicolumn{3}{c|}{\cellcolor[HTML]{EFEFEF}Mão dominante}                    & \multicolumn{3}{c}{\cellcolor[HTML]{EFEFEF}Mão não-dominante}                                                                                                                                  \\
            \rowcolor[HTML]{EFEFEF}
                                                          & Movimento                                                                     & Orientação                                                    & \multicolumn{1}{c|}{\cellcolor[HTML]{EFEFEF}Config. mão} & Movimento                  & Orientação           & Config. mão     \\ \hline
            Valores originais                             & \textit{right\_up}                                                            & \textit{left}                                                 & \multicolumn{1}{c|}{\textit{bentBL}}                     & \textit{left\_front\_down} & \textit{back\_right} & \textit{bentBL} \\
            Valores abreviados                            & \textit{ru}                                                                   & \textit{l}                                                    & \multicolumn{1}{c|}{\textit{bentBL}}                     & \textit{lfd}               & \textit{br}          & \textit{bentBL} \\ \hline
            {\color[HTML]{3531FF} Palavra}                & \multicolumn{6}{c}{{\color[HTML]{3531FF} \textit{ru-l-bentBL-lfd-br-bentBL}}}                                                                                                                                                                                                  \\ \hline
        \end{tabular}%
    }
    \nomefonte{}
\end{table}

% Experimento
%   Preparação das features (asl-phono -> palavras)
%       Transformação das sequências no dataset: frames -> palavras
%       Justificativa??
%   Preparação dos modelos (Transformer, LSTM, GRU, etc) -- por modelo
%       Arquitetura + Parâmetros
%       lr scheduler, optimizer, loss function
%       Busca de parâmetros (dimensionar os modelos/parâmetros)
