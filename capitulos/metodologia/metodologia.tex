% TODO: revisar se essa introdução continua abrangendo o conteúdo do capítulo
Neste capítulo, discutiremos a abordagem proposta por este trabalho, bem como as justificativas para sua adoção, suas contribuições para a área de \acrshort{slr} e os detalhes das técnicas e experimentos envolvidos.

% proposta
% justificativas
% beneficio
%   - desafios (da SLR) que resolve

% técnicas
% experimentos


esta pesquisa possui como proposta adotar uma abordagem centrada na linguística da língua de sinais para realizar o reconhecimento dos sinais
- ao analisar a linguistica, vemos que ela se divide o estudo da língua em três partes distintas, que descrevem níveis crescentes de significado: vão desde seus menores elementos constituintes (na fonologia), passam pela articulação desses elementos para um nível maior de significado -- que seriam as palavras (na morfologia) e alcançam um nível mais complexo, que articula essas palavras para compor sentenças e estruturas mais complexas (na sintaxe).
- como percebe-se, a fonologia é o nível mais fundamental para se abordar uma língua sob uma abordagem linguistica. na língua de sinais, ela descreve os parâmetros mínimos que precisam ser compreendidos e dominados antes de avançar para os próximos níveis de significado. 

- por este motivo, este trabalho adota a hipótese de que este é também o nível mais básico e fundamental para representar e abordar a língua de sinais em técnicas de \acrshort{slr}, o qual deve ser dominado antes de partirmos para explorar níveis de significado mais elevados e complexos. 
- dessa forma, ao invés de utilizar imagens RGB ou coordenadas brutas do corpo dos indivíduos como entrada para as técnicas aplicadas -- como observa-se em muitos estudos na área --, nesta pesquisa representando os sinais segundo o nível fonológico da língua.

imagens RGB e coordenadas do corpo humano não carregam semântica por si só. ao contrário, é necessário um nível de interpretação sobre elas para que seja possível inferir, por exemplo, que um conjunto de coordenadas de dedos e corpo se refere a "uma mão com orientação voltada para o corpo numa trajetória para a esquerda". 
além disso, a quantidade de pixels contidos nessas imagens ou de coordenadas corporais brutas são comumente grandes, mas apenas uma parcela pequena é realmente relevante para os sinais que se deseja interpretar.
de fato, essa necessidade de lidar com dados brutos tem exigido de pesquisadores um esforço adicional para que suas técnicas interpretem-os corretamente e consigam contornar essa complexidade antes de progredir para níveis semânticos maiores da língua de sinais. Por exemplo, é comum observar técnicas como fluxo óptico, 
\acrshort{mei}\footnote{
    \acrlong{mei}: é uma imagem binária, onde a região branca representa onde há movimento ocorrendo e o preto denota a região onde não há movimento \cite{ahad-2012-mhi-for-action}.
}
e
\acrshort{mhi}\footnote{
    \acrlong{mhi}: expressa a sequência de movimento de forma compacta, em uma única imagem, onde pixels em escala de cinza com intensidade menor descrevem frames de movimentos mais antigos nessa sequência \cite{ahad-2012-mhi}.
} 
sendo aplicadas para capturar os movimentos a partir dos pixels dos frames, ou ainda camadas extras sendo adicionadas aos modelos de aprendizagem profunda para lidar com isso, mas que os tornam mais caros e complexos. 


% - reduz a complexidade de aprender o mapeamento coordenadas -> linguistica / foca no aprendizado da linguistica
ao prover parâmetros fonológicos ao invés de dados brutos aos modelos de aprendizagem, retiramos o foco do aprender a interpretar pixels ou coordenadas sem significado e fazemos com que ele passe a aprender as relações e regras linguísticas contidas na língua. conforme discutimos na seção XXX, a linguística da língua já apresenta um conjunto de complexidades e regras que precisam ser abordadas para que a \acrshort{slr} avance de forma consistente. englobar a interpretação desses dados brutos como parte do problema faz com que a complexidade seja redobrada e limita a velocidade desses avanços.

apesar dessa complexidade, muitos desses parâmetros fonológicos poderiam ser computados algebricamente a partir de coordenadas 3d para formatos semanticamente mais elevados de forma mais barata antes de serem ingeridos por algoritmos.








sendo assim, a abordagem deste trabalho adotará parâmetros fonológicos como dados de entrada para o processo de reconhecimento dos sinais. como pode-se imaginar, atualmente não existem \textit{datasets} disponíveis com esse tipo de parâmetros e, portanto, o primeiro desafio nessa direção foi desenvolver um \textit{dataset} que pudesse suportar essa pesquisa. 

dentre as alternativas mais viáveis para isso, optamos por derivar um novo \textit{dataset} a partir do \acrfull{asllvd}, o qual é um dos mais relevantes da \acrshort{asl} e foi desenvolvido na Universidade de Boston por \citeonline{athitsos-2008-asllvd} e \citeonline{neidle-2012-asllvd}. % Ele é composto por um vocabulário de 2.745 sinais representados em cerca de 9.763 sequências de vídeos anotadas que, por sua vez, são articuladas por Surdos nativos nessa língua.

aplicamos um conjunto de processos para computar as representações fonológicas a partir dos frames RGB em 2D das amostras do \acrshort{asllvd}, o que incluiu a criação de uma representação intermediária em 3D sob a qual foi possível aplicar um conjunto de operações algébricas para alcançar nosso objetivo. Discutiremos esse processo em detalhes nas seções a seguir.

% para isso, primeiro foi necessário desenvolver um dataset -- desafios:
% - estabelecer técnicas para computar e representar no nível de parâmetros fonológicos as  coordenadas do corpo humano
%     - mas antes disso: sair de imagens 2D para coordenadas em 3D

% para isso, desafios:
%  - nao ha datasets da fonologia -> construir dataset -> necessário coordenadas 3d
%  - dataset produz imagens 2d -> estimar coordenadas 2d -> reconstruir coordenadas 3d
%  - computar parametros fonologicos a partir das coordenadas 3d
%     - estabelecer funções algebricas


uma vez que os parâmetros de entrada foram estabelecidos, podemos então prosseguir realizando o reconhecimento dos sinais. para isso, aplicaremos algumas arquiteturas clássicas de modelos sequenciais de aprendizagem profunda, além de um \textit{transformer} -- que é comumente adotado em tarefas de \acrshort{nlp} --. para que assim possamos quantificar a eficácia e estabelecer uma linha de base para a abordagem proposta. este processo será discutido em mais detalhes nas seções seguintes.

% TODO: criar imagem detalhando o processo/abordagem proposto
observamos na imagem XXX um diagrama que ilustra a abordagem proposta.


entre os desafios que esta proposta contribui a superar, temos os seguintes:

\begin{enumerate}
    \item aborda o \acrshort{slr} sob uma perspectiva orientada às particularidades da língua de sinais, considerando sua linguística e natureza visual intrínseca como fatores primordiais.
    
    \item por ser centrado na linguística da língua de sinais, este trabalho ajuda a criar consciência e reconhecimento dela como sendo um pilar fundamental para novas técnicas na área de \acrshort{slr}.

    \item é uma abordagem independente de variações nos traços e dimensões corporais, iluminação do ambiente, qualidade das câmeras, etc., bem como separada de complexidades atreladas à detecção partes do corpo ou interpretação de seus movimentos no espaço. isso porque, uma vez que o foco é colocado na fonologia da língua, tais complexidades são terceirizados para algoritmos especializados nesse tipo tarefa. sendo assim, o papel da \acrshort{slr} passa a englobar tão somente a língua de sinais e suas particularidades.
    
    \item as anotações contendo os parâmetros fonológicos são geradas automaticamente por meio de expressões algébricas que analisam o corpo no espaço tridimensional. isso pode ser replicado para outros \textit{datasets} (ou em aplicações do mundo real), o que contribuiria para reduzir a necessidade de produção de anotações manuais e, como consequência, as limitações ao combinar \textit{datasets} distintos com o intuito de elevar a quantidade e diversidade de amostras utilizadas pelas pesquisas em \acrshort{slr}.
    
    \item este trabalho contribui com dos novos \textit{datasets} para a língua de sinais, os quais podem ser utilizados para derivar outros novos \textit{datasets} ou para continuar evoluindo técnicas centradas na linguística. além disso, por se basearem no \acrshort{asllvd}, estes \textit{datasets} são criados a partir de sinais articulados por indivíduos Surdos nativos na língua.

    \item apesar do conjunto específico de parâmetros fonológicos contemplados e de lidar apenas com sinais isolados, entendemos que a abordagem proposta é passível de ser estendida para outros parâmetros e também para sinais contínuos.

\end{enumerate}










% \section{Metodologia}
% \label{sec:metodologia}

% A metodologia aplicada nesta dissertação concentra-se em compreender os desafios atuais da área de pesquisa para assim introduzir uma proposta que suporte avanços futuros coerentes com as necessidades do mundo real.
% As etapas percorridas aqui podem ser sumarizadas como:

% \begin{itemize}
%     \item Revisão do panorama atual da deficiência auditiva e do papel que as línguas de sinais desempenham aqui;
%     \item Revisão do panorama das pesquisas atuais em processamento de língua de sinais e das lacunas que têm limitado progressos mais expressivos na área.
%     \item Desenvolvimento de uma proposta que aborde as lacunas acima, contribuindo para preencher algumas delas e produzindo artefatos que suportem novas pesquisas a evoluir nessa direção;
%     \item Realização de experimentos e análise dos resultados coletados.
% \end{itemize}

% % - análise do panorama atual das línguas de sinais 
% % - análise do panorama atual das pesquisas na área e identificação de principais lacunas para seu avanço
% % - revisão da literatura das línguas de sinais e de sua linguísticas
% % ------
% % - seleção de um conjunto de atributos linguísticos
% % - análise de técnicas algébricas para suportar o processamento de atributos linguísticos selecionados
% % - seleção de modelos sequenciais de aprendizagem de máquina para os experimentos
% % - execução dos experimentos e análise dos resultados coletados






%   Introduzir abordagem (reconhecimento de língua de sinais baseada na linguística)
%       
%   Justificar caminho pelo uso da linguística
%       Abordagens anteriores costumam se limitar à classificação de imagens estáticas ou vídeos
%       Coordenadas 3D não carregam semântica por si só
%           Número de coordenadas é grande e nem todas as mudanças nelas são relevantes ??
%       Linguística considera a dinâmica / semântica da língua (agrupa as entradas em uma granularidade maior, com significado que é compreensível por humanos)
%       Elevar o nível semântico / prover um nível semântico para os algoritmos
%   Benefícios da abordagem (desafios abordados)
%       
%
%   Visão geral do método proposto 
%       Explicar método (reconhecimento de sinais baseado na fonologia)
%           representar informação no nível da fonologia (menor nível da linguística)
%           
%       ASL-Phono
%           Desafios / escassez de datasets diferentes
%           Falar do dataset
%       Modelos sequenciais
%           Transformer: breve introdução (arquitetura e funcionamento)
%           RNNs (GRU, LSTM, etc)
%
%   Experimento
%       Preparação dos dados
%           Transformação das sequências no dataset: frames -> palavras
%           Justificativa??
%       Setup dos modelos (Transformer, LSTM, GRU, etc)
%           Parâmetros
%               Buscar parâmetros (dimensionar os modelos/parâmetros)


