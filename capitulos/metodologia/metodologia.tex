% TODO: revisar se essa introdução continua abrangendo o conteúdo do capítulo
Neste capítulo, discutiremos a abordagem proposta por este trabalho, bem como as justificativas para sua adoção, suas contribuições para a área de \acrshort{slr} e os detalhes das técnicas e experimentos envolvidos.


% proposta



% justificativas
% técnicas
% experimentos




% \section{Metodologia}
% \label{sec:metodologia}

% A metodologia aplicada nesta dissertação concentra-se em compreender os desafios atuais da área de pesquisa para assim introduzir uma proposta que suporte avanços futuros coerentes com as necessidades do mundo real.
% As etapas percorridas aqui podem ser sumarizadas como:

% \begin{itemize}
%     \item Revisão do panorama atual da deficiência auditiva e do papel que as línguas de sinais desempenham aqui;
%     \item Revisão do panorama das pesquisas atuais em processamento de língua de sinais e das lacunas que têm limitado progressos mais expressivos na área.
%     \item Desenvolvimento de uma proposta que aborde as lacunas acima, contribuindo para preencher algumas delas e produzindo artefatos que suportem novas pesquisas a evoluir nessa direção;
%     \item Realização de experimentos e análise dos resultados coletados.
% \end{itemize}

% % - análise do panorama atual das línguas de sinais 
% % - análise do panorama atual das pesquisas na área e identificação de principais lacunas para seu avanço
% % - revisão da literatura das línguas de sinais e de sua linguísticas
% % ------
% % - seleção de um conjunto de atributos linguísticos
% % - análise de técnicas algébricas para suportar o processamento de atributos linguísticos selecionados
% % - seleção de modelos sequenciais de aprendizagem de máquina para os experimentos
% % - execução dos experimentos e análise dos resultados coletados






%   Introduzir abordagem (reconhecimento de língua de sinais baseada na linguística)
%       
%   Justificar caminho pelo uso da linguística
%       Abordagens anteriores costumam se limitar à classificação de imagens estáticas ou vídeos
%       Coordenadas 3D não carregam semântica por si só
%           Número de coordenadas é grande e nem todas as mudanças nelas são relevantes ??
%       Linguística considera a dinâmica / semântica da língua (agrupa as entradas em uma granularidade maior, com significado que é compreensível por humanos)
%       Elevar o nível semântico / prover um nível semântico para os algoritmos
%   Benefícios da abordagem (desafios abordados)
%       
%
%   Visão geral do método proposto 
%       Explicar método (reconhecimento de sinais baseado na fonologia)
%           representar informação no nível da fonologia (menor nível da linguística)
%           
%       ASL-Phono
%           Desafios / escassez de datasets diferentes
%           Falar do dataset
%       Modelos sequenciais
%           Transformer: breve introdução (arquitetura e funcionamento)
%           RNNs (GRU, LSTM, etc)
%
%   Experimento
%       Preparação dos dados
%           Transformação das sequências no dataset: frames -> palavras
%           Justificativa??
%       Setup dos modelos (Transformer, LSTM, GRU, etc)
%           Parâmetros
%               Buscar parâmetros (dimensionar os modelos/parâmetros)


