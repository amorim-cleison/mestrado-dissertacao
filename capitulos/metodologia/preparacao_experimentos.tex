\section{Preparação dos experimentos}
\label{sec:metodologia-preparacao-experimentos}

Os parâmetros para treinamento dos modelos foram selecionados com base na discussão apresentada por \citeonline{goodfellow-2016-deep-learning}, a qual aborda de forma prática temas como definição de métricas, modelos de base, otimização de parâmetros, entre outros para tal finalidade.

Dessa forma, utilizamos como algoritmo de otimização o \acrfull{sgd} com \textit{momentum} de 0,9. Ele foi combinado com uma estratégia de redução da taxa de aprendizagem por um fator de 0,2 sempre que o valor da perda calculada sobre os dados de validação atinge um platô por 5 épocas seguidas. A função de perda utilizada, por sua vez, foi a \textit{Cross-Entropy Loss} (ou Perda de Entropia Cruzada). Além disso, adotamos \textit{batches} com tamanho de 50 amostras e particionamos o \textit{dataset} numa proporção de 15\% para validação, 15\% para testes e o restante para treinamento dos modelos.

% TODO: detalhar arquitetura do seq2seq + lstm:
Para o modelo \textit{Encoder-Decoder} com \acrshort{lstm} utilizamos ...

% TODO: detalhar arquitetura do seq2seq + gru:
Para o \textit{Encoder-Decoder} com \acrshort{gru}, por sua vez, utilizamos ...

No caso do \textit{Transformer}, não aplicamos nenhuma customização e adotamos a arquitetura original introduzida por \cite{vaswani-2017-transformer} -- que consiste de \textit{embeddings} com dimensões de 512, camadas ocultas com dimensões de 2048, 6 camadas, 8 cabeças e \textit{dropout} de 0,1.
Realizamos uma otimização da taxa de aprendizagem para esse modelo utilizando a estratégia de \textit{grid search} com validação cruzada de 5 \textit{folds} por 50 épocas. O resultado, ilustrado na \autoref{fig:otim_lr_transformer}, nos mostra que o valor que melhor minimiza a perda é 0,1.

\figura[]
    {fig:otim_lr_transformer}
    {capitulos/metodologia/imagens/otim_lr_transformer}
    {height=4.0cm}
    {Otimização da taxa de aprendizagem com relação à perda calculada para o \textit{Transformer}. LR refere-se à taxa de aprendizagem (ou \textit{learning rate}).}
    {}

\figura[]
    {fig:otim_lr_encdec_lstm}
    {capitulos/metodologia/imagens/otim_lr_encdec_lstm}
    {height=4.0cm}
    {Otimização da taxa de aprendizagem com relação à perda calculada para o \textit{Encoder-Decoder (LSTM)}. LR refere-se à taxa de aprendizagem (ou \textit{learning rate}).}
    {}


\figura[]
    {fig:otim_lr_encdec_gru}
    {capitulos/metodologia/imagens/otim_lr_encdec_gru}
    {height=4.0cm}
    {Otimização da taxa de aprendizagem com relação à perda calculada para o \textit{Encoder-Decoder (GRU)}. LR refere-se à taxa de aprendizagem (ou \textit{learning rate}).}
    {}

O código fonte utilizado nos experimentos está disponível no endereço indicado abaixo\footnote{
    Disponível em \url{https://www.cin.ufpe.br/~cca5/sl-nlp}.
}.



% \cite{goodfellow-2016-deep-learning}

% --------
% A reasonable choice of optimization algorithm is SGD with momentum with a decaying learning rate (popular decay schemes that perform better or worse on different problems include decaying linearly until reaching a fixed minimum learning rate, decaying exponentially, or decreasing the learning rate by a factor of 2–10 each time validation error plateaus).

% - optimizer: SGD
%     nesterov: False
%     momentum: 0.9
% --------
% Loss function

% - criterion: CrossEntropyLoss

% --------
% Early stopping should be used almost universally.

% - training_args:
%     max_epochs: 50
%     batch_size: 50
%     lr: 0.01
%     test_size: 0.15
%     valid_size: 0.15
%     early_stopping:
%         patience: 10
%         threshold: 10e-4
%         threshold_mode: rel
%     gradient_clipping:
%         gradient_clip_value: 0.5
%     lr_scheduler: 
%         policy: ReduceLROnPlateau
%         factor: 0.2
%         patience: 5

% --------
% Learning rate - apresentar graficos?
% - learning rate (grid search)


% --------
% - dimensões dos modelos

%     Transformer (selecionamos as dimensões do modelo clássico \cite{vaswani-2017-transformer})
%         embedding_size: 512
%         hidden_size: 2048
%         num_layers: 6
%         dropout: 0.1
%         num_heads: 8

    