Neste capítulo discutiremos em mais detalhes a abordagem proposta, as justificativas para sua adoção, bem como as técnicas que foram aplicadas e a preparação dos experimentos realizados.

Esta pesquisa propõe-se a realizar o reconhecimento da linguagem de sinais a partir de uma perspectiva estritamente linguística, baseada nos constituintes fonológicos mínimos que descrevem os sinais, e centrada no aprendizado das complexidades e regras que convém contexto e dá significado a eles.

Isso assemelha-se à forma como hoje outras línguas são abordadas com sucesso pelo \acrfull{nlp} e diferencia-se daquela predominante no \acrfull{slr} que, por sua vez, trata os sinais como gestos não-estruturados, mapeados a partir de dados brutos capturados dos indivíduos -- como pixels de imagens ou \textit{frames} de vídeos, pontos lidos de luvas eletrônicas, coordenadas 2D ou 3D, entre outros -- e colocam em segundo plano a importância linguística do sinal.

% Esta pesquisa apresenta uma proposta de reconhecimento da linguagem de sinais que segue uma abordagem linguística, baseada em seus constituintes mínimos fonológicos, que se diferencia da abordagem predominantemente observada na \acrshort{slr} que, por sua vez, analisa os sinais a partir das informações brutas capturadas do corpo dos indivíduos -- como frames de imagens e vídeos, dados de luvas eletrônicas, coordenadas 3D, entre outras.

Nossa hipótese assume que, além de deixar de abordar uma parte muito importante dessa língua, lidar com esse tipo de dados brutos traz complexidades adicionais que extrapolam o escopo que deveria ser efetivamente abordado pelo \acrshort{slr}.
Em outras palavras, esse foco inadequado faz com que pesquisas em \acrshort{slr} deixem de solucionar um problema intrinsecamente de \acrshort{nlp} e passem a investir esforços consideráveis tentando lidar com um conjunto de desafios pertinentes à área de \acrfull{cv} -- os quais comumente já estão abordados ou solucionados por uma de suas subáreas, como a detecção, segmentação e rastreamento de partes do corpo em imagens e vídeos; a interação entre mãos e oclusões decorrentes disso; as variações de tom de pele, cores de roupa e luminosidade do ambiente; entre outros listados nas revisões literárias elaboradas no decorrer da última década para a \acrshort{slr} \cite{papastratis-2021-ai-technologies-sl,rastgoo-2021-slr-deep-survey,koller-2020-quantitative-survey-slr,bragg-2019-slr-interdisciplinary,wadhawan-2019-slr-literature-review,suharjito-2018-feature-extraction-survey,joksimoski-2022-scoping-review,cooper-2011-slr}.

% - problem statement / motivation

% hipótese principal: a área de reconhecimento de sinais hoje dedica muita energia tentando lidar com problemas que deveriam (e são) abordados por outras áreas (como detecção de partes do corpo, variações de luminosidade e tom de pele, interpretação de movimento a partir de coordenadas ou de pixels dos frames) mas ainda não chegaram a abordar efetivamente as complexidades da linguística da língua de sinais em si, que por si só é um universo extremamente complexo (e ilustro isso dando uma introdução da linguística na seção de fundamentação teórica). 
% Essa "perda de energia" atrasa progressos realmente importantes dessa área. 
%     -> essa argumentação precisa estar no documento com as referencias que corroboram essa análise. 


% confusão - reconhecimento de gestos (detecção/rastreamento de mãos) x SLR corrobora
%     - desperdiçam energia com: hand detection, hand segmentation, hand tracking, detecção de caracteristicas corporais (mas não da linguagem propriamente dita)

%     Copper - "Sign language offers a more complex challenge than the traditionally more confined domain of gesture recognition."
%     Copper - The task of recognition is often simplified by forcing the possible word sequence to conform to a grammar which limits the potential choices and thereby improves recognition rates [91, 104, 12, 45].

% problemas que precisam lidar
%     Suharjito, Copper - in natural scenario hands move fast / tracking hand, hands are dynamic, occlusion, interaction with each other / skin color detection / segmentation of body or hand / 

% tecnicas usadas
%     Suharjito, Copper - (tracking based: ) kalman filter, optical flow, MHI, HOG / (non-tracking based: ) PCA, ICA (independent component analysis), sobel filters, Haar, DTW (dynamic time warping)

% Copper, Joskimoski, Suharjito, Bragg, Rastgoo - data gloves, gloves, imagens, videos, RF sensor, armband, kinect, leap motion
%     Papastratis - electromyography (EMG), inertial measurement unit (IMU)

% Copper, Rastgoo, Joskimoski, Suharjito

% Joksimoski -> Copper: Various approaches using various types
% of Neural Networks, Hidden Markov Models and their
% variations (like Parallel HMM), decision trees, and self-organizing
% maps are utilized for various parts of SLR.


Alguns exemplos populares desses problemas sendo consistentemente endereçados dentro da \acrshort{cv} incluem as ferramentas OpenPose~\cite{wei-2016-conv-machines-openpose,cao-2017-openpose,simon-2017-openpose-hand-face} e MediaPipe~\cite{lugaresi-2019-mediapipe,bazarevsky-2019-mediapipe-blazeface,vakunov-2020-mediapipe-hands,bazarevsky-2020-mediapipe-blazepose}, desenvolvidas pela Carnegie Mellon University e Google Research, respectivamente.
Ambas são o resultado de anos de pesquisa em torno de tais questões, as quais alcançaram um nível de maturidade elevado capaz de abordar em tempo real tarefas de estimativa de pose e rastreamento do corpo, mãos e face (inclusive envolvendo múltiplos indivíduos) de forma robusta a variações corporais, de luminosidade e de ambientes, utilizando apenas uma câmera comum RGB.
Elas estão disponíveis abertamente\footnote{
    O OpenPose está disponível em \url{https://github.com/CMU-Perceptual-Computing-Lab/openpose} e o MediaPipe em \url{https://mediapipe.dev/}.
} e a reutilização desse conhecimento nas etapas para capturar e gerar \textit{features} de níveis mais elevados para o \acrshort{slr} certamente contribuirá para progressos mais efetivos.


Fazendo uma analogia com outras tarefas de \acrshort{nlp}, abordar a língua de sinais por meio dos dados brutos como discutido acima e lidar com os desafios apresentados, por exemplo, possui uma complexidade equivalente a tentar interpretar textos manuscritos apenas rastreando-se o movimento da mão do autor enquanto ele desenha as letras no papel -- ao invés de simplesmente escanear o texto final escrito como entrada para isso; ou ainda, tentar reconhecer a fala de um indivíduo apenas realizando a detecção e o rastreamento dos movimentos de seus lábios -- ao invés de considerar os sinais de áudio capturados para tal.

% avaliar isto: numa metáfora, hoje estamos caminhando como se tentássemos reconhecer um texto escrito ao mesmo tempo em que a mão está desenhando as letras no papel (ao invés de utilizar o artefato intermediário, que é o texto no papel) -- percebe a extensão do problema? detecção da mão, detecção do movimento, para finalmente alcançar o real objetivo.
% Fazendo uma metáfora, é como se a gente tivesse tentando reconhecer textos escritos a mão, mas ao invés de escanear o papel com o texto escrito (assumindo-se o texto escrito como um produto intermediário deste processo), estivéssemos dando vários passos para atrás e tentando fazer esse reconhecimento através da detecção do movimento da mão enquanto o autor escreve (o que abrange um escopo muito maior do que o problema de reconhecer a escrita, por si só).


% % TODO avaliar isto: a proposta aqui é quebrar o problema grande em problemas menores, delegando as partes aos algoritmos que são especializados naquilo: detecção do corpo humano/coordenadas em imagens, rastreamento do movimento, reconhecimento de sinais
%     Então, nessa minha abordagem eu proponho retirar o foco desses problemas (de reconhecer pixels, rastrear partes do corpo, etc.) que são objeto de estudo de outras áreas da visão computacional (e que hoje já tem muitos algoritmos muitos melhores do que eu poderia propor dentro de minha arquitetura)...
%         -> Acho válido. Um ponto que pode ajudar é talvez vc mostrar que essas coisas são bem resolvidas na outra área e que vc pode reusar esse conhecimento/métodos para essa fase de aquisição dos dados.
%             -> explorar os papers do OpenPose e do MediaPipe (Google)



Como resultado desse enquadramento inadequado por parte das pesquisas em \acrshort{slr}, no decorrer das últimas décadas constata-se um progresso pouco expressivo dessa área sobretudo nos aspectos da linguagem e aplicabilidade no mundo real, acerca do qual \citeonline{selvaraj-2022-openhands,yin-2021-sl-in-nlp,cooper-2011-slr} reiteram:

% apresentado acima, que se destaca entre os outros desafios que discutimos na \autoref{sec:slr-desafios}, constata-se um progresso muito pouco expressivo da área de \acrshort{slr} no decorrer das últimas décadas, sobretudo sob os aspectos de linguagem e de aplicabilidade no mundo real. 
% Na prática, há uma carência de trabalhos que abordem a língua de sinais como uma língua e, consequentemente, produzam resultados úteis ao contexto real.
% \citeonline{cooper-2011-slr} reiteram:

\begin{citacao}
    Quando comparado com a pesquisa de \acrlong{nlp} baseada em texto e fala, o progresso das pesquisas para línguas de sinais está significativamente atrasado. \cite[tradução nossa]{selvaraj-2022-openhands,yin-2021-sl-in-nlp}
\end{citacao}

\begin{citacao}
    Enquanto sistemas de reconhecimento da fala avançaram ao ponto de estarem comercialmente disponíveis, o reconhecimento de sinais ainda está em sua infância.
    Atualmente, todos os serviços comerciais de tradução de sinais são baseados em humanos e requerem que pessoal especializado esteja disponível, o que os tornam caros e pouco acessíveis. \cite[tradução nossa]{cooper-2011-slr}
\end{citacao}


% Apesar disso, Cooper, Holt e Bowden (2011) acreditam que o progresso apresentado por ela nas últimas décadas ainda foi pouco expressivo:
%     Enquanto sistemas de reconhecimento automático da fala avançaram ao
%     ponto de estarem comercialmente disponíveis, o reconhecimento de sinais
%     ainda está em sua infância. Atualmente, todos os serviços comerciais de tra
%     dução de sinais são baseados em humanos e requerem pessoal especializado
%     – o que os tornam caros (COOPER; HOLT; BOWDEN, 2011, tradução nossa).
% Bragg et al. (2019) e Cooper, Holt e Bowden (2011) acreditam que, de uma forma geral, isso está relacionado a fatores como os desafios particulares a essas línguas e a forma com que as pesquisas desenvolvidas na área têm abordando o problema no decorrer das últimas décadas.


Dessa forma, considerando a discussão desenvolvida até aqui, nesta pesquisa posicionaremos a \acrshort{slr} como uma tarefa de \acrshort{nlp}, delimitando seu escopo ao âmbito da linguística e representando os sinais através de seus fonemas. Além disso, aplicaremos o conhecimento disponibilizado pelas ferramentas acima para criar \textit{features} que representem estes fonemas, viabilizando este processo.
Com isso, objetivamos eliminar o escopo pertinente a outras áreas de pesquisa e concentrar a capacidade dos modelos aplicados ao aprendizado das regras e restrições linguísticas da língua de sinais.

Tal estratégia de abordar a linguagem por meio de suas unidades constituintes mínimas é também observada em outras tarefas de \acrshort{nlp}. \citeonline{jurafsky-2022-speech-lang-processing} afirmam que a ideia da palavra falada ser composta por unidades menores da fala é adotada, por exemplo, por algoritmos utilizados em tarefas de reconhecimento de fala e de conversão de texto em voz.
Observe na \autoref{fig:exemplo-waveform-phone} o exemplo ilustrado pelos autores da forma de onda da fala para a sentença em inglês ``\textit{she just had a baby}'' (ou ``ela acabou de ter um bebê'').
Cada trecho é rotulado na linha inferior com suas respectivas partículas mínimas de som (ou ``fones''), as quais são transcritas utilizando-se o ARPAbet\footnote{
    ARPAbet é um alfabeto fonético simples introduzido por \citeonline{shoup-1980-arpabet} que utiliza símbolos ASCII para representar um subconjunto do \acrshort{ipa} que se refere ao idioma inglês-americano. O \acrfull{ipa}, por sua vez, é a representação fonética padrão para a transcrição das línguas ao redor do mundo \cite{jurafsky-2022-speech-lang-processing}.
}. Esse tipo de partículas são comumente utilizadas como \textit{features} de entrada para tarefas envolvendo o processamento da fala.


% The characters that make up the texts we’ve been discussing in this book aren’t just random symbols. They are also an amazing scientific invention: a theoretical model of the elements that make up human speech.
% This implicit idea that the spoken word is composed of smaller units of speech underlies algorithms for both speech recognition (transcribing waveforms into text) and text-to-speech (converting text into waveforms). 
% In this chapter we give a computational perspective on phonetics, the study of the speech sounds used in the languages of the world, how they are produced in the human vocal tract, how they are realized acoustically, and how they can be digitized and processed.
% \cite{jurafsky-2022-speech-lang-processing}

\figura[p. 586]
{fig:exemplo-waveform-phone}
{capitulos/metodologia/imagens/exemplo_waveform_phone}
{width=0.90\textwidth}
{Formas de onda da fala para a sentença ``\textit{she just had a baby}'' (primeira linha) rotuladas com suas respectivas partículas de som transcritas em ARPAbet (linha inferior)}
{jurafsky-2022-speech-lang-processing}



% ... e começo a focar a língua de sinais a partir de sua fonologia (que é justamente a forma como essas línguas são descritas e aprendidas). Os fonemas ou parâmetros fonológicos são componentes mínimos em termos dos quais as línguas são representadas e ensinadas; e que permitem aos modelos começarem a aprender a semântica e regras linguísticas efetivamente, ao invés de lidar com pixels e coordenadas brutas sem significado. 
%     -> Essencial vc destacar a importância de focar nessa abordagem fonológica e os potenciais ganhos de ir nessa direção. Você pode inclusive usar exemplos de outras áreas como o reconhecimento de escrita. Um ponto que talvez você precise tocar é a questão que hoje o NLP está indo na direção de aprendizado a partir de dados, sem formalizar todas as regras de linguagem. Não sei se isso se aplica a seu contexto também, mas pode ser uma pergunta da banca.
%         % TODO avaliar isto: retirar a complexidade dos modelos de SLR torna-os menores e reduz os requisitos computacionais -- como consequência, eles podem ser rodados em browsers ou aparelhos moveis, por exemplo
%         -> potenciais ganhos disso:
%             Retira escopo (e complexidade) de outras áreas e foca na língua
%             Torna os modelos de SLR menos complexos (de partida)



No caso da abordagem proposta aqui, substituiremos essas partículas por alguns dos parâmetros fonológicos introduzidos na \autoref{sec:linguistica}. Uma vez que não há \textit{datasets} disponíveis com esse tipo de representação para as línguas de sinais, nosso primeiro passo consistirá em gerar este \textit{dataset}. Para isso, utilizaremos o \acrshort{asllvd} como \textit{dataset} base e processaremos suas amostras utilizando o OpenPose, o qual nos fornece coordenadas que possibilitam a extração de nossas \textit{features} fonológicas, de nível semântico mais elevado.
Em seguida, adotaremos alguns dos modelos de aprendizagem de máquina comumente utilizados em tarefas de \acrshort{nlp} e os aplicaremos aqui com o intuito de avaliar seu desempenho neste contexto e a eficácia da abordagem proposta. 
% Esse processo será discutido em mais detalhes na seção seguinte.

% Discutiremos em detalhes os processos de seleção dos modelos e criação do \textit{dataset} nas seções a seguir.


% Como dados de entrada, utilizaremos os sinais representados a partir seus parâmetros fonológicos. Contudo, como não há \textit{datasets} disponíveis com esse tipo de representação, nosso primeiro passo para viabilizar a pesquisa consistirá em gerar um novo \textit{dataset} da fonologia da língua de sinais. Para isso, utilizaremos o \acrshort{asllvd} como base e processaremos suas amostras utilizando o OpenPose, o qual fornecerá coordenadas que serão utilizadas para a extração de nossas \textit{features}, que possuem um nível mais elevado. 


% Em seguida, selecionaremos alguns dos modelos de aprendizagem de máquina comumente utilizados em tarefas de \acrshort{nlp} e os aplicaremos aqui com o intuito de avaliar seu desempenho no contexto em questão e a eficácia da abordagem proposta. 
% Isso também será melhor detalhado nas seções adiante.

% Para aplicar a abordagem proposta, selecionaremos alguns modelos de aprendizagem  profunda comumente utilizados em tarefas de \acrshort{nlp} e os aplicaremos ao contexto introduzido aqui afim de avaliar o seu desempenho.


% Discutiremos em detalhes os processos de seleção dos modelos e criação do \textit{dataset} nas seções a seguir.





A \autoref{fig:etapas-metodologia} ilustra essa abordagem, a qual é divida em duas etapas.
No bloco à esquerda, temos o processo envolvido na geração do \textit{dataset}, que inicia-se pelas amostras do \acrshort{asllvd}, contempla a obtenção de coordenadas 3D por meio de ferramentas de \acrshort{cv} e finaliza com a geração do ASL-Phono, que contém os respectivos atributos fonológicos.
No bloco à direita, está a etapa de \acrlong{slr}, que engloba a preparação das \textit{features}, o processamento dessas \textit{features} pelos modelos de aprendizagem de máquina e a classificação dos sinais.
Todas essas etapas serão discutidas em detalhes nas seções a seguir.

\figura
{fig:etapas-metodologia}
{capitulos/metodologia/imagens/etapas_metodologia}
{width=0.95\textwidth}
{Etapas envolvidas na metodologia proposta}
{}












% esta pesquisa possui como proposta adotar uma abordagem centrada na linguística da língua de sinais para realizar o reconhecimento dos sinais
% - ao analisar a linguistica, vemos que ela se divide o estudo da língua em três partes distintas, que descrevem níveis crescentes de significado: vão desde seus menores elementos constituintes (na fonologia), passam pela articulação desses elementos para um nível maior de significado -- que seriam as palavras (na morfologia) e alcançam um nível mais complexo, que articula essas palavras para compor sentenças e estruturas mais complexas (na sintaxe).
% - como percebe-se, a fonologia é o nível mais fundamental para se abordar uma língua sob uma abordagem linguistica. na língua de sinais, ela descreve os parâmetros mínimos que precisam ser compreendidos e dominados antes de avançar para os próximos níveis de significado. 

% - por este motivo, este trabalho adota a hipótese de que este é também o nível mais básico e fundamental para representar e abordar a língua de sinais em técnicas de \acrshort{slr}, o qual deve ser dominado antes de partirmos para explorar níveis de significado mais elevados e complexos. 
% - dessa forma, ao invés de utilizar imagens RGB ou coordenadas brutas do corpo dos indivíduos como entrada para as técnicas aplicadas -- como observa-se em muitos estudos na área --, nesta pesquisa representando os sinais segundo o nível fonológico da língua.

% imagens RGB e coordenadas do corpo humano não carregam semântica por si só. ao contrário, é necessário um nível de interpretação sobre elas para que seja possível inferir, por exemplo, que um conjunto de coordenadas de dedos e corpo se refere a "uma mão com orientação voltada para o corpo numa trajetória para a esquerda". 
% além disso, a quantidade de pixels contidos nessas imagens ou de coordenadas corporais brutas são comumente grandes, mas apenas uma parcela pequena é realmente relevante para os sinais que se deseja interpretar.
% de fato, essa necessidade de lidar com dados brutos tem exigido de pesquisadores um esforço adicional para que suas técnicas interpretem-os corretamente e consigam contornar essa complexidade antes de progredir para níveis semânticos maiores da língua de sinais. Por exemplo, é comum observar técnicas como fluxo óptico, 
% \acrshort{mei}\footnote{
%     \acrlong{mei}: é uma imagem binária, onde a região branca representa onde há movimento ocorrendo e o preto denota a região onde não há movimento \cite{ahad-2012-mhi-for-action}.
% }
% e
% \acrshort{mhi}\footnote{
%     \acrlong{mhi}: expressa a sequência de movimento de forma compacta, em uma única imagem, onde pixels em escala de cinza com intensidade menor descrevem frames de movimentos mais antigos nessa sequência \cite{ahad-2012-mhi}.
% } 
% sendo aplicadas para capturar os movimentos a partir dos pixels dos frames, ou ainda camadas extras sendo adicionadas aos modelos de aprendizagem profunda para lidar com isso, mas que os tornam mais caros e complexos. 


% % - reduz a complexidade de aprender o mapeamento coordenadas -> linguistica / foca no aprendizado da linguistica
% ao prover parâmetros fonológicos ao invés de dados brutos aos modelos de aprendizagem, retiramos o foco do aprender a interpretar pixels ou coordenadas sem significado e fazemos com que ele passe a aprender as relações e regras linguísticas contidas na língua. conforme discutimos na seção XXX, a linguística da língua já apresenta um conjunto de complexidades e regras que precisam ser abordadas para que a \acrshort{slr} avance de forma consistente. englobar a interpretação desses dados brutos como parte do problema faz com que a complexidade seja redobrada e limita a velocidade desses avanços.


% apesar dessa complexidade, muitos desses parâmetros fonológicos poderiam ser computados algebricamente a partir de coordenadas 3d para formatos semanticamente mais elevados de forma mais barata antes de serem ingeridos por algoritmos.



% sendo assim, a abordagem deste trabalho adotará parâmetros fonológicos como dados de entrada para o processo de reconhecimento dos sinais. como pode-se imaginar, atualmente não existem \textit{datasets} disponíveis com esse tipo de parâmetros e, portanto, o primeiro desafio nessa direção foi desenvolver um \textit{dataset} que pudesse suportar essa pesquisa. 

% dentre as alternativas mais viáveis para isso, optamos por derivar um novo \textit{dataset} a partir do \acrfull{asllvd}, o qual é um dos mais relevantes da \acrshort{asl} e foi desenvolvido na Universidade de Boston por \citeonline{athitsos-2008-asllvd} e \citeonline{neidle-2012-asllvd}. % Ele é composto por um vocabulário de 2.745 sinais representados em cerca de 9.763 sequências de vídeos anotadas que, por sua vez, são articuladas por Surdos nativos nessa língua.

% aplicamos um conjunto de processos para computar as representações fonológicas a partir dos frames RGB em 2D das amostras do \acrshort{asllvd}, o que incluiu a criação de uma representação intermediária em 3D sob a qual foi possível aplicar um conjunto de operações algébricas para alcançar nosso objetivo. Discutiremos esse processo em detalhes nas seções a seguir.

% para isso, primeiro foi necessário desenvolver um dataset -- desafios:
% - estabelecer técnicas para computar e representar no nível de parâmetros fonológicos as  coordenadas do corpo humano
%     - mas antes disso: sair de imagens 2D para coordenadas em 3D

% para isso, desafios:
%  - nao ha datasets da fonologia -> construir dataset -> necessário coordenadas 3d
%  - dataset produz imagens 2d -> estimar coordenadas 2d -> reconstruir coordenadas 3d
%  - computar parametros fonologicos a partir das coordenadas 3d
%     - estabelecer funções algebricas


% uma vez que os parâmetros de entrada foram estabelecidos, podemos então prosseguir realizando o reconhecimento dos sinais. para isso, aplicaremos algumas arquiteturas clássicas de modelos sequenciais de aprendizagem profunda, além de um \textit{transformer} -- que é comumente adotado em tarefas de \acrshort{nlp} --. para que assim possamos quantificar a eficácia e estabelecer uma linha de base para a abordagem proposta. este processo será discutido em mais detalhes nas seções seguintes.





% \section{Metodologia}
% \label{sec:metodologia}

% A metodologia aplicada nesta dissertação concentra-se em compreender os desafios atuais da área de pesquisa para assim introduzir uma proposta que suporte avanços futuros coerentes com as necessidades do mundo real.
% As etapas percorridas aqui podem ser sumarizadas como:

% \begin{itemize}
%     \item Revisão do panorama atual da deficiência auditiva e do papel que as línguas de sinais desempenham aqui;
%     \item Revisão do panorama das pesquisas atuais em processamento de língua de sinais e das lacunas que têm limitado progressos mais expressivos na área.
%     \item Desenvolvimento de uma proposta que aborde as lacunas acima, contribuindo para preencher algumas delas e produzindo artefatos que suportem novas pesquisas a evoluir nessa direção;
%     \item Realização de experimentos e análise dos resultados coletados.
% \end{itemize}

% % - análise do panorama atual das línguas de sinais 
% % - análise do panorama atual das pesquisas na área e identificação de principais lacunas para seu avanço
% % - revisão da literatura das línguas de sinais e de sua linguísticas
% % ------
% % - seleção de um conjunto de atributos linguísticos
% % - análise de técnicas algébricas para suportar o processamento de atributos linguísticos selecionados
% % - seleção de modelos sequenciais de aprendizagem de máquina para os experimentos
% % - execução dos experimentos e análise dos resultados coletados






%   Introduzir abordagem (reconhecimento de língua de sinais baseada na linguística)
%       
%   Justificar caminho pelo uso da linguística
%       Abordagens anteriores costumam se limitar à classificação de imagens estáticas ou vídeos
%       Coordenadas 3D não carregam semântica por si só
%           Número de coordenadas é grande e nem todas as mudanças nelas são relevantes ??
%       Linguística considera a dinâmica / semântica da língua (agrupa as entradas em uma granularidade maior, com significado que é compreensível por humanos)
%       Elevar o nível semântico / prover um nível semântico para os algoritmos
%   Benefícios da abordagem (desafios abordados)
%       
%
%   Visão geral do método proposto 
%       Explicar método (reconhecimento de sinais baseado na fonologia)
%           representar informação no nível da fonologia (menor nível da linguística)
%           
%       ASL-Phono
%           Desafios / escassez de datasets diferentes
%           Falar do dataset
%       Modelos sequenciais
%           Transformer: breve introdução (arquitetura e funcionamento)
%           RNNs (GRU, LSTM, etc)
%
%   Experimento
%       Preparação dos dados
%           Transformação das sequências no dataset: frames -> palavras
%           Justificativa??
%       Setup dos modelos (Transformer, LSTM, GRU, etc)
%           Parâmetros
%               Buscar parâmetros (dimensionar os modelos/parâmetros)


