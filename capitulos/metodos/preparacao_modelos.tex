\section{Preparação dos modelos}
\label{sec:metodos-preparacao-modelos}

% SELEÇÃO DOS MODELOS -------------------------------------------

Tomando como referência a discussão introduzida na \autoref{sec:am-ap}, serão adotados nos experimentos deste trabalho três das principais arquiteturas utilizadas em tarefas de \acrfull{nlp}: o \textit{Encoder-Decoder} em uma versão com \acrfull{lstm} e outra com \acrfull{gru}, e também o \textit{Transformer}.

Para estabelecer os parâmetros dessas arquiteturas, as estratégias de otimização e de treinamento, bem como as métricas utilizadas nos experimentos, foram consideradas as discussões apresentadas por \citeonline{goodfellow-2016-deep-learning} e pela \autoref{sec:am}.

Dessa forma, o algoritmo de otimização dos modelos será definido como o \acrfull{sgd} com \textit{momentum} de 0,9 \cite{robbins-2007-stochastic}. Ele será combinado a uma estratégia de redução da taxa de aprendizagem por um fator de 0,2 sempre que o valor do erro médio calculado atingir um platô por 5 épocas seguidas.

A função objetivo (ou função de perda), por sua vez, será a \acrfull{cel} \cite{mitchell-1997-ml}, que é apresentada na \autoref{eqn:cross-entropy-loss}. Nela, \(p\) representa as probabilidades ou pontuações estimadas pelo modelo para as amostras e \(y\) corresponde ao valor correto esperado para essas estimativas:

\begin{equation}
    \label{eqn:cross-entropy-loss}
    L_{\log}(y, p) = -(y \log (p) + (1 - y) \log (1 - p))
\end{equation}


Os dados serão particionados numa proporção de 15\% para o subconjunto de validação, 15\% para o de testes e o restante para o subconjunto treinamento. Os \textit{batches} (ou lotes), por sua vez, possuirão tamanho de 50 amostras.



A seleção dos hiperparâmetros dos modelos foi realizada utilizando-se o algoritmo \textit{Grid Search} (ou busca em grade) com validação cruzada de 5 \textit{folds}. O conjunto de valores de hiperparâmetros utilizados na busca estão apresentados na \autoref{tab:otim-params} e as combinações que melhor reduziram o erro médio para cada modelo foram as seguintes:

% Please add the following required packages to your document preamble:
% \usepackage{multirow}
% \usepackage{graphicx}
% \usepackage[table,xcdraw]{xcolor}
% If you use beamer only pass "xcolor=table" option, i.e. \documentclass[xcolor=table]{beamer}
\begin{table}[]
    \centering
    \caption{Hiperparâmetros otimizados para o \textit{Encoder-Decoder} (\acrshort{lstm}), \textit{Encoder-Decoder} (\acrshort{gru}) e o \textit{Transformer} com os respectivos erros médios calculados (\acrfull{cel})}
    \label{tab:otim-params}
    \resizebox{0.95\textwidth}{!}{%
        \begin{tabular}{lr|rrr}
            \hline
            \rowcolor[HTML]{EFEFEF}
            \multicolumn{2}{l|}{\cellcolor[HTML]{EFEFEF}}                                                             & \multicolumn{3}{c}{\cellcolor[HTML]{EFEFEF}Erro médio (\acrshort{cel})}                                                                                                                                                                                                     \\ \cline{3-5}
            \rowcolor[HTML]{EFEFEF}
            \multicolumn{2}{l|}{\multirow{-2}{*}{\cellcolor[HTML]{EFEFEF}Hiperparâmetro}}                                  & \multicolumn{1}{c|}{\cellcolor[HTML]{EFEFEF}\textbf{Encoder-Decoder (LSTM)}} & \multicolumn{1}{c|}{\cellcolor[HTML]{EFEFEF}\textbf{Encoder-Decoder (GRU)}} & \multicolumn{1}{c}{\cellcolor[HTML]{EFEFEF}\textbf{Transformer}}                                    \\ \hline
            \multicolumn{1}{l|}{}                                                                                     & 0,001                                                                        & \multicolumn{1}{r|}{7,734342}                                               & \multicolumn{1}{r|}{7,263981}                                    & 2,579863                         \\
            \multicolumn{1}{l|}{}                                                                                     & 0,01                                                                         & \multicolumn{1}{r|}{6,000810}                                               & \multicolumn{1}{r|}{\cellcolor[HTML]{FFF5E1}4,253146}            & 0,000009                         \\
            \multicolumn{1}{l|}{\multirow{-3}{*}{Taxa de aprendizagem}}                                               & 0,1                                                                          & \multicolumn{1}{r|}{\cellcolor[HTML]{FFF5E1}4,787887}                       & \multicolumn{1}{r|}{4,313777}                                    & \cellcolor[HTML]{FFF5E1}0,000000 \\ \hline
            \multicolumn{1}{l|}{}                                                                                     & 0,1                                                                          & \multicolumn{1}{r|}{\cellcolor[HTML]{FFF5E1}4,787887}                       & \multicolumn{1}{r|}{\cellcolor[HTML]{FFF5E1}4,253146}            & 0,000005                         \\
            \multicolumn{1}{l|}{\multirow{-2}{*}{Dropout}}                                                            & 0,5                                                                          & \multicolumn{1}{r|}{4,942306}                                               & \multicolumn{1}{r|}{4,336690}                                    & \cellcolor[HTML]{FFF5E1}0,000000 \\ \hline
            \multicolumn{1}{l|}{}                                                                                     & 128                                                                          & \multicolumn{1}{r|}{5,871544}                                               & \multicolumn{1}{r|}{5,102946}                                    & 0,000005                         \\
            \multicolumn{1}{l|}{}                                                                                     & 512                                                                          & \multicolumn{1}{r|}{5,096612}                                               & \multicolumn{1}{r|}{4,509590}                                    & \cellcolor[HTML]{FFF5E1}0,000000 \\
            \multicolumn{1}{l|}{\multirow{-3}{*}{Tamanho embeddings}}                                                 & 1024                                                                         & \multicolumn{1}{r|}{\cellcolor[HTML]{FFF5E1}4,787887}                       & \multicolumn{1}{r|}{\cellcolor[HTML]{FFF5E1}4,253146}            & 0,000000                         \\ \hline
            \multicolumn{1}{l|}{}                                                                                     & 128                                                                          & \multicolumn{1}{r|}{4,926501}                                               & \multicolumn{1}{r|}{4,481058}                                    & 0,000000                         \\
            \multicolumn{1}{l|}{}                                                                                     & 256                                                                          & \multicolumn{1}{r|}{4,822424}                                               & \multicolumn{1}{r|}{4,296523}                                    & 0,000000                         \\
            \multicolumn{1}{l|}{\multirow{-3}{*}{\begin{tabular}[c]{@{}l@{}}Tamanho camadas \\ ocultas\end{tabular}}} & 512                                                                          & \multicolumn{1}{r|}{\cellcolor[HTML]{FFF5E1}4,787887}                       & \multicolumn{1}{r|}{\cellcolor[HTML]{FFF5E1}4,253146}            & \cellcolor[HTML]{FFF5E1}0,000000 \\ \hline
            \multicolumn{1}{l|}{}                                                                                     & 2                                                                            & \multicolumn{1}{r|}{\cellcolor[HTML]{FFF5E1}4,787887}                       & \multicolumn{1}{r|}{\cellcolor[HTML]{FFF5E1}4,253146}            & \cellcolor[HTML]{FFF5E1}0,000000 \\
            \multicolumn{1}{l|}{}                                                                                     & 4                                                                            & \multicolumn{1}{r|}{7,734434}                                               & \multicolumn{1}{r|}{5,433316}                                    & 0,000391                         \\
            \multicolumn{1}{l|}{\multirow{-3}{*}{Nº de camadas}}                                                      & 6                                                                            & \multicolumn{1}{r|}{7,734549}                                               & \multicolumn{1}{r|}{7,178291}                                    & 0,016886                         \\ \hline
            \multicolumn{1}{l|}{}                                                                                     & 4                                                                            & \multicolumn{1}{r|}{}                                                       & \multicolumn{1}{r|}{}                                            & 0,000000                         \\
            \multicolumn{1}{l|}{\multirow{-2}{*}{Nº de cabeças}}                                                      & 8                                                                            & \multicolumn{1}{r|}{\multirow{-2}{*}{N/A}}                                  & \multicolumn{1}{r|}{\multirow{-2}{*}{N/A}}                       & \cellcolor[HTML]{FFF5E1}0,000000 \\ \hline
        \end{tabular}%
    }
    \nomefonte{}
\end{table}

\begin{itemize}
    \item \textit{Encoder-Decoder} com \acrshort{lstm}: taxa de aprendizagem de 0,1; \textit{dropout} de 0,1; \textit{embeddings} com dimensão de 1024; camadas ocultas com dimensão de 512; e utilização de 2 camadas de \acrshort{lstm} no \textit{encoder} e no \textit{decoder}.

    \item \textit{Encoder-Decoder} com \acrshort{gru}: taxa de aprendizagem de 0,01; \textit{dropout} de 0,1; \textit{embeddings} com dimensão de 1024; camadas ocultas com dimensão de 512; e utilização de 2 camadas de \acrshort{gru} no \textit{encoder} e no \textit{decoder}.

    \item \textit{Transformer}: taxa de aprendizagem de 0,1; \textit{dropout} de 0,5; \textit{embeddings} com dimensão de 512; camadas ocultas com dimensão de 512; utilização de 2 camadas e de 8 cabeças de \textit{attention}.
\end{itemize}



O código-fonte utilizado nos experimentos deste trabalho foi disponibilizado através do endereço indicado abaixo\footnote{
    Disponível em \url{https://www.cin.ufpe.br/~cca5/sl-nlp}.
}.
