\subsection{ASL-Skeleton3D}
\label{sec:metodos-datasets-3d}

O ASL-Skeleton3D é um \textit{dataset} intermediário que introduz a representação em coordenadas 3D das amostras do \acrshort{asllvd}. 
Esse tipo de informação fornece detalhes mais precisos acerca do corpo dos indivíduos enquanto eles articulam os sinais, o que possibilita a pesquisadores em \acrshort{slr} extrair diferentes tipos de \textit{features}, explorar novas técnicas, ou ainda derivar outros novos \textit{datasets}.
Além disso, essa representação é genérica e pode ser replicada para outras línguas de sinais.

Para que fosse possível projetar as amostras do \acrshort{asllvd} dentro do espaço tridimensional, foi adotada uma estratégia que consiste em combinar duas perspectivas 2D perpendiculares entre si -- a vista frontal e a vista lateral -- para reconstruir uma perspectiva 3D, conforme ilustra a \autoref{fig:our-strategy-3d}. Com isso, assume-se que enquanto a vista frontal fornece os eixos \(x\) e \(y\), a vista lateral fornecerá a dimensão de profundidade, correspondente ao eixo \( z\).


\begin{figure}[ht!]
    \centering
    \caption{\textmd{Estratégia adotada para representar as amostras no espaço 3D: as perspectivas frontal (\subref{subfig:our-strategy-3d-front}) e lateral (\subref{subfig:our-strategy-3d-side}) são posicionadas perpendicularmente para reconstruir uma perspectiva 3D (\subref{subfig:our-strategy-3d-persp})}}
    \borda[0.80\textwidth]{
        \subcaptionbox{\label{subfig:our-strategy-3d-front}}{
            \includegraphics[height=3.5cm]{capitulos/metodos/imagens/chair_front}
        }%
        \hfill
        \subcaptionbox{\label{subfig:our-strategy-3d-side}}{
            \includegraphics[height=3.5cm]{capitulos/metodos/imagens/chair_side}
        }%
        \hfill
        \subcaptionbox{\label{subfig:our-strategy-3d-persp}}{
            \includegraphics[height=3.5cm]{capitulos/metodos/imagens/chair_perspective}
        }%
    }
    \nomefonte{}
    \label{fig:our-strategy-3d}
\end{figure}


Uma vez definida essa estratégia, foi considerado o processo descrito por \citeonline{amorim-2019-stgcn-sl} para realizar a estimativa dos esqueletos para os indivíduos nas amostras. Esse processo é composto pelas etapas de obtenção de amostras, segmentação dos sinais, estimativa e normalização dos esqueletos, as quais sofreram adaptações no contexto do presente trabalho para acomodar a composição de esqueletos 3D e os desafios encontrados para isso. Serão discutidas a seguir as adaptações realizadas para superar as limitações dos \textit{datasets} existentes:

\begin{enumerate}
    \item \textbf{Obtenção das amostras}: nessa etapa as amostras de vídeos são recuperadas a partir dos servidores do \acrshort{asllvd}, mas no contexto atual, isso é realizado para ambas as câmeras frontal e lateral.

          A saber, existem dois formatos nos quais os vídeos dessas câmeras podem ser disponibilizados: o \textit{mov}, que é compacto e mais fácil de processar; e o \textit{.vid}, que consiste no vídeo bruto, que é maior e mais pesado para baixar e processar.
          Ao analisar as amostras, identificou-se que parte delas possuía ambas câmeras disponíveis nos dois formatos; para outras, cada câmera estava em um formato distinto; contudo, nos piores casos uma das câmeras estava ausente ou corrompida, fazendo com que essas amostras fossem perdidas. Lidar com essa falta de homogeneidade acrescentou uma complexidade inesperada, mas que foi contornada resultando numa perda de apenas 0,16\% das amostras originais -- ou seja, 16 amostras de um total inicial de 9.763.


    \item \textbf{Segmentação dos sinais}: compreende a segmentação das sequências de vídeos do \acrshort{asllvd} em pedaços menores, contendo um único sinal.
          Nessa etapa também reduziu-se a taxa de quadros de 60 para 3 FPS, uma vez que considera-se ser pouco provável a articulação de mais de três movimentos relevantes em um único segundo.
          Isso contribuiu para reduzir em cerca de 20 vezes o número de \textit{frames} a serem processados nas etapas seguintes.


    \item \textbf{Estimativa dos esqueletos 3D}: nesse momento os esqueletos dos sinalizadores são estimados utilizando-se o OpenPose para ambas as câmeras frontal e lateral. Dois esqueletos 2D são obtidos a partir disso:

          \begin{enumerate}
              \item \textit{Esqueleto frontal} (\autoref{subfig:front-side-persp-skeletons-front}), contendo as coordenadas plotadas sobre o eixos \(x\) e \(y\) que correspondem às mesmas coordenadas \(x\) e \(y\) de quando imagina-se o indivíduo representado no espaço tridimensional (vide \autoref{subfig:front-side-persp-skeletons-persp}).

              \item \textit{Esqueleto lateral} (\autoref{subfig:front-side-persp-skeletons-side}), que também é estimado com um par de coordenadas \(x\) e \(y\) mas que, quando posicionados perpendicularmente ao esqueleto frontal (vide \autoref{subfig:front-side-persp-skeletons-persp}), denotam a dimensão de profundidade e correspondem aos eixos \(z\) e \(y\) do indivíduo no espaço tridimensional, respectivamente. Uma vez que \(y\) já foi obtido por meio do esqueleto frontal, ele pode ser então descartado.

          \end{enumerate}

          Dessa forma, são combinadas as coordenadas \(x\), \(y\) e \(z\) obtidas para dar origem ao esqueleto 3D.

          \begin{figure}[ht!]
              \centering
              \caption{\textmd{Os esqueletos 2D frontal (\subref{subfig:front-side-persp-skeletons-front}) e lateral (\subref{subfig:front-side-persp-skeletons-side}) são posicionados perpendicularmente (\subref{subfig:front-side-persp-skeletons-persp}) e combinados para compor o esqueleto 3D final utilizado aqui}}
              \borda[0.85\textwidth]{
                  \subcaptionbox{\label{subfig:front-side-persp-skeletons-front}}{
                      \includegraphics[height=2.5cm]{capitulos/metodos/imagens/asllvd_example_front_skeleton}
                  }%
                  \hfill
                  \subcaptionbox{\label{subfig:front-side-persp-skeletons-side}}{
                      \includegraphics[height=2.5cm]{capitulos/metodos/imagens/asllvd_example_side_skeleton}
                  }%
                  \hfill
                  \subcaptionbox{\label{subfig:front-side-persp-skeletons-persp}}{
                      \includegraphics[height=4.0cm]{capitulos/metodos/imagens/asllvd_front_side_perspective_skeleton}
                  }%
              }
              \nomefonte{}
              \label{fig:front-side-persp-skeletons}
          \end{figure}


    \item \textbf{Normalização dos esqueletos 3D}: por fim, os esqueletos 3D são normalizados para remover variações decorrentes do posicionamento das câmeras e dos corpos dos indivíduos. Isso é importante porque o \acrshort{asllvd} foi capturado através de diferentes seções e envolveu diferentes sinalizadores.

          Adotou-se como referência para essa normalização a largura entre os ombros dos sinalizadores (vide \autoref{fig:shoulders-width}), a qual foi inspirada pela medida antropométrica de \textit{diâmetro biacromial} apresentada por \citeonline{stoudt-1970-skinfolds}. Dessa forma, a largura entre os ombros \(W_{shoulders}\) é definida aqui como a distância euclidiana \(d\) entre as coordenadas do ombro esquerdo \(S_{l}\) e do ombro direito \(S_ {r}\), conforme \autoref{eqn:shoulders-width}:

          \figura
          {fig:shoulders-width} % Label
          {capitulos/metodos/imagens/shoulders_width} % Path
          {height=3.5cm} % Size
          {A largura entre ombros foi utilizada para normalizar as coordenadas nos esqueletos 3D} % Caption
          {} % Citation

          \begin{equation}
              \label{eqn:shoulders-width}
              W_{shoulders} = d\left(S_{l}, S_{r}\right)
          \end{equation}

          Utilizando-se \(W_{shoulders}\) é possível transformar as coordenadas \(K\) do esqueleto 3D em coordenadas normalizadas \(K_{norm}\), conforme \autoref{eqn:normalized-keypoint}:

          % Normalização de pontos-chave:
          \begin{equation}
              \label{eqn:normalized-keypoint}
              K_{norm} = \frac{K}{W_{shoulders}}
          \end{equation}

\end{enumerate}


O \autoref{cod:sample-json-datasetTD} exemplifica uma amostra do ASL-Skeleton3D resultante do processamento acima, bem como suas propriedades. Nele, atributos com sufixo ``dh'' referem-se à \textit{dominant hand} (mão dominante) e aqueles com ``ndh'' referem-se à \textit{non-dominant hand} (mão não-dominante). Observa-se no início do arquivo informações básicas extraídas do \acrshort{asllvd}, como rótulo, nome do indivíduo, sessão, cena, \textit{frames} de início e fim, entre outras. Na propriedade ``\textit{frames}'' estão listados os \textit{frames} para aquela sequência e o esqueleto 3D estimado para cada um deles. Cada esqueleto contém grupos referentes ao ``\textit{body}'' (corpo), ``\textit{face}'' (face), ``\textit{hand left}'' (mão esquerda) e ``\textit{hand right}'' (mão direita) que, por sua vez, contêm propriedades que listam o nome da coordenada, o ``\textit{score}'' (ou acurácia) de sua estimativa e os respectivos eixos \(x\), \(y\) e \(z\).
Por exemplo, ao observar o primeiro índice das propriedades do grupo ``\textit{body}'' percebe-se que ele corresponde à coordenada ``\textit{nose}'' (nariz), apresenta um \textit{score} de aproximadamente 90\% e coordenadas \(x\), \(y\) e \(z\) localizadas em 4,488, 1,696 e 2,872.

\codigo
    {cod:sample-json-datasetTD}
    {capitulos/metodos/codigos/exemplo_3d.m}
    {ASLDataset}
    {Exemplo de amostra do \textit{dataset} ASL-Skeleton3D}
    {}

O \textit{dataset} final e o código-fonte utilizado para o processamento apresentado nesta seção estão disponíveis publicamente na URL listada abaixo\footnote{Disponível em \url{http://www.cin.ufpe.br/~cca5/asl-skeleton3d}}.
