Introduziu-se ao longo deste trabalho uma abordagem que concentra-se nos atributos linguísticos da língua de sinais para realizar o seu reconhecimento de forma automática.
Apesar dessa ser uma estratégia comum entre tarefas de \acrfull{nlp}, ela difere-se da abordagem que tem sido utilizado pela maioria das pesquisas em \acrfull{slr} que, por sua vez, concentra-se principalmente no processamento de dados brutos como pixels RGB ou coordenadas.

% Isso é diferente do que observamos para grande parte das pesquisas realizadas nas últimas décadas na área de \acrfull{slr} principalmente porque retiramos o foco do processamento de dados brutos e problemas que estão sob domínio da \acrfull{cv} e o colocamos no processamento da língua acima como \acrfull{nlp} propriamente dito.

Dessa maneira, o foco foi trocado do domínio da \acrfull{cv} para o domínio do \acrshort{nlp}, permitindo que os modelos aplicados aprendessem acerca da estrutura linguística em vez das relações entre pixels e coordenadas, por exemplo.
Pelos resultados apresentados no \autoref{cap:avaliacao}, percebe-se que essa abordagem apresentou uma eficácia bastante satisfatória quando combinada com o modelo \textit{Transformer} que, por sua vez, é uma arquitetura amplamente utilizada e bem-sucedida ao lidar com tarefas envolvendo linguagens.

% intro
% contributions and takeaways
% difficulties and limitations/challenges?
% future works

