% contribuições e aprendizados -------------
% entre os desafios que esta proposta contribui a superar, temos os seguintes:

Entre as principais contribuições do trabalho, pode-se enumerar que ele:

\begin{itemize}
      \item Aborda a língua de sinais efetivamente como uma língua, ao invés de considerá-la como um conjunto de gestos ou posições estáticas de mãos, conforme também observa-se em muitas pesquisas em \acrshort{slr}.
            Isso posiciona a linguística como o pilar fundamental para essa área e abrange de suas complexidades.
            %   Isso introduz uma mudança de paradigma para o \acrshort{slr} de técnicas baseadas em \acrfull{cv} para outras baseadas no \acrfull{nlp}.

      \item Conscientiza o leitor acerca da linguística e das particularidades da natureza visual das línguas de sinais.
            Isso permite que pesquisadores apropriem-se desses temas e motivem-se a desenvolver novas pesquisas que contribuam com avanços seguindo essa mesma direção.

      \item Traz clareza acerca das principais lacunas que têm limitado progressos expressivos na área de \acrshort{slr} ao longo das últimas décadas, os quais poderiam viabilizar soluções efetivamente aplicáveis ao contexto real da língua de sinais e do Surdo.

      \item Contribui com um novo \textit{dataset} de atributos linguísticos para a língua de sinais, que viabiliza com que novas pesquisas em \acrshort{slr} sejam desenvolvidas seguindo a mesma abordagem que introduzimos aqui.
            Além disso, este trabalho também contribui com um \textit{dataset} de coordenadas 3D que permite pesquisadores derivar outros tipos de representações linguísticas.
            Ambos baseiam-se no \acrshort{asllvd}, que é um importante \textit{dataset} da \acrshort{asl}.

      \item Introduz uma estratégia para computar atributos linguísticos a partir de dados brutos, como \textit{frames} de vídeos ou coordenadas, a qual foi aplicada na geração do \textit{dataset} acima.
            Nela, as complexidades relacionadas à \acrlong{cv} são delegadas para ferramentas que implementam seus respectivos estados da arte.
            Como consequência, essa estratégia é compatível com diferentes \textit{datasets} que, por sua vez, poderiam ser processados e combinados para produzir volumes ainda maiores de dados, que são atualmente escassos para as línguas de sinais.

            % %FIXME: [cz] acho que vc deveria citar como contribuição o seu paper publicado e o submetido --> [cca5] adicionado 
      \item Contribui com dois artigos para a área de \acrshort{slr}:
            em \citeonline{amorim-2019-stgcn-sl}, além do \acrshort{stgcn}, é introduzido o processamento do \acrshort{asllvd} utilizado como base para a obtenção dos \textit{datasets} apresentados neste trabalho;
            em \citeonline{amorim-2022-asl-datasets} (que foi submetido e encontra-se em revisão), são discutidos os detalhes da criação desses \textit{datasets}.

\end{itemize}


% - aborda a língua como uma língua, e com complexidades linguísticas, ao invés de um conjunto de gestos no espaço. isso faz com que ela deixe de ser tratada como uma tarefa de VC e passe a ser abordada como NLP.
% - posiciona a linguística como pilar essencial para o SLR e cria consciência acerca disso para que os leitores e pesquisas futuras na área se apropriem dela e produzam avanços verdadeiramente expressivos (perante o contexto real de uso)
% - realizou uma análise do panorama da área de SLR e identificou as principais lacunas que impedem avanços mais expressivos



% \begin{enumerate}
%     \item aborda o \acrshort{slr} sob uma perspectiva orientada às particularidades da língua de sinais, considerando sua linguística e natureza visual intrínseca como fatores primordiais.

%     \item por ser centrado na linguística da língua de sinais, este trabalho ajuda a criar consciência e reconhecimento dela como sendo um pilar fundamental para novas técnicas na área de \acrshort{slr}.

%     \item é uma abordagem independente de variações nos traços e dimensões corporais, iluminação do ambiente, qualidade das câmeras, etc., bem como separada de complexidades atreladas à detecção partes do corpo ou interpretação de seus movimentos no espaço. isso porque, uma vez que o foco é colocado na fonologia da língua, tais complexidades são terceirizados para algoritmos especializados nesse tipo tarefa. sendo assim, o papel da \acrshort{slr} passa a englobar tão somente a língua de sinais e suas particularidades.

%     \item as anotações contendo os parâmetros fonológicos são geradas automaticamente por meio de expressões algébricas que analisam o corpo no espaço tridimensional. isso pode ser replicado para outros \textit{datasets} (ou em aplicações do mundo real), o que contribuiria para reduzir a necessidade de produção de anotações manuais e, como consequência, as limitações ao combinar \textit{datasets} distintos com o intuito de elevar a quantidade e diversidade de amostras utilizadas pelas pesquisas em \acrshort{slr}.

%     \item este trabalho contribui com dos novos \textit{datasets} para a língua de sinais, os quais podem ser utilizados para derivar outros novos \textit{datasets} ou para continuar evoluindo técnicas centradas na linguística. além disso, por se basearem no \acrshort{asllvd}, estes \textit{datasets} são criados a partir de sinais articulados por indivíduos Surdos nativos na língua.

%     \item apesar do conjunto específico de parâmetros fonológicos contemplados e de lidar apenas com sinais isolados, entendemos que a abordagem proposta é passível de ser estendida para outros parâmetros e também para sinais contínuos.

% \end{enumerate}
