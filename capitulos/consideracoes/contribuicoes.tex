
Entre as principais contribuições do trabalho, pode-se enumerar que ele:

\begin{itemize}
      \item Aborda a língua de sinais efetivamente como uma língua, ao invés de considerá-la como um conjunto de gestos ou posições estáticas de mãos, conforme também observa-se em muitas pesquisas em \acrshort{slr}.
            Isso posiciona a linguística como o pilar fundamental para essa área e abrange suas complexidades.
 
      \item Conscientiza o leitor acerca da linguística e das particularidades da natureza visual das línguas de sinais.
            Isso permite que pesquisadores apropriem-se desses temas e motivem-se a desenvolver novas pesquisas que contribuam com avanços seguindo essa mesma direção.

      \item Traz clareza acerca das principais lacunas que têm limitado progressos expressivos na área de \acrshort{slr} ao longo das últimas décadas, os quais poderiam viabilizar soluções efetivamente aplicáveis ao contexto real da língua de sinais e do Surdo.

      \item Contribui com um novo \textit{dataset} de atributos linguísticos para a língua de sinais, que viabiliza com que novas pesquisas em \acrshort{slr} sejam desenvolvidas seguindo a mesma abordagem que introduzimos aqui.
            Além disso, este trabalho também contribui com um \textit{dataset} de coordenadas 3D que permite pesquisadores derivar outros tipos de representações linguísticas.
            Ambos baseiam-se no \acrshort{asllvd}, que é um importante \textit{dataset} da \acrshort{asl}.

      \item Introduz uma estratégia para computar atributos linguísticos a partir de dados brutos, como \textit{frames} de vídeos ou coordenadas, a qual foi aplicada na geração do \textit{dataset} acima.
            Nela, as complexidades relacionadas à \acrlong{cv} são delegadas para ferramentas que implementam seus respectivos estados da arte.
            Como consequência, essa estratégia é compatível com diferentes \textit{datasets} que, por sua vez, poderiam ser processados e combinados para produzir volumes ainda maiores de dados, que são atualmente escassos para as línguas de sinais.
 
      \item Contribui com dois artigos para a área de \acrshort{slr}:
            em \citeonline{amorim-2019-stgcn-sl}, além do \acrshort{stgcn}, é introduzido o processamento do \acrshort{asllvd} utilizado como base para a obtenção dos \textit{datasets} apresentados neste trabalho;
            em \citeonline{amorim-2022-asl-datasets} (que foi submetido e encontra-se em revisão), são discutidos os detalhes da criação desses \textit{datasets}.

\end{itemize}
