Por outro lado, enxergamos como passos futuros para a abordagem deste trabalho os seguintes itens:

\begin{itemize}
    \item Avaliar a eficácia dessa abordagem para o cenário de sinais contínuos que, por sua vez, é mais complexo por não haver uma segmentação clara entre os sinais -- como no nosso contexto.
          Apesar disso, sinais contínuos são muito mais próximos do contexto real de uso da língua de sinais e, por conta disso, possuem uma grande sinergia com a linha de pesquisa que iniciamos aqui.

    \item Explorar outros atributos linguísticos atualmente não considerados em nosso subconjunto incipiente, para ampliar as \textit{features} do \textit{dataset} introduzido aqui.
          Isso contribuirá para prover mais contexto sobre a estrutura linguística dos sinais aos modelos, os quais precisarão tornar-se gradualmente mais robustos à medida em que nos aprofundamos em cenários mais reais de uso da língua -- como no exemplo dos sinais contínuos.

    \item Produzir um corpus de sinais mais amplo e diversificado para a língua de sinais, utilizando a estratégia de anotação de atributos linguísticos introduzida aqui.
          Conforme discutimos anteriormente, os \textit{datasets} atualmente disponíveis para as línguas de sinais são pequenos e pouco diversificados em comparação àqueles utilizados pelo \acrfull{nlp} para alcançar o estado da arte em outros tipos de linguagens.
          Além disso, técnicas modernas e mais robustas de \acrlong{dl} são orientadas a dados e demandam vocabulários cada vez maiores.

    \item Por fim, avaliar o desempenho da abordagem deste trabalho no cenário online (ou em tempo real) de reconhecimento de sinais.
          Uma vez que é um objetivo de longo prazo contribuir para que a área de \acrshort{slr} consiga prover ferramentas para contextos reais da língua, é importante conhecer sua eficácia também para cenários online afim de assegurar que a direção percorrida pela linha de pesquisa permanece alinhada com esse propósito conforme evolui.
          %   tempo de inferência e potenciais gargalos
\end{itemize}

% próximos passos --------------
% - extender features fonologicas
% - utilizar sinais continuos
% - juntar datasets diferentes para produzir um corpora maior e mais diversificado (a nossa abordagem é genérica e pode ser usada em outros datasets ?)
% - ? pipeline de sistema real / online ?


