\chapter{Considerações finais}
\label{cap:consideracoes-finais}

%\section{XXX}
\label{xxx}

Reconhecimento de linguas faladas e texto X linguas de sinais
NLP: discutir evolução no tempo das abordagens utilizadas (para texto e voz) 
Utilização da fonologia + semântica das palavras para reconhecimento?
Linguas faladas e texto -> hoje são acessíveis, oferecidos em escala comercial e em produtos estáveis
Linguas de sinais -> não há ferramentas em escala comercial e há necessidade de contratar profissionais intérpretes para a tarefa
Torna-se cara e pouco acessível
Modelos sequenciais (aprendizagem de maquina)
Transformer: breve introdução (arquitetura e funcionamento)
RNNs (GRU, LSTM, etc)






% TODO revisar isto
entre os desafios que esta proposta contribui a superar, temos os seguintes:

\begin{enumerate}
    \item aborda o \acrshort{slr} sob uma perspectiva orientada às particularidades da língua de sinais, considerando sua linguística e natureza visual intrínseca como fatores primordiais.
    
    \item por ser centrado na linguística da língua de sinais, este trabalho ajuda a criar consciência e reconhecimento dela como sendo um pilar fundamental para novas técnicas na área de \acrshort{slr}.

    \item é uma abordagem independente de variações nos traços e dimensões corporais, iluminação do ambiente, qualidade das câmeras, etc., bem como separada de complexidades atreladas à detecção partes do corpo ou interpretação de seus movimentos no espaço. isso porque, uma vez que o foco é colocado na fonologia da língua, tais complexidades são terceirizados para algoritmos especializados nesse tipo tarefa. sendo assim, o papel da \acrshort{slr} passa a englobar tão somente a língua de sinais e suas particularidades.
    
    \item as anotações contendo os parâmetros fonológicos são geradas automaticamente por meio de expressões algébricas que analisam o corpo no espaço tridimensional. isso pode ser replicado para outros \textit{dataset}s (ou em aplicações do mundo real), o que contribuiria para reduzir a necessidade de produção de anotações manuais e, como consequência, as limitações ao combinar \textit{dataset}s distintos com o intuito de elevar a quantidade e diversidade de amostras utilizadas pelas pesquisas em \acrshort{slr}.
    
    \item este trabalho contribui com dos novos \textit{dataset}s para a língua de sinais, os quais podem ser utilizados para derivar outros novos \textit{dataset}s ou para continuar evoluindo técnicas centradas na linguística. além disso, por se basearem no \acrshort{asllvd}, estes \textit{dataset}s são criados a partir de sinais articulados por indivíduos Surdos nativos na língua.

    \item apesar do conjunto específico de parâmetros fonológicos contemplados e de lidar apenas com sinais isolados, entendemos que a abordagem proposta é passível de ser estendida para outros parâmetros e também para sinais contínuos.

\end{enumerate}


