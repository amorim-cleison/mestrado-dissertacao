
% ----------------------------------------------------------
% DADOS DO TRABALHO - CAPA e FOLHA DE ROSTO
% Configure os dados do trabalho aqui
% ----------------------------------------------------------
\titulo{\textbf{???:} Uma Abordagem Fonológica para o Reconhecimento de Línguas de Sinais}
\autor{CLEISON CORREIA DE AMORIM}
\local{Recife}
\data{\Year}
\areaconcentracao{\textbf{Área de Concentração}: Aprendizagem de Máquina e Mineração}
\orientador{\textbf{Orientador}: Cleber Zanchettin}
%\coorientador{\textbf{Coorientador (a)}: Texto Texto Texto}

\instituicao{UNIVERSIDADE FEDERAL DE PERNAMBUCO \\ CENTRO DE INFORMÁTICA \\ PROGRAMA DE PÓS-GRADUAÇÃO EM CIÊNCIA DA COMPUTAÇÃO}
\departamento{Centro de Informática}
\programa{Pós-graduação em Ciência da Computação}
\emailprograma{contato@cin.ufpe.br}
\siteprograma{http://www.cin.ufpe.br}

\tipotrabalho{Dissertação de Mestrado}

\preambulo{Trabalho apresentado ao Programa de Pós-Graduação em Ciência da Computação do Centro de Informática da Universidade Federal de Pernambuco, como requisito para obtenção do grau de Mestre em Ciência da Computação.}

\preambuloatadefesa{Dissertação apresentada ao Programa de Pós-Graduação em Ciência da Computação da Universidade Federal de Pernambuco, como requisito para a obtenção do título de Mestre em Ciência da Computação, na data de ?? de Junho de 2022.}
% FIXME: ajustar data

%\input{userlists}
