% -----------------------------------------------------------
% PORTUGUÊS
% -----------------------------------------------------------
\begin{resumo}[Resumo]
  \noindent
  A língua de sinais é uma ferramenta essencial na vida do Surdo, capaz de assegurar seu acesso à comunicação, educação, desenvolvimento cognitivo e socio-emocional.
  Na verdade, ela é a principal força que une essa comunidade e o símbolo de identificação entre seus membros.
  Por outro lado, o número de indivíduos ouvintes que conseguem se comunicar por meio dessa língua atualmente é pequeno e, na prática, isso acaba trazendo alguns obstáculos para o cotidiano do Surdo.
  Tarefas simples como utilizar o transporte público, comprar roupas, ir ao cinema ou obter assistência médica acabam se tornando um desafio por conta dessa comunicação limitada.
  O Reconhecimento de Língua de Sinais é uma das áreas de pesquisa que objetiva desenvolver tecnologias capazes de reduzir essas barreiras linguísticas e facilitar a comunicação entre esses indivíduos.
  Apesar disso, ao analisarmos sua evolução ao longo das últimas décadas, percebemos que seu progresso ainda não é suficiente para disponibilizar soluções efetivamente aplicáveis ao mundo real.
  Isso ocorre principalmente porque várias pesquisas nessa área acabam não se apropriando ou abordando adequadamente as particularidades linguísticas das línguas de sinais, que são decorrentes de sua natureza visual.
  Tendo isso em vista, este trabalho introduz uma abordagem linguística ao reconhecimento computacional da língua de sinais, a qual objetiva estabelecer uma direção capaz de conduzir a avanços mais efetivos para essa área e, consequentemente, contribuir com a superação dos obstáculos hoje enfrentados pelo Surdo.

%FIXME: acho que faltou dizer o que na pratica vc fez. Falta um paragrafo apresentando seu trabalho, o que o leitor vai encontrar se decidir ler


  % Em meio a tantos desafios, a língua de sinais surge como uma ferramenta poderosa que é capaz de assegurar o desenvolvimento cognitivo, facilitar a comunicação, e possibilitar que esses indivíduos obtenham educação e socio-emocional adequado (World Health Organization, 2021).

  % A língua de sinais é a língua utilizada pela maioria dos Surdos em sua vida diária. Muito mais do que isso, ela é a principal força que une essa comunidade e o símbolo de identificação entre seus membros

  % Apesar disso, Bragg et al. (2019), Agência Senado (2019) observam que atualmente ainda são poucos os ouvintes que conseguem se comunicar por meio dessa língua. Isso traz obstáculos adicionais aos Surdos e transforma muitas de suas atividades corriqueiras num grande desafio. 
  % Por exemplo, no transporte público é difícil solicitar ajuda ou ter acesso às instruções divulgadas nos alto-falantes; em lojas, é raro encontrar vendedores preparados para interagir através dessa língua ou que não os trate com preconceito; no cinema, eles apenas podem consumir filmes estrangeiros, uma vez que os nacionais não dispõem de legenda; no serviço de saúde, não são raros os relatos de pacientes que saem de consultas com prescrições médicas erradas porque o médico não entendeu corretamente seus sintomas; entre outras situações

  % Isso deve-se, de um modo geral, a um conjunto de particularidades que as línguas de sinais apresentam quando comparadas às línguas faladas, bem como à forma com que as pesquisas em RLS têm abordado elas, afirmam Bragg et al. (2019), Cooper, Holt e Bowden (2011). Diferentemente das faladas, as línguas sinalizadas possuem uma natureza visual e transmitem significado através de múltiplos canais ao mesmo tempo, como mãos, corpo, face, entre outros de granularidade ainda menor. Essa natureza faz com que sua linguística seja estruturada de uma forma muito específica, demandando que novas técnicas sejam desenvolvidas para abordar tais particularidades. Contudo, um grande número de pesquisas nessa área não aborda essa linguística ou suas complexidades e, como consequência, acabam não produzindo avanços realmente efetivos.

  % Tendo em vista isso, este trabalho busca contribuir com a área de Reconhecimento de Língua de Sinais (RLS) por meio de uma proposta que aborda a língua de sinais através de sua linguística e aplica técnicas de Processamento de Linguagem Natural (PLN) para estabelecer uma direção que possa conduzir a avanços efetivos na área e, consequentemente, ajude a superar alguns dos desafios cotidianos atualmente encarados pelos Surdos.

  \vspace{\onelineskip}

  \noindent
  \textbf{Palavras-chaves}: Língua de Sinais. Linguística. Processamento de Linguagem Natural.
\end{resumo}



% -----------------------------------------------------------
% INGLÊS
% -----------------------------------------------------------
\begin{resumo}[Abstract]
  \begin{otherlanguage*}{english}
    \noindent
    Sign language is an essential resource to ensure access to communication, education, cognitive and socio-emotional development for the Deaf. In fact, it is the main force that unites this community and the symbol of identification among its members.
    On the other hand, the number of hearing individuals who are able to communicate through this language is currently small and, in practice, this imposes some obstacles to the daily life of the Deaf.
    Simple tasks such as using public transportation, shopping for clothes, going to the movies, or getting medical assistance end up becoming challenges because of this limited communication.
    The Sign Language Recognition, in turn, is one of the research areas dedicated to developing technologies capable of reducing these language barriers and facilitating communication between these individuals.
    However, when analyzing its evolution over the last decades, we realize that it has not progressed enough to provide solutions effectively applicable to the real world.
    This is mainly due to the fact that several researches in this field do not appropriate or address the linguistic particularities presented by the sign languages, which stem from their visual nature.
    With this in mind, this work introduces a linguistic approach to sign language recognition that aims to establish a direction that can lead to more effective advances in this research area and, consequently, contribute to overcoming the obstacles faced by the Deaf today.

    \vspace{\onelineskip}

    \noindent
    \textbf{Keywords}: Sign Language. Linguistics. Natural Language Processing.
  \end{otherlanguage*}
\end{resumo}
