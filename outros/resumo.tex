% -----------------------------------------------------------
% PORTUGUÊS
% -----------------------------------------------------------
\begin{resumo}[Resumo]
  \noindent
  A língua de sinais é uma ferramenta essencial na vida do Surdo, capaz de assegurar seu acesso à comunicação, educação e desenvolvimento cognitivo e socio-emocional.
  Na verdade, ela é a principal força que une essa comunidade e o símbolo de identificação entre seus membros.
  Contudo, atualmente o número de indivíduos ouvintes que conseguem se comunicar por meio dessa língua é pequeno e, na prática, isso impõe obstáculos ao cotidiano do Surdo.
  Tarefas simples como utilizar o transporte público, comprar roupas, ir ao cinema ou obter assistência médica acabam tornando-se um desafio por conta dessa limitação.
  O Reconhecimento de Língua de Sinais é uma das áreas de pesquisa que objetiva desenvolver tecnologias capazes de reduzir essas barreiras linguísticas e facilitar a comunicação entre ambos indivíduos.
  Apesar disso, ao analisar sua evolução ao longo das últimas décadas, percebe-se que seu progresso ainda não é suficiente para disponibilizar soluções efetivamente aplicáveis ao mundo real.
  Isso ocorre principalmente porque várias pesquisas na área acabam não apropriando-se ou abordando adequadamente as particularidades linguísticas das línguas de sinais, decorrentes de sua natureza visual.
  Tendo isso em vista, este trabalho apresenta uma abordagem que aplica modelos sequenciais de aprendizagem de máquina para realizar o reconhecimento computacional dos sinais através de seus atributos linguísticos. Além disso, são introduzidos dois novos \textit{datasets} para a língua de sinais, dentre os quais está um \textit{dataset} de atributos linguísticos computados a partir do ASLLVD.
  Com isso, objetiva-se estabelecer uma direção capaz de conduzir a avanços mais efetivos para essa área e, consequentemente, contribuir com a superação dos obstáculos hoje enfrentados pelo Surdo.

  \vspace{\onelineskip}

  \noindent
  \textbf{Palavras-chaves}: Língua de Sinais. Linguística. Processamento de Linguagem Natural.
\end{resumo}



% -----------------------------------------------------------
% INGLÊS
% -----------------------------------------------------------
\begin{resumo}[Abstract]
  \begin{otherlanguage*}{english}
    \noindent
    Sign language is an essential resource to ensure the Deaf to have access to communication, education, as well as to cognitive and socio-emotional development. In fact, it is the main force that unites such community and the key trait that identifies its members.
    On the other hand, the number of hearing individuals who are able to communicate through this language is currently small and, in practice, this imposes obstacles to the daily life of the Deaf.
    Simple tasks like using public transportation, shopping for clothes, going to the movies, or getting medical assistance end up becoming challenges due to such limitation.
    The Sign Language Recognition, in turn, is one of the research areas dedicated to developing technologies that aim to reduce such language barriers and facilitate the communication between these individuals.
    However, when analyzing its evolution over the last decades, we realize that it has not progressed enough to provide solutions effectively applicable to the real world.
    This is mainly because several researches in this field do not appropriate or address the linguistic particularities presented by the sign languages, which stem from their visual nature.
    Considering this problem, the present work introduces an approach that adopts sequential machine learning models to recognize signs through their linguistic attributes. In addition, we introduce two new sign language datasets, among which is a novel dataset of linguistic attributes computed from the ASLLVD.
    Thus, we aim to establish a direction that can lead to more effective advances in this research area and, consequently, contribute to overcoming the obstacles faced by the Deaf today.

    \vspace{\onelineskip}

    \noindent
    \textbf{Keywords}: Sign Language. Linguistics. Natural Language Processing.
  \end{otherlanguage*}
\end{resumo}
