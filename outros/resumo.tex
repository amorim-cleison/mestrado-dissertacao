% -----------------------------------------------------------
% PORTUGUÊS
% -----------------------------------------------------------
\begin{resumo}[Resumo]
  \noindent
  A língua de sinais é uma ferramenta essencial na vida do Surdo, capaz de assegurar seu acesso à comunicação, educação e desenvolvimento cognitivo e socio-emocional.
  Na verdade, ela é a principal força que une essa comunidade e o símbolo de identificação entre seus membros.
  Contudo, atualmente o número de indivíduos ouvintes que conseguem se comunicar por meio dessa língua é pequeno e, na prática, isso impõe obstáculos ao cotidiano do Surdo.
  Tarefas simples como utilizar o transporte público, comprar roupas, ir ao cinema ou obter assistência médica acabam se tornando um desafio por conta dessa limitação.
  O Reconhecimento de Língua de Sinais é uma das áreas de pesquisa que objetiva desenvolver tecnologias capazes de reduzir essas barreiras linguísticas e facilitar a comunicação entre ambos indivíduos.
  Apesar disso, ao analisarmos sua evolução ao longo das últimas décadas, percebe-se que seu progresso ainda não é suficiente para disponibilizar soluções efetivamente aplicáveis ao mundo real.
  Isso ocorre principalmente porque várias pesquisas nessa área acabam não apropriando-se ou abordando adequadamente as particularidades linguísticas das línguas de sinais, decorrentes de sua natureza visual.
  % Tendo isso em vista, este trabalho introduz uma abordagem linguística ao reconhecimento computacional da língua de sinais, a qual objetiva estabelecer uma direção capaz de conduzir a avanços mais efetivos para essa área e, consequentemente, contribuir com a superação dos obstáculos hoje enfrentados pelo Surdo.
  Tendo isso em vista, este trabalho apresenta uma abordagem que aplica modelos sequenciais de aprendizagem de máquina para realizar o reconhecimento computacional dos sinais através de seus atributos linguísticos. Além disso, são introduzidos dois novos \textit{datasets} para a língua de sinais, dentre os quais está um \textit{dataset} de atributos linguísticos computados a partir do ASLLVD.
  Com isso, objetiva-se estabelecer uma direção capaz de conduzir a avanços mais efetivos para essa área e, consequentemente, contribuir com a superação dos obstáculos hoje enfrentados pelo Surdo.

  % %FIXME: [cz] acho que faltou dizer o que na pratica vc fez. Falta um paragrafo apresentando seu trabalho, o que o leitor vai encontrar se decidir ler --. [cca5] feito


  % Em meio a tantos desafios, a língua de sinais surge como uma ferramenta poderosa que é capaz de assegurar o desenvolvimento cognitivo, facilitar a comunicação, e possibilitar que esses indivíduos obtenham educação e socio-emocional adequado (World Health Organization, 2021).

  % A língua de sinais é a língua utilizada pela maioria dos Surdos em sua vida diária. Muito mais do que isso, ela é a principal força que une essa comunidade e o símbolo de identificação entre seus membros

  % Apesar disso, Bragg et al. (2019), Agência Senado (2019) observam que atualmente ainda são poucos os ouvintes que conseguem se comunicar por meio dessa língua. Isso traz obstáculos adicionais aos Surdos e transforma muitas de suas atividades corriqueiras num grande desafio. 
  % Por exemplo, no transporte público é difícil solicitar ajuda ou ter acesso às instruções divulgadas nos alto-falantes; em lojas, é raro encontrar vendedores preparados para interagir através dessa língua ou que não os trate com preconceito; no cinema, eles apenas podem consumir filmes estrangeiros, uma vez que os nacionais não dispõem de legenda; no serviço de saúde, não são raros os relatos de pacientes que saem de consultas com prescrições médicas erradas porque o médico não entendeu corretamente seus sintomas; entre outras situações

  % Isso deve-se, de um modo geral, a um conjunto de particularidades que as línguas de sinais apresentam quando comparadas às línguas faladas, bem como à forma com que as pesquisas em RLS têm abordado elas, afirmam Bragg et al. (2019), Cooper, Holt e Bowden (2011). Diferentemente das faladas, as línguas sinalizadas possuem uma natureza visual e transmitem significado através de múltiplos canais ao mesmo tempo, como mãos, corpo, face, entre outros de granularidade ainda menor. Essa natureza faz com que sua linguística seja estruturada de uma forma muito específica, demandando que novas técnicas sejam desenvolvidas para abordar tais particularidades. Contudo, um grande número de pesquisas nessa área não aborda essa linguística ou suas complexidades e, como consequência, acabam não produzindo avanços realmente efetivos.

  % Tendo em vista isso, este trabalho busca contribuir com a área de Reconhecimento de Língua de Sinais (RLS) por meio de uma proposta que aborda a língua de sinais através de sua linguística e aplica técnicas de Processamento de Linguagem Natural (PLN) para estabelecer uma direção que possa conduzir a avanços efetivos na área e, consequentemente, ajude a superar alguns dos desafios cotidianos atualmente encarados pelos Surdos.

  \vspace{\onelineskip}

  \noindent
  \textbf{Palavras-chaves}: Língua de Sinais. Linguística. Processamento de Linguagem Natural.
\end{resumo}



% -----------------------------------------------------------
% INGLÊS
% -----------------------------------------------------------
\begin{resumo}[Abstract]
  \begin{otherlanguage*}{english}
    \noindent
    Sign language is an essential resource to ensure that the Deaf have access to communication, education, as well as to cognitive and socio-emotional development. In fact, it is the main force that unites this community and an identifying trait among its members.
    On the other hand, the number of hearing individuals who are able to communicate through this language is currently small and, in practice, this imposes obstacles to the daily life of the Deaf.
    Simple tasks like using public transportation, shopping for clothes, going to the movies, or getting medical assistance end up becoming challenges due to such limitation.
    The Sign Language Recognition, in turn, is one of the research areas dedicated to developing technologies that aim to reduce such language barriers and facilitate communication between these individuals.
    However, when analyzing its evolution over the last decades, we realize that it has not progressed enough to provide solutions effectively applicable to the real world.
    This is mainly because several researches in this field do not appropriate or address the linguistic particularities presented by the sign languages, which stem from their visual nature.
    % With this in mind, this work introduces a linguistic approach to sign language recognition that aims to establish a direction that can lead to more effective advances in this research area and, consequently, contribute to overcoming the obstacles faced by the Deaf today.
    Considering this problem, the present work introduces an approach that adopts sequential machine learning models to recognize signs through their linguistic attributes. In addition, we introduce two new sign language datasets, among which is a novel dataset of linguistic attributes computed from the ASLLVD.
    Thus, we aim to establish a direction that can lead to more effective advances in this research area and, consequently, contribute to overcoming the obstacles faced by the Deaf today.

    \vspace{\onelineskip}

    \noindent
    \textbf{Keywords}: Sign Language. Linguistics. Natural Language Processing.
  \end{otherlanguage*}
\end{resumo}
